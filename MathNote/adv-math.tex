\chapter{Advanced Calculus}
\section{文科类作业集}
本节汇总面向文科生的高等数学作业, 因作业过于简单以至于各种传统大语言模型都能不易出错的给出正确解答, 因此这里只做列举而不给出具体解答.
\subsection{Indefinite Integral}
\bq{}{}
若 $f(x)$ 的一个原函数为 $x^3-\ue^x$, 则 $\int f(x) \ud x=$

A. $x^3-e^x+C$\qquad
B. $x^3-e^x$\qquad
C. $3 x^2-e^x$\qquad
D. $3 x^2-e^x+C$
\eq

\bq{}{}
设 $f^{\prime}(x)=2$, 且 $f(0)=0$, 则 $\int f(x) \ud x=$

A. $x^2+x+C$\qquad
B. $x^2+C$\qquad
C. $2 x+C$\qquad
D. $x^2+C x$
\eq

\bq{}{}
\ $\ln x$ 是 $f(x)$ 的一个原函数, 则 $f^{\prime}(x)=$

A. $\frac{1}{x^2}$\qquad
B. $x^2$\qquad
C. $\frac{1}{x}$\qquad
D. $-\frac{1}{x^2}$
\eq

\bq{}{}
下列等式正确的是

A. $\frac{\ud}{\ud x} \int f(x) \ud x=f(x) \ud x$\qquad
B. $\int \ud f(x)=f(x)$\qquad
C. $\int f^{\prime}(x) \ud x=f(x)+C$\qquad
D. $\ud\left(\int f(x) \ud x\right)=f(x)$
\eq

\bq{}{}
\ $
\frac{\ud}{\ud x}\left(\int \ue^{2 x} \ud x\right)=
$

A. $\ue^{2 x}$\qquad
B. $2 \ue^{2 x}$\qquad
C. $\frac{1}{2} \ue^{2 x}$\qquad
D. $\ue^x$
\eq

\bq{}{}
\ $
\int f^{\prime \prime}(x) \ud x=
$


A. $f(x)+c$\qquad
B. $f^{\prime}(x)+c$\qquad
C. $f^{\prime \prime}(x)+c$\qquad
D. 无法确定
\eq

\bq{}{}
设 $f(x)<g(x)$, 则 $\int f(x) \ud x$ $\underline{\qquad}$ $\int g(x) \ud x$

A. $\leq$\qquad
B. $\geq$\qquad
C. $<$\qquad
D. 无法确定
\eq

\bq{}{}
若 $\int \frac{f(x)}{x^2+1} \ud x=\ln \left(x^2+1\right)+c$, 则 $f(x)=$

A. $x^2$\qquad
B. $2 x$\qquad
C. $x$\qquad
D. $\frac{x}{2}$
\eq

\bq{}{}
已知 $f(x)$ 的一个原函数为 $\frac{\sin x}{x}$, 则 $f(x)=$

A. $\frac{\cos x}{x}$\qquad
B. $\frac{\cos x}{x^2}$\qquad
C. $\frac{x \cos x-\sin x}{x^2}$\qquad
D. 无法确定
\eq

\subsubsection{第一类换元积分法}

\bq{}{}
设 $f(x)$ 是连续的偶函数, 则其原函数一定是

A. 偶函数\qquad
B. 奇函数\qquad
C. 非奇非偶函数\qquad
D. 有一个是奇函数
\eq

\bq{}{}
\ $
\int \sin x \ud x=
$

A. $\sec ^2 x$\qquad
B. $\sec ^2 x+{C}$\qquad
C. $-\cos x$\qquad
D. $-\cos x+{C}$
\eq

\bq{}{}
\ $
\int \frac{\ln x}{x} \ud x=
$

A. $\frac{1}{2} \ln ^2 x+C$
B. $\frac{1}{2} \ln ^2 x$
C. $\ln ^2 x+C$
D. $\ln ^2 x$
\eq

\bq{}{}
\ $
\int \frac{\sec^2 x}{\tan x} \ud x=
$

A. $\ln |\tan x|$\qquad
B. $\ln |\tan x|+{C}$\qquad
C. $\ln\tan x$\qquad
D. $\ln\tan x+{C}$
\eq

\bq{}{}
\ $
\int \frac{\arctan^2 x}{1+x^2} \ud x=
$

A. $\frac{\arctan^3 x}{3}+C$\qquad
B. $\frac{\arctan^3 x}{3}$\qquad
C. $\arctan^3 x$\qquad
D. $\arctan^3 x+{C}$
\eq

\bq{}{}
计算
$
\int \frac{1}{x \sqrt{1+\ln x}} {\ud} x
$


A. 答案是 $\sqrt{1+\ln x}+C$.

B. 答案是 $2 \sqrt{1+\ln x}+C$.

C. 答案是 $\frac{1}{2} \sqrt{1+\ln x}+C$.

D. ABC 都不对
\eq

\bq{}{}
判断对错: 若 $f(x)$ 是连续的奇函数, 则 $\int f(x) {\ud} x$ 全部是偶函数. 
\eq

\bq{}{}
若 $f(x)$ 为连续函数, 则 $\int f'(2x)\ud x=$

A. $f(2x)+C$\quad
B. $f(x)+C$\quad
C. $\frac12f(2x)+C$\quad
D. $2f(2x)+C$
\eq

\bq{}{}
若 $f(x)$ 的一个原函数为 $\ue^{-x^2}$, 则 $\int xf(x^2+1)\ud x=$

A. $f(2x)+C$\quad
B. $f(x)+C$\quad
C. $\frac12f(2x)+C$\quad
D. $2f(2x)+C$\quad
E. ABCD 都不对
\eq

\bq{}{}
设 $f(x)$ 的导数为 $\cos x$, 则 $f(x)$ 有一原函数为

A. $1-\sin x$\quad
B. $1+\sin x$\quad
C. $1-\cos x$\quad
D. $1+\cos x$
\eq

\bq{}{}
\ $\int\tan x\ud x=$

A. $\sec^2 x$\quad
B. $\sec^2 x+C$\quad
C. $-\ln|\cos x|$\quad
D. $-\ln|\cos x|+C$
\eq

\bq{}{}
\ $\int \ue^{\sin x}\cos x\ud x=$

A. $\ue^{\sin x}+C$\quad
B. $\ue^{\sin x}$\quad
C. $-\ue^{\sin x}+C$\quad
D. $-\ue^{\sin x}$
\eq

\bq{}{}
\ $\int\frac{\ln x}{x}\ud x=$

A. $\frac12\ln^2x+C$\quad
B. $\frac12\ln^2x$\quad
C. $\ln^2x+C$\quad
D. $\ln^2x$
\eq

\bq{}{}
求不定积分:

1. $\int\frac{x^3}{1+x^2}\ud x$;\quad
2. $\int(3x+2)^{2025}\ud x$.
\eq

\subsubsection{第二类换元积分法}
\bq{D}{}
做变换 $x=\kong$, 可使被积表达式 $\sqrt{x^2-a^2}\ud x$ 中的被积函数转化为三角函数有理式.

A. $x=a\sin t$\quad
B. $x=a\cos t$\quad
C. $x=a\tan t$\quad
D. $x=a\sec t$
\eq

\bq{C}{}
做变换 $x=\kong$, 可使表达式 $\sqrt{a^2+x^2}\ud x$ 转化为三角函数有理式.

A. $x=a\sin t$\qquad
B. $x=a\cos t$\qquad
C. $x=a\tan t$\qquad
D. $x=a\sec t$
\eq

\bq{B}{}
做变换 $x=\kong$, 可使表达式 $\sqrt{x^2-3x+5}\ud x$ 转化为三角函数有理式.

A. $x=\frac32-\frac{\sqrt{11}}{2}\tan t$\qquad
B. $x=\frac32+\frac{\sqrt{11}}{2}\tan t$\qquad
C. $x=\frac{\sqrt{11}}{2}\tan t-\frac32$\qquad
D. $x=\frac{\sqrt{11}}{2}\tan t+\frac32$
\eq

\bq{A}{}
做变换 $x=\kong$, 可使表达式 $\sqrt{4+2x-x^2}\ud x$ 转化为三角函数有理式.

A. $x=1+\sqrt{5}\sin t$\qquad
B. $x=1-\sqrt{5}\sin t$\qquad
C. $x=1+\sqrt{5}\tan t$\qquad
D. $x=1-\sqrt{5}\tan t$
\eq

\bq{C}{}
若 $x=\tan t$, 则 $\sec t=\kong$, $\sin t=\kong$.

A. $\sqrt{x^2+1}$, $\frac1{\sqrt{x^2+1}}$\qquad
B. $\sqrt{x^2-1}$, $\frac{x}{\sqrt{x^2+1}}$\qquad
C. $\sqrt{x^2+1}$, $\frac{x}{\sqrt{x^2+1}}$\qquad
D. $\sqrt{x^2-1}$, $\frac1{\sqrt{x^2+1}}$
\eq

\bq{D}{}
若 $x=a\sin t$, 则 $\tan t=\kong$.

A. $\frac1{\sqrt{a^2-x^2}}$\qquad
B. $\frac{x}{\sqrt{x^2-a^2}}$\qquad
C. $\frac1{x^2-a^2}$\qquad
D. $\frac{x}{\sqrt{a^2-x^2}}$
\eq

\bq{A}{}
若 $x-3=\sec t$, 则 $\sin t=\kong$.

A. $\frac{\sqrt{x^2-6x+8}}{x-3}$\qquad
B. $\frac{\sqrt{x^2-6x+8}}{3-x}$\qquad
C. $\frac{\sqrt{x^2+6x+8}}{x-3}$\qquad
D. $\frac{\sqrt{x^2+6x+8}}{3-x}$
\eq

\bq{AB}{}
做变换 $x=\kong$, 可使表达式 $\sqrt{a^2-x^2}\ud x$ 转化为三角函数的有理式.

A. $x=a\sin t$\qquad
B. $x=a\cos t$\qquad
C. $x=a\tan t$\qquad
D. $x=a\sec t$
\eq

\bq{ACD}{}
计算 $\int\sqrt{1+\ue^x}\ud x$.

A. 答案是 $2\sqrt{1+\ue^x}-\ln\left|\frac{\sqrt{1+\ue^x}+1}{\sqrt{1+\ue^x}-1}\right|+C$\qquad
B. 答案是 $\sqrt{1+\ue^x}+\ln\left|\frac{\sqrt{1+\ue^x}+1}{\sqrt{1+\ue^x}-1}\right|+C$

C. 答案是 $2\sqrt{1+\ue^x}+2\ln(\sqrt{1+\ue^x}-1)-x+C$\qquad
D. 答案是 $2\sqrt{1+\ue^x}+\ln\left|\frac{\sqrt{1+\ue^x}-1}{\sqrt{1+\ue^x}+1}\right|+C$
\eq

\bq{}{}
计算不定积分 $\int\frac1{\sqrt{(x^2+2025)^3}}\ud x$.
\eq

\bq{}{}
计算不定积分 $\int\frac1{\sqrt{x}(1+\sqrt[3]{x})}\ud x$.
\eq

\bq{}{}
计算不定积分 $\int\frac1{x\sqrt{2025+\ln x}}\ud x$.
\eq

\bq{}{}
计算不定积分 $\int\frac1{\sqrt{50x-x^2}}\ud x$.
\eq

\bq{$-2/3$}{}
$\sin\left(\frac32x\right)\ud x=\kong\ud\left(\cos\left(\frac32x\right)\right)$.
\eq

\bq{$-1$}{}
$\frac{x\ud x}{\sqrt{1-x^2}}=\kong\ud\sqrt{1-x^2}$.
\eq

\bq{}{}
求不定积分 $\int\frac{\cos(2x)}{\cos x-\sin x}\ud x$.
\eq

\bq{}{}
求不定积分 $\int \tan^3x\sec x\ud x$.
\eq

\bq{}{}
求不定积分 $\int\frac1{1+\sin x}\ud x$.
\eq

\bq{}{}
求不定积分 $\int\frac1{1+\cos x}\ud x$.
\eq

\subsubsection{分部积分法}

\bq{ADE}{}
设 $f(x)$ 有连续导数, 则下列正确的是

A. $\frac{\ud}{\ud x}\int f(x)\ud x=f(x)$\qquad
B. $\int f'(x)\ud x=f(x)$

C. $\ud\int f(x)\ud x=f(x)$\qquad
D. $\int\frac{\ud}{\ud x}f(x)\ud x=f(x)+C$

E. $\int\ud f^2(x)=f^2(x)+C$
\eq

\bq{}{}
判断题: $\ln|x|$ 与 $\frac1a\ln|ax|$ 是同一个函数的原函数.
\eq

\bq{}{}
求不定积分 $\int x\sin^2x\ud x$.
\eq

\bq{}{}
求不定积分 $\int 2025x\sin^2 x\ud x$.
\eq

\bq{}{}
求不定积分 $\int 2024x\sin^2 x\ud x$.
\eq

\bq{}{}
求不定积分 $\int 2024x\ue^{2x}\ud x$.
\eq

\bq{}{}
求不定积分 $\int 9x\cdot\ue^{13x}\ud x$.
\eq

\bq{}{}
求不定积分 $\int x\cdot\ue^{t x}\ud x$, 其中 $t$ 为非零参数; 如果几分钟的 $t=0$, 积分的结果当如何?
\eq

\bq{}{}
求不定积分 $\int\frac{\arcsin\ue^x}{\ue^x}\ud x$.
\eq

\subsection{ Definite Integral }
\subsection{ Differential Equations }
\section{导数与不等式}
\bq{根据 MSE160248 改编}{}
证明以下结论:

\begin{enumerate}[1.]
	\item 对于任意的 $t\in[0,1]$, 有 $(1-t)\ue^{t}\le\ue^{-t^{2}/2}$.
	\item 对于任意的 $t\ge0$, 函数 $(1-t)\ue^{t}$ 是单调递减的.
\end{enumerate}
\eq
\section{ Definite Integral - web }
\subsection{ Quora }

本节主要记录来自互联网上出现的各种奇形怪状的定积分问题, 其中的解答过程会因为化简到某个简单的现成的问题后而直接给予引用, 而不是再次重复已有的工作, 这在解决问题的篇幅上会更为简略, 从而展示更多可能感兴趣的问题.

\bq{https://www.quora.com/What-is-int\_-0-frac-pi-2-frac-tan-x-4-ln-2-tan-x-pi-2}{}
求定积分 $\int_{0}^{\pi/2}\frac{\tan x}{4\ln^{2}\tan x+\pi^{2}}\ud x$.
\eq
\ba
做变量替换 $t=\ln\left(\tan x\right)$, 则有 $x=\arctan\left(e^{t}\right)$,
所以 $dx=\frac{1}{1+e^{2t}}\cdot e^{t}\ud t$, 原定积分等于
\[
\int_{-\infty}^{+\infty}\frac{e^{t}}{4t^{2}+\pi^{2}}\cdot\frac{e^{t}}{1+e^{2t}}\ud t,
\]
做变量替换 $t\longrightarrow-t$, 则原定积分又等于
\[
\int_{-\infty}^{+\infty}\frac{1}{4t^{2}+\pi^{2}}\cdot\frac{1}{1+e^{2t}}\ud t.
\]
于是原积分等于上面两个积分的平均值, 也即, 原积分等于
\[
\frac{1}{2}\int_{-\infty}^{\infty}\frac{1}{4t^{2}+\pi^{2}}\ud t=\left[\frac{1}{2}\cdot\frac{1}{2\pi}\arctan\left(\frac{2t}{\pi}\right)\right]_{-\infty}^{\infty}=\frac{1}{4}.
\]
\ea
