\chapter{抽象代数}
\bt{}{}
同类置换有相同的循环结构.
\et
\ba
$P,Q,T\in S_n$且$Q=TPT^{-1}$, $Q,P$属同类, 则$Q,P$有相同的循环结构.

$P(v)=(1^{v_1}2^{v_2}\cdots m^{v_m})$, $P=C_1C_2\cdots C_r$, $r=\sum_iv_i$, $n=\sum_iiv_i$.
$Q=TPT^{-1}=\prod_iTC_iT^{-1}=\prod_iC_i'$. 

$T$是一一映射, $(C_i')_i$两两不交, 因$(C_i)_i$两两不交,
$C_i'$与$C_i$同阶, 所以$Q(v)=P(v)$.
\ea

\bq{}{}
在环$R$中, 对于任意的$x\in R$, 都存在$n\in\pNN$, 使得$x=x^{n+1}$, 证明: 对于任意的$y\in R$, $yx^n=x^ny$.
\eq
\ba
先证$x^n$是幂等的, $(x^n)^2=x^n$.

再证, 若$ab=0$, 则$ba=(ba)^{\tilde{n}+1}=b(ab)^{\tilde{n}}a=0$.

再证, $x=x^{n+1}$, 则$yx^n=yx^{2n}$, 所以$(y-yx^n)x^n=0$, 所以$x^n(y-yx^n)=0$, 即$x^ny=x^nyx^n$.

最后, 同上面的做法, 由$x^ny=x^{2n}y$, 有$yx^n=x^nyx^n$, 所以$x^ny=yx^n$.
\ea

\bq{AMM, E.C.Johnsen, D.L. Outcalt and Adil Yaqub, An Elementary Commutativity Theorem For Rings, Vol. 75, No. 3, 288-289}{}
有幺元的非结合环$R$中, 若对于任意的$x,y\in R$, 有$(xy)^2=x^2y^2$, 则$R$是交换环.
\eq
\ba
由$(xy)^2=x^2y^2$, $(x(y+1))^2=x^2y^2+2x^2y+x^2$, 而$(x(y+1))^2=(xy+x)^2=(xy)^2+(xy)x+x(xy)+x^2$, 
所以$xyx+xxy+2x^2y$, 将$x+1$代换$x$的位置, 有$xyx+yx+xxy=2x^2y+xy$, 即得$xy=yx$.

注. 含幺性不可省, $R=\left\{
\begin{pmatrix}
 a & b\\
 0 & 0
\end{pmatrix}
\vert
a,b\in\ZZ
\right\}$, 或$\left\{
\begin{pmatrix}
 0 & 0\\
 0 & 0
\end{pmatrix},
\begin{pmatrix}
 0 & 1\\
 0 & 1
\end{pmatrix},
\begin{pmatrix}
 1 & 0\\
 1 & 0
\end{pmatrix},
\begin{pmatrix}
 1 & 1\\
 1 & 1
\end{pmatrix}
\right\}\le \GF(2)$
有左幺元.

注2. $(xy)^k=x^ky^k$在$k>2$时有反例, $k\ge3$固定, $p$素且满足: $k$奇时, $p\mid k$, $k$偶时, $p\mid\frac{k}{2}$.
\bee
R=\left\{
\begin{pmatrix}
 a & b & c\\
 0 & a & d\\
 0 & 0 & a
\end{pmatrix}:
a,b,c d\in\GF(p)
\right\}\le\GF(p),
\eee
这里的$R$不可交换.
\ea

\bq{}{}
$R$是含幺环, 若对于任意的$x,y\in R$, 存在$m,n\in \NN$, 使得$x^{m+1}y^{n+1}=x^myxy^n$, 则$R$是交换环.
\eq
\ba
$x^m(xy-yx)y^n=0$, $x^l(xy-yx)(y+1)^k=0$, 定义$r=\max\{m,l\}$, 则
\bee
{\color{red}{x^r(xy-yx)y^n=0}},\ x^r(xy-yx)(y+1)^k=0, \ (y,y+1)=1\Longrightarrow (y^n, (y+1)^k)=1.
\eee
所以$(y^{2n-1}, (y+1)^ky^{n-1})=y^{n-1}$, 所以存在$A(y),B(y)$使得$Ay^{2n-1}+B(y+1)^ky^{n-1}=y^{n-1}$, 
所以{\color{red}{$x^r(xy-yx)y^{n-1}=0$}}, 注意到红色部分的$y^{n}$得到了降次, 所以存在$s$满足$x^s(xy-yx)=0$.

再设$(x+1)^t(xy-yx)=0$, 同样辗转相除得到$xy-yx=0$, 即$R$可交换.
\ea

\section{群的定义问题集(MA2008)}
%
\begin{question}{}{}
	Prove that the set
	$$
	G=\left\{3^{k} / 2^{2 k} ; k \in \mathbb{Z}\right\}
	$$
	forms a group with respect to multiplication.
	You may assume that multiplication is associative.
\end{question}

%
\begin{question}{}{}
	Consider the following group of congruence classes of integers modulo 14 with respect to multiplication:
	$$
	G=\left\{[1]_{14},[3]_{14},[5]_{14},[9]_{14},[11]_{14},[13]_{14}\right\}
	$$
	Given that $[1]_{14}$ is the identity of the group, find
	(a) the order of $[13]_{14}$,
	(b) the order of $[3]_{14}$,
	(c) the inverse of $[9]_{14}$.
\end{question}
%
\begin{question}{}{}
	Suppose that $G$ is a group, $x \in G$ is an element of order 3 and $y \in G$ is an element of order $N$. Prove that
	$$
	\left(x^{2} y x\right)^{N}=1
	$$
\end{question}
%
\begin{question}{}{}
	Let $G=\left\{1, a, a^{2}, b, b^{2}, a b, a b^{2}, a^{2} b, a^{2} b^{2}\right\}$ be a group where $a^{3}=b^{3}=1$ and $a b=b a$.
	(a) Find $\left\langle b^{2}\right\rangle$, the subgroup of $G$ generated by $b^{2}$.
	(b) It is given that $H=\left\{1, a b, a^{2} b^{2}\right\}$ is a subgroup of $G$.
	Find the distinct right cosets of $H$.
\end{question}
%
\begin{question}{}{}
	Suppose $G$ is an abelian group with its binary operation written as multiplication. Show that the mapping $\theta: G \rightarrow G$ defined by $g \theta=g^{-1}$ is a homomorphism.
\end{question}
%
\begin{question}{}{}
	Prove that the set of vectors
	$$
	\left\{\left(\begin{array}{c}
		a+b \\
		a-b \\
		a
	\end{array}\right): a, b \in \mathbb{R}\right\}
	$$
	is a vector subspace of $\mathbb{R}^{3}$.
	Calculate a basis and the dimension of this subspace.
\end{question}
%
\begin{question}{}{}
	Reduce the matrix $A=\left(\begin{array}{rrrr}2 & 0 & 1 & 1 \\ 1 & 0 & -1 & 0 \\ 3 & 0 & 0 & 1 \\ 3 & 0 & -3 & 0\end{array}\right)$ to row-echelon form. Give a basis for the row-space of $A$.
\end{question}
%
\begin{question}{}{}
	Prove that the following vector subspaces $V$ and $W$ of $\mathbb{R}^{3}$ are equal:
	$$
	V=\operatorname{span}\left(\left(\begin{array}{l}
		1 \\
		2 \\
		3
	\end{array}\right),\left(\begin{array}{l}
		4 \\
		5 \\
		6
	\end{array}\right)\right), \quad W=\operatorname{span}\left(\left(\begin{array}{l}
		0 \\
		2 \\
		4
	\end{array}\right),\left(\begin{array}{r}
		1 \\
		0 \\
		-1
	\end{array}\right),\left(\begin{array}{l}
		5 \\
		7 \\
		9
	\end{array}\right)\right)
	$$
\end{question}
%
\begin{question}{}{}
	Explain carefully which of the following functions define homomorphisms of vector spaces:
	
	(a) $f: \mathbb{R} \rightarrow \mathbb{R}$ defined by $f(a)=a+1$,
	
	(b) $f: \mathbb{R}^{2} \rightarrow \mathbb{R}$ defined by $f\begin{pmatrix}a \\ b\end{pmatrix}=a+b$,
	
	(c) $f: \mathcal{P}_{2} \rightarrow \mathcal{P}_{5}$ defined by $f\left(a+b x+c x^{2}\right)=\left(1-3 x^{2}+7 x^{3}\right)\left(a+b x+c x^{2}\right)$.
\end{question}
%
\begin{question}{}{}
	State the rank-nullity theorem for a homomorphism of finite-dimensional vector spaces.
	For any linear map $f: M_{2 \times 2}(\mathbb{R}) \rightarrow \mathbb{R}^{2}$, prove there are at least two linearly independent matrices $A \in M_{2 \times 2}(\mathbb{R})$ which satisfy $f(A)=\begin{bmatrix}
		0\\
		0
	\end{bmatrix}$.
\end{question}
%
\begin{question}{}{}
	(a) Suppose $G$ is a finite group and $g$ is an element of $G$. Define the terms
	i. order of $G$,
	ii. order of $g$.
	
	(b) Use Lagrange's theorem to prove the following.
	If $G$ is a finite group and $g \in G$ then the order of $g$ divides the order of $G$.
	
	(c) Now suppose that $G$ is an abelian group and let
	$$
	H_{k}=\left\{x \in G: x^{k}=1_{G}\right\}
	$$
	where $k$ is a positive integer. Show that $H_{k}$ is a subgroup of $G$.
	
	(d) Let $G$ be the group with the following operation table.
	
	\begin{center}
		\begin{tabular}{c|cccccccccccc} 
		& 1 & $a$ & $a^{2}$ & $a^{3}$ & $b$ & $a b$ & $a^{2} b$ & $a^{3} b$ & $b^{2}$ & $a b^{2}$ & $a^{2} b^{2}$ & $a^{3} b^{2}$ \\
		\hline 1 & 1 & $a$ & $a^{2}$ & $a^{3}$ & $b$ & $a b$ & $a^{2} b$ & $a^{3} b$ & $b^{2}$ & $a b^{2}$ & $a^{2} b^{2}$ & $a^{3} b^{2}$ \\
		$a$ & $a$ & $a^{2}$ & $a^{3}$ & 1 & $a b$ & $a^{2} b$ & $a^{3} b$ & $b$ & $a b^{2}$ & $a^{2} b^{2}$ & $a^{3} b^{2}$ & $b^{2}$ \\
		$a^{2}$ & $a^{2}$ & $a^{3}$ & 1 & $a$ & $a^{2} b$ & $a^{3} b$ & $b$ & $a b$ & $a^{2} b^{2}$ & $a^{3} b^{2}$ & $b^{2}$ & $a b^{2}$ \\
		$a^{3}$ & $a^{3}$ & 1 & $a$ & $a^{2}$ & $a^{3} b$ & $b$ & $a b$ & $a^{2} b$ & $a^{3} b^{2}$ & $b^{2}$ & $a b^{2}$ & $a^{2} b^{2}$ \\
		$b$ & $b$ & $a b^{2}$ & $a^{2} b$ & $a^{3} b^{2}$ & $b^{2}$ & $a$ & $a^{2} b^{2}$ & $a^{3}$ & 1 & $a b$ & $a^{2}$ & $a^{3} b$ \\
		$a b$ & $a b$ & $a^{2} b^{2}$ & $a^{3} b$ & $b^{2}$ & $a b^{2}$ & $a^{2}$ & $a^{3} b^{2}$ & 1 & $a$ & $a^{2} b$ & $a^{3}$ & $b$ \\
		$a^{2} b$ & $a^{2} b$ & $a^{3} b^{2}$ & $b$ & $a b^{2}$ & $a^{2} b^{2}$ & $a^{3}$ & $b^{2}$ & $a$ & $a^{2}$ & $a^{3} b$ & 1 & $a b$ \\
		$a^{3} b$ & $a^{3} b$ & $b^{2}$ & $a b$ & $a^{2} b^{2}$ & $a^{3} b^{2}$ & 1 & $a b^{2}$ & $a^{2}$ & $a^{3}$ & $b$ & $a$ & $a^{2} b$ \\
		$b^{2}$ & $b^{2}$ & $a b$ & $a^{2} b^{2}$ & $a^{3} b$ & 1 & $a b^{2}$ & $a^{2}$ & $a^{3} b^{2}$ & $b$ & $a$ & $a^{2} b$ & $a^{3}$ \\
		$a b^{2}$ & $a b^{2}$ & $a^{2} b$ & $a^{3} b^{2}$ & $b$ & $a$ & $a^{2} b^{2}$ & $a^{3}$ & $b^{2}$ & $a b$ & $a^{2}$ & $a^{3} b$ & 1 \\
		$a^{2} b^{2}$ & $a^{2} b^{2}$ & $a^{3} b$ & $b^{2}$ & $a b$ & $a^{2}$ & $a^{3} b^{2}$ & 1 & $a b^{2}$ & $a^{2} b$ & $a^{3}$ & $b$ & $a$ \\
		$a^{3} b^{2}$ & $a^{3} b^{2}$ & $b$ & $a b^{2}$ & $a^{2} b$ & $a^{3}$ & $b^{2}$ & $a$ & $a^{2} b^{2}$ & $a^{3} b$ & 1 & $a b$ & $a^{2}$
	\end{tabular}
	\end{center}

	i. Write down the elements of $H_{4}=\left\{x \in G: x^{4}=1_{G}\right\}$.
	
	ii. Is $H_{k}$ always a subgroup of $G$, even if $G$ is not abelian? Explain your answer.
\end{question}
% answer
\begin{proof}[解]
	$H_4=\{1,a,a^2,a^3,b,ab,a^3b,ab^2,a^3b^2\}$. 其中$(ab)^4=(a^2)^2=1$, $(a^3b)^4=(a^2)^2=1$, $(ab^2)^4=(a^2)^2=1$, $(a^3b^2)^4=a^2=1$.
\end{proof}
%
\begin{question}{}{}
	(a) Suppose that
	$$
	G=\left\{1, a, a^{2}, a^{3}, b, a b, a^{2} b, a^{3} b\right\}
	$$
	where
	$$
	a^{4}=1, \quad a^{2}=b^{2}, \quad b a=a^{3} b
	$$
	and
	$$
	G^{\prime}=\left\{1, c, c^{2}, c^{3}, d, c d, c^{2} d, c^{3} d\right\}
	$$
	where
	$$
	c^{4}=d^{2}=1 \text { and } d c=c d .
	$$
	Construct a non-trivial homomorphism $\theta: G \rightarrow G^{\prime}$.
	Find the image and kernel of your homomorphism.
	
	(b) Consider the linear map $f: \mathcal{P}_{2} \rightarrow \mathcal{P}_{2}$ defined by
	$$
	f\left(a+b x+c x^{2}\right)=(a+b)+(2 b+c) x+3 c x^{2} .
	$$
	Find three eigenvectors of $f$ which form a basis $\mathcal{B}$ of $\mathcal{P}_{2}$.
	Write down the following matrices and the relation between them:
	
	- The change of basis matrix $P$, from $\mathcal{B}$ to the standard basis $\left\{1, x, x^{2}\right\}$.
	
	- The matrix $C$ which represents $f$ with respect to the standard basis.
	
	- The matrix $D$ which represents $f$ with respect to the basis $\mathcal{B}$.
\end{question}
%
\begin{question}{}{}
	Consider the function
	$$
	f: \mathcal{P}_{2} \longrightarrow M_{2 \times 2}(\mathbb{R})
	$$
	defined by
	$$
	f\left(a+b x+c x^{2}\right)=\left(\begin{array}{cc}
		a+b & 0 \\
		a-c & b+c
	\end{array}\right)
	$$
	
	(a) Prove that the nullspace of $f$ has dimension one, and give a basis $\{v\}$ for it.
	
	(b) Prove that the matrices $f(x)$ and $f\left(x^{2}\right)$ form a basis for the image of $f$.
	
	(c) Find the matrix $A$ which represents the linear map $f$ with respect to the standard bases for $\mathcal{P}_{2}$ and $M_{2 \times 2}(\mathbb{R})$.
	Using part (b), write down a basis for the column-space of this matrix.
	
	(d) Extend the set $\left\{f(x), f\left(x^{2}\right)\right\}$ to a basis $\mathcal{B}$ for $M_{2 \times 2}(\mathbb{R})$.
	Write down:
	
	i. the change of basis matrix $P$ from the basis $\mathcal{B}$ to the standard basis for $M_{2 \times 2}(\mathbb{R})$,
	
	ii. The change of basis matrix $Q$ from the basis $\left\{v, x, x^{2}\right\}$, where $v$ is the polynomial you gave in part (a), to the standard basis for $\mathcal{P}_{2}$.
	Show that
	$$
	AQ=P\left(\begin{array}{lll}
		0 & 1 & 0 \\
		0 & 0 & 1 \\
		0 & 0 & 0 \\
		0 & 0 & 0
	\end{array}\right)
	$$
\end{question}
