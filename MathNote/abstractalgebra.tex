\chapter{抽象代数}
\bt{}{}
同类置换有相同的循环结构.
\et
\ba
$P,Q,T\in S_n$且$Q=TPT^{-1}$, $Q,P$属同类, 则$Q,P$有相同的循环结构.

$P(v)=(1^{v_1}2^{v_2}\cdots m^{v_m})$, $P=C_1C_2\cdots C_r$, $r=\sum_iv_i$, $n=\sum_iiv_i$.
$Q=TPT^{-1}=\prod_iTC_iT^{-1}=\prod_iC_i'$. 

$T$是一一映射, $(C_i')_i$两两不交, 因$(C_i)_i$两两不交,
$C_i'$与$C_i$同阶, 所以$Q(v)=P(v)$.
\ea

\bq{}{}
在环$R$中, 对于任意的$x\in R$, 都存在$n\in\pNN$, 使得$x=x^{n+1}$, 证明: 对于任意的$y\in R$, $yx^n=x^ny$.
\eq
\ba
先证$x^n$是幂等的, $(x^n)^2=x^n$.

再证, 若$ab=0$, 则$ba=(ba)^{\tilde{n}+1}=b(ab)^{\tilde{n}}a=0$.

再证, $x=x^{n+1}$, 则$yx^n=yx^{2n}$, 所以$(y-yx^n)x^n=0$, 所以$x^n(y-yx^n)=0$, 即$x^ny=x^nyx^n$.

最后, 同上面的做法, 由$x^ny=x^{2n}y$, 有$yx^n=x^nyx^n$, 所以$x^ny=yx^n$.
\ea

\bq{AMM, E.C.Johnsen, D.L. Outcalt and Adil Yaqub, An Elementary Commutativity Theorem For Rings, Vol. 75, No. 3, 288-289}{}
有幺元的非结合环$R$中, 若对于任意的$x,y\in R$, 有$(xy)^2=x^2y^2$, 则$R$是交换环.
\eq
\ba
由$(xy)^2=x^2y^2$, $(x(y+1))^2=x^2y^2+2x^2y+x^2$, 而$(x(y+1))^2=(xy+x)^2=(xy)^2+(xy)x+x(xy)+x^2$, 
所以$xyx+xxy+2x^2y$, 将$x+1$代换$x$的位置, 有$xyx+yx+xxy=2x^2y+xy$, 即得$xy=yx$.

注. 含幺性不可省, $R=\left\{
\begin{pmatrix}
 a & b\\
 0 & 0
\end{pmatrix}
\large\vert
a,b\in\ZZ
\right\}$, 或$\left\{
\begin{pmatrix}
 0 & 0\\
 0 & 0
\end{pmatrix},
\begin{pmatrix}
 0 & 1\\
 0 & 1
\end{pmatrix},
\begin{pmatrix}
 1 & 0\\
 1 & 0
\end{pmatrix},
\begin{pmatrix}
 1 & 1\\
 1 & 1
\end{pmatrix}
\right\}\le \GF(2)$
有左幺元.

注2. $(xy)^k=x^ky^k$在$k>2$时有反例, $k\ge3$固定, $p$素且满足: $k$奇时, $p\mid k$, $k$偶时, $p\mid\frac{k}{2}$.
\bee
R=\left\{
\begin{pmatrix}
 a & b & c\\
 0 & a & d\\
 0 & 0 & a
\end{pmatrix}:
a,b,c d\in\GF(p)
\right\}\le\GF(p),
\eee
这里的$R$不可交换.
\ea

\bq{}{}
$R$是含幺环, 若对于任意的$x,y\in R$, 存在$m,n\in \NN$, 使得$x^{m+1}y^{n+1}=x^myxy^n$, 则$R$是交换环.
\eq
\ba
$x^m(xy-yx)y^n=0$, $x^l(xy-yx)(y+1)^k=0$, 定义$r=\max\{m,l\}$, 则
\bee
{\color{red}{x^r(xy-yx)y^n=0}},\ x^r(xy-yx)(y+1)^k=0, \ (y,y+1)=1\Longrightarrow (y^n, (y+1)^k)=1.
\eee
所以$(y^{2n-1}, (y+1)^ky^{n-1})=y^{n-1}$, 所以存在$A(y),B(y)$使得$Ay^{2n-1}+B(y+1)^ky^{n-1}=y^{n-1}$, 
所以{\color{red}{$x^r(xy-yx)y^{n-1}=0$}}, 注意到红色部分的$y^{n}$得到了降次, 所以存在$s$满足$x^s(xy-yx)=0$.

再设$(x+1)^t(xy-yx)=0$, 同样辗转相除得到$xy-yx=0$, 即$R$可交换.
\ea