\chapter{math.stackexchange.com}
%%%%%%%%%%%%%%%%%%%%%%%%%%%%%%%%%%%%%%%%%%%%%%%
%%%%%%%%%%%%%%%%%%%%%%%%%%%%%%%%%%%%%%%%%%%%%%%
\bqq[breakable]{1. What Does it Really Mean to Have Different Kinds of Infinities?}{1}
What Does it Really Mean to Have Different Kinds of Infinities?

Can someone explain to me how there can be different kinds of infinities?

I was reading \href{http://en.wikipedia.org/wiki/The_Man_Who_Loved_Only_Numbers}{The man who loved only numbers}
by \href{http://en.wikipedia.org/wiki/Paul_Hoffman_(science_writer)}{Paul Hoffman}
and came across the concept of countable and uncountable infinities,
but they're only words to me.

Any help would be appreciated. 
\bs
Suppose no one ever taught you the names for ordinary numbers. Then
suppose that you and I agreed that we would trade one bushel of corn
for each of my sheep. But there's a problem, we don't know how to
count the bushels or the sheep! So what do we do?

We form a  bijection between the two
sets. That's just fancy language for saying you pair things up by
putting one bushel next to each of the sheep. When we're done we swap.
We've just proved that the number of sheep is the same as the number
of bushels without actually counting.

We can try doing the same thing with infinite sets. So suppose you
have the set of positive integers and I have the set of rational numbers
and you want to trade me one positive integer for each of my rationals.
Can you do so in a way that gets all of my rational numbers?

Perhaps surprisingly the answer is yes! You make the rational numbers
into a big square grid with the numerator and denominators as the
two coordinates. Then you start placing your  bushels
along diagonals of increasing size, \href{http://en.wikipedia.org/wiki/File:Pairing_natural.svg}{see wikipedia}.

This says that the rational numbers are  countable
that is you can find a clever way to count them off in the above fashion.

The remarkable fact is that for the real numbers there's no way at
all to count them off in this way. No matter how clever you are you
won't be able to scam me out of all of my real numbers by placing
a natural number next to each of them. The proof of that is Cantor's
clever \href{http://en.wikipedia.org/wiki/Cantor's_diagonal_argument}{diagonal argument}.
\bm
Fantastic answer! -- Allain Lalonde

I like this so far, but maybe add a bit on uncountable to distinguish
the difference. -- BBischof

That's a really good answer, thanks :D -- fbstj

Why can't lecturers at Uni explain things in this way? -- Sachin
Kainth

In the case of positives and rationals how you match them? How diagonals
become  bushels . Can u explain more on
that figure -- user5507

+1 for  fancy language -- Tyler Langan

Wow, great way to explain it. -- Abhimanyu Pallavi Sudhir

One bushel of corn for each sheep is a little too generous for me.
:P -- BlackAdder

OMG I love the bushels and the sheep. Very great way to explain it.
-- Brian Cheung

I assume with  positive numbers you mean
 positive integers . Because, after all,
$\pi$ is a positive number as well. -- celtschk
\em
\es
\bs
\textbf{How there can be different kinds of infinities?}

This is very simple to see. This is because of:

Claim: A given set $X$ and its power set $P(X)$ can never be in
bijection. 

Proof: By contradiction. Let $f$ be any function from $X$ to $P(X)$.
It suffices to prove $f$ cannot be surjective. That means that some
member of $P(X)$ i.e., some subset of $S$, is not in the image of
$f$. Consider the set:

$T=\{ x\in X: x\not\in f(x) \}.$

For every $x$ in $X$ , either $x$ is in $T$ or not. If $x$ is
in $T$, then by definition of $T$, $x$ is not in $f(x)$, so $T$
is not equal to $f(x)$. On the other hand, if $s$ is not in $T$,
then by definition of $T$, $x$ is in $f(x)$, so again $T$ is not
equal to $f(x)$. Q.E.D.

Thus take any infinite set you like. Then take its power set, its
power set, and so on. You get an infinite sequence of sets of increasing
cardinality(Here I am skipping a little; but a use of the Schroeder-Bernstein
theorem will fix things).

\href{http://en.wikipedia.org/wiki/Hilbert\%27s_paradox_of_the_Grand_Hotel}{Hilbert's Hotel}
is a classic demonstration.
\es
\bs
\href{http://en.wikipedia.org/wiki/Hilbert\%27s_paradox_of_the_Grand_Hotel}{Hilbert's Hotel} is a classic demonstration.
\bm
A really good book on the subject was written by David Wallace Foster, 
\href{http://www.amazon.co.uk/Everything-More-Compact-History-Infinity/dp/0753818825/ref=ntt_at_ep_dpt_10}{Everything and More: A Compact History of Infinity} – FordBuchanan

David Foster Wallace. (RIP :-( ) – Jason S
\em
\es
\bs
A \textbf{countably infinite} set is a set for which you can list
the elements $a_1,a_2,a_3,...$

For example, the set of all integers is countably infinite since I
can list its elements as follows: 

$0,1,-1,2,-2,3,-3,...$ 

So is the set of rational numbers, but this is more difficult to see.
Let's start with the positive rationals. Can you see the pattern in
this listing?

$\frac{1}{1},\frac{1}{2},\frac{2}{1},\frac{1}{3},\frac{2}{2},\frac{3}{1},\frac{1}{4},\frac{2}{3},\frac{3}{2},\frac{4}{1},\frac{1}{5},\frac{2}{4},...$

(Hint: Add the numerator and denominator to see a different pattern.) 

This listing has lots of repeats, e.g. $\frac{1}{1}, \frac{2}{2}$
and $\frac{1}{2}, \frac{2}{4}$. That's ok since I can condense the
listing by skipping over any repeats.

$\frac{1}{1},\frac{1}{2},\frac{2}{1},\frac{1}{3},\frac{3}{1},\frac{1}{4},\frac{2}{3},\frac{3}{2},\frac{4}{1},\frac{1}{5},...$

Let's write $q_n$ for the $n$-th element of this list. Then $0,q_1,-q_1,q_2,-q_2,q_3,-q_3,...$
is a listing of all rational numbers.

A \textbf{countable set} is a set which is either finite or countably
infinite; an \textbf{uncountable set} is a set which is not countable.

Thus, an uncountable set is an infinite set which has no listing of
all of its elements (as in the definition of countably infinite set).

An example of an uncountable set is the set of all real numbers. To
see this, you can use the \textbf{diagonal method}. Ask another question
to see how this works...
\es
\bs
You can see that there are infinitely many natural numbers $1, 2, 3, \ldots$, 
and infinitely many real numbers, such as $0, 1, \pi$, etc. 
But are these two infinities the same?

Well, suppose you have two sets of objects, e.g. people and horses, 
and you want to know if the number of objects in one set is the same as in the other. 
The simplest way is to find a way of corresponding the objects one-to-one. 
For instance, if you see a parade of people riding horses, 
you will know that there are as many people as there are horses, 
because there is such a one-to-one correspondence.

We say that a set with infinitely many things is ``countable", 
if we can find a one-to-one correspondence between the things in this set and the natural numbers.

E.g., the integers are countable: $1 \leftrightarrow 0$, $2 \leftrightarrow -1$, $3 \leftrightarrow 1$, 
$4 \leftrightarrow -2$, $5 \leftrightarrow 2$, etc, gives such a correspondence.

However, the set of real numbers is NOT countable! 
This was proven for the first time by Georg Cantor. 
Here is a proof using the so-called \href{http://en.wikipedia.org/wiki/Cantor's\_diagonal\_argument}{diagonal argument}.
\es
\bs
Infinity is an overloaded term that can mean many things.

One common non-mathematical use of infinity is to refer to 
everything in the universe. This is {\bf not} what mathematicians mean 
when they say infinity. That would be a kin to the set of all sets, 
which is a paradoxical concept that is not part of mathematical discourse.

Mathematicians will use infinity as a way to represent a process that 
continues indefinitely. This is a kin to saying ``take the limit as $n$ goes 
to infinity", which is close to saying ``continue this process indefinitely."

Infinity is also use infinity to talk about size. All sets are either infinite or finite.

The story doesn't stop there. There is something fundamentally different about 
sets like the points on a line, where there are no holes, and sets like the 
integers where there are holes. They are both infinite but one seems denser 
than the other. 

That's where whole countable uncountable thing comes in. Infinite sets 
have a size, but it is not a number in the traditional sense. Its more 
like ``relative size". Bijections are how we determine size for infinite sets, 
which are explained well on this page, so I won't repeat the explanation. 

A more in-depth, but still understandable explanation is given in 
Computability and Logic by George Boolos.
\es
\bs
The basic concept is thus:
\begin{itemize}
 \item A 'countable' infinity is one where you can give each item 
 in the set an integer and 'count' them (even though there are 
 an infinite number of them)
 \item An 'uncountable' infinity defies this. You cannot assign 
 an integer to each item in the set because you will miss items.
\end{itemize}

The key to seeing this is using the 'diagonal slash' argument as 
originally put forward by Cantor. With a countable infinity, you can 
create a list of all the items in the set and assign each one a different natural number. 
This can be done with the naturals (obviously) and the complete range of 
integers (including negative numbers) and even the rational numbers 
(so including fractions). It cannot be done with the reals due to the diagonal slash argument:

\begin{enumerate}[1.]
 \item Create your list of all real numbers and assign each one an integer
 \item Create a real number with the rule that the first digit after the 
 decimal point is different from the first digit of your first number, 
 the second digit is different from the second digit of your second number, 
 and so on for all digits
 \item Try and place this number in your list of all numbers. it can't be the 
 first number, or the second or the third and so on down the list. 
 \item Reductio Ad Absurdium, your number does not exist in your countable 
 list of all real numbers and must be added on to create a new list. 
 The same process can then be done again to show the list still isn't complete.
\end{enumerate}

This shows a difference between two obviously infinite sets and leads to the 
somewhat scary conclusion that there are (at least) 2 different forms of infinity.
\es
\eqq
