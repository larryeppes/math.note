\chapter{math.stackexchange.com}
%%%%%%%%%%%%%%%%%%%%%%%%%%%%%%%%%%%%%%%%%%%%%%%
%%%%%%%%%%%%%%%%%%%%%%%%%%%%%%%%%%%%%%%%%%%%%%%
\bqq[breakable]{1. What Does it Really Mean to Have Different Kinds of Infinities?}{1}
What Does it Really Mean to Have Different Kinds of Infinities?

Can someone explain to me how there can be different kinds of infinities?

I was reading \href{http://en.wikipedia.org/wiki/The_Man_Who_Loved_Only_Numbers}{The man who loved only numbers}
by \href{http://en.wikipedia.org/wiki/Paul_Hoffman_(science_writer)}{Paul Hoffman}
and came across the concept of countable and uncountable infinities,
but they're only words to me.

Any help would be appreciated. 
\bs
Suppose no one ever taught you the names for ordinary numbers. Then
suppose that you and I agreed that we would trade one bushel of corn
for each of my sheep. But there's a problem, we don't know how to
count the bushels or the sheep! So what do we do?

We form a  bijection between the two
sets. That's just fancy language for saying you pair things up by
putting one bushel next to each of the sheep. When we're done we swap.
We've just proved that the number of sheep is the same as the number
of bushels without actually counting.

We can try doing the same thing with infinite sets. So suppose you
have the set of positive integers and I have the set of rational numbers
and you want to trade me one positive integer for each of my rationals.
Can you do so in a way that gets all of my rational numbers?

Perhaps surprisingly the answer is yes! You make the rational numbers
into a big square grid with the numerator and denominators as the
two coordinates. Then you start placing your  bushels
along diagonals of increasing size, \href{http://en.wikipedia.org/wiki/File:Pairing_natural.svg}{see wikipedia}.

This says that the rational numbers are  countable
that is you can find a clever way to count them off in the above fashion.

The remarkable fact is that for the real numbers there's no way at
all to count them off in this way. No matter how clever you are you
won't be able to scam me out of all of my real numbers by placing
a natural number next to each of them. The proof of that is Cantor's
clever \href{http://en.wikipedia.org/wiki/Cantor's_diagonal_argument}{diagonal argument}.
\bm
Fantastic answer! -- Allain Lalonde

I like this so far, but maybe add a bit on uncountable to distinguish
the difference. -- BBischof

That's a really good answer, thanks :D -- fbstj

Why can't lecturers at Uni explain things in this way? -- Sachin
Kainth

In the case of positives and rationals how you match them? How diagonals
become  bushels . Can u explain more on
that figure -- user5507

+1 for  fancy language -- Tyler Langan

Wow, great way to explain it. -- Abhimanyu Pallavi Sudhir

One bushel of corn for each sheep is a little too generous for me.
:P -- BlackAdder

OMG I love the bushels and the sheep. Very great way to explain it.
-- Brian Cheung

I assume with  positive numbers you mean
 positive integers . Because, after all,
$\pi$ is a positive number as well. -- celtschk
\em
\es
\bs
\textbf{How there can be different kinds of infinities?}

This is very simple to see. This is because of:

Claim: A given set $X$ and its power set $P(X)$ can never be in
bijection. 

Proof: By contradiction. Let $f$ be any function from $X$ to $P(X)$.
It suffices to prove $f$ cannot be surjective. That means that some
member of $P(X)$ i.e., some subset of $S$, is not in the image of
$f$. Consider the set:

$T=\{ x\in X: x\not\in f(x) \}.$

For every $x$ in $X$ , either $x$ is in $T$ or not. If $x$ is
in $T$, then by definition of $T$, $x$ is not in $f(x)$, so $T$
is not equal to $f(x)$. On the other hand, if $s$ is not in $T$,
then by definition of $T$, $x$ is in $f(x)$, so again $T$ is not
equal to $f(x)$. Q.E.D.

Thus take any infinite set you like. Then take its power set, its
power set, and so on. You get an infinite sequence of sets of increasing
cardinality(Here I am skipping a little; but a use of the Schroeder-Bernstein
theorem will fix things).

\href{http://en.wikipedia.org/wiki/Hilbert\%27s_paradox_of_the_Grand_Hotel}{Hilbert's Hotel}
is a classic demonstration.
\es
\bs
\href{http://en.wikipedia.org/wiki/Hilbert\%27s_paradox_of_the_Grand_Hotel}{Hilbert's Hotel} is a classic demonstration.
\bm
A really good book on the subject was written by David Wallace Foster, 
\href{http://www.amazon.co.uk/Everything-More-Compact-History-Infinity/dp/0753818825/ref=ntt_at_ep_dpt_10}{Everything and More: A Compact History of Infinity} – FordBuchanan

David Foster Wallace. (RIP :-( ) – Jason S
\em
\es
\bs
A \textbf{countably infinite} set is a set for which you can list
the elements $a_1,a_2,a_3,...$

For example, the set of all integers is countably infinite since I
can list its elements as follows: 

$0,1,-1,2,-2,3,-3,...$ 

So is the set of rational numbers, but this is more difficult to see.
Let's start with the positive rationals. Can you see the pattern in
this listing?

$\frac{1}{1},\frac{1}{2},\frac{2}{1},\frac{1}{3},\frac{2}{2},\frac{3}{1},\frac{1}{4},\frac{2}{3},\frac{3}{2},\frac{4}{1},\frac{1}{5},\frac{2}{4},...$

(Hint: Add the numerator and denominator to see a different pattern.) 

This listing has lots of repeats, e.g. $\frac{1}{1}, \frac{2}{2}$
and $\frac{1}{2}, \frac{2}{4}$. That's ok since I can condense the
listing by skipping over any repeats.

$\frac{1}{1},\frac{1}{2},\frac{2}{1},\frac{1}{3},\frac{3}{1},\frac{1}{4},\frac{2}{3},\frac{3}{2},\frac{4}{1},\frac{1}{5},...$

Let's write $q_n$ for the $n$-th element of this list. Then $0,q_1,-q_1,q_2,-q_2,q_3,-q_3,...$
is a listing of all rational numbers.

A \textbf{countable set} is a set which is either finite or countably
infinite; an \textbf{uncountable set} is a set which is not countable.

Thus, an uncountable set is an infinite set which has no listing of
all of its elements (as in the definition of countably infinite set).

An example of an uncountable set is the set of all real numbers. To
see this, you can use the \textbf{diagonal method}. Ask another question
to see how this works...
\es
\eqq
