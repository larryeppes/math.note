\chapter{math.stackexchange.com}
%%%%%%%%%%%%%%%%%%%%%%%%%%%%%%%%%%%%%%%%%%%%%%%
%%%%%%%%%%%%%%%%%%%%%%%%%%%%%%%%%%%%%%%%%%%%%%%
\section{questions}
\bf 1. What Does it Really Mean to Have Different Kinds of Infinities?

\rm Can someone explain to me how there can be different kinds of infinities?

I was reading ``\href{http://en.wikipedia.org/wiki/The_Man_Who_Loved_Only_Numbers}{The man who loved only numbers}'' 
by \href{http://en.wikipedia.org/wiki/Paul_Hoffman_(science_writer)}{Paul Hoffman} and came accross the concept of countable
and uncountable infinities, but they're only words to me.

Any help would be appreciated\footnote{赏识, 感激}.

\bf Answers 1

\rm Suppose no one ever taught you the names for ordinary numbers. Then suppose that you and I 
agreed that we would trade one bushel of corn for each of my sheep. 
But there's a problem, we don't know how to count the bushels or the sheep! 
So what do we do?

We form a "bijection" between the two sets. That's just fancy language for saying 
you pair things up by putting one bushel next to each of the sheep. When we're 
done we swap. We've just proved that the number of sheep is the same as 
the number of bushels without actually counting.

We can try doing the same thing with infinite sets. So suppose you have the set 
of positive integers and I have the set of 
rational numbers and you want to trade me one positive integer for each of my rationals. 
Can you do so in a way that gets all of my rational numbers?

Perhaps surprisingly the answer is yes! You make the rational numbers into a big square 
grid with the numerator and denominators as the two coordinates. 
Then you start placing your "bushels" along diagonals of increasing size, 
\href{http://en.wikipedia.org/wiki/File:Pairing_natural.svg}{see wikipedia.}

This says that the rational numbers are "countable" that is you can 
find a clever way to count them off in the above fashion.

The remarkable fact is that for the real numbers there's no way at all to count them off in this way. 
No matter how clever you are you won't be able to scam me out of all of my real numbers by placing 
a natural number next to each of them. The proof of that is Cantor's 
clever "\href{http://en.wikipedia.org/wiki/Cantor's_diagonal_argument}{diagonal argument}."

\it I like this so far, but maybe add a bit on uncountable to distinguish the difference.