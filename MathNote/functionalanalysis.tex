\chapter{泛函分析}
\bq{}{}
设$X=\{f(z): f\textrm{在}|z|<1\textrm{内解析, 且在}|z|\le1\textrm{上连续}\}$, 令
\bee
d(f,g)=\max_{|z|=1}|f(z)-g(z)|,\quad f,g\in X.
\eee
求证: $(X, d)$是度量空间.
\eq
\ba
正则性: $d(f,g)=0\Longleftrightarrow\max_{|z|=1}|f(z)-g(z)|=0$, 而由最大模原理对于任意的$|z|\le 1$, 有$|f(z)-g(z)|\le\max_{|z|=1}|f(z)-g(z)|=0$, 
所以$f\equiv g, \forall |z|\le 1$.
\ea

\bq{}{}
求证: $l^1\subsetneqq l^p\subsetneqq l^q\subsetneqq c\subsetneqq l^{\infty}$, ($1<p<q<+\infty$).
\eq

\bq{}{}
求证: $C[a,b]\subsetneqq L^{\infty}[a,b]\subsetneqq L^p[a,b]\subsetneqq L^q[a,b]\subsetneqq L[a,b]$, ($1<q<p<+\infty$).
\eq

\bq{}{}
$x,y\in\mathbb{R}^n$或$l^1$, $x=(\xi_i)$, $y=(\eta_i)$, 
\bee
d_p(x,y)=\left(\sum|\xi_i-\eta_i|^p\right)^{1/p}.
\eee
则$d_p(x,y)\ge d_q(x,y)$, $\forall 1\le p\le q<+\infty$.
\eq

\bq{}{}
设$X(n)=\{P(D)=p_nD^n+p_{n-1}D^{n-1}+\cdots+p_1D+p_0: p_n,\cdots, p_0\in\mathbb{R}\}$, 其中$D=\frac{\ud}{\ud t}$, 令
\bee
d(P(D), Q(D))=\sum_{i=0}^{n}|p_i-q_i|
\eee
其中$Q(D)=\sum_{i=0}^{n}q_iD^i$, $D^0=1$, 求证: $(X(n), d)$是度量空间.
\eq

\bq{}{}
求证: 若$\rho: X\times X\to \mathbb{R}$满足
\begin{enumerate}[(I)]
 \item $\rho(x,x)=0$, $\forall x\in X$, $\rho(x,y)>0$, $\forall x\ne y$.
 \item $\rho(x,y)\le\rho(x,z)+\rho(y,z)$, $\forall x,y,z\in X$.
\end{enumerate}
则$(X, \rho)$是度量空间.
\eq

\bq{}{}
求证: 度量空间中的闭球,
\bee
V[x_0, r]=\{x\in X: d(x, x_0)\le r\}, r>0, x_0\in X
\eee
总是闭集, 球面$S(x_0, r)=\{x\in X: d(x, x_0)=r\}$也是闭集, 试问: 是否恒有$V[x_0, r]=\overline{V(x_0, r)}$, 
及$V[x_0, r]-V(x_0, r)=S(x_0, r)\ne\emptyset$?
\eq
\ba
通常的离散拓扑, 取$r=1$.
\ea

\bq{}{}
设$A\subset(X, d)$, 令$F(x)=d(x, A)=\inf_{y\in A}d(x,y)$, 求证: $F: X\to\mathbb{R}$是连续泛函且一致连续.
\eq
\ba
设$x_1, x_2\in X$, 由$d(x_1, A)\le d(x_1, x_2)+d(x_2, A)\Longrightarrow |F(x_1)-F(x_2)|\le d(x_1, x_2)$, 
于是$F(x)$是$X$上 Lipschitz 连续的, 从而一致连续.
\ea

\bq{}{}
求证: $K: C[a,b]\to C[a,b]$, $x(t)\mapsto\int_{a}^{t}x(\tau)\ud \tau$一致连续.
\eq
\ba
$\|Kx-Ky\|_C=\max_{t\in[a,b]}\left|\int_{a}^{t}(x(\tau)-y(\tau))\ud \tau\right|\le(b-a)\|x-y\|_C$.
\ea

\bq{}{}
求证: 度量空间$(X,d)$中互不相交的闭集$A,B$, 必有互不相交的开集$G, V$使得$A\subset G$, $B\subset V$.
\eq
\ba
$A\subset B^c$故$x\in A$必有$\delta_x>0$使$V(x, \delta_x)\subset B^c$, 所以$A\subset\bigcup_{x\in A}V(x, \delta_x)\subset B^c$, 故
\bee
A\subset\bigcup_{x\in A}V(x, \frac12\delta_x)\subset\overline{\bigcup_{x\in A}V(x, \frac12\delta_x)}\subset\bigcup_{x\in A}V(x, \delta_x)\subset B^c
\eee
令$G=\bigcup_{x\in A}V(x,\frac12\delta_x)$, $V=\left(\bigcup_{x\in A}V(x,\frac12\delta_x)\right)^c$.
\ea
\ba
$F(x)=d(x, A)$是连续泛函, 令$G=\{x\in X; d(x, A)< d(x,B)\}$, 则$G=G^{-1}((-\infty,0))$, 其中$G(x)=d(x, A)-d(x, B)$是$X$上的连续泛函, 从而$G$开且$A\subset G$.
同理$B\subset V=\{x\in X: d(x, B)< d(x, A)\}$开于$X$.
\ea

\bq{}{}
设$Y=\{f; f:(X,d)\to\mathbb{R}\textrm{为有界连续泛函}\}$, 令
\bee
\rho(f,g)=\sup_{x\in X}|f(x)-g(x)|,
\eee
设$x_0$为$X$中固定点, 令$G: X\to(Y,\rho)$, $y\mapsto d(x,y)-d(x, x_0)$, 求证:
$G:(X,d)\to(G(x), \rho)$是等距映射.
\eq
\ba
先证$G(X)$是$(Y,\rho)$的子空间, 然后证$\rho(G(y), G(z))=d(y,z)$.
\ea

\bq{}{}
设$A, M\subset(X,d)$, 求证: $A$在$M$中稠密, 即$\overline{A}\supset M$的充要条件为如下任何一条成立:
\begin{enumerate}[(I)]
 \item $\forall x\in M$的任一邻域$V_{\varepsilon}(x)$有$A\cap V_{\varepsilon}(x)\ne\emptyset$.
 \item $\forall x\in M$, 存在$\{y_n\}\subset A$, 使$y_n\to x$.
 \item 若$A$在$B$中稠密, 且$B$在$M$中稠密.
\end{enumerate}
\eq
\ba
(I).  若$x\in M$, $\forall \varepsilon>0$, $A\cap V_{\varepsilon}(x)\ne \emptyset$, 必有$x\in\overline{A}$, 即$M\subset\overline{A}$.

(II). 因$x\in\overline{A}\Longleftrightarrow\exists y_n\in A$, 使得$y_n\to x$, 故若$\forall x\in M$, $\exists y_n\in A$使得$y_n\to x$, 则$x\in\overline{A}$.

(III). 因$A$在$B$中稠密, 故$\overline{A}\supset B$, $B$在$M$中稠密, 故$\overline{B}\supset M$, 从而$\overline{A}\supset M$.
\ea

\bq{}{}
求证: $l^p$($1\le p<+\infty$)是可分的.
\eq
\ba
首先$\mathbb{R}^n$是可分的, 稠子集为$\mathbb{Q}^n=\{r=(r_1,\cdots,r_n):r_i\in\mathbb{Q},i=1,\cdots n\}$, 
令
\bee
M=\{r=(r_1, \cdots, r_n, 0, 0,\cdots): n\in \mathbb{N}, r_i\in \mathbb{Q}, i=1,2,\cdots, n\}, 
\eee
则$M$是$l^p$的可列子集,
然后证$\overline{M}=l^p$.
\ea

\bq{}{}
求证: $l^{\infty}$是不可分的.
\eq
\ba
令$K=\{x=(x_1, x_2,\cdots, x_n, \cdots): x_i\textrm{或为}0\textrm{或为}1\}$, 则$K$不可数且$K$中人两个不同元素$x\ne y$, 
有$d_{\infty}(x,y)=1$, 若$l^{\infty}$可分且其稠子集为$B$, $B$可数, 作集类
\bee
\left\{V\left(x,\frac13\right): x\in K\right\}.
\eee
则它为不可列集, 且两两不交, 但对于任意的$x\in K$必有$B\cap V\left(x,\frac13\right)\ne\emptyset$, 从而$B$不可数, 矛盾.
\ea

\bq{}{}
设$(X, d)$是离散度量空间, 求证: 任一映射$F: (X, d)\to(Y, \rho)$都是一致连续的.
\eq

\bq{}{}
设$(X, d)$为度量空间, $Y=\{f; f: X\to \mathbb{R}\}\textrm{为 Lipschitz 连续泛函}\}$, 即
\bee
f\in Y\Longleftrightarrow\textrm{存在常数}k\textrm{使}|f(x)-f(y)|\le kd(x,y), \forall x,y\in X.
\eee
\begin{enumerate}[(I)]
 \item $Y$是线性空间.
 \item 问$\sigma(f,g)=\inf\{k:|f(x)-g(x)-f(y)+g(y)|\le kd(x,y), \forall x,y\in X\}$是否为$Y$上的度量?
 \item 设$Y_0=\{f: f\in Y, f(x_0)=0\}$, 其中$x_0$为$X$中定点, 问$\sigma$是否为$Y_0$上的度量.
\end{enumerate}
\eq
\ba
(II). $\sigma(f,g)=0$未必有$f=g$, 设$f\in Y$, 令$g=f(x)+c$($c$非零常数), 则$g\in Y$但$\sigma(f,g)=0$, 因此$\sigma$不是$Y$上的度量(称为$Y$上的拟度量).

(III). 因
\bee
\sigma(f,g)=\sup_{x\ne y}\frac{|f(x)-g(x)-f(y)+g(y)|}{d(x,y)},
\eee
故$\sigma(f,g)=0\Longleftrightarrow f(x)-g(x)=f(y)-g(y)=f(x_0)-g(x_0)$, 对于任意的$x,y\in X$成立, 即$f(x)\equiv g(x)$, 对于任意的$x\in X$, 
即$f=g$, 从而易得$\sigma$是$Y_0$上的度量.
\ea

\bq{}{}
设$f_n, f\in C[a,b]$且$d(f_n, f)\to0$, $t_n\in[a,b]$, $t_n\to t_0$, 求证:
\bee
\lim_{n\to\infty}f_n(t_n)=f(t_0).
\eee
\eq

\bq{}{}
设$f:X\to X$连续, $(X, d)$为度量空间, 设$X\times X$中度量为
\bee
d((x_1, y_1),(x_2, y_2))=d(x_1,x_2)+d(y_1,y_2),\quad (x_i,y_i)\in X\times X.
\eee
定义$X\times X$的对角线集为
\bee
\Delta=\{(x,x): x\in X\},
\eee
令
\bee
g:\Delta\to \mathrm{Graph}(f),\quad (x,x)\mapsto (x,f(x)).
\eee
求证: $g$连续, 可逆, $g^{-1}$也连续.
\eq
\ba
$x_n, x\in X$, 因$(x_n, x_n)\to (x, x)\Longleftrightarrow d(x_n, x)\to 0$, 
所以由$f: X\to X$连续, 故$(x_n, x_n)\to(x,x)$时, $d(x_n,x)\to0$, 从而$d(f(x_n),f(x))\to0$, 从而
\bee
d((x_n, f(x_n)), (x,f(x)))=d(x_n, x)+d(f(x_n),f(x))\to 0
\eee
所以$g$连续. 然后证$g$是双射, 所以可逆.
\bee
g^{-1}: \mathrm{Graph}(f)\to \Delta,\quad (x,f(x))\mapsto (x,x),
\eee
于是
\bee
(x_n, f(x_n))\to(x,f(x))
  \Longleftrightarrow d((x_n, f(x_n)), (x,f(x)))\to 0
  \Longleftrightarrow d(x_n, x)\to0\ \textrm{且}\ d(f(x_n), f(x))\to 0
  \Longrightarrow (x_n, x_n)\to(x,x)
\eee
所以$g^{-1}$连续.
\ea

\bq{}{}
求证: 同胚映射$F:(X,d)\to(Y,\rho)$, 若$(X,d)$可分, 则$(Y,\rho)$可分.
\eq

\bq{}{}
设 Hilbert 立方体 $A=\{x=(\xi_1,\cdots,\xi_n,\cdots): |\xi_n|\le\frac{1}{n}, \forall n\}$, 
求证$A$闭于$l^2$.
\eq
\ba
显然$A\subset l^2$, 设$x_0\in \overline{A}$, 
则存在$x_k=\left(\xi_i^{(k)}\right)\in A$使得$d(x_k, x_0)\to 0$,
即$\forall i$, $\left|\xi_i^{(k)}-\xi_i^{(0)}\right|\to 0$, 
从而$\left|\xi_i^{(0)}\right|\le\frac{1}{i}$, 
于是$x_0\in A$. 其实, $A$闭于$l^p$($1<p\le+\infty$).
\ea

\bq{http://math.stackexchange.com/questions/1087885}{}
定义: 设$(S,\rho)$是一距离空间, $T: S\to S$, 若存在$\beta\in(0,1)$使对于任意$x,y\in S$, 有$\rho(Tx, Ty)\le\beta\rho(x,y)$称$T$为压缩映射.
\bt{Blackwell's 压缩映射的充分条件}{}
让$X\subset \mathbb{R}^l$, $B(X)$是带有上确界范数的有界函数 $f: X\to \mathbb{R}$ 的全体组成的空间, 让 $T: B(X)\to B(X)$ 满足
\begin{enumerate}[1)]
 \item (单调性), $f,g\in B(X)$ 且若 $f\le g$, $\forall x\in X$ 则 $Tf\le Tg$, $\forall x\in X$.
 \item (discounting) 存在 $\beta\in(0,1)$ 使 $\forall f\in B(X)$, $a\ge 0$, $x\in X$ 有 $[T(f+a)](x)\le(Tf)(x)+\beta a$, 其中 $(f+a)(x):= f(x)+a$.
\end{enumerate}
则 $T$ 是模 $\beta$ 的压缩映射.
\et
\eq
\ba
若 $f\le g$, $\forall x\in X$, 则 $\forall f,g\in B(X)$, $f\le g+\|f-g\|$, 从而
\bee
Tf\le T(g+\|f-g\|)\le Tg+\beta \|f-g\|.
\eee
交换 $f,g$的位置, 从而有 $\|Tf-Tg\|\le\beta\|f-g\|$.
\ea

\bq{http://math.stackexchange.com/questions/1125691}{}
让 $A=(a_{ij})$ 为一 $n\times n$ 实矩阵, $b\in\mathbb{R}^n$, 并有 $\|A-I\|_2<1$, 证明映射 $T:\mathbb{R}^n\to\mathbb{R}^n$, $x\mapsto x-Ax+b$ 是压缩映射,
其中距离定义为普通 Euclid 距离.
\eq
\ba
$Tx=b-(A-I)x$, 所以 $Tx-Ty=(A-I)(y-x)$, 又因为 $\|A-I\|_2<1$, 所以存在 $\alpha\in(0,1)$ 使 $\|A-I\|_2\le \alpha<1$, 
\bee
\|Tx-Ty\|\le \|A-I\|_2\cdot\|y-x\|<\alpha\|y-x\|.
\eee
所以 $T$ 是压缩映射.
\ea

\bq{http://math.stackexchange.com/questions/1124660}{}
让$(S,\rho)$为一紧距离空间, 映射$T:S\to S$使对于任意$x\ne y$, 有$\rho(Tx, Ty)<\rho(x,y)$.
证明: $\phi(x)=\rho(x,Tx)$连续, 且$T$有唯一不动点.
\eq
\ba
因$|\rho(x, Tx)-\rho(y,Ty)|\le \rho(x,y)+\rho(Tx, Ty)<2\rho(x,y)$, 所以$\phi(x)$连续. 任取$x\in S$, 构造点列$\{T^nx\}$, 并利用$S$的紧性.
唯一性用反证法.
\ea
