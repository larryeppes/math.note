\chapter{Inequality}
\section{Elementary Inequality}
%%%%%%%%%%%%%%%%%%%%%%%%%%%%%%%%%%%%%%%%%%%%%%%%%%%%%%%%
%%%%%%%%%%%%%%%%%%%%%%%%%%%%%%%%%%%%%%%%%%%%%%%%%%%%%%%%
\bq{}{}
求证: 
\bee
\paren{1+\frac11}\paren{1+\frac14}\paren{1+\frac17}\cdots\paren{1+\frac1{3n-2}}>\sqrt[3]{4}.
\eee
\eq
\ba
先证$\prod_{m=1}^n\paren{1+\frac{1}{3m-2}}>\sqrt[3]{3n-2}$.
\ea

\bq{}{}
设$f(x)=x^n+a_1x^{n-1}+\cdots+a_{n-1}x+1$有$n$个实根, 且系数$a_1,\cdots,a_{n-1}$都是非负的. 证明$f(2)\ge3^n$.
\eq

\bq{Ho Joo Lee}{}
$a,b,c$是三个正实数, 证明:
\bee
a+b+c\le\frac{a^2+bc}{b+c}+\frac{b^2+ca}{c+a}+\frac{c^2+ab}{a+b}.
\eee
\eq
\ba
不等式等价于$\sumcyc\frac{(a-c)(a-b)}{b+c}\ge0$, 用Shur不等式\ref{th:GSI}即得.
\ea

\bq{}{}
设正实数$a,b,c$的和为$3$. 证明:
\bee
\frac1a+\frac1b+\frac1c\ge\frac3{2a^2+bc}+\frac3{2b^2+ac}+\frac3{2c^2+ab}.
\eee
\eq
\ba
不等式等价于$\sumcyc\frac{(a-c)(a-b)}{2a^3+abc}\ge0$, 用Shur不等式\ref{th:GSI}即得.
\ea

\bq{Italian Winter Camp 2007}{}
设$a,b,c$为三角形的三条边长, 证明:
\bee
\frac{\sqrt{a+b-c}}{\sqrt{a}+\sqrt{b}-\sqrt{c}}+\frac{\sqrt{b+c-a}}{\sqrt{b}+\sqrt{c}-\sqrt{a}}+\frac{\sqrt{c+a-b}}{\sqrt{c}+\sqrt{a}-\sqrt{b}}\le 3.
\eee
\eq
\ba
\begin{align*}
\sumcyc\left(1-\frac{\sqrt{a+b-c}}{\sqrt{a}+\sqrt{b}-\sqrt{c}}\right)\ge0
  & \Longleftrightarrow\sumcyc\frac{\sqrt{a}+\sqrt{b}-\sqrt{c}-\sqrt{a+b-c}}{\sqrt{a}+\sqrt{b}-\sqrt{c}}\ge0\\
  & \Longleftrightarrow\sumcyc\frac{\sqrt{ab}-\sqrt{c(a+b-c)}}{(\sqrt{a}+\sqrt{b}-\sqrt{c})(\sqrt{a}+\sqrt{b}+\sqrt{c}+\sqrt{a+b-c})}\ge0\\
  & \Longleftrightarrow\sumcyc\frac{(c-a)(c-b)}{S_c}\ge0.
\end{align*}
其中
\bee
S_c=(\sqrt{a}+\sqrt{b}-\sqrt{c})(\sqrt{a}+\sqrt{b}+\sqrt{c}+\sqrt{a+b-c})(\sqrt{ab}+\sqrt{c(a+b-c)}),
\eee
不妨设$b\ge c$, 则
\begin{align*}
\sqrt{a}+\sqrt{b}-\sqrt{c}&\ge\sqrt{c}+\sqrt{a}-\sqrt{b}\\
\sqrt{a}+\sqrt{b}+\sqrt{c}+\sqrt{a+b-c}&\ge\sqrt{a}+\sqrt{b}+\sqrt{c}+\sqrt{c+a-b}\\
\sqrt{ab}+\sqrt{c(a+b-c)}&\ge\sqrt{ca}+\sqrt{b(c+a-b)}.
\end{align*}
所以$S_c\ge S_b$, 由Schur不等式推论\ref{co:GSIC}即得.
\ea

\bq{APMO 2007}{}
正实数$x,y,z$满足$\sqrt{x}+\sqrt{y}+\sqrt{z}=1$. 证明:
\bee
\frac{x^2+yz}{x\sqrt{2(y+z)}}+\frac{y^2+zx}{y\sqrt{2(z+x)}}+\frac{z^2+xy}{z\sqrt{2(x+y)}}\ge1.
\eee
\eq
\ba
先用幂平均不等式证明$\sumcyc\sqrt{x+y}\ge\sqrt{2}$, 原不等式等价于$\sumcyc\frac{x^2+yz}{x\sqrt{y+z}}\ge\sumcyc\sqrt{x+y}\ge\sqrt{2}$.
而不等式左边部分又等价于证明$\sumcyc\frac{(x-y)(x-z)}{x\sqrt{y+z}}\ge0$. 当$x\ge y\ge z$时, 有$y\sqrt{x+z}\ge z\sqrt{x+y}$, 
利用Schur不等式推论\ref{co:GSIC}即得.
\ea

\bq{}{}
设$a,b,c$为正实数, 证明:
\bee
\frac{a^2+2bc}{(b+c)^2}+\frac{b^2+2ac}{(a+c)^2}+\frac{c^2+2ab}{(a+b)^2}\ge\frac94.
\eee
\eq
\ba
原不等式等价于
\bee
\sumcyc\frac{(a-b)(a-c)+(ab+bc+ca)}{(b+c)^2}\ge\frac94.
\eee
即证
\bee
\sumcyc\frac{(a-b)(a-c)}{(b+c)^2}\ge0,\quad \sumcyc\frac{ab+bc+ca}{(b+c)^2}\ge\frac94.
\eee
前者用Shur不等式, 后者是著名的Iran 96不等式.
\ea

\bq{Nguyen Van Thach}{}
设$a,b,c$为正实数, 证明:
\bee
\sqrt{\frac{a^3+abc}{(b+c)^3}}+\sqrt{\frac{b^3+abc}{(c+a)^3}}+\sqrt{\frac{c^3+abc}{(a+b)^3}}\ge\frac{a}{b+c}+\frac{b}{c+a}+\frac{c}{a+b}.
\eee
\eq
\ba
注意到
\bee
\sqrt{\frac{a^3+abc}{(b+c)^3}}-\frac{a}{b+c}=\frac{\sqrt{a}(a-b)(a-c)}{(b+c)\sqrt{b+c}(\sqrt{a^2+bc}+\sqrt{a(b+c)})}.
\eee
所以原不等式等价于证明$\sumcyc S_{a}(a-b)(b-c)$, 其中
\bee
S_{a}=\frac{\sqrt{a}}{(b+c)\sqrt{b+c}(\sqrt{a^2+bc}+\sqrt{a(b+c)})}.
\eee
不妨设$a\ge b\ge c$, 则由$\sqrt{\frac{a}{b+c}}\ge\sqrt{\frac{b}{c+a}}$, $(b+c)\sqrt{a^2+bc}\le(a+c)\sqrt{b^2+ac}$, $(b+c)\sqrt{a(b+c)}\le(a+c)\sqrt{b(a+c)}$. 
知$S_{a}\ge S_{b}$. 根据Shur不等式的推论\ref{co:GSIC}即得证明.
\ea

\bq{}{20170325001}
设$a, b, c, k$为正实数, 证明:
\bee
\frac{a}{b}+\frac{b}{c}+\frac{c}{a}\ge\frac{a+k}{b+k}+\frac{b+k}{c+k}+\frac{c+k}{a+k}.
\eee
\eq
\ba
不妨设$c=\min(a,b,c)$, 并注意
\bee
LHS-3=\left(\frac{a}{b}+\frac{b}{a}-2\right)+\left(\frac{b}{c}+\frac{c}{a}-\frac{b}{a}-1\right)
 = \frac{(a-b)^2}{ab}+\frac{(a-c)(b-c)}{ac}.
\eee
\ea

\bq{}{20170325002}
设$a,b,c$为正实数, 若$k\ge\max(a^2, b^2, c^2)$, 证明:
\bee
\frac{a}{b}+\frac{b}{c}+\frac{c}{a}\ge\frac{a^2+k}{b^2+k}+\frac{b^2+k}{c^2+k}+\frac{c^2+k}{a^2+k}.
\eee
\eq
\ba
同\ref{q:20170325001}不妨设$c=\min(a,b,c)$, 只需证:
\bee
\frac{(a-b)^2}{ab}+\frac{(a-c)(b-c)}{ac}\ge\frac{(a-b)^2(a+b)^2}{(a^2+k)(b^2+k)}+\frac{(a-c)(b-c)(a+c)(b+c)}{(a^2+k)(c^2+k)}.
\eee
所以只需由$k\ge\max(a^2, b^2, c^2)$来证:
\bee
\begin{dcases}
 (a^2+k)(b^2+k)\ge ab(a+b)^2\\
 (a^2+k)(c^2+k)\ge ac(a+c)(b+c).
\end{dcases}
\eee
不等式的弱化见\ref{q:20170325002}.
\ea

\bq{}{}
设$a,b,c$为正实数, 若$k\ge\max(ab,bc,ca)$, 证明:
\bee
\frac{a}{b}+\frac{b}{c}+\frac{c}{a}\ge\frac{a^2+k}{b^2+k}+\frac{b^2+k}{c^2+k}+\frac{c^2+k}{a^2+k}.
\eee
\eq
\ba
证法同\ref{q:20170325002}, 但更弱.
\ea

\bq{}{}
设$a,b,c$为正实数, 证明:
\bee
\frac{a^2}{b^2}+\frac{b^2}{c^2}+\frac{c^2}{a^2}+\frac{8(ab+bc+ca)}{a^2+b^2+c^2}\ge 11.
\eee
\eq
\ba
由恒等式$\frac{a}{b}+\frac{b}{c}+\frac{c}{a}=\frac{(a-b)^2}{ab}+\frac{(a-c)(b-c)}{ac}$, 不妨设$c=\min(a, b, c)$. 原不等式可化为:
\bee
\frac{(a-b)^2(a+b)^2}{a^2b^2}+\frac{(c-a)(c-b)(c+a)(c+b)}{a^2c^2}-\frac{8(a-b)^2+8(c-a)(c-b)}{a^2+b^2+c^2}\ge0.
\eee
易证$\frac{(a+b)^2}{a^2b^2}\ge\frac{8}{a^2+b^2+c^2}$. 而
\bee
 \frac{(c+a)(c+b)}{a^2c^2}\ge\frac{8}{a^2+b^2+c^2}\Longleftrightarrow
 (a+c)(c+b)(a^2+b^2+c^2)\ge(a+c)2c(a^2+c^2)\ge4c^2(a^2+c^2)
\eee
所以只需证$a\ge c$时最后的不等式.
\ea

\bq{2006年CMO}{}
实数列$\{a_n\}$满足$a_1=\frac12$, $a_{k+1}=-a_{k}+\frac1{2-a_k}$, $k\in\mathbb{N}$, 证明: 
\bee
\left(\frac{n}{2\sum_{i=1}^{n}a_i}-1\right)^n\le\left(\frac{\sum_{i=1}^{n}a_i}{n}\right)^n\prod_{i=1}^{n}\left(\frac{1}{a_i}-1\right).
\eee
\eq
\ba
先用$y=-x+\frac{1}{2-x}$归纳证明$0<a_n\le\frac{1}{2}$. 则原命题等价于证明:
\bee
\left(\frac{n}{\sum a_i}\right)^n\left(\frac{n}{2\sum a_i}-1\right)^n\le\prod\left(\frac{1}{a_i}-1\right).
\eee
对函数$y=\ln\left(\frac{1}{x}-1\right)$用Jensen不等式, 有$\left(\frac{n}{\sum a_i}-1\right)^n\le\prod\left(\frac{1}{a_i}-1\right)$.
另外, 用Cauchy不等式证
\bee
\sum_{i=1}^{n}(1-a_i)=\sum\frac{1}{a_i+a_{i+1}}-n\ge\frac{n^2}{2\sum a_i}-n, 
\eee
所以
\bee
\frac{n}{\sum a_i}-1=\frac{\sum(1-a_i)}{\sum a_i}\ge\frac{1}{\sum a_i}\left(\frac{n^2}{2\sum a_i}-n\right).
\eee
\ea

%%%%%%%%%%%%%%%%%%%%%%%%%%%%%%%%%%%%%%%%%%%%%%%%%%%%%%%%
%%%%%%%%%%%%%%%%%%%%%%%%%%%%%%%%%%%%%%%%%%%%%%%%%%%%%%%%
\section{Combinatorics}
\bq{}{}
证明
\begin{align*}
\paren{\binom{m}{m}a_m+\binom{m+1}{m}a_{m+1}+\cdots+\binom{n}{m}a_n}^2
\ge&\paren{\binom{m-1}{n-1}a_{m-1}+\binom{m}{m-1}a_m+\cdots+\binom{n}{m-1}a_n}\\
&\quad\cdot\paren{\binom{m+1}{m+1}a_{m+1}+\binom{m+2}{m+1}a_{m+2}+\cdots+\binom{n}{m+1}a_n},
\end{align*}
其中$a_i>0$, $i=1,2,\cdots,n$, 且
\bee
a_1\le a_2\le\cdots\le a_n, \quad 0<m<n, \quad m,n\in\pNaturals.
\eee
\eq
\ba
若$\{a_n\}$是常数列, 只要证$\paren{\sum_{i=m}^n\binom{i}{m}}^2\ge\sum_{k=m-1}^{n}\binom{k}{m-1}\sum_{k=m+1}^{n}\binom{k}{m+1}$, 
这等价于证$\paren{\binom{n+1}{m+1}}^2\ge\binom{n+1}{m}\binom{n+1}{m+2}\Longleftrightarrow\frac1{(m+1)(n-m)}\ge\frac1{(n-m+1)(m+2)}
\Longleftrightarrow n+2\ge0$, ... 

设调整若干项相等, 即$a_1=\cdots=a_{i-1}=t$, $a_i=\cdots=a_n=k$, ($k>t$), 则
\begin{align*}
 &\paren{t\sum_{j=m}^{i-1}\binom{j}{m}+k\sum_{j=i}^n\binom{j}{m}}^2\ge\paren{t\sum_{j=m+1}^{i-1}\binom{j}{m+1}+k\sum_{j=i}^n\binom{j}{m+1}}\paren{t\sum_{j=m+1}^{i-1}\binom{j}{m-1}+k\sum_{j=i}^n\binom{j}{m-1}}\\
 &\Longleftrightarrow\paren{t\binom{i}{m+1}+k\binom{n+1}{m+1}}^2\ge\paren{t\binom{i}{m+2}+k\binom{n+1}{m+2}}\cdot\paren{t\binom{i}{m}+k\binom{n+1}{m}}\\
 &\Longleftrightarrow\paren{\paren{\binom{i}{m+1}}^2-\binom{i}{m+2}\binom{i}{m}}t^2-2tk\left[\binom{i}{m+1}\binom{n+1}{m+1}-\paren{\binom{i}{m}\binom{n+1}{m+2}+\binom{i}{m+2}\binom{n+1}{m}}\right]\\
  &\qquad\qquad+\left[\paren{\binom{n+1}{m+1}}^2-\binom{n+1}{m+2}\binom{n+1}{m}\right]k^2\ge0.
\end{align*}
因$\paren{\binom{i}{m+1}}^2-\binom{i}{m+2}\binom{i}{m}\ge0$, $\paren{\binom{n+1}{m+1}}^2-\binom{n+1}{m+2}\binom{n+1}{m}\ge0$,
但$\binom{i}{m+1}\binom{n+1}{m+1}-\paren{\binom{i}{m}\binom{i}{m+2}+\binom{i}{m+2}\binom{n+1}{m}}$的符号不一定.
\ea

%%%%%%%%%%%%%%%%%%%%%%%%%%%%%%%%%%%%%%%%%%%%%%%%%%%%%%%%
%%%%%%%%%%%%%%%%%%%%%%%%%%%%%%%%%%%%%%%%%%%%%%%%%%%%%%%%
\section{Analysis}
\bq{}{}
求证积分形式的Holder不等式与Minkowski不等式.
\bee
\|x(t)y(t)\|_1\le\|x(t)\|_{p}\cdot\|y(t)\|_q,
\eee
其中$\frac1p+\frac1q=1$, $p>1$且$x\in L^p[a,b]$, $y\in L^q[a,b]$;
\bee
\|x\pm y\|_p\le\|x\|_p+\|y\|_p,
\eee
其中$p\ge 1$, $x(t),y(t)\in L^p[a,b]$.
\eq
\ba
(1). 若$\|x(t)\|_p=0$或$\|y(t)\|_q=0$, 则$x(t)y(t)=0, a.e. x\in[a,b]$, 不等式显然. 否则令$A=\frac{|x(t)|}{\|x(t)\|_p}$, $B=\frac{|y(t)|}{\|y(t)\|_q}$.
由Young不等式$AB\le\frac{A^p}{p}+\frac{B^q}{q}$, 两边积分即得.

(2). 若$p=1$, 则用绝对值的三角不等式. 若$p>1$, 
\begin{align*}
 \|x\pm y\|_p^p & \le \||x\pm y|^{p-1}\cdot(|x|+|y|)\|_1=\||x|\cdot|x\pm y|^{p-1}\|_1+\||y|\cdot|x\pm y|^{p-1}\|_1\\
  & \le\|x\|_p\cdot\||x\pm y|^{p-1}\|_q+\|y\|_p\cdot\||x\pm y|^{p-1}\|_q\\
  & = (\|x\|_p+\|y\|_p)\cdot\|x\pm y\|_{(p-1)q}^{p-1}=(\|x\|_p+\|y\|_p)\cdot\|x\pm y\|_p^{p-1}.
\end{align*}
即得.
\ea