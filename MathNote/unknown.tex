\chapter{未知的问题与解答}

\bu{2014-05-25}{}
\bee
\sum_{k=1}^{n}\paren{\frac{\sum_{i=1}^{k}x_i}{k}}^2\le(1+\sqrt{2})^2\sum_{i=1}^nx_i^2.
\eee
\eu
\ba
\bee
A=\begin{pmatrix}
   1 & & & & \\
   \frac12 & \frac12 & & & \\
   \frac13 & \frac13 & \frac13 & & \\
   \vdots & \vdots & \vdots & \ddots & \\
   \frac1n & \frac1n & \frac1n & \cdots & \frac1n
  \end{pmatrix}
\eee
证明$\norm{A}_2\le\sqrt{2}+1$.
\ea

\bu{2014-05-24}{}
设$p_n$是第$n$个素数; $a_n=\sum_{i=1}^{n}p_i$. 
则$[a_n, a_{n+1}]$中至少有一个平方数.
\eu

\bu{2015-05-23}{}
对于任意的拓扑空间$X$, $\Reals$与$X\times X$不同胚.
\eu

\bu{2015-05-23}{}
平方和因子定理.

$(a,b)=1$, 则素数$p=4k+3\nmid(a^2+b^2)$.
\eu

\bu{2014-05-23}{}
\bee
\left\{
\begin{array}{l}
 a,b\not\in\Integers;\\
 \Re(c+d-a-b)>1.
\end{array}
\right.\Longrightarrow
\sum_{n=-\i}^{\i}\frac{\Gamma(a+n)\Gamma(b+n)}{\Gamma(c+n)\Gamma(d+n)}=\frac{\pi^2}{\sin(\pi a)\sin(\pi b)}\cdot
  \frac{\Gamma(c+d-a-b-1)}{\Gamma(c-a)\Gamma(d-a)\Gamma(c-b)\Gamma(d-b)}.
\eee
\eu

\bu{Enestrom-Kakeya定理}{}
\bee
\left.
\begin{array}{l}
 a_0\ge a_1\ge\cdots\ge a_n>0,\\
 p(z):=a_nz^n+a_{n-1}z^{n-1}+\cdots+a_0
\end{array}
\right\}\Longrightarrow
p=0\textrm{的根在开单位圆外.}
\eee
\eu
\ba
对$(1-z)p(z)$用反证法/直接证明.
\ea

\bu{2014-05-21}{}
\bee
\left.
\begin{array}{l}
 a_n>0\\
 b_n>0
\end{array}
\right\}\Longrightarrow
\sqrt[n]{\prod_{i=1}^n(a_i+b_i)}\ge\sqrt[n]{\prod_{i=1}^na_i}+\sqrt[n]{\prod_{i=1}^nb_i}.
\eee
\eu

\bu{2014-05-21}{}
\bee
\begin{array}{l}
 \lim_{\b\to\i}F\paren{1,\b,1;\frac{z}{\b}}=\ue^z;\\
 \lim_{\a\to\i}F\paren{\a,\a,\frac12;\frac{z^2}{4\a^2}};\\
 F\paren{\frac{n}{2},-\frac{n}{2},\frac12;\sin^2x}=\cos nx.
\end{array}
\eee
\eu

\bu{2014-05-21}{}
\bee
\left.
\begin{array}{l}
 x_0\in(a,b).\\
 \textrm{迭代公式}f_1(x_n)=f_2(x_{n-1})\\
 f_1(\xi)=f_2(\xi)\\
 \abs{\frac{f'_2(y)}{f'_1(x)}}\le q<1, (a<x,y<b)
\end{array}
\right\}\Longrightarrow
x_n\to\xi, n\to\i.
\eee
\eu

\bu{2014-05-20}{}
$\forall n$, $a_n>0$,$\sum_{i=1}^{n}a_i\ge\sqrt{n}$,
则$\forall n$, 
\bee
\sum_{i=1}^{n}a_i^2>\frac14\sum_{j=1}^n\frac1j.
\eee
\eu

\bu{2014-05-20}{}
关于$x$的方程$x^2-2\arcsin(\cos x)+a^2=0$有唯一解, 求$a$.
\eu

\bu{2014-05-19}{}
\bee
\left.
\begin{array}{l}
 f\textrm{定义域:} \Reals\\
 f^2(x)\le 2x^2f\paren{\frac{x}{2}}\\
 A=\{a\mid f(a)>a^2\}\ne\emptyset
\end{array}
\right\}\Longrightarrow
A\textrm{是无限集.}
\eee
\eu

\bu{2014-05-18}{}
\bee
\left.
\begin{array}{l}
 M_1=\{x^2+Ax+B\mid x\in\Integers\}\\
 M_2=\{2x^2+2x+C\mid x\in\Integers\}
\end{array}
\right\}\Longrightarrow
\forall A,B\in\Integers, \ \exists C\in\Integers\ni M_1\cap M_2=\emptyset.
\eee
\eu

\bu{2014-05-18}{}
\bee
(0,1)\nsubseteq\bigcup_{\frac{p}{q}\in(0,1)\cap\Rationals}\paren{\frac{p}{q}-\frac{1}{4q^2}, \frac{p}{q}+\frac{1}{4q^2}}.
\eee
\eu

\bu{}{}
设数列$\{x_n\}$满足条件
\bee
0\le x_{m+n}\le x_m+x_n\quad(m,n\in\pNaturals),
\eee
证明: $\lim_{n\to\infty}\frac{x_n}{n}$存在.
\eu

\bu{}{}
设$x_n\ge0$且$y_n\ge0$, ($n\in\pNaturals$), 证明: (在以下各极限均存在的情况下)
\bee
\liminf_{n\to\infty}x_n\cdot\liminf_{n\to\infty}y_n
\le\liminf_{n\to\infty}(x_ny_n)
\le\liminf_{n\to\infty}x_n\cdot\limsup_{n\to\infty}y_n
\le\limsup_{n\to\infty}(x_ny_n)
\le\limsup_{n\to\infty}x_n\cdot\limsup_{n\to\infty}y_n
\eee
\eu

\bu{}{}
已知$x,y,z\in \Reals$, 且$x+y+z=0$.
\begin{enumerate}
 \item 求证:
 \bee
 6(x^3+y^3+z^3)^2\le(x^2+y^2+z^2)^3.
 \eee
 \item 求最佳常数$\l,\m$, 使得:
 \bee
 \l(x^6+y^6+z^6)\le(x^2+y^2+z^2)^3\le\m(x^6+y^6+z^6).
 \eee
\end{enumerate}

\eu

\bu{}{}
证明对于任意的$x>-1$, 有$\ue^x-\ln(x+1)-2x>0$
\eu
\ba
先证$1+\frac{x}{1-\frac{x}{2}+\frac{x^2}{12}}-\ln(1+x)-2x>0$.
\ea
\ba
用 Taylor 公式.
\ea

\bu{}{}
证明: $\nabla\times\nabla\times A=\nabla(\nabla \cdot A)-\nabla^2A$.
\eu

\bu{印度, 巴斯卡拉}{}
当$x\in \left[0,\frac{\pi}{2}\right]$时, 有$\sin x\approx\frac{16x(\pi-x)}{5\pi^2-4x(\pi-x)}$.
\eu

\bu{柏拉图体, 正多面体}{}

\eu

\bu{阿基米德体, 半正多面体}{}

\eu

\bu{}{}
置换$P(\nu)$的循环结构为$(\nu)=(1^{\nu_1}2^{\nu_2}\cdots m^{\nu_m})$, 在$S_n$群中属于与$P(\nu)$共轭的置换数目为
\bee
N(P)=\frac{n!}{\prod_i(i^{\nu_i}\nu_i!)}
\eee
\eu

\bu{}{}
正整数列$(v_i)$满足$\sum_iiv_i=n$, 则$\prod_ii^{v_i}v_i!\mid n!$
\eu

\bu{}{}
设$f(x)\in C[0,+\infty)$, 且对任何非负实数$a$, 有
\bee
\lim_{x\to\infty}(f(x+a)-f(x))=0.
\eee
证明: 存在$g(x)\in C[0,+\infty)$和$h(x)\in C^1[0,+\infty)$, 使得: $f(x)=g(x)+h(x)$, 且满足
\bee
\lim_{x\to\infty}g(x)=0, \quad \lim_{x\to\infty}h'(x)=0.
\eee
\eu

\bu{}{}
设$p$是奇素数, 将$1+\frac12+\frac13+\cdots+\frac1{p-1}$写成最简分数$A_p/B_p$.
\begin{enumerate}[(a)]
 \item 求$A_p\pmod{p}$的值.
 \item 给出$A_p\pmod{p^2}$的值.
\end{enumerate}
\eu

\bu{}{}
设$m$是正整数, $a_1, a_2,\cdots, a_{\phi(m)}$是$1$与$m$之间且与$m$互素的整数,
记
\bee
\frac1{a_1}+\frac1{a_2}+\cdots+\frac1{a_{\phi(m)}}
\eee
的最简分数为$A_m/B_m$.
\begin{enumerate}[(a)]
 \item 求$A_m\pmod{m}$的值.
 \item 求$A_m\pmod{m^2}$的值.
\end{enumerate}
\eu

\bu{Hardy-Ramanujan asymptotic}{}
整数$n$的分划函数$p(n)$有如下渐近公式
\bee
p(n)\sim\frac{1}{4n\sqrt{3}}\ue^{\pi\sqrt{2n/3}}.
\eee
\eu

\bu{Fubini's Theorem}{}

\eu

\bu{Tonelli's Theorem}{}

\eu

\bu{陕西省第七次大学生高等数学竞赛复赛}{}
设函数$f(x,y)$在闭圆域$D=\{(x,y)\vert x^2+y^2\le R^2, R>0\}$上有连续偏导数, 
而且$f\left(\frac{R}{2},0\right)=f\left(0,\frac{R}{2}\right)$. 证明: 在$D$的内部至少存在两点$(x_1, y_1)$和$(x_2, y_2)$, 
使
\bee
x_if'_y(x_i,y_i)-y_if'_x(x_i,y_i)=0, i=1,2.
\eee
\eu

\bu{陕西省第七次大学生高等数学竞赛复赛}{}
设在$[a,b]$上, $f''(x)\ne0$, $f(a)=f(b)=0$, 且有$x_0\in(a,b)$, 使$y_0=f(x_0)>0$, $f'(x_0)=0$. 证明:
\begin{enumerate}[(1)]
 \item 存在$x_1\in(a,x_0)$及$x_2\in(x_0,b)$, 使$f(x_1)=f(x_2)=\frac{y_0}{2}$;
 \item $\int_a^bf(x)\ud x<y_0(x_2-x_1)$.
\end{enumerate}
\eu

\bu{}{}
已知对于任意的$a_1, a_2, \cdots, a_n\in[p,q]$, $p>0$, 证明:
\bee
\sum_{i=1}^{n}a_i\sum_{i=1}^{n}\frac{1}{a_i}\le n^2+\frac{k(p-q)^2}{4pq},
\eee
其中,
\bee
k=\begin{dcases}
   n^2-1, \quad n\textrm{ 是奇数};\\
   n^2,\quad n\textrm{ 是偶数}.
  \end{dcases}
\eee
\eu

\bu{http://math.stackexchange.com/questions/66473}{}
a Fourier series $\sum c_n\ue^{2\pi\ui n x}$ to be $k$-fold termwise
differentiable is for the Fourier coefficients to be ``appropriately small"
in the following sense: if
\bee
\sum |c_n|\cdot|n|^k<\infty
\eee
holds for some $k$, then then function represented by the Fourier series will be $k$-times
differentiable, and will be differentiable termwise. If this holds for all $k$,
then the function is smooth.
\eu

\bu{http://math.stackexchange.com/questions/1992808}{}
已知$\sum_{n=1}^{\infty}(|a_n|+|b_n|)$存在有限, 
\bee
f(x)=\sum_{n=1}^{\infty}(a_n\cos nx+b_n\sin nx)
\eee
能否逐项求导?
\eu

\bu{http://math.stackexchange.com/questions/420878}{}
John B. Conway's 'Function of one complex variable', Proposition 2.5.
\eu

\bu{http://math.stackexchange.com/questions/1922228}{}
\bt{}{}
$f_n$在区域$D$上序列解析(sequence analytic), 若$f_n$在$D$的任一紧子集上一致收敛, 
则$f_n$在$D$上解析
\et
于是
\bee
\sum_{n\ge1}(-1)^n\frac{z^{2n+1}}{(2n+1)!}
\eee
在$\Bbb C$上解析.
\bc{}{}
若幂级数在$z_0$为圆心, $R$为半径的圆盘上收敛, 则幂级数在其收敛域内的任一子集上一致收敛.
\ec
\bt{}{}
$f_n$是区域$D$上的序列, 若$\sum f_n$在$D$上收敛, 且在$D$内任一紧子集上一致收敛, 
则$\sum f_n$在$D$上解析且可逐项求导.
\et
\eu

\bu{http://math.stackexchange.com/questions/294383}{20170405un1}
求$\int_0^{+\infty}\left(\frac{x}{\ue^x-\ue^{-x}}-\frac12\right)\frac1{x^2}\ud x$.
\eu
\ba
重积分法, 利用Laplace变换: $\int_0^{\infty}t\ue^{-xt}\ud t=\frac{1}{x^2}$.
\begin{align*}
 \int_{0}^{\infty}\left(\frac{x}{\ue^x-\ue^{-x}}-\frac{1}{2}\right)\frac{1}{x^2}\ud x
  & = \int_0^{\infty}\left(\frac{x}{\ue^x-\ue^{-x}}-\frac{1}{2}\right)\int_{0}^{\infty}t\ue^{-xt}\ud t\ud x\\
  & =\frac{1}{2}\int_{0}^{\infty}t\int_{0}^{\infty}\frac{2x\ue^{-x}-1+\ue^{-2x}}{1-\ue^{-2x}}\ue^{-xt}\ud x\ud t\\
  & = \frac{1}{2}\int_0^{\infty}t\int_0^{\infty}(2x\ue^{-x}-1+\ue^{-2x})\ue^{-xt}\sum_{n=0}^{\infty}\ue^{-2nx}\ud x\ud t\\
  & = \frac{1}{2}\int_0^{\infty}t\sum_{n=1}^{\infty}\left(-\frac{1}{2n+t}+\frac{2}{(2n+t+1)^2}+\frac{1}{2n+t+2}\right)\ud t+\log 2-1\\
  & = \frac{1}{2}\int_0^{\infty}\sum_{n=1}^{\infty}\left(\frac{2n}{2n+t}+\frac{2t}{(2n+t+1)^2}-\frac{2(n+1)}{2n+t+2}\right)\ud t + \log 2 -1\\
  & = \log 2-1-\sum_{n=1}^{\infty}\left(1+n\ln n+\ln\left(n+\frac12\right)-(n+1)\ln(n+1)\right)=-\frac12\ln 2.
\end{align*}
另外留数法见\ref{q:20170405001}.
\ea

\bu{}{20170117002}
对于任意给定的正奇数$n$, 对于任意正有理数$r$, 都存在正整数$a,b,c,d$满足$r=\frac{a^n+b^n}{c^n+d^n}$. 强化版见\ref{q:20170117001}.
\eu

\bu{}{20170116001}
一个自然数$N$可以写成两个整数的平方和当且仅当$N$的素因子分解中每个可以写成$4n-1$的素数出现次数为偶数.
\eu
\ba
抽象代数有证明, 主要工具是环理论.
\ea

\bu{}{}
Fermat数$F_n=2^{2^n}+1$, $n\ge5$时是否有素数? 尙未发现素数.

1801年Gauss证明: 正$N$边形可尺规作图当且仅当$N=2^e p_1\cdots p_s$, $p_i$是Fermat素数.
\eu

\bu{}{}
设$a, b$为正整数且$(a,b)=1$. 证明对于给定的$n>ab-a-b$, 方程$ax+by=n$有非负整数解, 且$n=ab-a-b$时没有非负整数解.
\eu

\bu{http://math.stackexchange.com/questions/428663/closed-form-of-sum-limits-i-1n-k1-i-or-asymptotic-equivalent-when-n-to\#}{}
\bee
\sum_{i=1}^{n}k^{\frac{1}{i}} = n + \ln(k)\ln(n) + \gamma\ln(k) + \sum_{r=2}^{\infty}\frac{\zeta(r)\ln(k)^r}{r!} + O\left(\frac1n\right).
\eee
\eu

\bu{http://tieba.baidu.com/p/4819379251}{}
求所有符合条件的$x, y, z$.
\bee
x+y+z=x^2+y^2+z^2=x^3+y^3+z^3.
\eee
\eu

\bu{陈计的不等式}{}
设$x,y,z\in\mathbb{R}^+$. 求证:
\bee
(xy+yz+zx)\left(\frac1{(x+y)^2}+\frac1{(y+z)^2}+\frac1{(z+x)^2}\right)\ge\frac94.
\eee
\eu
\ba
因为
\begin{align*}
4\sumcyc yz & \left( \sumcyc(x+y)^2(z+x)^2\right)-9\prodcyc(y+z)^2\\
  & =\sumcyc yz(y-z)^2(4y^2+7yz+4z^2)+\frac{xyz}{x+y+z}\sumcyc(y-z)^2(2yz+(y+z-x)^2)\ge0.
\end{align*}
\ea

\bu{http://tieba.baidu.com/p/4005256822 18届东令营第3题 2003CMO3}{}
设$x_i\in\mathbb{R}^{+}$($i=1,2,\cdots,n$)且$x_1x_2\cdots x_n=1$, $n$给定($n\ge1$).
求最小正数$\lambda$使得
\bee
\lambda\ge\sum_{i=1}^{n}\frac{1}{\sqrt{1+2x_i}}
\eee
恒成立.
\eu

\bu{http://tieba.baidu.com/p/4811340691}{}
设$x_1, x_2, \cdots, x_n$是正实数, 求证:
\bee
x_1^3+\left(\frac{x_1+x_2}{2}\right)+\cdots+\left(\frac{x_1+x_2+\cdots+x_n}{n}\right)^3\le\frac{27}{8}(x_1^3+x_2^3+\cdots+x_n^3).
\eee
\eu
\ba
由Holder不等式($a_i, b_i, c_i\ge0$)
\bee
(a_1^3+\cdots+a_n^3)(b_1^3+\cdots+b_n^3)(c_1^3+\cdots+c_n^3)\ge(a_1b_1c_1+\cdots+a_nb_nc_n)^3
\eee
得
\bee
\left(\sum_{j=1}^{k}\left(j-\frac12\right)^{2/3}x_j^3\right)\left(\sum_{j=1}^{k}\frac1{\left(j-\frac12\right)^{1/3}}\right)^2\ge\left(\sum_{j=1}^kx_j\right)^3.
\eee
两边同时除以$k^3$, 求和有
\bee
\sum_{k=1}^{n}\left(\frac{x_1+\cdots+x_k}{k}\right)^3
  \le \sum_{k=1}^n\frac1{k^3} \left( \sum_{i=1}^k \frac1{\left(i-\frac12\right)^{1/3}} \right)^2 \left( \sum_{j=1}^k\left( j-\frac12\right)^{2/3}x_j^3\right)
  =\sum_{j=1}^{n}\sum_{k=j}^{n}\left(j-\frac12\right)^{2/3}x_{j}^3\frac1{k^3}\left(\sum_{i=1}^{k}\frac1{\left(i-\frac12\right)^{1/3}}\right)^2
\eee
由于
\bee
\sum_{k=j}^n\frac1{k^3}\left(\sum_{i=1}^{k}\frac1{\left(i-\frac12\right)^{1/3}}\right)^2
  < \sum_{k=j}^{n}\frac1{k^3}\left(\int_{0}^{k}\frac1{x^{1/3}}\ud x\right)^2
  = \sum_{k=j}^{n}\frac9{4k^{5/3}} < \int_{j-\frac12}^{n+\frac12}\frac9{4x^{5/3}}\ud x < \frac{27}{8\left(j-\frac12\right)^{2/3}}.
\eee
所以
\bee
\sum_{k=1}^{n}\left(\frac{x_1+\cdots+x_k}{k}\right)^3
  < \frac{27}{8}\sum_{k=1}^{n}x_j^3
\eee
\ea

\bu{http://tieba.baidu.com/p/4740384715}{}
\bee
\sqrt{\frac{1+\sqrt{5}}{2}+2}-\frac{1+\sqrt{5}}{2}=\frac{\ue^{-\frac{2\pi}{5}}}{1+\frac{\ue^{-2\pi}}{1+\frac{\ue^{-4\pi}}{1+\frac{\ue^{-6\pi}}{1+\cdots}}}}.
\eee
\eu

\bu{}{}
设$a,b,c$是三角形的三边长, 证明:
\bee
4\left(\frac{a}{b}+\frac{b}{c}+\frac{c}{a}\right)\ge9+\frac{a^2+c^2}{c^2+b^2}+\frac{c^2+b^2}{b^2+a^2}+\frac{b^2+a^2}{a^2+c^2}.
\eee
\eu

\bu{http://tieba.baidu.com/p/4850101496}{}
1. 有界闭区间$[a,b]$上函数$f(x)$满足对任意$x,y\in [a,b]$, $\lambda\in(0,1)$都有$f(\lambda x+(1-\lambda)y)\le\lambda f(x)+(1-\lambda)f(y)$. 求证$f(x)$在$[a,b]$上有界.

2. 设$f(x)$在$0$附近有$2$阶连续导数, 且$f''(0)\ne0$. 

(1) 求证$|x|$充分小时, 对任意这样的$x$, 存在唯一$\theta\in(0,1)$使得$f(x)=f(0)+f'(\theta x)x$.

(2) 求$\lim\limits_{x\to 0}\theta$.
\eu
\ba
2.(2) $f'(\theta x)=\frac{f(x)-f(0)}{x}$, 所以$\frac{f'(\theta x)-f'(0)}{x}=\frac{f(x)-f(0)-xf'(0)}{x^2}$, 两边令$x\to0$, 右侧用罗比达法则. $\lim\limits_{x\to 0}\theta=\frac12$
\ea

\bu{http://tieba.baidu.com/p/4849247837}{}
已知关于$x$的方程$x^3-4x^2+6x+c=0$有三个实根$r, s, t$, 且
\bee
\frac1{r^2+s^2}+\frac1{s^2+t^2}+\frac1{t^2+r^2}=1,
\eee
求正数$c$的值.
\eu

\bu{http://tieba.baidu.com/p/4850318851}{}
试举出反例: 函数$f(x)$在$x=x_0$处可导与如下几个式子存在不等价:
\bee
\lim_{h\to0}\frac{f(x_0+h)-f(x_0-h)}{2h},\quad
\lim_{h\to0}\frac{f(x_0+2h)-f(x_0+h)}{h},\quad
\lim_{n\to\infty}n\left[f\left(x_0+\frac1n\right)-f(x0)\right]
\eee
\eu

\bu{http://tieba.baidu.com/p/4846868296}{}
三角形$ABC$中, 求证: 
\bee
\prodcyc\cos A\le\frac18\cot\frac{\pi^3}{27}\tan(ABC)\le\prodcyc\sin\frac{A}{2}.
\eee
\eu