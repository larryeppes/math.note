\chapter{神奇的反例}

\bq{}{}
试指出无限维欧式空间中正交变换不一定是满射变换. 从而不一定有逆变换.
\eq
\ba
考虑$\RR[x]$(内积为多项式对应系数乘积的和)中的
\bee
Tf(x)=xf(x).
\eee
\ea

\bq{}{}
举例说明: $\lim_{x\to a}\phi(x)=A$, $\lim_{x\to A}\psi(x)=B$, 但$\lim_{x\to a}\psi(\phi(x))\ne B$.
\eq
\ba
\bee
\phi(x)=\left\{
\begin{array}{ll}
 \frac1q, & x=\frac{p}{q}, (p,q\in\ZZ, (p,q)=1)\\
 0, & x\not\in\QQ.
\end{array}
\right.\quad
\psi(x)=\left\{
\begin{array}{ll}
 1, & x\ne0,\\
 0, & x=0.
\end{array}
\right.\quad a=0.
\eee
\ea

\bq{http://math.stackexchange.com/questions/2190498}{}
给出如下论断的反例.

存在$x_0\in[a,b]$使函数项级数的和函数$\sum_{n=1}^{\infty}f_n(x)$收敛到$f(x)$, 
在$[a,b]$上$\sum_{n=1}^{\infty}f'_n(x)$收敛到$g(x)$. 则$f'=g$
\eq
\ba
定义$f_n: [0,1]\to\Bbb R$为
\bee
f_n(x)=n(n+1)(n+2)\left(\frac{x^{n+1}}{n+1}-\frac{x^{n+2}}{n+2}\right).
\eee
则$f_n$在$[0,1)$上点态收敛到$0$, 且$f_n(1)=n$. 其导数$f'_n(x)=n(n+1)(n+2)x^n(1-x)$, 在$[0,1]$上逐点收敛到$0$.
\ea

\bq{http://math.stackexchange.com/questions/294383}{}
给出一个积分号下不能求导的例子.
\eq
\ba
令$F(s)=\int_{0}^{\infty}\ue^{-sx}\left(\frac{x}{\ue^x-\ue^{-x}}-\frac12\right)\frac{1}{x^2}\ud x$,
则
\bee
F''(s)=\int_{0}^{\infty}\ue^{-sx}\left(\frac{x}{\ue^x-\ue^{-x}}-\frac12\right)\ud x
  =\frac14\psi'\left(\frac12+\frac12s\right)-\frac{1}{2s}
\eee
积分一次, $F'(s)$的常数项不能确定是有限值, 且积分两次后的特解使$F(0)=\frac{\log\pi}{2}$与正确的$F(0)=-\frac{\log2}{2}$不同.
\ea

\bq{http://math.stackexchange.com/questions/494145}{}
给出一个不能逐项求导的收敛级数.
\eq
\ba
定义$u_k(x)$为
\bee
\sum_{k=1}^{n}u_k(x)=\frac{nx}{1+n^2x^2}
\eee
则$\sum_{k=1}^{\infty}u_k(x)=0$. 若逐项求导则有
\bee
\sum_{k=1}^{n}u'_k(x)=\frac{n(1-n^2x^2)}{(1+n^2x^2)^2}
\eee
其和在$x=0$处发散
\ea
