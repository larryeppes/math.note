\section{连续函数}
\subsection{函数的极限}

设$x_{0}\in\RR$, $\delta>0$, $\left(x_{0}-\delta,x_{0}+\delta\right)$称为$x_{0}$的开邻域.

$\left(x_{0}-\delta,x_{0}\right)\cup\left(x_{0},x_{0}+\delta\right)$称为$x_{0}$的去心开邻域.

\paragraph{定义3.1.1}

设$f(x)$定义在$x_{0}$的某个去心开邻域上, 若$\exists A\in\RR$, s.t. $\forall\epsilon>0$,
$\exists\delta=\delta(\epsilon,x_{0})\in(0,\delta_{0})$, 当$0<\left|x-x_{0}\right|<\delta$时,
有
\[
\left|f(x)-A\right|<\epsilon.
\]
则称$f(x)$在$x_{0}$处有极限$A$, 记为
\[
\lim_{x\to x_{0}}f(x)=A,\text{ or }f(x)\to A,\ x\to x_{0}.
\]


\paragraph{注: }

去心邻域说明$f(x)$在$x=x_{0}$处可能没有定义.

可以类似的定义左右极限.

\paragraph{命题3.1.1}

$f$在$x_{0}$处有极限的充要条件是$f$在$x_{0}$的左右极限存在且相等.

\paragraph{命题3.1.2 (夹逼原理)}

设在$x_{0}$的一个空心领域内有
\[
f_{1}(x)\le f(x)\le f_{2}(x).
\]
若$f_{1}$, $f_{2}$在$x_{0}$处的极限存在且等于$A$, 则$f(x)$在$x_{0}$处极限为$A$.

\paragraph{命题3.1.3 (极限唯一性)}

函数极限存在必然唯一.

$\epsilon-\delta$语言是证明函数极限的最简单框架, 其难点仅在于对不等式的掌握情况, 比如重要极限
\[
\lim_{x\to0}\frac{\sin x}{x}=1
\]
对应不等式
\[
\cos x<\frac{\sin x}{x}<1.
\]

同样类似地给出涉及无穷大与无穷远时的函数极限的定义.

\paragraph{习题12}

设 $f,g$ 为两个周期函数, 如果 $\lim_{x\rightarrow+\infty}[f(x)-g(x)]=0$, 则
$f=g$. 

注: $f,g$的周期比可能是无理数, 所以$f-g$可能不是周期函数.

pf. 设$f$的周期为$T$, $g$的周期为$S$. 则
\begin{align*}
	f(x)-g(x) & =f(x+nT)-g(x+nS)\\
	& =f(x+nT)-g(x+nT)\\
	& \quad+g(x+nT{\color{magenta}+nS})-f(x+nS{\color{magenta}+nT})\\
	& \quad+f(x+nS)-g(x+nS)\\
	& =\lim_{n\to\infty}f(x+nT)-g(x+nT)\\
	& \quad+\lim_{n\to\infty}g(x+nT{\color{magenta}+nS})-f(x+nS{\color{magenta}+nT})\\
	& \quad+\lim_{n\to\infty}f(x+nS)-g(x+nS)\\
	& =0+0+0.
\end{align*}


\paragraph{习题13}

设 $\lim_{x\rightarrow0}f(x)=0,\lim_{x\rightarrow0}\frac{1}{x}[f(2x)-f(x)]=0$,
证明 $\lim_{x\rightarrow0}\frac{f(x)}{x}=0$.

pf. $\forall\epsilon>0$, $\exists X>0$, s.t. $\forall x>X$, $\left|f(x)\right|<\epsilon$,
且
\[
\left|f(2x)-f(x)\right|\le\epsilon x.
\]
所以
\[
\left|f(x)-f\left(\frac{x}{2^{n}}\right)\right|\le\epsilon x.
\]
令$n\to\infty$即可.

\subsection{函数极限的性质}

\paragraph{定理3.1.5 (Heine, 归结原则)}

设$f$定义在$x_{0}$的某个去心领域上, $f$在$x_{0}$处极限为$A$的充要条件是$\forall x_{n}\to x_{0}$,
($n\to\infty$), 且$x_{n}\ne x_{0}$, ($\forall n$), 均有
\[
\lim_{n\to\infty}f(x_{n})=A.
\]


\paragraph{证明:}

$\Longrightarrow$: 是容易的; $\Longleftarrow$: 用反证法, 则
\[
\exists\epsilon_{0}>0,\ s.t.\ \forall\delta>0,\ \exists x_{\delta},\ s.t.\ 0<\left|x_{\delta}-x_{0}\right|<\delta,
\]
但$\left|f(x_{\delta})-A\right|\ge\epsilon_{0}$. 取$\delta=\frac{1}{n}$,
构造子列$(x_{n})\to x_{0}$, 但$\left|f(x_{n})-A\right|\ge\epsilon_{0}$.
矛盾.

Heine定理可改述为: $f(x)$在$x_{0}$处有极限当且仅当, $\forall x_{n}\to x_{0}$,
($x_{n}\ne x_{0}$), $\lim_{n\to\infty}f(x_{n})$存在.

\paragraph{定理3.1.6 (Cauchy准则)}

设$f$在$x_{0}$的空心领域上有定义, 则$f$在$x_{0}$处有极限, iff, $\forall\epsilon>0$,
$\exists\delta>0$, s.t. 当$0<\left|x'-x_{0}\right|<\delta$, $0<\left|x''-x_{0}\right|<\delta$时,
有
\[
\left|f(x')-f(x'')\right|<\epsilon.
\]


\paragraph{注:}

对于无穷远处极限有限时, Cauchy准则仍然成立.

给出Cauchy准则的否定表述.

\paragraph{定理3.1.7 (单调有界原理)}

设$f$定义在$(x_{0}-\delta,x_{0})$上, 若$f$单调上升有上界, 或$f$单调下降有下界, 则$f$在$x_{0}$有左极限.

\paragraph{定理3.1.8}

(1). (局部有界原理) 若$f$在$x_{0}$处有有限极限, 则$f$在$x_{0}$的某空心领域内有界.

(2). (保序性) 
\begin{align*}
	\lim_{x\to x_{0}}f(x)=A,\ \lim_{x\to x_{0}}g(x)=B,\ f(x)\ge g(x), & \Longrightarrow A\ge B.\\
	\lim_{x\to x_{0}}f(x)=A,\ \lim_{x\to x_{0}}g(x)=B,\ A>B, & \Longrightarrow\exists U_{x_{0}}^{\circ}\ s.t.\ f(x)>g(x),\ \forall x\in U_{x_{0}}^{\circ}.
\end{align*}

(3). (四则运算)

\paragraph{定理3.1.9 (复合函数极限)}

设$f(y)\to A$, $y\to y_{0}$; $g(x)\to y_{0}$, $x\to x_{0}$, 且$\exists U_{x_{0}}^{\circ}$,
s.t. $\forall x\in U_{x_{0}}^{\circ}$, $g(x)\ne y_{0}$, 则$f(g(x))\to A$,
$x\to x_{0}$.

这个定理说明极限定义中去心领域的重要性, 定理中$y_{0}\not\in g(U_{x_{0}}^{\circ})$不可以弱化为:
存在收敛于$x_{0}$的序列$\left(x_{n}\right)$, 使得$g(x_{n})\ne y_{0}$, $\forall n$.
比如
\[
f(y)=\begin{cases}
	1, & y\ne0,\\
	0, & y=0,
\end{cases}\quad g(x)\equiv0,\ y_{0}=0.
\]

当$f$在$y_{0}$处连续时, 这个去心领域的条件又可以去掉, 这说明研究连续函数是有价值的.

\subsection{无穷小量与无穷大量的阶}

\paragraph{定义3.2.1 (无穷小量与无穷大量)}

若函数$f$在$x_{0}$处的极限是$0$, 则称$f$在$x\to x_{0}$时为无穷小量, 记为$f(x)=o(1)$,
($x\to x_{0}$); 若$x\to x_{0}$时, $\left|f\right|\to+\infty$, 则称$f$在$x\to x_{0}$时为无穷大量.

在无穷远处也可以定义无穷小量和无穷大量, 数列也可以定义无穷小量和无穷大量.

\paragraph{定理3.2.1 (等价代换)}

设$x\to x_{0}$时, $f\sim f_{1}$, $g\sim g_{1}$, 若$\frac{f_{1}}{g_{1}}$在$x_{0}$处有极限,
则$\frac{f}{g}$在$x_{0}$处有极限, 且极限相等.

几个常用的等价代换:
\[
\tan x\sim\sin x\sim x\sim e^{x}-1\sim\ln(1+x).
\]


\paragraph{无穷小量的性质:}

习题4. 设$f(x)=o(1)$, ($x\to x_{0}$), 证明, 当$x\to x_{0}$时, 有

(1) $o(f(x))+o(f(x))=o(f(x))$.

(2) $o(cf(x))=o(f(x))$, 其中$c$是常数.

(3) $g(x)\cdot o(f(x))=o(f(x)g(x))$, 其中$g(x)$是有界函数.

(4) $[o(f(x))]^{k}=o\left(f^{k}(x)\right)$.

\subsection{连续函数}

用来刻画连续变化的量

\paragraph{定义3.3.1 (连续性)}

若$f$在$x_{0}$的某领域上有定义, 且$f$在$x_{0}$处的极限是$f(x_{0})$, 则称$f$在点$x_{0}$处连续,
$x_{0}$称为$f$的连续点. 类似地可以定义左右连续. 在定义域上每一点都连续的函数称为连续函数.

$f$在$x_{0}$处下半连续: $\forall\epsilon>0$, $\exists\delta>0$, 当$\left|x-x_{0}\right|<\delta$时,
总有$f(x)>f(x_{0})-\epsilon$.

$f$在$x_{0}$处上半连续: $\forall\epsilon>0$, $\exists\delta>0$, 当$\left|x-x_{0}\right|<\delta$时,
总有$f(x)<f(x_{0})+\epsilon$.

\paragraph{连续函数的基本性质:}

(1). 保持四则运算;

(2). 若$f,g$连续, 则$\max\left\{ f,g\right\} $和$\min\left\{ f,g\right\} $均连续.

\paragraph{定理3.3.2 (复合函数连续性)}

设$f$在$y_{0}$处连续, $g(x)\to y_{0}$, $x\to x_{0}$, 则
\[
\lim_{x\to x_{0}}f(g(x))=f\left(\lim_{x\to x_{0}}g(x)\right)=f(y_{0}).
\]
当$g$在$x_{0}$处连续时, $f(g(x))$在$x_{0}$处连续.

\paragraph{定义3.3.2}

设$x_{0}$是$f(x)$的间断点, 如果$f(x_{0}-0)$, $f(x_{0}+0)$均存在且有限, 则称$x_{0}$是第一类间断点,
否则, 称为第二类间断点. 按照左右极限不相等和相等来区分跳跃间断点和可去间断点.

\paragraph{命题3.3.3}

设$f(x)$在$[a,b]$上单调, $x_{0}\in\left(a,b\right)$是$f(x)$的间断点, 则$x_{0}$是跳跃间断点.

\paragraph{命题3.3.4}

设$f(x)$定义在区间$I$上的单调函数, 则$f(x)$的间断点至多可数.

\paragraph{证明:}

间断点$x$与开区间$\left(f(x_{0}-0),f(x_{0}+0)\right)$一一对应且至多可数.

\paragraph{命题3.3.5}

若$f(x)$定义在区间$I$上严格单调, 则$f(x)$连续当且仅当$f(I)$也是区间.

\paragraph{证明:}

$\Longrightarrow$: 用介值定理. $\Longleftarrow$: 反证法, 则有$x_{0}$使得$\left(f(x_{0}-0),f(x_{0}+0)\right)$不在区间$f(I)$内,
矛盾.

\paragraph{推论3.3.6}

定义在区间$I$上的严格单调连续函数$f(x)$一定可逆, 且其逆严格单调连续.

\subsection{闭区间上连续函数的性质}

依赖实数系的基本性质

\paragraph{定理3.4.1 (有界性定理)}

设$f\in C[a,b]$, 则$f$在$[a,b]$上有界.

\paragraph{证法一:}

反证法, 取$\left|f(x_{n})\right|\ge n$, 由聚点定理$\Longrightarrow x_{n_{k}}\to x_{0}\in[a,b]$,
$f$连续使得$f(x_{n_{k}})\to f(x_{0})$有界, 矛盾.

\paragraph{证法二:}

连续点$x$处有领域$U_{x}(\delta_{x})$, 使得其上$\left|f-f(x)\right|\le1$, 这样的$\left(U_{x}\right)$形成$[a,b]$的覆盖,
用有限覆盖定理.

\paragraph{定理3.4.2 (最值定理)}

设$f(x)\in C[a,b]$, 则$f(x)$在$[a,b]$上取到最值.

\paragraph{证法一:}

$f$有界, $[a,b]$闭, 所以逼近上确界的点列$\left\{ x_{n}\right\} $有收敛子列$\left\{ x_{n_{k}}\right\} $,
再由$f$连续得到最值点.

\paragraph{证法二:}

反证法, 设$f<M\coloneqq\sup f$, 构造$F(y)=\frac{1}{M-f(y)}\in C[a,b]$,
同样有界$F(y)<K$, $K>0$, 则$\frac{1}{M-f(y)}<K$得出
\[
\sup_{x}f(x)=M>f(y)+\frac{1}{K}\Longrightarrow\sup_{x}f(x)\ge\sup_{y}f(y)+\frac{1}{K}>\sup_{x}f(x).
\]

一般区间上连续函数最值判别法:

\paragraph{命题5.1.3}

设$f\in C(\RR)$, 且
\[
\lim_{x\to-\infty}f(x)=\lim_{x\to+\infty}f(x)=+\infty\text{ (或}-\infty\text{)}.
\]
则$f$在$\RR$上达到最小(大)值.

\paragraph{定理3.4.3 (零点定理, Bolzano)}

设$f\in C[a,b]$, $f(a)f(b)<0$, 则存在$\xi\in\left(a,b\right)$, s.t.
$f(\xi)=0$.

\paragraph{证法一:}

区间二分法+区间套定理.

\paragraph{证法二:}

用连续函数的保号性, 构造
\[
A=\left\{ x\in[a,b]:f(x)<0\right\} ,
\]
取$\xi=\sup A$, 则有$x_{n}\in A$, $x_{n}\to\xi$, $f(x_{n})<0$, 所以$f(\xi)\le0$.
反之, 在$(\xi,b)$上$f\ge0$, 取$x_{n}\downarrow\xi$, 则$f(x_{n})\ge0\Longrightarrow f(\xi)=0$.

\paragraph{定理3.4.4 (介值定理)}

设$f\in C[a,b]$, $\mu$严格介于$f(a)$与$f(b)$之间, 则存在$\xi\in(a,b)$, s.t.
$f(\xi)=\mu$.

\paragraph{推论3.4.5}

设$f(x)$是$[a,b]$上的连续函数, 则$f([a,b])=[m,M]$, 其中$m,M$是$f$在$[a,b]$上的最小值,
最大值.

\paragraph{推论3.4.6}

设区间$I$上, $f(x)\in C(I)$, 则$f(I)$是区间. (可以退化为单点集)

\paragraph{注: }

区间$I$可以无界, 可以是开集.

\paragraph{推论3.4.7}

设$f(x)$是区间$I$上的连续函数, 则$f(x)$可逆的充要条件是$f(x)$严格单调.

\subsection{一致连续性}

\paragraph{定义3.4.1 (一致连续)}

设$f(x)$定义在区间$I$上, 若$\forall\epsilon>0$, $\exists\delta=\delta(\epsilon)>0$,
s.t. 当$x_{1},x_{2}\in I$, $\left|x_{1}-x_{2}\right|<\delta$时, 有$\left|f(x_{1})-f(x_{2})\right|<\epsilon$,
则称$f(x)$在$I$中一致连续.

否定表述: $f(x)$在$I$中不一致连续: $\exists\epsilon_{0}>0$, 以及$\left(a_{n}\right)$,
$\left(b_{n}\right)\subseteq I$, 且$a_{n}-b_{n}\to0$, ($n\to\infty$).
有$\left|f(a_{n})-f(b_{n})\right|\ge\epsilon_{0}$.

\paragraph{定理3.4.9 (Cantor定理)}

闭区间上, 连续函数一致连续.

\paragraph{证法一:}

(反证), $\exists\epsilon_{0}>0$, $\left(a_{n}\right)$, $\left(b_{n}\right)\subseteq[a,b]$,
$a_{n}-b_{n}\to0$, $n\to\infty$, 且$\left|f(a_{n})-f(b_{n})\right|\ge\epsilon_{0}$.
用聚点定理取$\left(b_{n}\right)$的收敛子列$b_{n_{k}}\to x_{0}\in[a,b]$, 则$a_{n_{k}}\to x_{0}$,
取极限.

\paragraph{证法二:}

用连续性构造有限覆盖开集.

\paragraph{定义3.4.2 (振幅, 连续性模)}

设$f(x)$在$x_{0}$的开邻域内有定义, 称
\[
\omega_{f}(x_{0},r)=\sup\left\{ \left|f(x')-f(x'')\right|:x',x''\in B_{x_{0}}(r)\right\} ,\quad r>0.
\]
为$f$在$(x_{0}-r,x_{0}+r)$上的振幅, 显然, $\omega_{f}(x_{0},r)$关于$r\to0+$递减,
故
\[
\omega_{f}(x_{0})=\lim_{r\to0+}\omega_{f}(x_{0},r)
\]
存在, (不一定有限). 称为$f$在点$x_{0}$处的振幅.

注: 定义提到的是两种振幅, 分别表征函数$f$在\textquotedbl 区间\textquotedbl 上和在\textquotedbl 一点\textquotedbl 上的振幅.

\paragraph{命题3.4.10}

$f(x)$在$x_{0}$处连续的充要条件是$\omega_{f}(x_{0})=0$.

\paragraph{命题3.4.11}

$f(x)$在$I$中一致连续的充要条件是
\[
\lim_{r\to0+}\omega_{f}(r)=0,
\]
其中
\[
\omega_{f}(r)=\sup\left\{ \left|f(x')-f(x'')\right|:\forall x',x''\in I,\ \left|x'-x''\right|\le r\right\} .
\]


\subsection{连续函数的积分}

\paragraph{积分定义:}

设$f\in C[a,b]$, 直线$x=a$, $x=b$, $y=0$与曲线$f(x)$的图像在平面上所围成图形的面积用$\int_{a}^{b}f(x)\ud x$来表示,
称为$f$在$[a,b]$上的积分.

\paragraph{命题3.5.1}

设$f\in C[a,b]$, $f_{n}(x)$是分段线性函数, 是将$[a,b]$进行$n$等分, 分点$x_{i}=a+\frac{i}{n}(b-a)$,
在$x\in[x_{i-1},x_{i}]$时
\[
f_{n}(x)=l_{i}(x)=f(x_{i-1})+\frac{f(x_{i})-f(x_{i-1})}{x_{i}-x_{i-1}}(x-x_{i-1}).
\]
则$\forall\epsilon>0$, $\exists N=N(\epsilon)$, 当$n>N$时, 
\[
\left|f(x)-f_{n}(x)\right|<\epsilon,\quad\forall x\in[a,b].
\]
即$f_{n}(x)\rightrightarrows f(x)$. 进而
\begin{align*}
	\int_{a}^{b}f(x)\ud x & =\lim_{n\to\infty}\int_{a}^{b}f_{n}(x)\ud x\\
	& =\lim_{n\to\infty}\sum_{i=1}^{n}\frac{1}{2}\left[f(x_{i-1})+f(x_{i})\right]\cdot\frac{b-a}{n}\\
	& =\lim_{n\to\infty}\sum_{i=1}^{n}f(x_{i})\Delta x_{i}.
\end{align*}


\paragraph{积分的基本性质}

约定$\int_{a}^{a}f(x)\ud x=0$, $\int_{b}^{a}f(x)\ud x=-\int_{a}^{b}f(x)\ud x$.

(1) (线性性) $f,g\in C[a,b]$, $\alpha,\beta\in\RR$, 则
\[
\int_{a}^{b}\left(\alpha f(x)+\beta g(x)\right)\ud x=\alpha\int_{a}^{b}f(x)\ud x+\beta\int_{a}^{b}g(x)\ud x.
\]

(2) 若$\left|f(x)\right|\le M$, $\forall x\in[a,b]$, 则
\[
\left|\int_{a}^{b}f(x)\ud x\right|\le M(b-a).
\]

(3) (保序性) 若$f\ge g$, $f,g\in C[a,b]$, 则$\int_{a}^{b}f\ge\int_{a}^{b}g$.
特别地, 
\[
\left|\int_{a}^{b}f(x)\ud x\right|\le\int_{a}^{b}\left|f(x)\right|\ud x.
\]

(4) (区间可加性) 设$f\in C(I)$, $a,b,c\in I$, 则
\[
\int_{a}^{c}f(x)\ud x=\int_{a}^{b}f(x)\ud x+\int_{b}^{c}f(x)\ud x.
\]

(5) 若$f(x)\in C[a,b]$, 非负, 则$\int_{a}^{b}f(x)\ge0$, 等号仅当$f\equiv0$时取到.

\paragraph{例3.5.4}

设$f\in C[a,b]$, $c\in[a,b]$, 定义$F(x)=\int_{c}^{x}f(t)\ud t$, $x\in[a,b]$,
则$F$是Lipschitz函数.

\paragraph{证明:}

\[
\left|F(x_{2})-F(x_{1})\right|\le M\int_{x_{1}}^{x_{2}}\ud t.
\]


\paragraph{命题3.5.2 (积分中值定理)}

设$f,g\in C[a,b]$, 若$g$不变号, 则$\exists\xi\in[a,b]$, s.t.
\[
\int_{a}^{b}f(x)g(x)\ud x=f(\xi)\int_{a}^{b}g(x)\ud x.
\]


\paragraph{证明:}

用连续函数介值定理.

\paragraph{例3.5.10}

设$f\in C[0,a]$, 定义
\[
f_{0}(x)=f(x),\quad f_{n}(x)=\int_{0}^{x}f_{n-1}(t)\ud t,\quad n=1,2,\cdots.
\]
证明: 存在$\xi=\xi_{n,x}\in[0,x]$, s.t. $f_{n}(x)=f(\xi)\frac{x^{n}}{n!}$.

\paragraph{证明:}

\[
m\frac{x^{n}}{n!}\le\int_{0}^{x}f_{n-1}(t)\ud t\le M\frac{x^{n}}{n!},
\]
用介值定理.

\paragraph{例3.5.12}

求连续函数$f$满足
\[
f(x+y)=f(x)+f(y),\qquad\forall x,y\in\RR.
\]


\paragraph{证明:}

此题的条件可以弱化为$f$是可积函数. 取积分
\[
\int_{0}^{y}f(x+t)\ud t=\int_{x}^{x+y}f(t)\ud t=\int_{0}^{y}f(x)\ud t+\int_{0}^{y}f(t)\ud t,
\]
所以
\[
\int_{0}^{x+y}f(t)\ud t=yf(x)+\int_{0}^{x}f(t)\ud t+\int_{0}^{y}f(t)\ud t.
\]
交换$x,y$的位置即得$yf(x)=xf(y)$, 再取$y=1$.

\paragraph{例3.5.15}

设$f\in C[a,b]$, $g$是周期为$T$的连续函数, 则
\[
\lim_{n\to\infty}\int_{a}^{b}f(x)g(nx)\ud x=\frac{1}{T}\int_{a}^{b}f(x)\ud x\int_{0}^{T}g(x)\ud x.
\]


\subsection{作业}

16. 设 $0<a<b,$ $a_{1}=a$, $b_{1}=b$. 

(1) 如果 $a_{n+1}=\sqrt{a_{n}b_{n}}$, $b_{n+1}=\frac{1}{2}\left(a_{n}+b_{n}\right)$,
则 $\left\{ a_{n}\right\} $ 和 $\left\{ b_{n}\right\} $ 收敛于同一极限. 

(2) 如果 $a_{n+1}=\frac{1}{2}\left(a_{n}+b_{n}\right),b_{n+1}=\frac{2a_{n}b_{n}}{a_{n}+b_{n}}$,
则 $\left\{ a_{n}\right\} $ 和 $\left\{ b_{n}\right\} $ 收敛于同一极限.

pf. (1). 因为$a_{1},b_{1}>0$, 由$a_{n},b_{n}>0$可以得到, 
\[
a_{n+1}=\sqrt{a_{n}b_{n}}>0,\ b_{n+1}=\frac{a_{n}+b_{n}}{2}>0.
\]
即由归纳法, $\left\{ a_{n}\right\} $, $\left\{ b_{n}\right\} $均是正数数列.

由均值不等式, 
\[
a_{n+1}=\sqrt{a_{n}b_{n}}\le\frac{a_{n}+b_{n}}{2}=b_{n+1},\quad n\ge1.
\]
又由$a_{1}=a<b=b_{1}$, 故对于任意的$n\in\NN$, 有$a_{n}\le b_{n}$. 于是
\[
b_{n+1}=\frac{a_{n}+b_{n}}{2}\le\frac{b_{n}+b_{n}}{2}=b_{n},
\]
知$\left\{ b_{n}\right\} $单调递减; 同理, 由
\[
a_{n+1}=\sqrt{a_{n}b_{n}}\ge\sqrt{a_{n}a_{n}}=a_{n},
\]
知$\left\{ a_{n}\right\} $单调上升. 于是
\[
a_{1}\le a_{2}\le\cdots\le a_{n}\le b_{n}\le b_{n-1}\le\cdots\le b_{1}.
\]
这说明$\left\{ a_{n}\right\} $是单调上升的有界序列, 上界是$b_{1}$; $\left\{ b_{n}\right\} $是单调下降的有界序列,
下界是$a_{1}$.

由于单调有界序列必然收敛, 可设$a_{n}\to a$, $b_{n}\to b$, ($n\to\infty$). 对$b_{n+1}=\frac{a_{n}+b_{n}}{2}$两边同时取极限$\lim_{n\to\infty}$得到
\[
b=\frac{a+b}{2}\Longleftrightarrow a=b.
\]
即$\left\{ a_{n}\right\} $和$\left\{ b_{n}\right\} $收敛于同一极限.

13. 设 $f(x)$ 是 $(a,b)$ 中定义的无第二类间断点的函数, 如果对任意的两点 $x$, $y\in(a,b)$,
均有 
\[
f\left(\frac{x+y}{2}\right)\leqslant\frac{1}{2}[f(x)+f(y)],
\]
则 $f$ 为 $(a,b)$ 中的连续函数.

pf. 用反证法, 若$f(x)$在$(a,b)$上不连续, 则存在$x_{0}\in(a,b)$, 使得$f$在$x_{0}$处为第一类间断点(因为$f$没有第二类间断点).

若$f$在$x_{0}$处为跳跃间断点, 则$f(x_{0}-0)$和$f(x_{0}+0)$均存在且不相等, 不妨设$f(x_{0}-0)<f(x_{0}+0)$,
取$\epsilon<\frac{f(x_{0}+0)-f(x_{0}-0)}{4}$, 则存在$\delta>0$, 使得对于任何$x\in(x_{0}-\delta,x_{0})$,
\[
f(x_{0}-0)-\epsilon<f(x)<f(x_{0}-0)+\epsilon;
\]
且对于任何$y\in(x_{0},x_{0}+\delta)$,
\[
f(x_{0}+0)+\epsilon>f(y)>f(x_{0}+0)-\epsilon.
\]
特别的
\[
f\left(x_{0}+\frac{\delta}{8}\right)=f\left(\frac{x_{0}-\frac{\delta}{4}+x_{0}+\frac{\delta}{2}}{2}\right)>f(x_{0}+0)-\epsilon>\frac{3f(x_{0}+0)+f(x_{0}-0)}{4}.
\]
而
\[
\frac{1}{2}\left(f\left(x_{0}-\frac{\delta}{4}\right)+f\left(x_{0}+\frac{\delta}{2}\right)\right)<\frac{1}{2}\left(f(x_{0}-0)+\epsilon+f(x_{0}+0)+\epsilon\right)\le\frac{3f(x_{0}+0)+f(x_{0}-0)}{4},
\]
这与
\[
f\left(\frac{x_{0}-\frac{\delta}{4}+x_{0}+\frac{\delta}{2}}{2}\right)\le\frac{1}{2}\left(f\left(x_{0}-\frac{\delta}{4}\right)+f\left(x_{0}+\frac{\delta}{2}\right)\right)
\]
矛盾.

若$f$在$x_{0}$处是跳跃间断点, 则$f(x_{0}\pm0)$均存在且不等于$f(x_{0})$. 取$x=x_{0}-t$,
$y=x_{0}+t$, 并令$t\to0$, 则
\[
f\left(\frac{x+y}{2}\right)=f(x_{0})\le\frac{f(x_{0}+t)+f(x_{0}-t)}{2}\to\lim_{x\to x_{0}}f(x),\qquad t\to0.
\]
即$f(x_{0})<\lim_{x\to x_{0}}f(x)$. 取$x=x_{0}$, $y=x_{0}+t$, 并令$t\to0+$,
则
\[
f\left(\frac{t}{2}+x_{0}\right)=f\left(\frac{x+y}{2}\right)\le\frac{f(x_{0})+f(x_{0}+t)}{2}\Longrightarrow\lim_{x\to x_{0}}f(x)\le\frac{f(x_{0})+\lim_{x\to x_{0}}f(x)}{2}\Longrightarrow\lim_{x\to x_{0}}f(x)\le f(x_{0}).
\]
又矛盾.

15. 设 $f$ 为 $\mathbb{R}$ 上的连续函数, 如果对任意的 $x,y\in\mathbb{R}$ 均有 $f(x+y)=f(x)f(y)$,
则要么 $f$ 恒为零, 要么存在常数 $a>0$, 使得 $f(x)=a^{x}$, $\forall x\in\mathbb{R}$.

pf. 若$f$不恒为零, 则由于$f$连续, $f$在$\RR$上保持符号. 不然则由连续函数的介值定理, 存在$x_{0}$使得$f(x_{0})=0$,
则对于任意的$x\in\RR$,
\[
f(x)=f(x_{0})f(x-x_{0})\equiv0.
\]
与$f$不恒为零矛盾.

$f$不能恒为负值, 不然$f(2x)=f(x)^{2}\Longrightarrow f(2x)>0$导致$f$在$\RR$上不再保持符号,
与上述推导矛盾. 所以$f$在$\RR$恒为正.

取$g(x)=\ln f(x)$, 由于$f$的连续性, $g(x)$也是$\RR$上的连续函数. 并有
\[
g(x+y)=\ln f(x+y)=\ln\left(f(x)f(y)\right)=\ln f(x)+\ln f(y)=g(x)+g(y).
\]
由于$g$连续, 所以$g(x)=g(1)x$. 令$c=g(1)$, $a=f(1)>0$, 即有$\ln f(x)=cx=x\ln f(1)=\ln a^{x}$,
$f(x)=a^{x}$. (14题的结论是经典结论, 任何时候都可以直接拿来用)

1. 设 $f$ 在 $[a,+\infty)$ 中连续, 且 $\lim_{x\rightarrow+\infty}f(x)=A$,
则 $f$ 在 $[a,+\infty)$ 中有界, 且最大值和最小值中的一个必定能被 $f$ 达到.

pf1. 考虑函数
\[
g(t)\coloneqq\sup_{x\ge t}f(x),
\]
则$g(t)$是$[a,+\infty)$上连续的递减函数. 且
\[
\lim_{t\to\infty}g(t)=\limsup_{x\to\infty}f(x)=\alpha.
\]
所以$g(t)\ge\alpha$.

如果存在$t$使得$g(t)>\alpha$, 也即$\sup_{x\ge t}f(x)>\alpha$, 取$\epsilon<\sup_{x\ge t}f(x)-\alpha$,
则有$X>t$, s.t. $\forall x>X$, 
\[
\left|f(x)-\alpha\right|<\epsilon\Longrightarrow f(x)<\alpha+\epsilon\Longrightarrow\sup_{x>X}f(x)\le\alpha+\epsilon<\sup_{x\ge t}f(x),
\]
这说明$f(x)$在$[t,\infty)$上的上确界不可能在$[X,\infty)$上取到, 即
\[
\sup_{x\ge t}f(x)=\sup_{t\le x\le X}f(x)\Longrightarrow\sup_{x\ge a}f(x)=\sup_{a\le x\le X}f(x)
\]
即$f(x)$在$[a,X]$上存在最大值, 也即$f(x)$在$[a,\infty)$上存在最大值.

若对于任何$t$, $g(t)=\alpha$. 也即
\[
\sup_{x\ge t}f(x)=\alpha,\quad\forall t\ge a.
\]
当$f(x)$不为常数时, 则必然有$f(x_{0})<\alpha$. 取$\epsilon<\alpha-f(x_{0})$,
则有$X>x_{0}$, s.t. $\forall x>X$
\[
\left|f(x)-\alpha\right|<\epsilon\Longrightarrow f(x)>\alpha-\epsilon>f(x_{0}).
\]
所以$f(x)$必然在$[a,X]$中取到最小值.

pf2. $f(x)$在$[a,\infty)$不恒为常数.

因为$f(x)\to\alpha$, $x\to\infty$, 则有点列$\left\{ x_{n}\right\} $,
使得$f(x_{n})\to\alpha$, $f(x_{1})\ne\alpha$(因为$f$不恒为常数).

因为$f(x_{1})\ne\alpha$, 则取$\epsilon<\left|f(x_{1})-\alpha\right|$,
有$X>x_{1}$, s.t. $\forall x>X$,
\[
\left|f(x)-\alpha\right|<\epsilon<\left|f(x_{1})-\alpha\right|,
\]
这意味着在$[a,X]$内存在点$x_{1}$的函数值$f(x_{1})$远离$(\alpha-\epsilon,\alpha+\epsilon)$.

当$f(x_{1})$>$\alpha$时, $f(x)$在$[a,X]$上存在最大值$M\ge f(x_{1})$, 而在$(X,\infty)$,
$f(x)\le\alpha+\epsilon<f(x_{1})\le M$. 最大值存在性得证.

当$f(x_{1})$<$\alpha$时, $f(x)$在$[a,X]$上存在最小值$m\le f(x_{1})$, 而在$(X,\infty)$,
$f(x)\ge\alpha-\epsilon>f(x_{1})\ge m$. 最小值存在性得证.

3. 设$b,\alpha>0$, 求积分$\int_{0}^{b}x^{\alpha}\ud x$, 利用积分计算极限
\[
\lim_{n\to\infty}\frac{1^{\alpha}+2^{\alpha}+\cdots+n^{\alpha}}{n^{\alpha+1}}.
\]

8. 设$f\in C[a,b]$, 如果对于任意的$g\in\left\{ g\in C[a,b]:g(a)=g(b)=0\right\} $,
均有$\int_{a}^{b}f(x)g(x)\ud x=0$, 则$f\equiv0$.

此题类比变分基本定理.

9. 设$f\in C[a,b]$, 如果对于任意的$g\in\left\{ g\in C[a,b]:\int_{a}^{b}g(x)\ud x=0\right\} $,
均有$\int_{a}^{b}f(x)g(x)\ud x=0$, 则$f=C$为常函数.

提示: 设$f$的平均值为$C$, 考虑$g=f-C$和$g^{2}$的积分.

12. 设$f\in C[a,b]$, 则
\[
\lim_{n\to+\infty}\left(\int_{a}^{b}\left|f(x)\right|^{n}\ud x\right)^{1/n}=\max_{x\in[a,b]}\left|f(x)\right|.
\]

14. 设$f,g\in C[a,b]$, 且$f,g>0$, 证明
\[
\lim_{n\to\infty}\frac{\int_{a}^{b}f^{n+1}(x)g(x)\ud x}{\int_{a}^{b}f^{n}(x)g(x)\ud x}=\max_{x\in[a,b]}f(x).
\]

17. 设$f(x)\in C[0,+\infty)$严格单调递增, 则
\[
F(x)=\begin{cases}
	\frac{1}{x}\int_{0}^{x}f(t)\ud t, & x>0,\\
	f(0), & x=0
\end{cases}
\]
也是严格单调递增连续函数.

18. 设$f(x)\in C[a,b]$, $f>0$, 则
\[
\lim_{r\to0+}\left(\frac{1}{b-a}\int_{a}^{b}f^{r}(x)\ud x\right)^{1/r}=\exp\left(\frac{1}{b-a}\int_{a}^{b}\ln f(x)\ud x\right),
\]
并用H\"{o}lder不等式说明上式左端关于$r$单调递增.