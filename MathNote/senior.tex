\chapter{高中数学}

%%%%%%%%%%%%%%%%%%%%%%%%%%%%%%%%%%%%%%%%%%%%%%%%%%%%%%%%%%%%%%%%%%%%%%%%%%%%%%
%%%%%%%%%%%%%%%%%%%%%%%%%%%%%%%%%%%%%%%%%%%%%%%%%%%%%%%%%%%%%%%%%%%%%%%%%%%%%%
%%%%%%%%%%%%%%%%%%%%%%%%%%%%%%%%%%%%%%%%%%%%%%%%%%%%%%%%%%%%%%%%%%%%%%%%%%%%%%

\section{2016年中科大入学数学考试}
\bq{}{}
在四面体$ABCD$中, $AD=BD=CD$, $AB=BC=CA=1$. 若二面角$A-BC-D$等于$75\degree$, 求二面角$A-BD-C$的余弦值.
\eq
\ba
用空间余弦定理. 答案是: $\frac{3\sqrt{3}-2}{8}$.
\ea

\bq{}{}
设$m,n$非负整数, 证明$\frac{(2m)!(2n)!}{m!n!(m+n)!}$是整数.
\eq
\ba
记所讨论的数为$f(m,n)$, 对$m$归纳证明$f(m+1,n)=4f(m,n)-f(m,n+1)$.
\ea

\bq{}{}
设正数$a, b, c$满足$ab+bc+ca=1$. 求$\frac{a}{\sqrt{1+a^2}}+\frac{b}{\sqrt{1+b^2}}+\frac{c}{\sqrt{1+c^2}}$的取值范围.
\eq
\ba
联想到$\triangle ABC$中, $\tan A+\tan B+\tan C=\tan A\tan B\tan C$,
所以可取$a=\cot A$, $\sum\frac{a}{\sqrt{1+a^2}}=\sum\cos A$, 用琴生不等式证明
$\sum\cos A\le\frac32$. 另一方面, $\sum\cos A=1+4\sin\frac{A}{2}\sin\frac{B}{2}\sin\frac{C}{2}>1$.
故所求取值范围为$\left(1,\frac32\right]$.
\ea

\bq{}{}
正项数列$\{a_n\}$满足$a_1=1$, $f(a_n)=a_{n+1}(n>1)$, 其中$f(x)=\ue^x-\cos x$, 求证: 存在正整数$K$使得
$\sum\limits_{k=1}^{K}a_{k}>2016$.
\eq
\ba
函数$f(x)$的一次导数为正, 数列$\{a_{n}\}$是递增数列.
\ea

%%%%%%%%%%%%%%%%%%%%%%%%%%%%%%%%%%%%%%%%%%%%%%%%%%%%%%%%%%%%%%%%%%%%%%%%%%%%%%
%%%%%%%%%%%%%%%%%%%%%%%%%%%%%%%%%%%%%%%%%%%%%%%%%%%%%%%%%%%%%%%%%%%%%%%%%%%%%%
%%%%%%%%%%%%%%%%%%%%%%%%%%%%%%%%%%%%%%%%%%%%%%%%%%%%%%%%%%%%%%%%%%%%%%%%%%%%%%

\section{初中代数题}
\bq{}{}
设$a$是实数, 试确定多项式$x^4-2ax^2-x+a^2-a$的实根的个数.
\eq
\ba
把多项式看作$a$的多项式, 因式分解得$(x^2-x-a)(x^2+x-a+1)$.
\ea

\bq{}{20170518005}
设 $f(x)=a_nx^n+\cdots+a_1x+a_0\in\Bbb C[x]$, $a_n\ne0$, $\alpha$ 是 $f(x)$ 任一根, 则
\bee
|\alpha|<1+\max_{0\le k\le n-1}\left|\frac{a_k}{a_n}\right|.
\eee
\eq
\ba
反证法, 由
\bee
 \left|\alpha^n+\frac{a_{n-1}}{a_{n}}\alpha^{n-1}+\cdots+\frac{a_1}{a_n}\alpha+\frac{a_0}{a_n}\right|
 \ge |\alpha|^n-\left|\frac{a_{n-1}}{a_n}\right|\cdot|\alpha|^{n-1}-\cdots-\left|\frac{a_1}{a_n}\right|\cdot|\alpha|+\left|\frac{a_0}{a_n}\right|
 \ge 1.
\eee
\ea
\ba
只证$|\alpha|>1$时的情况, 记$M=\max_{0\le k\le n-1}\left|\frac{a_k}{a_n}\right|$. 由$a_0+\cdots+a_{n-1}\alpha^{n-1}=-a_{n}\alpha^n$得
\bee
|\alpha|^n=\left|\frac{a_0}{a_n}+\cdots+\frac{a_{n-1}}{a_{n}}\alpha^{n-1}\right|
\le M(1+|\alpha|+\cdots+|\alpha|^{n-1})<M\frac{|\alpha|^n}{|\alpha|-1}.
\eee
\ea

\bq{}{20170518003}
设$f(x)=a_0x^n+a_1x^{n-1}+\cdots+a_{n-1}x+a_n$中, $0<a_0<a_1<\cdots<a_{n-1}<a_{n}$. 则$f(x)$的根的模均大于1.
\eq
\ba
反证法, 设$|\alpha|\le1$是$xf(x)-f(x)=a_0x^{n+1}+(a_1-a_0)x^n+\cdots+(a_{n}-a_{n-1})x-a_{n}$的零点. 所以
\bee
|a_{n}|=|a_{0}\alpha^{n+1}+(a_1-a_0)\alpha^n+\cdots+(a_{n}-a_{n-1})\alpha|\le a_0|\alpha|^{n+1}+(a_1-a_0)|\alpha|^n+\cdots\le a_n.
\eee
等号当且仅当$\alpha=1$时取到. 这不可能.
\ea

\bq{}{20170518004}
设多项式$f(x)=a_0x^n+a_1x^{n-1}+\cdots+a_{n-1}x+a_n\in\Bbb Z[x]$, 且满足
\begin{enumerate}[(i)]
 \item $0<a_0<a_1<\cdots<a_{n-1}<a_{n}$;
 \item $a_n=p^m$, 这里$p$是一个素数($m$是正整数), 且$p\nmid a_{n-1}$.
\end{enumerate}
证明: $f(x)$在$\Bbb Z$上不可约.
\eq
\ba
反证法, $f=gh$. $g=b_0x^r+\cdots+b_r$, $h=c_0x^s+\cdots+c_s$. 则由条件不妨设$p\nmid b_r$, $\abs{c_s}=p^m$, $\abs{b_r}=1$. 
由\ref{q:20170518003}可知, $f$的根的模均大于1, 所以$g$的根$\alpha_{1},\cdots,\alpha_{r}$的模均大于1.
$\abs{g(0)}=\abs{b_r}=\abs{b_0}\abs{\alpha_1}\cdots\abs{\alpha_r}>\abs{b_0}\ge1$, 与$\abs{b_r}=1$矛盾.
\ea

\bq{}{20170518006}
设$f(x)=a_nx^n+\cdots+a_1x+p\in\Bbb Z[x]$, $a_n\ne 0$, $p$是素数, 且
\bee
\abs{a_1}+\cdots+\abs{a_n}<p,
\eee
证明: $f(x)$在$\Bbb Z$上不可约.
\eq
\ba
反证, $f=gh$, $f$的根的模均大于1, 若$|g(0)|=1$, 则其实线性分解的模大于1. 矛盾.
\ea

\bq{}{}
设$f(x)=a_nx^n+\cdots+a_1x+a_0$是一个整系数多项式, $a_n\ne0$. 记
\bee
M=\max_{0\le i\le n-1}\abs{\frac{a_i}{a_n}}.
\eee
若有一个整数$m\ge M+2$, 使$|f(m)|$为素数, 则$f(x)$在$\Bbb Z$上不可约.
\eq
\ba
反证, $f=gh$, 由\ref{q:20170518005}知, 当$\abs{g(m)}=1$时, $\abs{g(m)}$的实线性分解的每项都大于1而导致矛盾.
\ea

\bq{}{}
判别$x^5+4x^4+2x^3+3x^2-x+5$在$\Bbb Q$上是否可约.
\eq
\ba
让$x+6$代替$x$化简得$x^5+34x^4+458x^3+3063x^2+10187x+13499$, 由于$13499$是素数, 由\ref{q:20170518004}即得.
\ea
\ba
待定系数法分解为二次多项式与三次多项式的乘积.
\ea
\ba
用\ref{q:20170518006}.
\ea

\bq{}{20170518001}
设$f(x)=a_nx^n+\cdots+a_1x+a_0\in\Bbb Z[x]$, 其中$0\le a_i\le 9$($i=0,1,\cdots,n$), 且$a_n\ne0$. 如$\alpha$是$f(x)$的一个复根, 则
$\Re(\alpha)\le0$或$|\alpha|<4$.
\eq
\ba
若$\Re(\alpha)\le0$或$|\alpha|\le1$, 则无需证明. 对于$\Re(\alpha)>0$且$|\alpha|>1$, 有$\Re\left(\frac{1}{\alpha}\right)>0$, 故从
\bee
0=\abs{\frac{f(\alpha)}{\alpha^n}}\ge\abs{a_0+\frac{a_1}{\alpha}}-\frac{a_2}{|\alpha|^2}-\cdots-\frac{a_n}{|\alpha|^n}
  \ge \Re\left(a_0+\frac{a_1}{\alpha}\right)-\frac{9}{|\alpha|^2}-\cdots-\frac{9}{|\alpha|^n}>1-\frac{9}{|\alpha|^2-|\alpha|}.
\eee
可得.
\ea

\bq{}{20170518002}
设$f(x)$是$n$次整系数多项式($n\ge1$), $\alpha_1,\cdots,\alpha_n$是其全部复根. 若存在整数$k$, $k>1+\Re(\alpha_j)$, ($j=1,\cdots,n$), 
使$|f(k)|$是素数, 则$f(x)$在$\Bbb Z$上不可约.
\eq
\ba
反证, $f=gh$, 对$g$进行$\Bbb R[x]$上的标准分解, 则有$\abs{g(k)}>1$, 同理$\abs{h(k)}>1$. 与$f(k)$素矛盾.
\ea

\bq{http://tieba.baidu.com/p/4811340691}{}
求证: $148<\sum\limits_{n=1}^{1000}\frac{1}{\sqrt[3]{n}}<150$.
\eq
\ba
由R-S积分, 
\bee
\sum_{n=1}^{1000}\frac{1}{\sqrt[3]{n}}=\int_{1^{-}}^{1000^{+}}\frac{1}{\sqrt[3]{x}}\ud[x]
  = \int_{1^{-}}^{1000^{+}}\frac{1}{\sqrt[3]{x}}\ud x-\int_{1^{-}}^{1000^{+}}\frac{1}{\sqrt[3]{x}}\ud\{x\}
  = 148.5-\frac13\int_{1^{-}}^{1000^{+}}\frac{\{x\}}{x^{4/3}}\ud x-\frac{\{x\}}{\sqrt[3]{x}}\Bigg|_{1^{-}}^{1000^{+}}.
\eee
求极限后得到
\bee
\sum_{n=1}^{1000}\frac{1}{\sqrt[3]{n}}=149.5-\frac13\int_{1}^{1000}\frac{\{x\}}{x^{4/3}}\ud x.
\eee
上式小于$150$是显然的, 然后利用$\{x\}<1$, 得到$\sum\limits_{n=1}^{1000}\frac{1}{\sqrt[3]{n}}>148.6$. 

贴吧答案: 先证放缩公式:
\bee
\frac{3}{2}\left(\sqrt[3]{(n+1)^2}-\sqrt[3]{n^2}\right)<\frac{1}{\sqrt[3]{n}}<\frac32\left(\sqrt[3]{n^2}-\sqrt[3]{(n-1)^2}\right)
\eee
由中值定理可证上式.
\ea

\bq{http://tieba.baidu.com/p/4932376161}{}
求
\bee
\sum_{k=1}^{2014}\left[\frac{-3+\sqrt{8k+1}}{4}\right]
\eee
\eq
\ba
令$t=\frac{-3+\sqrt{8k+1}}{4}$, 则$k=2t^2+3t+1$.
所以$\frac{-3+\sqrt{8k+1}}{4}=n$当且仅当$2n^2+3n+1\le k<2(n+1)^2+3(n+1)+1$, $n\in\mathbb{N}$.
另一方面$2\times30^2+3\times30+1=1891$, $2\times31^2+3\times31+1=2016$.
所以
\begin{align*}
\sum_{k=1}^{2014}\left[\frac{-3+\sqrt{8k+1}}{4}\right]
  & = \sum_{n=1}^{31}n[2(n+1)^2+3(n+1)+1-(2n^2+3n+1)]-30\\
  & = \sum_{n=1}^{31}(4n^2+50n)-30\\
  & = 4(1^2+2^2+\cdots+30^2)+5(1+2+3+\cdots+30)-30=40115.
\end{align*}

\ea

%%%%%%%%%%%%%%%%%%%%%%%%%%%%%%%%%%%%%%%%%%%%%%%%%%%%%%%%%%%%%%%%%%%%%%%%%%%%%%
%%%%%%%%%%%%%%%%%%%%%%%%%%%%%%%%%%%%%%%%%%%%%%%%%%%%%%%%%%%%%%%%%%%%%%%%%%%%%%
%%%%%%%%%%%%%%%%%%%%%%%%%%%%%%%%%%%%%%%%%%%%%%%%%%%%%%%%%%%%%%%%%%%%%%%%%%%%%%

\section{全国高中数学联赛}
\bq{陕西省第七次大学生高等数学竞赛复赛}{}
求函数$f(x,y)=\max\{|x-y|, |x+y|,|x-2|\}$的最小值.
\eq

\bq{}{}
设$n,a,b$是整数, $n>0$且$a\ne b$, 若$n\mid(a^n-b^n)$, 则$n\mid\frac{a^n-b^n}{a-b}$.
\eq
\ba
设$p^{\alpha}\| n$, 只需证$p^{\alpha}\mid\frac{a^n-b^n}{a-b}$. 记$t=a-b$, 如果$p\nmid t$,
则$(p^{\alpha}, t)=1$, 于是$p^{\alpha}\mid \frac{a^n-b^n}{a-b}$.
若$p\mid t$, 由二项式定理有
\bee
\frac{a^n-b^n}{t}=\sum_{i=1}^{n}\binom{n}{i}b^{n-i}t^{i-1}.
\eee
设$p^{\beta}\| i$, 则易知$\beta\le i-1$. 因此$\binom{n}{i}t^{i-1}=\frac{n}{i}\binom{n-1}{i-1}t^{i-1}$中所含的$p$的幂次至少是$\alpha$, 
即上式右边每一项均被$p^{\alpha}$整除.
\ea

\bq{2016年全国高中数学联赛四川预赛}{}
已知$a, b, c$为正实数, 求证:
\bee
abc\ge \frac{a+b+c}{\frac1{a^2}+\frac1{b^2}+\frac1{c^2}}\ge(a+b-c)(b+c-a)(c+a-b).
\eee
\eq
\ba
先证左边, 左边等价于$(ab)^2+(bc)^2+(ca)^2\ge abc(a+b+c)$, 而这等价于$x^2+y^2+z^2\ge xy+yz+zx$.

再证右边, 右边等价于$a+b+c\ge(a+b-c)(b+c-a)(c+a-b)\left(\frac1{a^2}+\frac1{b^2}+\frac1{c^2}\right)$. 不妨设$a=\max\{a,b,c\}$, 则只需考虑$b+c-a>0$的情况, 令$a=y+z$, 上式等价于$2(x+y+z)\ge8xyz\left(\frac1{(y+z)^2}+\frac1{(z+x)^2}+\frac1{(x+y)^2}\right)$. 即
\bee
\frac1{yz}+\frac1{zx}+\frac1{xy}\ge\frac4{(y+z)^2}+\frac4{(z+x)^2}+\frac4{(x+y)^2}.
\eee
最后用均值不等式.
\ea

\bq{2009年全国高中数学联赛加试}{}
证明:
\bee
-1<\sum_{k=1}^{n}\frac{k}{k^2+1}-\ln n\le\frac12, \quad n=1,2,\cdots.
\eee
\eq
\ba
利用不等式$\frac{x}{1+x}<\ln (1+x) < x$, $x_n=\sum_{k=1}^{n}\frac{k}{k^2+1}-\ln n$, 则$x_{n+1}<x_n<\cdots <x_1=\frac12$.
在由$\ln n=\sum_{k=1}^{n-1}\ln\left(1+\frac1k\right)$, $x_n>-\sum_{k=1}^{n-1}\frac{1}{k(k+1)}>-1$.
\ea

\bq{}{}
设集合$S_{k}=\{A_k\mid k\in\mathbb{N}_+\}$, 其中$A_{k}=\frac{2^{2^{k+1}}+1}{7^{7^k}+1}$, 求: $S_1, \cdots, S_{2015}$中任取两个数都是素数的概率.
\eq
\ba
用恒等式$\frac{x^7+1}{x+1}=(x+1)^6-7x(x^2+x+1)^2$, 当$x=7^{2m-1}$时, $(x+1)^6-7x(x^2+x+1)^2$有平方差公式.
\ea

\bq{IMO预选题}{20170117001}
证明: 对于任意正有理数$r$, 都存在正整数$a,b,c,d$满足$r=\frac{a^3+b^3}{c^3+d^3}$.
\eq
\ba
用恒等式$\frac{(x+y)^2+(2x-y)^2}{(x+y)^2+(2y-x)^2}=\frac{x}{y}$, $\frac{x}{y}\in\left(\frac12, 2\right)$. 推广见\ref{un:20170117002}.
\ea

\bq{Titu problem from book}{}
设$x_n$表示如下含有素数$2$的幂指数
\bee
\frac{2}{1}+\frac{2^2}{2}+\cdots+\frac{2^n}{n}. 
\eee
求证: $x_{2^n}\ge 2^n-n+1$.
\eq
\ba
用恒等式$\frac{2}{1}+\frac{2^2}{2}+\cdots+\frac{2^n}{n}=\frac{2^n}{n}\sum_{k=0}^{n-1}\frac1{\binom{n-1}{k}}$. 
由Lucas定理$\binom{2^n-1}{k}\equiv 1\pmod{2}$, $0\le k\le 2^n-1$. 故$\sum_{k=0}^{n-1}\frac1{\binom{n-1}{k}}\pmod{2}$意义下是$2^n$个奇数相加, 
所以一定是偶数. 那么$\frac{2^n}{n}\sum_{k=0}^{n-1}\frac1{\binom{n-1}{k}}$的和含有的$2$的指数至少为$2^n-n+1$, 故有$x_{2^n}\ge2^n-n+1$.
\ea

\bq{匈牙利征解}{}
设$P(x)$为次数不超过$2n$的多项式, 求证:
\bee
|P(n)|\le(2\sqrt{n}-1)\max(|P(0)|,|P(1)|, \cdots, |P(n-1)|, |P(n+1)|, \cdots, |P(2n)|).
\eee
\eq
\ba
用差分公式, 当$m\ge2n$时有, $\Delta^mP(0)=(E-I)^mP(0)=0$, $E$为移位算子$EP(x)=P(x+1)$, $I$为恒等算子, 得到
\bee
\sum_{k=0}^{m}(-1)^k\binom{m}{k}P(k)=0.
\eee
令$m=2n$. 并用绝对值的三角不等式可证命题. 其中需要用归纳法证明
\bee
\binom{2n}{n}\ge\frac{2^{2n-1}}{\sqrt{n}}.
\eee
\ea

%%%%%%%%%%%%%%%%%%%%%%%%%%%%%%%%%%%%%%%%%%%%%%%%%%%%%%%%%%%%%%%%%%%%%%%%%%%%%%
%%%%%%%%%%%%%%%%%%%%%%%%%%%%%%%%%%%%%%%%%%%%%%%%%%%%%%%%%%%%%%%%%%%%%%%%%%%%%%
%%%%%%%%%%%%%%%%%%%%%%%%%%%%%%%%%%%%%%%%%%%%%%%%%%%%%%%%%%%%%%%%%%%%%%%%%%%%%%

\section{数学竞赛}
\bq{Law of Tangents}{}
\bee
\frac{a-b}{a+b}=\frac{\tan\frac{A-B}{2}}{\tan\frac{A+B}{2}}.
\eee
\eq

\bq{}{}
求
\bee
(x+1)^x+(x+2)^x=(x+3)^x
\eee
的实解.
\eq
\ba
只能$x>-1$, 令$f(x)=\paren{1-\frac2{x+3}}^x+\paren{1-\frac1{x+3}}^x-1$,
\bee
f'(x)=\cdots(\textrm{用}\ln(1-x)<-x)<-\frac2{(x+1)(x+3)}\paren{1-\frac2{x+3}}^x-\frac2{(x+2)(x+3)}\paren{1-\frac1{x+3}}^x<0,
\eee
在$x>-1$时, $f(x)$单调递减, 而$f(2)=0$, 所以$x$只有实解$x=2$.
\ea

\bq{}{}
求满足如下恒等式的所有实函数$f(x):\Bbb R\to\Bbb R$,
\be\label{20170703}
f(f(x))f(y)-xy=f(x)+f(f(y))-1.
\ee
\eq
\ba
让$x,y$互换, 则有$f(f(y))f(x)-xy=f(y)+f(f(x))-1$. 将\ref{20170703}式两边同时乘以$f(x)$与前式相加化简得
\bee
(f(x)f(f(x))-1)f(y)=xy(f(x)+1)+f^2(x)-f(x)+f(f(x))-1.
\eee
若存在$x$使$f(x)f(f(x))-1\ne0$, 则$f(y)=ay+b$, 代入\ref{20170703}知$a,b$无解;
若对于任意的$x$都有$f(x)f(f(x))=1$, 则
\bee
0=xy(f(x)+1)+f^2(x)-f(x)+f(f(x))-1.
\eee
于是对于$y$的系数, 在$x\ne0$时$f(x)=-1$, 这与上式矛盾.
\ea


\bq{2005年中国西部数学奥林匹克}{}
已知$\alpha^{2005}+\beta^{2005}$可以表示成以$\alpha+\beta$, $\alpha\beta$为变元的二元多项式, 求这个多项式的系数之和.
\eq
\ba
让$(\sigma_1, \sigma_2)=(\alpha+\beta, \alpha\beta)$, 用牛顿等幂公式, 并令$\sigma_1=\sigma_2=1$得结果为1.
\ea

\bq{2005年中国西部数学奥林匹克}{}
设正实数$a,b,c$满足$a+b+c=1$. 证明:
\bee
10(a^3+b^3+c^3)-9(a^5+b^5+c^5)\ge1.
\eee
\eq
\ba
令$S_k=a^k+b^k+c^k$($k\in\pNN$), $(\sigma_1,\sigma_2,\sigma_3)=(a+b+c,ab+bc+ca,abc)$. 用牛顿等幂公式,
$S_3=1-3\sigma_2+3\sigma_3$, $S_5=1-5\sigma_2+5\sigma_3+5\sigma_2^2-5\sigma_2\sigma_3$, 原式等价于
$(\sigma_2-\sigma_3)(1-3\sigma_2)\ge0$. 其中$1-3\sigma_2=\sigma_1^2-3\sigma_2\ge0$.
\ea

\bq{2006年全国高中数学联赛}{}
解方程组
\bee
\left\{
\begin{array}{l}
 x-y+z-w=2,\\
 x^2-y^2+z^2-w^2=6,\\
 x^3-y^3+z^3-w^3=20,\\
 x^4-y^4+z^4-w^4=66.
\end{array}
\right.
\eee
\eq
\ba
令$a_n=x^n+z^n$, $b_n=y^n+w^n$($n\in\pNN$). 由牛顿等幂公式, 
$a_3=\frac12a_1(3a_2-a_1^2)$, $a_4=\frac12(2a_1^2a_2-a_1^4+a_2^2)$,
$b_3=\frac12b_1(3b_2-b_1^2)$, $b_4=\frac12(2b_1^2b_2-b_1^4+b_2^2)$. 
解得$a_1=4$, $a_2=10$, $b_1=2$, $b_2=4$. 故原方程有4个解, $(x,y,z,w)=(1,2,3,0)$, $(1,0,3,2)$,
$(3,2,1,0)$, $(3,0,1,2)$.
\ea

\bq{25届IMO}{}
求一对正整数$a,b$满足:
\begin{enumerate}
 \item $7\nmid ab(a+b)$;
 \item $7^7\mid((a+b)^7-a^7-b^7)$.
\end{enumerate}
\eq
\ba
令$(x,y,z)=(a+b,-a,-b)$, $S_n=x^n+y^n+z^n$, $(\sigma_1,\sigma_2,\sigma_3)=(0,xy+yz+zx,xyz)=(0,-a^2-ab-b^2,ab(a+b))$.
由牛顿等幂公式, $S_7=7\sigma_2^2\sigma_3=7ab(a+b)(a^2+ab+b^2)^2$. 由(1), (2)知$7^3\mid a^2+ab+b^2$.
取$a^2+ab+b^2=7^3\ge3ab$, 则有$ab\le114$. 解出一个$(a,b)=(18,1), (1,18)$.
\ea

\bq{}{}
求$\cos^5\frac{\pi}{9}+\cos^5\frac{5\pi}{9}+\cos^5\frac{7\pi}{9}$的值.
\eq
\ba
令$(x_1, x_2, x_3)=(\cos\pi/9,\cos5\pi/9,\cos7\pi/9)$, 
则$(\sigma_1, \sigma_2, \sigma_3)=(\sum x_i,\sum x_ix_j, x_1x_2x_3)=(0,-3/4,1/8)$,
最后用牛顿等幂公式得$\frac{15}{32}$.
\ea

\bq{}{}
解方程组
\bee
\left\{
\begin{array}{l}
 x_1+x_2+\cdots+x_n=n,\\
 x_1^2+x_2^2+\cdots+x_n^2=n,\\
 \cdots,\\
 x_1^n+x_2^n+\cdots+x_n^n=n.
\end{array}
\right.
\eee
\eq
\ba
由牛顿等幂公式, $S_n=\sum x_i^n$, $S_1=S_2=\cdots=S_n=n$, 得
$\sigma_1-\sigma_2+\cdots+(-1)^{n-2}\sigma_{n-1}+(-1)^{n-1}\sigma_n=1$.
从而$x_1, \cdots, x_n$中必有一个是1. 不妨让$x_n=1$. 于是对剩下的$(n-1)$个方程重复以上过程得$x_{n-1}=1$,
如此下去知, $x_1=x_2=\cdots=x_n=1$.
\ea

\bq{}{}
已知实数$x,y,z,w$满足$x+y+z+w=x^7+y^7+z^7+w^7=0$. 求
\bee
f=w(w+x)(w+y)(w+z).
\eee
\eq
\ba
$S_n=w^n+x^n+y^n+z^n$, ($n\in\pNN$), 
由牛顿等幂公式有$S_7=7(\sigma_2^2-\sigma_4)\sigma_3$,
所以$\sigma_4=\sigma_2^2$或$\sigma_3=0$.

若$\sigma_4=\sigma_2^2$, 则$S_4=-2\sigma_2^2\le0$, 得$x=y=z=w=f=0$.

若$\sigma_3=0$, 则$x,y,z,w$是$t^4+\sigma_2t^2+\sigma_4=0$的四个实根,
$w$比为$x,y,z$之一的相反数, 故$f=0$.
\ea

\bq{}{}
求方程组
\bee
\left\{
\begin{array}{l}
 x+y+z=0,\\
 x^3+y^3+z^3=-18.
\end{array}
\right.
\eee
的整数解$(x,y,z)$.
\eq
\ba
由牛顿等幂公式得$xyz=-6$. 
故$(x,y,z)=(1,2,-3)$, $(2,1,-3)$,
$(1,-3,2)$, $(2,-3,1)$, $(-3,1,2)$,
$(-3,2,1)$.
\ea

\bq{第9届美国数学奥林匹克}{}
定义
\bee
f_n=x^n\sin nA+y^n\sin nB+z^n\sin nC\quad (n\in\pNN),
\eee
其中$\angle A, \angle B, \angle C\in \RR$, 且$\angle A+\angle B+\angle C=k\pi$,
($k\in\ZZ$), 如果$f_1=f_2=0$, 求证$f_n=0$对于一切$n\in\pNN$成立.
\eq
\ba
令$(\alpha, \beta, \gamma)=(x\ue^{\ui A}, y\ue^{\ui B}, z\ue^{\ui C})$,
$S_n=\alpha^n+\beta^n+\gamma^n$, ($n\in \NN$),
则$\Im(S_n)=f_n$. 补充定义$S_0=3$, 
由于$(\sigma_1, \sigma_2,\sigma_3)=(\alpha+\beta+\gamma,\alpha\beta+\beta\gamma+\gamma\alpha,\alpha\beta\gamma)\in\RR^3$,
所以$S_1,S_2\in\RR$. 由牛顿等幂公式得$S_n\in\RR$, 即得.
\ea