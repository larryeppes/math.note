\chapter{定义集}

\section{初等数论}
\bd{模$p$同余}{}
若两多项式$f(x)$与$g(x)$同次幂系数均关于模$p$同余, 则称$f(x)$和$g(x)$对模$p$同余或模$p$恒等.
\bee
f(x)\equiv g(x)\pmod{p}.
\eee
\ed

\bd{多项式模$p$的次数}{}
若$f(x)$的系数不全被$p$整除, 其中系数不被$p$整除的最高幂次称为$f(x)$模$p$的次数.
\ed

\bd{容度, 本原多项式}{}
设$f(x)=a_nx^n+\cdots+a_1x+a_0\in\Bbb Z[x]$, 且$f(x)\ne0$, 将$a_0,a_1,\cdots,a_n$的最大公约数$(a_0,a_1,\cdots,a_n)$, 称为$f(x)$的容度.

容度为1的多项式称为本原多项式.
\ed

\section{数学分析}


\section{微分方程}
\bd{标准形式下的边值问题}{}
二阶线性微分方程: $y''+P(x)y'+Q(x)y=\phi(x)$, $P(x), Q(x), \phi(x)\in C[a,b]$在满足边界条件:
\bee
\left\{
\begin{array}{ll}
 \a_1y(a)+\b_1y'(a)=\g_1, & \a_1,\a_2,\b_1,\b_2,\g_1,\g_2\in\Reals,\\
 \a_2y(b)+\b_2y'(b)=\g_2, & \a_1^2+\b_1^2\ne0, \a_2^2+\b_2^2\ne0.
\end{array}
\right.
\eee
的问题称为标准形式下的边值问题. 边值问题是{\color{red}{齐次的}}, 若$\phi(x)\equiv0$, $\g_1=\g_2=0$. 
否则称为非齐次的.
\ed

\bd{更一般的齐次边值问题}{}
更一般的齐次边值问题是有如下形式的问题
\bee
\left\{
\begin{array}{l}
 y''+P(x,\l)y'+Q(x,\l)y=0;\\
 \a_1y(a)+\b_1y'(a)=0;\\
 \a_2y(b)+\b_2y'(b)=0
\end{array}
\right.
\eee
\ed

\section{泛函分析}
\bd{紧算子}{}
设 $X$ 是 Banach 空间, 若线性算子 $T$ 把每一有界集映成列紧集, 则称线性算子 $T$ 为紧算子.
\ed

\bd{Banach空间中的凸集}{}
设 $X$ 是 Banach 空间, 集合 $K\subset X$称为是凸的, 若 $(1-t)K+tK\subset K$, ($0\le t\le 1$).
\ed

\bd{拟半范数, 半范数}{}
设$\mathbb{K}$是$\mathbb{R}$或$\mathbb{C}$, $\mathcal{X}$是域$\mathbb{K}$上的向量空间.
\begin{enumerate}[A.]
 \item 映射$q: \mathcal{X}\to\mathbb{R}$称为拟半范数, 如果
 \begin{enumerate}[(i)]
  \item $q(x+y)\le q(x)+q(y)$, 对于任意$x,y\in\mathcal{X}$.
  \item $q(tx)=tq(x)$, 对任意的$x\in\mathcal{X}$和$t\in\mathbb{R}$, $t\ge0$.
 \end{enumerate}
  \item 映射$q:\mathcal{X}\to\mathbb{R}$称为是半范数, 如果上面的两个条件中(ii)改为
  \begin{itemize}
   \item[(ii')] $q(\lambda x)=|\lambda|q(x)$, 对任意$x\in\mathcal{X}$和$\lambda\in\mathbb{K}$.
  \end{itemize}
\end{enumerate}
注: 若$q:\mathcal{X}\to\mathbb{R}$是半范数, 则对于任意的$x\in\mathcal{X}$, $q(x\ge0)$. 
(因$2q(x)=q(x)+q(-x)\ge q(0)=0$).
\ed