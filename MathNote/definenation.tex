\chapter{定义集}

\section{初等数论}
\bd{模$p$同余}{}
若两多项式$f(x)$与$g(x)$同次幂系数均关于模$p$同余, 则称$f(x)$和$g(x)$对模$p$同余或模$p$恒等.
\bee
f(x)\equiv g(x)\pmod{p}.
\eee
\ed

\bd{多项式模$p$的次数}{}
若$f(x)$的系数不全被$p$整除, 其中系数不被$p$整除的最高幂次称为$f(x)$模$p$的次数.
\ed

\bd{容度, 本原多项式}{}
设$f(x)=a_nx^n+\cdots+a_1x+a_0\in\Bbb Z[x]$, 且$f(x)\ne0$, 将$a_0,a_1,\cdots,a_n$的最大公约数$(a_0,a_1,\cdots,a_n)$, 称为$f(x)$的容度.

容度为1的多项式称为本原多项式.
\ed

\section{数学分析}
\bd{数学分析习题集}{}
\begin{itemize}
 \item 分析引论
 \begin{enumerate}[1.]
  \item 实数
  \item 数列理论
  \item 函数的概念
  \item 函数图像表示法
  \item 函数的极限
  \item 符号$\mathscr{O}$
  \item 函数连续性
  \item 反函数, 用参数形式表示的函数
  \item 函数的一致连续性
  \item 函数方程
 \end{enumerate}
\end{itemize}
\ed

\bd{分析引论}{}
\begin{itemize}
 \item 实数
 \begin{itemize}
  \item 数学归纳法
  \item 分割$\rightarrow$实数
  \item 绝对值(模)$\rightarrow$三角不等式, 开区间, 半开区间, 闭区间
  \item 上, 下确界的定义
  \item 绝对误差, 相对误差$\rightarrow$精确数字
 \end{itemize}
 \item 数列理论
 \begin{itemize}
  \item 数列极限的概念
  \item 收敛, 发散, 无穷小量, 无穷极限
  \item 极限存在的判别法
  \begin{itemize}
   \item 夹逼定理
   \item 单调有界
   \item Cauchy准则
  \end{itemize}
  \item 数列极限的基本定理
  \begin{itemize}
   \item 保序性
   \item 唯一性
   \item 保加, 减, 乘, 除(分母不为$0$)运算
   \item Stolz公式
  \end{itemize}
  \item 极限点, 上下极限, 运算和不等式
  \item 重要极限
  \begin{itemize}
   \item $\ue=\lim_{n\to\i}\paren{1+\frac1n}^n$
   \item $\g=\lim_{n\to\i}(H_n-\log n)$
  \end{itemize}
 \end{itemize}
 \item 函数的概念
 \begin{itemize}
  \item (单值)函数的定义, 定义域(存在域), 值域
  \item 反函数
  \item (严格)单调
  \item 复合函数
 \end{itemize}
 \item 函数的图像表示, 函数的零点
 \item 函数的极限
 \begin{itemize}
  \item 函数的有界性, 上确界, 下确界, 振幅
  \item 函数在某一点的极限, (与数列极限的关系)
  \item 重要极限
  \begin{itemize}
   \item $\lim_{x\to0}\frac{\sin x}{x}=1$
   \item $\lim_{x\to0}\paren{1+x}^{\frac1x}=\ue$
  \end{itemize}
  \item Cauchy准则(函数极限存在的充要条件)
  \item 单侧极限, 左右极限
  \item 无穷极限, $\lim_{x\to a}f(x)=\i\Longleftrightarrow\forall E>0$, $\exists \delta=\delta(E)>0$, $\ni\forall0<\abs{x-a}<\delta(E)$, 均有$\abs{f(x)}>E$.
  \item 子列极限, 下极限, 上极限
 \end{itemize}
 \item 函数的连续性
 \begin{itemize}
  \item (点)连续
  \item 间断点
  \begin{itemize}
   \item 第一类间断点
   \begin{itemize}
    \item 可去间断点
    \item 跳跃间断点
   \end{itemize}
   \item 第二类间断点(无穷型间断点)
  \end{itemize}
  \item 左右连续
  \item (点)连续, 保加, 减, 乘, 除(分母不为$0$)
  \item 复合函数的连续
  \item 初等函数的连续
  \item 基本定理
  \begin{itemize}
   \item 闭区间上连续函数有界
   \item 闭区间上连续函数达到上下确界(Weierstrass定理)
   \item 闭区间上连续函数定义在$(\a,\b)\subset[a,b]$, $f$取到$f(\a)$, $f(\b)$之间的所有值(Cauchy定理)
   \item 闭区间上连续函数的零点定理
  \end{itemize}
 \end{itemize}
 \item 反函数
 \begin{itemize}
  \item 反函数的存在性和连续性
  \item 单值连续分支
  \item 参数形式表示的函数的连续性
 \end{itemize}
 \item 函数的一致连续性
 \begin{itemize}
  \item 一致连续性的定义
  \item Cantor定理
 \end{itemize}
\end{itemize}
\ed

\section{微分方程}
\bd{标准形式下的边值问题}{}
二阶线性微分方程: $y''+P(x)y'+Q(x)y=\phi(x)$, $P(x), Q(x), \phi(x)\in C[a,b]$在满足边界条件:
\bee
\left\{
\begin{array}{ll}
 \a_1y(a)+\b_1y'(a)=\g_1, & \a_1,\a_2,\b_1,\b_2,\g_1,\g_2\in\Reals,\\
 \a_2y(b)+\b_2y'(b)=\g_2, & \a_1^2+\b_1^2\ne0, \a_2^2+\b_2^2\ne0.
\end{array}
\right.
\eee
的问题称为标准形式下的边值问题. 边值问题是{\color{red}{齐次的}}, 若$\phi(x)\equiv0$, $\g_1=\g_2=0$. 
否则称为非齐次的.
\ed

\bd{更一般的齐次边值问题}{}
更一般的齐次边值问题是有如下形式的问题
\bee
\left\{
\begin{array}{l}
 y''+P(x,\l)y'+Q(x,\l)y=0;\\
 \a_1y(a)+\b_1y'(a)=0;\\
 \a_2y(b)+\b_2y'(b)=0
\end{array}
\right.
\eee
\ed

\section{泛函分析}
\bd{紧算子}{}
设 $X$ 是 Banach 空间, 若线性算子 $T$ 把每一有界集映成列紧集, 则称线性算子 $T$ 为紧算子.
\ed

\bd{Banach空间中的凸集}{}
设 $X$ 是 Banach 空间, 集合 $K\subset X$称为是凸的, 若 $(1-t)K+tK\subset K$, ($0\le t\le 1$).
\ed

\bd{拟半范数, 半范数}{}
设$\mathbb{K}$是$\mathbb{R}$或$\mathbb{C}$, $\mathcal{X}$是域$\mathbb{K}$上的向量空间.
\begin{enumerate}[A.]
 \item 映射$q: \mathcal{X}\to\mathbb{R}$称为拟半范数, 如果
 \begin{enumerate}[(i)]
  \item $q(x+y)\le q(x)+q(y)$, 对于任意$x,y\in\mathcal{X}$.
  \item $q(tx)=tq(x)$, 对任意的$x\in\mathcal{X}$和$t\in\mathbb{R}$, $t\ge0$.
 \end{enumerate}
  \item 映射$q:\mathcal{X}\to\mathbb{R}$称为是半范数, 如果上面的两个条件中(ii)改为
  \begin{itemize}
   \item[(ii')] $q(\lambda x)=|\lambda|q(x)$, 对任意$x\in\mathcal{X}$和$\lambda\in\mathbb{K}$.
  \end{itemize}
\end{enumerate}
注: 若$q:\mathcal{X}\to\mathbb{R}$是半范数, 则对于任意的$x\in\mathcal{X}$, $q(x\ge0)$. 
(因$2q(x)=q(x)+q(-x)\ge q(0)=0$).
\ed