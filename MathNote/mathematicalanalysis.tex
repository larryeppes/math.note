\chapter{数学分析}
\section{极限\index{极限}}
\bq{}{}
设函数$\varphi(x)$可导, 且满足$\varphi(0)=0$, 又设$\varphi'(x)$单调减少. 
\begin{enumerate}
\item 证明: 对$x\in(0,1)$, 有$\varphi(1)x<\varphi(x)<\varphi'(0)x$.
\item 若$\varphi(1)\ge0$, $\varphi'(0)\le1$, 任取$x_{0}\in(0,1)$, 令$x_{n}=\varphi(x_{n-1})$,
($n=1,2,\cdots$). 证明: $\lim_{n\to\infty}x_{n}$ 存在, 并求该极限值.
\end{enumerate}
\eq
\ba
对于任意的$x\in[0,1],$在$[0,x]$上用拉格朗日定理,
\bee
\varphi(x)-\varphi(0)=\varphi'(\xi_{1})x<\varphi'(0)x
\eee
在$[x,1]$上用拉格朗日定理
\bee
\varphi(1)-\varphi(x)=\varphi'(\xi_{2})(1-x)<\varphi'(\xi_{1})(1-x)=\varphi'(\xi_{1})-\varphi(x)
\eee
所以$\varphi(1)x<\varphi'(\xi_{1})x=\varphi(x)$.
\ea

\bq{https://www.zhihu.com/question/636352059}{}
已知实数列$\left\{ x_{n}\right\} $使得$3x_{n}-x_{n-1}$收敛, 证明$x_{n}$收敛.
\eq
\ba
由问题\ref{q:2018053110}, 设
\[
\limsup x_{n}=\overline{X},\quad\liminf x_{n}=\underline{X}.
\]
则$+\infty\ge\overline{X}\ge\underline{X}\ge-\infty$. 如果设$\lim(3x_{n}-x_{n-1})=L$.
取上面的数列$\left(x_{n},y_{n}\right)$对为原问题中的$\left(3x_{n},-x_{n-1}\right)$代入上面的不等式,
得到
\[
3\underline{X}-\overline{X}\le L\le3\underline{X}-\underline{X}\le L\le3\overline{X}-\underline{X},
\]
这说明$\underline{X}=\frac{L}{2}$. 当取上面的数列$\left(x_{n},y_{n}\right)$对为原问题中的$\left(-x_{n-1},3x_{n}\right)$时,
则得到
\[
-\overline{X}+3\underline{X}\le L\le-\overline{X}+3\overline{X}\le L\le-\underline{X}+3\overline{X},
\]
这说明$\overline{X}=\frac{L}{2}$. 所以问题中的数列$x_{n}$的上下极限相等且有限. 
\ea

\section{导数}

\bq{}{}
函数$f(x)\in C[0,1]$, 在$(0,1)$上可微, 对于任意的$x\in(0,1)$, $\abs{xf'(x)-f(x)+f(x)}<Mx^2$, 问$f'(0)$的存在性.
\eq
\ba
不妨设$f(0)=0$, 定义$h(x)=\frac{f(x)}{x}$($0<x<1$), 即证$\lim_{x\to0}h(x)$存在, 则$\abs{x^2h'(x)}<Mx^2$,
所以$\abs{h'(x)}<M$, 所以若$\{x_n\}\to0$, 则有
\bee
\abs{h(x_m)-h(x_n)}=\abs{h'{\xi}(x_m-x_n)}<M\abs{x_m-x_n}<\varepsilon.
\eee
所以$\{h(x_n)\}$是Cauchy列, 从而$\lim_{x\to0}h(x)$存在.
\ea

\bq{}{}
构造有界单调函数$f(x)$使得对于任意的$x\in\RR$, $f'(x)$存在, 且$\lim_{x\to\pm\i}f'(x)\ne0$.
\eq
\ba
取$a_n=1-2^{-n}$, ($n\in\NN$), $f(n)=a_n$, $f\paren{n+\frac12}=\frac12\paren{a_n+a_{n+1}}$, 
且$f'(n)=0$, $f'\paren{n+\frac12}=1$, 将其它点处可微连接$f(n)$这些离散点, 知$\lim_{x\to\pm\i}f'(x)$不存在, 从而不为0.
\ea

\section{积分}

\bq{}{}
\bee
\int_{-1}^1 \frac{\sqrt{\frac{x+1}{1-x}} \log \left(\frac{2 x^2+2 x+1}{2 x^2-2
		x+1}\right)}{x} \, dx
\eee
\eq
\ba
$4\pi \mathrm{arccot}(\sqrt{\phi}).$
\ea

\bq{}{}
设函数$f(x)$在$[a,b]$上有连续的导数, 且$f(a)=0$, 证明
\bee
\int_{a}^{b}f^{2}(x)\mathrm{d}x\le\frac{(b-a)^{2}}{2}\int_{a}^{b}[f'(x)]^{2}\mathrm{d}x.
\eee
\eq
\ba
令$F(b)=RHS-LHS,$证$F'(b)\ge0$即可.
\ea

\bq{}{}
设函数$f(x)$在$[0,1]$上有二阶连续的导数, 证明:
\begin{enumerate}
\item 对任意$\xi\in\paren{0,\frac{1}{4}}$和$\eta\in\paren{\frac{3}{4},1}$有
$$
\abs{f'(x)}<2\abs{f(\xi)-f(\eta)}+\int_{0}^{1}\abs{f''(x)}\mathrm{d}x\quad x\in[0,1]
$$
\item 当$f(0)=f(1)=0$及$f(x)\ne0$, ($x\in(0,1)$)时有
$$
\int_{0}^{1}\abs{\frac{f''(x)}{f(x)}}\mathrm{d}x\ge4.
$$
\end{enumerate}
\eq
\ba
用中值定理, 
\begin{align*}
\abs{f'(x)}-2\abs{f(\xi)-f(\eta)} & =\abs{f'(x)}-2\abs{f'(\theta)}(-\xi+\eta)\\
 & \le\abs{f'(x)}-\abs{f'(\theta)}\\
 & \le\abs{f'(x)-f'(\theta)}=\abs{\int_{\theta}^{x}f''(t)\mathrm{d}t}\\
 & \le\int_{0}^{1}\abs{f''(t)}\mathrm{d}t
\end{align*}
最后取$f(x_{0})=\max_{x\in[0,1]}f(x)$, 则
\bee
f(x_0)=f'(\xi_1)x_0=f'(\xi_2)(x_0-1),
\eee
所以
\begin{align*}
\int_0^1\abs{\frac{f''}{f}}\ud x & \ge\frac1{\abs{f(x_0)}}\ud x\\
 & \ge \frac1{\abs{f(x_0)}}\abs{\int_{\xi_1}^{\xi_2}f''\ud x}\\
 & = \frac{1}{\abs{f(x_0)}}\abs{f'(\xi_2)-f'(\xi_1)}\\
 & = \frac1{x_0}+\frac1{1-x_0}\ge4.
\end{align*}
\ea

\bq{}{}
设函数$f(x)$在$\left[-\frac1a,a\right]$上连续(其中$a>0$), 且$f(x)\ge0$, $\int_{-\frac1a}^{a}xf(x)\ud x=0$, 求证: $\int_{-\frac1a}^ax^2f(x)\ud x\le\int_{-\frac1a}^af(x)\ud x$.
\eq
\ba
因$(a-x)\paren{x+\frac1a}\ge0$, 对$(a-x)\paren{x+\frac1a}f(x)\ge0$两边同时积分.
\ea

\bq{}{}
设函数$f(x)$在$[0,1]$上连续, $\int_0^1f(x)\ud x=0$, $\int_0^1xf(x)\ud x=1$. 求证: 
\begin{enumerate}[1.]
 \item 存在$\xi\in[0,1]$, 使得$|f(\xi)|\ge4$;
 \item 存在$\eta\in[0,1]$, 使得$\abs{f(\eta)}=4$.
\end{enumerate}
\eq
\ba
用反证法, 
\bee
1=\abs{\int_0^1\paren{x-\frac12}f(x)\ud x}
  \le \int_0^1\abs{x-\frac12}\cdot\abs{f}\ud x
  \le 1.
\eee
等号取不到, 否则, 
\bee
\int_0^1\paren{4-\abs{f}}\abs{x-\frac12}\ud x=0.
\eee
\ea

\bq{}{}
求$\int_0^{2\pi}\frac{\ud \t}{3-\sin 2\t}$.
\eq
\ba
$\int_0^{2\pi}\frac{\ud \t}{3-\sin 2\t}
=\int_0^{2\pi}\frac{\ud \t}{2+2\sin^2\left(\t-\frac{\pi}{4}\right)}
=\frac12\int_0^{2\pi}\frac{\ud \t}{1+\sin^2\t}
=2\int_{0}^{2\pi}\frac{\ud \t}{1+\sin^2\t}
=2\int_0^{+\i}\frac{\ud t}{1+2t^2}=\frac{\sqrt{2}}{2}\pi$.
\ea
\ba
\begin{align*}
 \int_0^{2\pi}\frac{\ud \t}{3-\sin 2\t}
 &= \frac12\int_0^{4\pi}\frac{\ud t}{3-\sin t}\\
 &= \int_0^{2\pi}\frac{\ud t}{3-\sin t}\\
 &= \int_0^{\pi}\frac{\ud t}{3-\sin t}+\int_{0}^{\pi}\frac{\ud t}{3+\sin t}\\
 &= 2\int_{0}^{\pi/2}\frac{\ud x}{3-\sin x}+2\int_{0}^{\pi/2}\frac{\ud x}{3+\sin x}\\
 &= 12\int_{0}^{\pi/2}\frac{\ud x}{9-\sin^2x}\\
 &= 12\int_0^{\pi/2}\frac{\ud\tan x}{8\tan^2x+9}\\
 &= \frac{12}{6\sqrt{2}}\arctan\left(\frac{2\sqrt{2}}{3}\tan x\right)\big\vert_0^{\pi/2}
 =\frac{\sqrt{2}}{2}\pi.
\end{align*}
\ea
\ba
\begin{align*}
 \int_0^{2\pi}\frac{\ud \t}{3-\sin2\t}
 &=\frac12\int_0^{4\pi}\frac{\ud \t}{3\sin\t}\\
 &=\int_0^{2\pi}\frac{\ud \t}{3-\sin\t}\\
 &=\oint_{\abs{z}=1}\frac{\ud z}{\ui z\left(3-\frac{z-z^{-1}}{2i}\right)}\\
 &=2\oint_{\abs{z}=1}\frac{\ud z}{z^2+6\ui z-1}\\
 &=2\oint_{\abs{z}=1}\frac{\ud z}{(z-(-3+2\sqrt{2})\ui)(z-(-3-2\sqrt{2})\ui)}\\
 &=2\cdot2\pi\ui\Res(f(z))\big\vert_{z=(-3+2\sqrt{2})\ui}
 =\frac{\sqrt{2}}{2}\pi.
\end{align*}
\ea
\ba
\begin{align*}
 \int_0^{2\pi}\frac{\ud \t}{3-\sin 2\t}
 &=\int_0^{2\pi}\frac{\ud \tan\t}{3\tan^2\t-2\tan\t+3}\\
 &=\int_0^{\pi/2}+\int_{\pi/2}^{\pi}+\int_{\pi}^{3\pi/2}+\int_{3\pi/2}^{2\pi}=\cdots
\end{align*}
于是
\bee
\int_0^{\pi/2}
=\int_{0}^{\i}\frac{\ud\t}{3t^2-2t+3}
=\frac13\int_0^{+\i}\frac{\ud\left(t-\frac13\right)}{\left(t-\frac13\right)^2+\frac89}
=\frac13\cdot\frac3{\sqrt{8}}\arctan\left(\frac{3\left(t-\frac13\right)}{\sqrt{8}}\right)\big\vert_{0}^{+\i}
=\frac{\sqrt{2}}{4}\left(\frac{\pi}{2}+\arctan\frac{\sqrt{2}}{4}\right).
\eee
其它三个同理. 所以$\sum=\frac{\sqrt{2}}{4}\left(\frac{\pi}{2}\cdot4\right)=\frac{\sqrt{2}}{2}\pi$.
\ea

\bq{}{}
求
\bee
\int_0^{\pi/4}\frac{(\cot x-1)^{p-1}}{\sin^2x}\ln\tan x\ud x=-\frac{\pi}{p}\csc p\pi, \ (-1<p<0).
\eee
\eq
\ba
令$u=\cot x-1$, 原式等价于
\bee
\int_{0}^{+\i}u^{p-1}\ln(u+1)\ud u=\frac{\pi}{p}\csc p\pi,
\eee
而
\bee
\int_{0}^{\i}u^{p-1}\ln(u+1)\ud u
=\int_{0}^{\i}u^{p-1}\int_{1}^{1+u}\frac1y\ud y\ud u
\eee
交换积分次序, 用Beta函数
\ea

\section{级数}



\bq{}{}
证明
\bee
1+x+\frac{x^2}{2!}+\cdots+\frac{x^{2n}}{(2n)!}=0
\eee
无实根.
\eq
\ba
设$-y<0$, 则$y>0$, 所以$1-y+\frac{y^2}{2!}-\frac{y^3}{3!}+\cdots+\frac{y^{2n}}{(2n)!}>\ue^{-y}>0$.
\ea

\bq{F.F.Abi=Khuzam and A.B.Boghossian, Some recent geometric inequalities, AMM Vol 96(1989), No. 7:576-589}{}
函数$f(x)=\cot x-\frac1x$, 则$f^{(k)}(x)<0$, $0<x<\pi$, $k\in\NN$.
\eq
\ba
\bee
\cot x=\frac1x+\sum_{k=1}^{\i}\paren{\frac1{k\pi+x}-\frac1{k\pi-x}}, \quad x\in(0,\pi).
\eee
将真分式展开有
\bee
f(x)=-2x\sum_{k=0}^{\i}c_kx^{2k}, \  x\in(0,\pi), \quad c_k=\sum_{n=1}^{\i}\frac1{(n\pi)^{2k+2}},\ (k\in\NN).
\eee
\ea

\bq{}{}
对$n\in\pNN$, 确定$(0,1)$的子集, 使在此子集上$\paren{\frac{\ud}{\ud x}}^n\paren{\ln x\ln(1-x)}<0$.
\eq
\ba
\bee
f'(x)=(\ln x\ln(1-x))'=\sum_{m=1}^{\i}\frac{(1-x)^{m-1}-x^{m-1}}{m},
\eee
当$n$是偶数时, 所有项都是负的, 当$n$是奇数时, 仅当$1-x<x$即$x>\frac12$时, $f^{(n)}(x)$时负的.
\ea

\bq{}{}
已知$S_n=\frac{n+1}{2^{n+1}}\sum_{i=1}^{n}\frac{2^i}{i}$, 证明$\lim_{n\to\i}S_n$存在并求其值.
\eq
\ba
因$S_{n+1}=\frac{n+2}{2(n+1)}(S_n+1)$, 所以
\bee
S_{n+2}-S_{n+1}=\frac{(n+2)^2(S_{n+1}-S_{n})-S_{n+1}-1}{2(n+1)(n+2)},
\eee
$S_4-S_3=0$, 当$n\ge3$时, $S_n$不增, 所以$S=\lim_{n\to\i}S_{n}$存在, 所以$S=\lim_{n\to\i}\frac{n+2}{2(n+1)}(S+1)$, 
得$S=1$.
\ea

\section{其他}

\bq{}{}
证明:
\bee
\lim_{n\to\infty}\left(\frac12\cdot\frac34\cdots\frac{2n-1}{2n}\right)=0.
\eee
\eq
\ba
用$\frac12\cdot\frac34\cdots\frac{2n-1}{2n}<\frac1{\sqrt{2n+1}}$.
\ea

\bq{}{}
证明:
\bee
0<\ue-\left(1+\frac1n\right)^n<\frac{3}{n}.
\eee
\eq
\ba
先证: $x_n=\left(1+\frac1n\right)^n$单调上升且有界. ($1\cdot x_n\le\left(\frac{1+n(1+1/n)}{n+1}\right)^{n+1}=x_{n+1}$), 则
\begin{align*}
 x_n & = \sum_{k=0}^{n}\binom{n}{k}\frac1{n^k}=1+1+\binom{n}{2}\frac1{n^2}+\cdots\\
 &= 2+\frac1{2!}\left(1-\frac1n\right)+\frac1{3!}\left(1-\frac1n\right)\left(1-\frac2n\right)+\cdots+\frac1{n!}\left(1-\frac1n\right)\cdots\left(1-\frac{n-1}{n}\right)\\
 &< 2+\frac1{2\cdot 1}+\frac1{3\cdot 2}+\cdots+\frac1{n(n-1)}=3.
\end{align*}
再证$y_n=\left(1+\frac1n\right)^{n+1}$单调下降且有界. (用$(1+x)^n>1+nx$, $x>-1$证单调性).

由$\ue=\lim_{n\to\infty}x_n=\lim_{n\to\infty}y_n$, $x_n<\ue<y_n$, (这可以证得{\color{red}{$\frac1{n+1}<\ln\left(1+\frac1n\right)<\frac1n$}}).

故$\ue-x_n<y_n-x_n=\frac{x_n}{n}<\frac3n$.
\ea

\bq{}{}
证明不等式
\bee
\left(\frac{n}{\ue}\right)^n<n!<\ue\left(\frac{n}{2}\right)^n.
\eee
\eq
\ba
用$2<\left(1+\frac1n\right)^n<\ue$及归纳法.
\ea

\bq{}{}
设$a^{[n]}=a(a-h)\cdots[a-(n-1)h]$及$a^{[0]}=1$, 证明:
\bee
(a+b)^{[n]}=\sum_{m=0}^{n}\binom{n}{m}a^{[n-m]}b^{[m]}.
\eee
并由此推出牛顿二项式公式.
\eq

\bq{}{}
证明不等式
\bee
n!<\left(\frac{n+1}{2}\right)^n\quad (n>1).
\eee
\eq
\ba
用均值不等式.
\ea
\ba
用伯努利不等式证$\left(\frac{n+2}{n+1}\right)^{n+1}=\left(1+\frac1{n+1}\right)^{n+1}>2$, ($n\in\pNN$).
\ea

\bq{}{}
设$p_n$($n\in\pNN$)为趋于正无穷的任意数列, 而$q_n$($n\in\pNN$)为趋于负无穷的任意数列($p_n, q_n\not\in[-1,0]$), 求证:
\bee
\lim_{n\to\infty}\left(1+\frac1{p_n}\right)^{p_n}=\lim_{n\to\infty}\left(1+\frac1{q_n}\right)^{q_{n}}=\ue.
\eee
\eq
\ba
注意$[x]\le x<[x]+1$.
\ea

\bq{}{201805317}
已知$\lim_{n\to\infty}\left(1+\frac1n\right)^n=\ue$, 求证:
\bee
\lim_{n\to\infty}\left(1+1+\frac{1}{2!}+\frac{1}{3!}+\cdots+\frac{1}{n!}\right)=\ue.
\eee
并推出
\bee
\ue=1+1+\frac{1}{2!}+\frac{1}{3!}+\cdots+\frac{1}{n!}+\frac{\t_n}{n!n},
\eee
其中$0<\t_n<1$.
\eq

\bq{}{}
证明: $\ue$是无理数.
\eq
\ba
用反证法及\ref{q:201805317}有, 对于任意的$n$, $n!n\cdot\ue$不是整数.
\ea

\bq{}{}
证明不等式:
\begin{enumerate}[(a)]
 \item $\frac{1}{n+1}<\ln\left(1+\frac1n\right)<\frac1n$, ($n\in\pNN$);
 \item $1+\a<\ue^{\a}$, ($\a\ne0, \a\in\RR$).
\end{enumerate}
\eq
\ba
\begin{enumerate}[(a)]
 \item 原式等价于$\left(1+\frac1n\right)^n<\ue<\left(1+\frac1n\right)^{n+1}$;
 \item $\a>-1$时, 用伯努利不等式, $\ue^{\a}>\left(1+\frac1n\right)^{\a n}>1+\a$.
\end{enumerate}
\ea

\bq{}{2018053110}
证明: (在以下各极限均存在的情况下)
\begin{enumerate}[(a)]
 \item $\liminf_{n\to\infty}x_n+\liminf_{n\to\infty}y_n\le\liminf_{n\to\infty}(x_n+y_n)\le\liminf_{n\to\infty}x_n+\limsup_{n\to\infty}y_n$;
 \item $\liminf_{n\to\infty}x_n+\limsup_{n\to\infty}y_n\le\limsup_{n\to\infty}(x_n+y_n)\le\limsup_{n\to\infty}x_n+\limsup_{n\to\infty}y_n$.
\end{enumerate}
\eq
\ba
用$\liminf_{n\to\infty}x_n=-\limsup_{n\to\infty}(-x_n)$.
\ea

\bq{}{}
证明: 若$\lim_{n\to\infty}x_n$存在, 则对于任何数列$y_n$($n\in\pNN$), $\limsup_{n\to\infty}y_n$有限且有:
\begin{enumerate}[(a)]
 \item $\limsup_{n\to\infty}(x_n+y_n)=\lim_{n\to\infty}x_n+\limsup_{n\to\infty}y_n$;
 \item $\limsup_{n\to\infty}x_ny_n=\lim_{n\to\infty}x_n\cdot\limsup_{n\to\infty}y_n$, ($x_n\ge0$).
\end{enumerate}
\eq
\ba
用\ref{q:2018053110}.
\ea

\bq{}{}
证明: 若对于某数列$x_n$($n\in\pNN$), 无论数列$y_n$($n\in\pNN$)如何选取, 以下两个等式中都至少有一个成立:
\begin{enumerate}[(a)]
 \item $\limsup_{n\to\infty}(x_n+y_n)=\limsup_{n\to\infty}x_n+\limsup_{n\to\infty}y_n$.
 \item $\limsup_{n\to\infty}(x_ny_n)=\limsup_{n\to\infty}x_n\cdot\limsup_{n\to\infty}y_n$, ($x_n\ge0$).
\end{enumerate}
则数列$x_n$收敛或发散于正无穷.
\eq

\bq{}{}
证明: 若$x_n>0$($n\in\pNN$)及
\bee
\limsup_{n\to\infty}x_n\cdot\limsup_{n\to\infty}\frac1{x_n}=1.
\eee
则数列$x_n$是收敛的.
\eq

\bq{}{}
证明: 若数列$x_n$($n\in\pNN$)有界, 且
\bee
\lim_{n\to\infty}(x_{n+1}-x_n)=0.
\eee
则此数列的子列极限充满于下极限和上极限
\bee
l=\liminf_{n\to\infty}x_n\  \textrm{和}\ L=\limsup_{n\to\infty}x_n
\eee
之间.
\eq

\bq{}{2018053117}
证明: 若$x_n>0$($n\in\pNN$)且$\lim_{n\to\infty}\frac{x_{n+1}}{x_n}$存在, 则
\bee
\lim_{n\to\infty}\sqrt[n]{x_n}=\lim_{n\to\infty}\frac{x_{n+1}}{x_n}.
\eee
\eq
\ba
用结论: 若$\{x_n\}\to x$, $x_n>0$, 则$\lim_{n\to\infty}\sqrt[n]{x_1x_2\cdots x_n}=\lim_{n\to\infty}x_n=x$.
\ea

\bq{}{}
证明: $\lim_{n\to\infty}\frac{n}{\sqrt[n]{n!}}=\ue$.
\eq
\ba
用\ref{q:2018053117}.
\ea

\bq{数$a$和$b$的算术几何平均值}{}
证明: 由下列各式
\bee
x_1=a,\quad y_1=b,\quad x_{n+1}=\sqrt{x_ny_n},\quad y_{n+1}=\frac{x_n+y_n}{2},
\eee
确定的数列$x_n$和$y_n$($n\in\pNN$)有共同的极限.
\bee
\m(a,b)=\lim_{n\to\infty}x_n=\lim_{n\to\infty}y_n.
\eee
\eq
\ba
用幂平均不等式, $\sqrt{x_{n+1}+y_{n+1}}=\frac{\sqrt{x_n}+\sqrt{y_{n}}}{\sqrt{2}}\le\sqrt{x_n+y_n}$.
即$\{y_n\}$单调有界, 从而有极限, 从而$x_n=2y_{n+1}-y_n$有相同的极限.
\ea

\bq{}{}
设
\bee
f\left(x+\frac1x\right)=x^2+\frac1{x^2}\quad(\abs{x}\ge2),
\eee
求$f(x)$.
\eq
\ba
$x^2-2$, ($\abs{x}\ge\frac52$).
\ea

\bq{}{}
证明: 若
\begin{enumerate}[(1)]
 \item 函数$f(x)$定义于区域$x>a$;
 \item $f(x)$在每一个有限区间$a<x<b$内是有界的;
 \item 对于某一个整数$n$, 存在有限的或无穷的极限
 \bee
 \lim_{x\to+\infty}\frac{f(x+1)-f(x)}{x^n}=l,
 \eee
\end{enumerate}
则
\bee
\lim_{x\to+\infty}\frac{f(x)}{x^{n+1}}=\frac{l}{n+1}.
\eee
\eq
{\color{red}{\bf{能否用Cauchy定理\ref{th:cauchy-theorem}证明它.}}}

\bq{}{}
利用定理
\bt{}{}
设
\bee
\lim_{x\to 0}\frac{\phi(x)}{\psi(x)}=1,
\eee
其中$\psi(x)>0$, 再设当$n\to\infty$时$\a_{mn}\rightrightarrows0$($m=1,2,\cdots,n$), 
换言之, 对于任意$\varepsilon>0$, 存在正整数$N(\varepsilon)$, 当$m=1,2,\cdots,n$且$n>N(\varepsilon)$时, 
$0<\abs{\a_{mn}}<\varepsilon$. 证明:
\bee
\lim_{n\to\infty}[\phi(\a_{1n})+\phi(\a_{2n})+\cdots+\phi(\a_{mn})]
=\lim_{n\to\infty}[\psi(\a_{1n})+\psi(\a_{2n})+\cdots+\psi(\a_{mn})],
\eee
此处同时还要假设上式右端的极限存在.
\et
求:
\begin{enumerate}[(1)]
 \item $\lim_{n\to\infty}\sum_{k=1}^{n}\left(\sqrt[n]{1+\frac{k}{n^2}}-1\right)$;
 \item $\lim_{n\to\infty}\sum_{k=1}^{n}\left(\sin\frac{ka}{n^2}\right)$;
 \item $\lim_{n\to\infty}\sum_{k=1}^{n}\left(a^{\frac{k}{n^2}}-1\right)$, ($a>0$);
 \item $\lim_{n\to\infty}\prod_{k=1}^{n}\left(1+\frac{k}{n^2}\right)$;
 \item $\lim_{n\to\infty}\prod_{k=1}^{n}\cos\frac{ka}{n\sqrt{n}}$.
\end{enumerate}
\eq

\bq{}{}
设函数$f(x)$在区间$(x_0,+\infty)$上连续并有界. 证明: 对于任何数$T$, 可求得数列$x_n\to+\i$, 
使
\bee
\lim_{n\to\infty}[f(x_n+T)-f(x_n)]=0.
\eee
\eq

\bq{}{}
证明: 在有限区间$(a,b)$上有定义且连续的函数$f(x)$, 可用连续的方法延拓到闭区间$[a,b]$上, 
其充分必要条件是函数$f(x)$在区间$(a,b)$上一致连续.
\eq

\bq{}{}
$x_n$满足$x_n^n+x_n-1=0$, $0<x_n<1$, 求$\lim_{n\to\infty}x_n$.
\eq
\ba
$y=x^n+x-1$则有$y'>0$, $y\vert_{x=0}=-1<0$, $y\vert_{x=1}=1>0$.

$x_n$是$x^n+x-1$的唯一零点. 由于
\bee
x_{n}^{n+1}+x_n-1=(x_n-1)(1-x_n)<0
\eee
及$y$的单调性, 知$x_{n+1}$在$x_n$与$1$之间, 故$\{x_n\}$单调有界.
反证$\{x_n\}$的极限$A=1$, 否则$0\le A<1$矛盾.
\ea
\ba
$y^x+y-1=0$是隐函数, 确定$y=f(x)$, $x_n=f(n)$, 求导
\bee
y'=-y^x\cdot\frac{\ln y}{\left(1+\frac{x}{y}\cdot y^x\right)}
\eee
当$x\ge1$时, $y'>0$, $y$单调增加, 以下同上.
\ea

\bq{}{}
若级数$\sum_{n=1}^{\infty}a_n^2$, $\sum_{n=1}^{\infty}b_n^2$都收敛, 则以下不成立的是?

A. $\sum_{n=1}^{\infty}(a_n+b_n)^2$收敛;\qquad\qquad B.$\sum_{n=1}^{\infty}\frac{|a_n|}{n}$收敛;\\
C. $\sum_{n=1}^{\infty}a_nb_n$收敛;\qquad\qquad\qquad D. $\sum_{n=1}^{\infty}a_nb_n$发散.
\eq

\bq{}{}
设$f(x,y)$与$\varphi(x,y)$均为可微函数, 且$\varphi'_y(x,y)\ne0$. 已知$(x_0,y_0)$是$f(x,y)$在约束条件$\varphi(x,y)=0$下的一个极值点,
下列选项正确的是(D)

(A) 若$f'_x(x_0,y_0)=0$, 则$f'_y(x_0,y_0)=0$;\quad (B) 若$f'_{x}(x_0,y_0)=0$, 则$f'_{y}(x_0,y_0)\ne0$;\\
(C) 若$f'_{x}(x_0,y_0)\ne0$, 则$f'_{y}(x_0,y_0)=0$;\quad (D) 若$f'_{x}(x_0,y_0)\ne0$, 则$f'_{y}(x_0,y_0)\ne0$.
\eq

\bq{}{}
\bee
\frac{\int_0^1\frac{1}{\sqrt{1-t^4}}\ud t}{\int_0^1\frac{1}{\sqrt{1+t^4}}\ud t}=\sqrt{2}
\eee
\eq
\ba
用Beta函数.
\ea

\bq{陕西省第七次大学生高等数学竞赛复赛}{}
计算$I=\int_{\pi/8}^{3\pi/8}\frac{\sin^2x}{x(\pi-2x)}\ud x$.
\eq

\bq{陕西省第七次大学生高等数学竞赛复赛}{}
设$\varphi(x)$在$(-\infty, 0]$可导, 且函数
\bee
f(x)=\begin{cases}
      \int_{x}^{0}\frac{\varphi(t)}{t}\ud t, & x<0,\\
      \lim_{n\to\infty}\sqrt[n]{(2x)^n+x^{2n}}, & x\ge0.
     \end{cases}
\eee
在点$x=0$可导, 求$\varphi(0)$, $\varphi'(0^-)$, 并讨论$f'(x)$的存在性.
\eq

\bq{陕西省第七次大学生高等数学竞赛复赛}{}
已知函数$f(x)$与$g(x)$满足$f'(x)=g(x)$, $g'(x)=2\ue^x-f(x)$, 且$f(0)=0$, 求
\bee
\int_0^{\pi}\left(\frac{g(x)}{1+x}-\frac{f(x)}{(1+x)^2}\right)\ud x.
\eee
\eq

\bq{陕西省第七次大学生高等数学竞赛复赛}{}
设$y_1$和$y_2$是方程$y''+p(x)y'+2\ue^xy=0$的两个线性无关解, 而且$y_2=(y_1)^2$. 若有$p(0)>0$,
求$p(x)$及此方程的通解.
\eq

\bq{陕西省第七次大学生高等数学竞赛复赛}{}
设$f(x)$在$\left[-\frac1a,a\right]$($a>0$)上非负可积, 且$\int_{-1/a}^{a}xf(x)\ud x=0$. 求证:
\bee
\int_{-1/a}^{a}x^2f(x)\ud x\le\int_{-1/a}^{a}f(x)\ud x.
\eee
\eq
\ba
分$0<a\le1$和$a>1$两种情况讨论.
\ea

\bq{陕西省第七次大学生高等数学竞赛复赛}{}
设在点$x=0$的某邻域$U$内, $f(x)$可展成泰勒级数, 且对任意正整数$n$, 皆有
\bee
f\left(\frac1n\right)=\frac1{n^2}.
\eee
证明: 在$U$内, 恒有$f(x)=x^2$.
\eq

\bq{陕西省第七次大学生高等数学竞赛复赛}{}
设$\frac{\ud y}{\ud x}=\frac1{1+x^2+y^2}$, 证明: $\lim_{x\to+\infty}y(x)$和$\lim_{x\to-\infty}y(x)$都存在.
\eq
\ba
用单调有界定理.
\ea

\bq{陕西省第七次大学生高等数学竞赛复赛}{}
求幂级数$\sum_{n=1}^{\infty}\left(1+\frac12+\frac13+\cdots+\frac1n\right)x^n$的收敛域与和函数.
\eq

\bq{陕西省第七次大学生高等数学竞赛复赛}{}
求极限
\bee
\lim_{n\to\infty}\sum_{i=1}^n\left(\frac1{(n+i+1)^2}+\frac1{(n+i+2)^2}+\cdots+\frac1{(n+i+i)^2}\right).
\eee
\eq
\ba
用重积分得$\ln\frac{2}{\sqrt{3}}$.
\ea

\bq{}{20170404012}
设$\{f_n(x)\}_{n=1}^{\infty}\subset C_{[a,b]}$, 且$f_n(x)$在$[a,b]$上一致收敛于$f(x)$, 则$\lim_{n\to\infty}\int_{a}^{b}f_n(x)\ud x=\int_a^bf(x)\ud x$.
\eq
\ba
因$f_n(x)\rightrightarrows f(x)$, $x\in[a,b]$, 所以$f\in C_{[a,b]}$且$\forall \varepsilon>0$, $\exists N$, 当$n>N$时, $\forall x\in[a,b]$均有$|f_n(x)-f(x)|<\varepsilon$.
$f(x), f_n(x)$连续必可积, 有
\bee
\left|\int_a^b f_{n}(x)\ud x-\int_a^bf(x)\ud x\right| < (b-a)\varepsilon.
\eee
其实当$a\le x\le b$时, 有$\left|\int_a^xf_n(t)\ud t-\int_a^xf(t)\ud t\right|<(b-a)\varepsilon$对$x$一致成立, 
所以$\int_a^xf_n(x)\ud t\rightrightarrows\int_a^xf(t)\ud t$.
\ea

\bq{}{}
$f_n(x)$在$[a,b]$上都有连续导数, 且$f_n(x)\to f(x)$, $f'_n(x)\rightrightarrows g(x)$, 则$f'(x)=g(x)$, 
即$\frac{\ud}{\ud x}\left(\lim_{n\to\infty}f_n(x)\right)=\lim_{n\to\infty}\frac{\ud}{\ud x}f_n(x)$.
\eq
\ba
因$f'_n\rightrightarrows g$, 所以$g$连续, 可积, 由\ref{q:20170404012}, $\int_a^xg(t)\ud t=\lim_{n\to\infty}\int_a^xf'_n(t)\ud t=f(x)-f(a)$,
所以$f'(x)=g(x)$. 其实$f_n(x)=f_n(a)+\int_a^xf'_n(t)\ud t$在$[a,b]$上也一致收敛.

若\ref{q:20170404012}和$\sum_{k=1}^{\infty}u_k(x)$的前$n$项部分和, 就有函数项级数的相应命题.
\begin{enumerate}[(1). ]
 \item 若$[a,b]$上$\sum_{k=1}^{\infty}u_k(x)$中每项$u_k$均连续, 且$\sum_{k=1}^{\infty} u_k(x)\rightrightarrows f(x)$, 则$f(x)\in C_{[a,b]}$且
 \bee
 \sum_{k=1}^{\infty}\int_a^bu_k(x)\ud x=\int_a^bf(x)\ud x.
 \eee
 \item 若$[a,b]$上, $\sum_{k=1}^{\infty}u_k(x)$的每项都有连续导数$u'_k(x)$且$\sum_{k=1}^{\infty}u'_k(x)\rightrightarrows g(x)$, 
 而$\sum_{k=1}^{\infty}u_k(x)\to f(x)$. 则
 \bee
 \frac{\ud}{\ud x}\left(\sum_{k=1}^{\infty}u_k(x)\right)=\sum_{k=1}^{\infty}u'_k(x).
 \eee
\end{enumerate}
\ea

\bq{}{}
举反例:
\begin{enumerate}[(1)]
 \item 积分的极限不等于极限的积分的函数列.
 \item 导数的极限不等于极限的导数的函数列.
\end{enumerate}
\eq
\ba
(1). 在$[0,1]$上极限函数为0, 但$f_n$在$\left[0,\frac1n\right]$上面积为1的脉冲函数.

(2). $f_n(x)=\frac{x}{1+n^2x^2}$, $x\in[-1,1]$.
\ea

\bq{http://math.stackexchange.com/questions/2143014}{}
证明: 对于任意的$\alpha\in\mathbb{Q}\setminus\mathbb{Z}$, $\sum_{n=1}^{\infty}\ln\left|\frac{\alpha-n}{\alpha+n}\right|$发散.
\eq

\bq{http://math.stackexchange.com/questions/472007}{}
判断$\sum_{n=10}^{\infty}\frac{\sin n}{n+10\sin n}$的敛散性.
\eq
\ba
其实, 若$f(n)$是有界函数, 且级数$\sum_{n=1}^{n}\frac{f(n)}{n}$收敛, $a$是使任意的$n$都有$n+af(n)\ne 0$, 则$\sum_{n=1}^{\infty}\frac{f(n)}{n+af(n)}$.

这是因为
\bee
\sum_{n=1}^{m}\frac{f(n)}{n+af(n)}
  = \sum_{n=1}^{m}\frac{f(n)}{n}+\sum_{n=1}^{m}\frac{-af^2(n)}{n(n+af(n))}
\eee
后一个和式用比较判别法.
\ea

\bq{http://math.stackexchange.com/questions/273559}{}
判断$\sum_{n=1}^{\infty}\frac{\sin^2(n)}{n}$的敛散性.
\eq
\ba
比较判别法, $[k\pi+\frac{\pi}{6}, (k+1)\pi-\frac{\pi}{6}]$中总有至少一个整数.

或用$\sum\frac{\sin^2(n)}{n}=\sum\frac{1}{2n}-\sum\frac{\cos(2n)}{2n}$, 前者发散, 后者用Dirichlet判别法.
\ea

\bq{http://math.stackexchange.com/questions/991652}{}
证明$\int_0^1\left|\frac{1}{x}\sin\frac{1}{x}\right|\ud x$发散.
\eq
\ba
变量替换, 积分化为$\int_1^{\infty}\frac{|\sin x|}{x}\ud x$, 然后在子区间$(k\pi, (k+1)\pi)$上求下界. 
\ea

\bq{http://math.stackexchange.com/questions/620449}{}
证明: 求$p\in\mathbb{R}$, 使积分$\int_0^{\infty}\frac{x^p}{1+x^p}\ud x$发散.
\eq
\ba
$p\ge-1$.
\ea

\bq{http://math.stackexchange.com/questions/596511}{}
计算
\bee
\int_0^{\infty}\frac{\ue^{-x}-\ue^{-2x}}{x}\ud x
\eee
\eq
\ba
用Frullani积分. 结果为$\log 2$. 或用重积分: $\ue^{-x}-\ue^{-2x}=x\int_1^2\ue^{-xt}\ud t$. 这里给出Frullani积分证明过程的做法.
\begin{align*}
 \int_a^b\frac{\ue^{-x}-\ue^{-2x}}{x}\ud x
  & =\int_a^b\frac{\ue^{-x}}{x}\ud x - \int_{2a}^{2b}\frac{\ue^{-x}}{x}\ud x\\
  & =\left(\int_a^b-\int_{2a}^{2b}\right)\frac{\ue^{-x}}{x}\ud x\\
  & = \left(\int_a^{2a}-\int_{b}^{2b}\right)\frac{\ue^{-x}}{x}\ud x\to\log(2)-0, \quad a\to0, b\to\infty
\end{align*}
上式最后一步用$\ue^{-2c}\log(2)\le\int_c^{2c}\frac{\ue^{-x}}{x}\ud x\le\ue^{-c}\log(2)$.
\ea

\bq{http://math.stackexchange.com/questions/590774}{}
已知$a>b>0$, 求$\lim_{t\to0^+}(a^{-t}-b^{-t})\Gamma(t)$.
\eq
\ba
注意到$\Gamma(t)=a^t\int_0^{\infty}\frac{\ue^{-as}}{s^{1-t}}\ud s$, 所以
\bee
\lim_{t\to0^+}(a^{-t}-b^{-t})\Gamma(t)
  = \int_0^{\infty}\frac{\ue^{-ax}-\ue^{-bx}}{x}\ud x
\eee
然后用Frullani积分算得$\log\frac{a}{b}$.
\ea

\bq{http://math.stackexchange.com/questions/590774}{20170626001}
已知$a>b>0$, 求$\int_0^{\infty}\frac{\ue^{-ax}-\ue^{-bx}}{x}\ud x$.
\eq
\ba
可以用Frullani积分. 这里用含参积分求导的方法, 定义$I(t)=\int_0^{\infty}\frac{\ue^{-x}-\ue^{-tx}}{x}\ud x$, 被积函数记为$f(x,t)$, 
由$f(x,t)$和$f_t(x,t)$在定义域均连续, $I(t)$关于$t$收敛且$\int_0^{\infty}f_t(x,t)\ud x$关于$t$一致收敛, 则满足积分号下求导条件, 
所以$\frac{\ud I}{\ud t}=-\frac{1}{a}$, $I=-\log t$.
\ea
\ba
\begin{align*}
\int_{0}^{\infty}\frac{\exp(-ax) - \exp(-bx)}{x}dx &= \lim_{\epsilon\to 0}\int_{\epsilon}^{\infty}\frac{\exp(-ax) - \exp(-bx)}{x}dx\\
&=\lim_{\epsilon\to 0}\left[\int_{\epsilon}^{\infty}\frac{\exp(-ax)}{x}dx - \int_{\epsilon}^{\infty}\frac{\exp(-bx)}{x}dx\right]\\
&=\lim_{\epsilon\to 0}\left[\int_{a\epsilon}^{\infty}\frac{\exp(-t)}{t}dt - \int_{b\epsilon}^{\infty}\frac{\exp(-t)}{t}dt\right]\\
&=\lim_{\epsilon\to 0}\int_{a\epsilon}^{b\epsilon}\frac{\exp(-t)}{t}dt=\lim_{\epsilon\to 0}\int_{a}^{b}\frac{\exp(-\epsilon u)}{u}du
\end{align*}
最后的被积函数一致收敛到$\frac{1}{u}$.
\ea
\ba
用Laplace变换, $F(s)=\int_0^{\infty}f(x)\ue^{-sx}\ud x$, 则
\bee
F(s) = \int_{0}^{\infty} \frac{e^{-bx}-e^{-ax}}{x} e^{-sx} dx \implies  F'(s) = -\int_{0}^{\infty} ({e^{-bx}-e^{-ax}}) e^{-sx} dx.
\eee
计算最后一个积分, 然后求积分并令$s\to0$, 其中积分出来的积分常数用极限$\lim_{s\to\infty}F(s)=0$计算.
\ea

\bq{http://math.stackexchange.com/questions/164400}{}
求$\int_{0}^{\infty}\frac{\ue^{-x}-\ue^{-xt}}{x}\ud x=\log t$, 其中$t>0$.
\eq
\ba
让$u=\ue^{-x}$, 得
\bee
\int_{0}^{1}\frac{u^{t-1}-1}{\log u}\ud u = \int_{0}^{1}\int_{1}^{t}u^{s-1}\ud s\ud u
  =\int_{1}^{t}\int_{0}^{1}u^{s-1}\ud u\ud s=\log t.
\eee
\ea
\ba
同\ref{q:20170626001}的解法一.
\ea
\ba
用重积分求解, $I(t)=\int_{0}^{\infty}\frac{\ue^{-x}-\ue^{-xt}}{x}\ud x=\int_{0}^{\infty}\int_{1}^{t}\ue^{-xs}\ud s\ud x$.
这里验证积分次序可交换, 则
\bee
LHS=\int_{1}^{t}\int_{0}^{\infty}\ue^{-xs}\ud x\ud s=\ln t.
\eee
\ea
\ba
用Laplace变换, $g(s)=L[f(x)]=L\left[\frac{\ue^{-ax}-\ue^{-bx}}{x}\right]$, 则$-g'(s)=L[xf(x)]=\frac{1}{s+a}-\frac{1}{s+b}$.
所以$g(s)=\log\frac{s+b}{s+a}+c$, 由于$g(\infty)=0$, 所以$c=0$. 从而
\bee
\int_{0}^{\infty}\frac{\ue^{-ax}-\ue^{-bx}}{x}\ue^{-sx}\ud x=\log\frac{s+b}{s+a}.
\eee
令$s=0$即可.
\ea
\ba
用Laplace变换, $L[1]=\int_{0}^{\infty}\ue^{-st}\ud t=\frac1s$, 则
\bee
LHS = \int_{0}^{\infty}\int_{0}^{\infty}(\ue^{-x}-\ue^{-xt})\ue^{-xs}\ud s\ud x
  = \int_{0}^{\infty}\left(\frac{1}{s+1}-\frac{1}{s+t}\right)
  = \ln\frac{s+1}{s+t}\Big\vert_{0}^{\infty}
  = \ln t
\eee
\ea

\bq{}{}
求积分
\bee
I(a)=\int_{0}^{\frac{\pi}{2}}\ln\abs{\sin^2x-a}\ud x.
\eee
\eq
\ba
当$a\le0$时, 设$a=-t$,
\begin{align*}
	I(a) & =\int_{0}^{\pi/2}\ln\left(t+\sin^{2}x\right)\ud x\\
	& =\int_{0}^{\pi/2}\left[\ln\left(\sin^{2}x\right)+\int_{0}^{t}\frac{1}{s+\sin^{2}x}\ud s\right]\ud x\\
	& =-\pi\ln2+\int_{0}^{t}\int_{0}^{\pi/2}\frac{1}{s+\sin^{2}x}\ud x\ud s\\
	& =-\pi\ln2+\frac{\pi}{2}\int_{0}^{t}\frac{1}{\sqrt{s(1+s)}}\ud s\\
	& =-\pi\ln2+\pi\arctanh\sqrt{\frac{-a}{1-a}}.
\end{align*}
当$a\ge1$时, 取$a=t+1$, 同上面的解法
\begin{align*}
	I(a) & =\int_{0}^{\pi/2}\ln\left(t+\cos^{2}x\right)\ud x\\
	& =\int_{0}^{\pi/2}\left[\ln\cos^{2}x+\int_{0}^{t}\frac{1}{s+\cos^{2}x}\ud s\right]\ud x\\
	& =-\pi\ln2+\pi\cdot\arctanh\sqrt{\frac{t}{1+t}}=-\pi\ln2+\pi\cdot\arctanh\sqrt{\frac{a-1}{a}}.
\end{align*}
当$0<a<1$时, 
\bee
I(a)=\frac{1}{4}\int_{0}^{2\pi}\ln\left|a-\sin^{2}x\right|\ud x=\frac{1}{4}\int_{0}^{2\pi}\ln\left|\frac{1-2a}{2}-\frac{\cos2x}{2}\right|\ud x.
\eee
令$\cos2\alpha=2a-1$, 则
\bee
I(a)=\frac{1}{4}\int_{0}^{2\pi}\ln\left|\frac{\cos2\alpha+\cos2x}{2}\right|\ud x=\frac{1}{4}\int_{0}^{2\pi}\ln\left|\cos(x+\alpha)\cos(x-a)\right|\ud x=\frac{1}{2}\int_{0}^{2\pi}\ln\left|\cos x\right|\ud x=-\pi\ln2.
\eee
\ea

\bq{http://tieba.baidu.com/p/2686576086}{}
设$f(x)$在实轴$\mathbb{R}$上有二阶导数, 且满足方程
\bee
2f(x)+f''(x)=-xf'(x).
\eee
求证$f(x)$和$f'(x)$都在$\mathbb{R}$上有界.
\eq
\ba
构造$L=f^2+\frac12f'^2$, 研究$L'$. 方程可改成$f''(x)+\frac{x}{2}f'(x)+f(x)=0$.
\ea

\bq{http://tieba.baidu.com/p/3846349760}{}
证明: $\sum_{n=1}^{\infty}\frac{1}{n}$发散.
\eq
\ba
由于$\lim_{n\to\infty}\left(1+\frac{1}{n}\right)^n=\ue$, 且$\left(1+\frac{1}{n}\right)^n$单调增加, 则$\left(1+\frac{1}{n}\right)^n<\ue$, 
于是$\ln\left(1+\frac{1}{n}\right)<\frac{1}{n}$, 从而$1+\frac12+\frac13+\cdots+\frac1n>\ln2+\ln\frac32+\cdots+\ln\left(1+\frac{1}{n}\right)=\ln(n+1)>\ln n$, 
即得.
\ea

\bq{http://tieba.baidu.com/p/4931607145}{}
设$f(x)$在$[a,b]$上可导, $f'(x)$在$[a,b]$上可积. 令
\bee
A_{n}=\sum_{i=1}^{n}f\left(a+i\cdot \frac{b-a}{n}\right)-\int_{a}^{b}f(x)\ud x.
\eee
试证
\bee
\lim_{n\to\infty}nA_n=\frac{b-a}{2}[f(b)-f(a)].
\eee
\eq
\ba
令$x_i=a+i\cdot\frac{b-a}{n}$, 则
\begin{align*}
nA_n & = n\left[\sum_{i=1}^{n}f(x_i)\cdot\frac{b-a}{n}-\sum_{i=1}^{n}\int_{x_{i-1}}^{x_{i}}f(x)\ud x\right]\\
  & = n\sum_{i=1}^{n}\int_{x_{i-1}}^{x_{i}}[f(x_i)-f(x)]\ud x\\
  & = n\sum_{i=1}^{n}\int_{x_{i-1}}^{x_{i}}\frac{f(x_{i})-f(x)}{x_{i}-x}(x_i-x)\ud x.
\end{align*}
在$[x_{i-1}, x_{i}]$上, $(x_i-x)$保号, 而$g(x)=\frac{f(x_i)-f(x)}{x_i-x}$连续(补充定义$g(x_i)=f'(x_i)$). 由积分第一中值定理知, 
存在$\eta_{i}\in(x_{i-1},x_i)$, 使得
\bee
\int_{x_{i-1}}^{x_{i}}\frac{f(x_{i})-f(x)}{x_i-x}(x_i-x)\ud x = g(\eta_i)\int_{x_{i-1}}^{x_i}(x_i-x)\ud x.
\eee
再由Lagrange中值定理, 以上$g(\eta_i)=\frac{f(x_{i})-f(\eta_i)}{\eta_{i}-x}=f'(\xi_{i})$, ($\xi_i\in(\eta_i, x_i)\subset(x_{i-1},x_i)$). 
于是
\begin{align*}
nA_n & = n\sum_{i=1}^{n}f'(\xi_{i})\int_{x_{i-1}}^{x_i}(x_i-x)\ud x = \frac{n}{2}\sum_{i=1}^{n}f'(\xi_i)(x_i-x_{i-1})^2\\
  & = \frac{n}{2}\cdot\frac{b-a}{n}\sum_{i=1}^{n}f'(\xi_i)(x_i-x_{i-1})=\frac{b-a}{2}\sum_{i=1}^{n}f'(\xi_i)(x_i-x_{i-1})\\
  & \to \frac{b-a}{2}\int_a^bf'(x)\ud x = \frac{b-a}{2}(f(b)-f(a)), \textrm{当}n\to\infty.
\end{align*}
\ea

\bq{}{}
$\int_0^1\frac{\sqrt[4]{x(1-x)^3}}{(1+x)^3}\ud x = \frac{3}{64}\sqrt[4]{2}\pi$.
\eq

\bq{}{}
求证:
\bee
\int_0^{\infty}\frac{\sin^3x}{x^3}\ud x=\frac{3\pi}{8}
\eee
\eq
\ba
让$f(y)=\int_{0}^{\infty}\frac{\sin^3yx}{x^3}\ud x$, 判断积分号下可求导, 有
\bee
f''(y)=\frac{3}{4}\int_{0}^{\infty}\frac{-\sin yx+3\sin 3yx}{x}\ud x=\frac{3\pi}{4}\mathrm{sign} y.
\eee
\ea
\ba
用$\sum_{k=1}^{\infty}\frac{\sin k x}{k}=\frac{\pi-x}{2}$, 由
\bee
\frac{\ud^2}{\ud x^2}\frac{\sin^3(kx)}{k^3}=\frac{9\sin(3kx)-3\sin(kx)}{4k}.
\eee
则
\bee
\sum_{k=1}^{\infty}\frac{9\sin(3kx)-3\sin(kx)}{4k}=\frac{3\pi}{4}-3x.
\eee
积分后得:
\bee
\sum_{k=1}^{\infty}\frac{\sin^3(kx)}{k^3}=\frac{3\pi}{8}x^2-\frac{1}{2}x^3.
\eee
取$x=\frac{1}{n}$后两边同乘$n^2$得
\bee
\sum_{k=1}^{\infty}\frac{\sin^3\frac{k}{n}}{(k/n)^3}\frac{1}{n}=\frac{3\pi}{8}-\frac{1}{2n}.
\eee
而广义 Riemann 和显示
\bee
\lim_{n\to\infty}\sum_{k=1}^{\infty}\frac{\sin^3\frac{k}{n}}{(k/n)^3}\frac{1}{n}=\int_{0}^{\infty}\frac{\sin^3x}{x^3}\ud x.
\eee
广义 Riemann 和成立的条件是 $\sum_{k=0}^{\infty}\sup_{x\in[k,k+1]}|f'(x)|<\infty$? 这不等式蕴含 
$\int_{0}^{\infty}|f'|\ud x<+\infty$, $f'\in L^1(0,\infty)$.
\ea
\ba
用 Parseval 定理
\bee
\int_{-\infty}^{\infty}f(x)g(x)\ud x=\frac{1}{2\pi}\int_{-\infty}^{\infty}F(s)G(s)\ud s,
\eee
其中$F(s)=\mathcal{F}[f]$, $G(s)=\mathcal{F}[g]$, $\mathcal{F}[f]=\int_{\mathbb{R}}f(x)\ue^{\ui sx}\ud x$.
若$f(x)=\frac{\sin x}{x}$, 则
\bee
F(s)=
\begin{dcases}
 \pi, \quad |s|\le 1\\
 0, \quad |s|>1,
\end{dcases}
\eee
若$g(x)=\frac{\sin^2x}{x^2}$, 则
\bee
G(s)=
\begin{dcases}
 \pi\left(1-\frac{|s|}{2}\right), |s|\le 2\\
 0, |s|>2,
\end{dcases}
\eee
所以$\int_{\mathbb{R}}\frac{\sin^3x}{x^3}\ud x=\frac{1}{2\pi}\int_{-1}^{1}\pi^2\left(1-\frac{|s|}{2}\right)\ud s=\frac{3\pi}{4}$.
即$\int_{0}^{\infty}\frac{\sin^3x}{x^3}\ud x=\frac{3\pi}{8}$.
\ea
\ba
用 Laplace 变换 $F(s)=\mathcal{L}[f]=\int_{0}^{\infty}f(x)\ue^{-sx}\ud x$, 对于$f(x)=\frac{\sin^3x}{x^3}$有
\bee
F(s)=\frac{\pi s^2}{8}+\frac{3\pi}{8}-\frac{3(s^2-1)}{8}\arctan s+\frac{s^2-9}{8}\arctan\frac{s}{3}+\frac{3s}{8}\ln\frac{s^2+1}{s^2+9}.
\eee
\ea
\ba
用 Laplace 恒等式
\bee
\int_{0}^{\infty}F(u)g(u)\ud u=\int_{0}^{\infty}f(u)G(u)\ud u, F(s)=\mathcal{L}[f(t)], G(s)=\mathcal{L}[g(t)].
\eee
让 $G(u)=\frac{1}{u^3}$ 得 $g(u)=\frac{u^2}{2}$, 让 $f(u)=\sin^3 u$ 得 $F(u)=\frac{6}{(u^2+1)(u^2+9)}$, 则
\bee
\int_{0}^{\infty}\frac{\sin^3 x}{x^3}\ud x=\frac{6}{2}\int_{0}^{\infty}\frac{u^2}{(u^2+1)(u^2+9)}\ud u = \frac{3\pi}{8}.
\eee
\ea
\ba
留数定理, 由 $\sin^3 x=\frac{3\sin x-\sin(3x)}{4}$ 得 $\int_{\mathbb{R}}\left(\frac{\sin x}{x}\right)^3\ud x=\int_{\mathbb{R}}\frac{3\sin x-\sin(3x)}{4x^3}\ud x$.
围道 $\gamma=\gamma_1\cup\gamma_2\cup\gamma_3\cup\gamma_4$, $\gamma_1=[r,R]$, $\gamma_2$ 是以 $(0,0)$ 为心 $R$ 为径的上半圆,
$\gamma_3=[-R,-r]$, $\gamma_4$ 为以 $(0,0)$ 为心 $r$ 为径的上半圆, $\gamma$ 取逆时针方向为正方向. 取 $f(z)=\frac{3\ue^{\ui z}-\ue^{3\ui z}}{z^3}$, 
则 $f$ 在 $\gamma$ 内解析, 由 Cauchy-Goursat 公式 
\be
\oint_{\gamma}f(z)=0\label{Cauchy}, 
\ee
并用 Laurent 展开或留数定理得
\bee
\int_{\gamma_4}f(z)\ud z=-\frac{3\pi\ui}{4}, 
\lim_{R\to\infty}\int_{\gamma_2}f(z)\ud z=0, 
\lim_{{r\to0}\atop{R\to\infty}}\int_{\gamma_1\cup\gamma_3}f(z)\ud z=2\ui\int_{0}^{\infty}\left(\frac{\sin x}{x}\right)^3\ud x.
\eee
代入\ref{Cauchy}即得.
\ea

\bq{}{}
设函数$f:(a,b)\to\mathbb{R}$连续可微, 又设对于任意的$x,y\in(a,b)$, 存在唯一的$z\in(a,b)$使得$\frac{f(y)-f(x)}{y-x}=f'(z)$.
证明: $f(x)$严格凸或严格凹.
\eq
\ba
反证法, 构造$\lambda$的函数
\bee
\Lambda(\lambda)=f(\lambda\alpha+(1-\lambda)\beta)-\lambda f(\alpha)-(1-\lambda)f(\beta), \quad \lambda\in(0,1)
\eee
若结论不真, 则存在$\alpha, \beta\in(a,b)$使$\Lambda(\lambda)$在$(0,1)$的某两点异号, 由于$f$连续, 所以存在$\lambda_0\in(0,1)$使$\Lambda(\lambda_0)=0$. 

设$\gamma=\lambda_0\alpha+(1-\lambda_0)\beta$, 则$\gamma\in(\alpha,\beta)\subset(a,b)$且
$\frac{f(\gamma)-f(\alpha)}{\gamma-\alpha}=\frac{f(\beta)-f(\gamma)}{\beta-\gamma}$. 再由拉格朗日中值定理得出矛盾.
\ea

\newpage
\bq{}{}
设函数$f(x)$在$\mathbb{R}$上无限可微, 且:
\begin{enumerate}[a)]
 \item 存在$L>0$, 使得对于任意的$x\in\mathbb{R}$及$n\in\mathbb{N}$有$|f^{(n)}(x)|\le L$.
 \item $f\left(\frac1n\right)=0$, 对所有$n=1,2,3\cdots$.
\end{enumerate}
求证: $f(x)\equiv 0$.
\eq
\ba
由$f$在$\mathbb{R}$上无限次可微, 且由a)知$f$有在$x=0$处的Taylor展开
\bee
f(x)=f(0)+\frac{f'(0)}{1!}x+\frac{f''(0)}{2!}x^2+\cdots.
\eee
设$N$是使$f^{(N)}(0)\ne0$的最小者, 取正整数$M>1$使$|f^{(N)}(0)|>\frac{L}{M-1}$.
则
\begin{align*}
 0=\left|f\left(\frac{1}{M}\right)\right| &= \left|f^{(N)}(0)\frac{x^N}{N!}+f^{(N+1)}(0)\frac{x^{N+1}}{(N+1)!}+\cdots\right|\\
    &\ge |f^{(N)}(0)|\frac{1}{N!M^N}-L\left(\frac{1}{(N+1)!M^{N+1}}+\frac{1}{(N+2)!M^{N+2}}+\cdots\right)\\
    &\ge |f^{(N)}(0)|\frac{1}{N!M^N}-L\left(\frac{1}{N!M^{N+1}}+\frac{1}{N!M^{N+2}}+\cdots\right)\\
    &= |f^{(N)}(0)|\frac{1}{N!M^N}-\frac{L}{N!}\frac{1}{M^{N+1}}\frac{1}{1-\frac1M}\\
    &= \left(|f^{(N)}(0)|-\frac{L}{M-1}\right)\frac{1}{N!M^N}>0.
\end{align*}
这导致矛盾, 即所有$f^{(n)}(0)=0$, $f(x)\equiv0$.
\ea

\chapter{数学分析 - 梅加强}

梅加强的大名在我上本科的时候就已经听说了, 现在去读他写的书到第二章的时候突然感受到他被奉为巨佬的原因, 不同于国内大多数数学分析教科书惯有的教学顺序,
这本书先讲掉了定积分再引进的导数, 不得不说国内这么干的, 这是我见过的第一本(虽然我也没读过几本国内大学自行出版的教材, 好在北大,
中科大的等等都大概翻过). 写这一小段文字只是突发感慨, 因为在读此之前确实也见过别的教科书这么干, 比如柯朗的《微积分和数学分析引论》也是先引入的积分再引入的导数,
这种与原来按部就班的学习形成对比, 微分和积分是实数完备性发展出来的两条支线, 最后在微积分基本定理它们融合了.

\section{NULL}
\section{数列极限}
\bd{}{}
给定序列$\left\{ a_{n}\right\} $, 实数$A\in\RR$, 如果
$$
\forall\epsilon>0,\exists N=N(\epsilon),s.t.\forall n>N,\left|a_{n}-A\right|<\epsilon,
$$
就称$a_{n}\to A$, $n\to\infty$. 反面表述序列$a_{n}\not\to A$: 
$$
\exists\epsilon>0,s.t.\forall N=N(\epsilon),\exists n>N,\left|a_{n}-A\right|\ge\epsilon.
$$
如果
$$
\forall A>0,\exists N,s.t.\forall n>N,a_{n}>A,
$$
则称$a_{n}\to+\infty$. 如果
$$
\forall A<0,\exists N,s.t.\forall n>N,a_{n}<A,
$$
则称$a_{n}\to-\infty$. 如果
$$
\forall A>0,\exists N,s.t.\forall n>N,\left|a_{n}\right|>A,
$$
则称$a_{n}\to\infty$.
\ed
%

\paragraph{性质 (极限的性质)}

1. 序列极限如果存在, 必然唯一.

2. 序列极限收敛于有限实数, 必然有界. (这个性质可以弱化, 比如允许序列前几项中有$\infty$出现, 这时本条的序列极限性质可以表述为,
从某项开始序列有界)

3. 保序性, 当$a_{n}\to A$, $b_{n}\to B$, $a_{n}\ge b_{n}$, 则$A\ge B$.
不等式$a_{n}\ge b_{n}$可以换成$a_{n}>b_{n}$.

\subsection{求极限的方法}

求极限没有通用方法, 不要有能学到通用方法的任何期待. 我们所能做的只有从最简单的方法到最复杂的方法进行逐个尝试.

\subsubsection{$\epsilon-N$法}

也就是定义法, 这个方法要求事先知道所求极限为何, 然后套用这一框架. 方法比较基本, 不再举例.

\subsubsection{夹逼原理}

这也是一个求解框架, 找到满足$a_{n}\le b_{n}\le c_{n}$, 且$a_{n}\to A$, $c_{n}\to A$的上下界来求解$b_{n}$的极限.

\subsubsection{单调有界原理}

主要用于求解抽象型极限问题.

\paragraph{例2.2.3}

设$a_{1}>0$, $a_{n+1}=\frac{1}{2}\left(a_{n}+\frac{1}{a_{n}}\right)$,
$n\ge1$, 求$a_{n}$的极限.

求解递推公式的极限问题常用不动点法先找到极限是什么. 也就是求解$x=\frac{1}{2}\left(x+\frac{1}{x}\right)$,
得到$x=\pm1$.

注意到$a_{1}>0$, 所以数列的每一项$a_{n}>0$. 并计算
$$
a_{n+1}-1=\frac{1}{2a_{n}}(a_{n}-1)^{2}\ge0\Longrightarrow a_{n+1}\ge1,\quad\forall n\ge1.
$$
这表明当$n\ge2$时, $\frac{1}{a_{n}}\le1\le a_{n}$, 所以
$$
a_{n+1}=\frac{1}{2}\left(a_{n}+\frac{1}{a_{n}}\right)\le a_{n}.
$$
序列$\left\{ a_{n}\right\} $从第二项起单调递减, 有下界$1$. 递推方程的正不动点只有$x=1$,
所以$a_{n}\to1$.

事实上, 对于递推公式型极限问题也可以尝试求解它的通项公式, 比如
$$
\frac{a_{n+1}-1}{a_{n+1}+1}=\left(\frac{a_{n}-1}{a_{n}+1}\right)^{2}.
$$


\subsubsection{重要极限}

\paragraph{$e$相关}

$$
\left(1+\frac{1}{n}\right)^{n}<\left(1+\frac{1}{n+1}\right)^{n+1}<e<\left(1+\frac{1}{n+1}\right)^{n+2}<\left(1+\frac{1}{n}\right)^{n+1},\quad\forall n\ge1.
$$

$$
\left(1+\frac{1}{n}\right)^{n}\cdot1\le\left(\frac{n\left(1+\frac{1}{n}\right)+1}{n+1}\right)^{n+1}=\left(1+\frac{1}{n+1}\right)^{n+1},
$$
$$
\left(\frac{1+\frac{1}{n-1}}{1+\frac{1}{n}}\right)^{n}=\left(\frac{n^{2}}{n^{2}-1}\right)^{n}=\left(1+\frac{1}{n^{2}-1}\right)^{n}>1+\frac{n}{n^{2}-1}>1+\frac{1}{n}\Longrightarrow\left(1+\frac{1}{n-1}\right)^{n}>\left(1+\frac{1}{n}\right)^{n+1}.
$$

Bernoulli不等式
$$
x\ge-1\Longrightarrow(1+x)^{n}\ge1+nx,
$$
等号当且仅当$x=0$取到. 推广形式
$$
\prod_{k=1}^{n}\left(1+x_{k}\right)\ge1+\sum_{k=1}^{n}x_{k},
$$
其中$x_{k}\ge0$. 注意, 这里给出的两种Bernoulli不等式的前提条件不同, 后者不能是$x_{k}\ge-1$,
因为$(1-x)(1+x)=1-x^{2}\le1$.

\paragraph{Euler常数}

$$
H_{n}=\ln n+\gamma+o(1).
$$


\paragraph{例2.2.7}

求
$$
\lim_{n\to\infty}\left(\frac{1}{n+1}+\cdots+\frac{1}{2n}\right).
$$

1. 单调上升有上界.

2. $H_{2n}-H_{n}=\ln2n-\ln n+o(1)$.

3. 
$$
\lim_{n\to\infty}\frac{1}{n}\sum_{k=1}^{n}\frac{1}{1+\frac{k}{n}}=\int_{0}^{1}\frac{1}{1+x}\ud x.
$$

4. Euler求和公式
$$
\sum_{k=1}^{n}\frac{1}{n+k}=\ln\frac{2n}{n+1}+\frac{1}{2}\left(\frac{1}{n+1}+\frac{1}{2n}\right)-\int_{1}^{n}\frac{\left\langle x\right\rangle }{(n+x)^{2}}\ud x.
$$


\paragraph{Stirling公式}

$$
n!\sim\sqrt{2\pi n}\left(\frac{n}{e}\right)^{n}e^{\frac{\theta_{n}}{12n}},\quad\theta_{n}\in\left(0,1\right).
$$


\subsubsection{上下极限}

这种方法也常用于求解抽象型序列极限问题. 

序列收敛的一个充要条件是, 序列的上下极限相等.

定义
$$
\limsup_{n\to\infty}a_{n}\coloneqq\lim_{n\to\infty}\left(\sup_{k\ge n}a_{k}\right),\quad\liminf_{n\to\infty}a_{n}\coloneqq\lim_{n\to\infty}\left(\inf_{k\ge n}a_{k}\right)
$$


\paragraph{命题2.2.4}

1. 存在$N_{0}$, 当$n>N_{0}$时, $a_{n}\ge b_{n}$, 则
$$
\liminf_{n\to\infty}a_{n}\ge\liminf_{n\to\infty}b_{n},\quad\limsup_{n\to\infty}a_{n}\le\limsup_{n\to\infty}b_{n}.
$$

2. 
$$
\limsup_{n\to\infty}\left(a_{n}+b_{n}\right)\le\limsup_{n\to\infty}a_{n}+\limsup_{n\to\infty}b_{n}.
$$


\paragraph{例2.2.12}

设序列$\left(a_{n}\right)$, $a_{n}\ge0$, 满足$a_{m+n}\le a_{m}+a_{n}$,
$\forall m,n\ge1$. 证明$\left(\frac{a_{n}}{n}\right)$收敛.

证明: 设$n>m$, 则$n=mk+l$, 其中$0\le l\le m-1$. 

对于任意固定的$m$, 有
$$
a_{n}=a_{mk+l}\le ka_{m}+a_{l}\Longrightarrow\frac{a_{n}}{n}\le\frac{ka_{m}}{km+l}+\frac{a_{l}}{n}.
$$
不等式两边同时取上极限
$$
\limsup_{n\to\infty}\frac{a_{n}}{n}\le\frac{a_{m}}{m}.
$$
所以
$$
\limsup_{n\to\infty}\frac{a_{n}}{n}\le\liminf_{m\to\infty}\frac{a_{m}}{m}.
$$


\subsubsection{Cauchy收敛准则}

\paragraph{定义}

序列$\left(a_{n}\right)$, 如果$\forall\epsilon>0$, $\exists N=N(\epsilon)$,
s.t. $\forall m,n>N$, $\left|a_{m}-a_{n}\right|<\epsilon$, 则称$\left(a_{n}\right)$为Cauchy列.

其它表述: $\forall\epsilon>0$, $\exists N=N(\epsilon)$, s.t. $\forall n>N$,
$\forall p>0$, $\left|a_{n+p}-a_{n}\right|<\epsilon$.

反面描述: $\exists\epsilon_{0}>0$, s.t. $\forall N$, $\exists m_{0},n_{0}>N$,
使得$\left|a_{m_{0}}-a_{n_{0}}\right|\ge\epsilon_{0}$.

Cauchy列均有解. (这里仍然可以允许序列的前几项可以取$\infty$, 此时Cauchy除了开始的有限项外是有界序列).

序列$\left(a_{n}\right)$收敛当且仅当$\left(a_{n}\right)$是Cauchy列.

\paragraph{习题2}

设序列$\left(a_{n}\right)$满足
$$
\lim_{n\to\infty}\left|a_{n+p}-a_{n}\right|=0,\quad\forall p\ge1.
$$
问$\left(a_{n}\right)$是否是Cauchy列?

$a_{n}=H_{n}$就是反例.

\subsubsection{Stolz公式}

\paragraph{定理2.4.2}

设序列$(x_{n})$, $(y_{n})$, 其中$y_{n}$单调上升趋向$\infty$, 若
$$
\lim_{n\to\infty}\frac{x_{n}-x_{n-1}}{y_{n}-y_{n-1}}=A,
$$
则
$$
\lim_{n\to\infty}\frac{x_{n}}{y_{n}}=A.
$$


\paragraph{注: }

和洛必达法则的情况一样, 当
$$
\lim_{n\to\infty}\frac{x_{n}-x_{n-1}}{y_{n}-y_{n-1}}
$$
不存在时, 
$$
\lim_{n\to\infty}\frac{x_{n}}{y_{n}}
$$
仍可能存在.

\paragraph{定理2.4.3}

设$\left(y_{n}\right)$单调下降趋向于0, $\left(x_{n}\right)\to0$, 若
$$
\lim_{n\to\infty}\frac{x_{n}-x_{n-1}}{y_{n}-y_{n-1}}=A,
$$
则
$$
\lim_{n\to\infty}\frac{x_{n}}{y_{n}}=A.
$$


\paragraph{注}

条件$\left(x_{n}\right)\to0$是必要的, 比如$x_{n}\equiv C$是一个矛盾.

\paragraph{例2.1.15}

设$\lim_{n\to\infty}a_{n}=A$, 证明
$$
\lim_{n\to\infty}\frac{a_{1}+\cdots+a_{n}}{n}=A.
$$

这个例子有多种证法, 使用Stolz公式只需一步.

\paragraph{例2.4.3}

设$x_{1}\in\left(0,1\right)$, $x_{n+1}=x_{n}(1-x_{n})$, $\forall n\ge1$.
证明$nx_{n}\to1$, $n\to\infty$.

\paragraph{证明:}

单调收敛证明$x_{n+1}<x_{n}$. 假设极限为$x$, 则$x=x(1-x)$, 解的$x=0$. 所以$x_{n}\to0$,
$n\to\infty$.

从递推公式得到
$$
\frac{1}{x_{n+1}}-\frac{1}{x_{n}}=\frac{1}{1-x_{n}}\to1,\quad n\to\infty.
$$
由Stolz公式
$$
\lim_{n\to\infty}nx_{n}=\lim_{n\to\infty}\frac{n}{x_{n}^{-1}}=\lim_{n\to\infty}\frac{n-(n-1)}{x_{n}^{-1}-x_{n-1}^{-1}}=\lim_{n\to\infty}\frac{1}{\frac{1}{1-x_{n-1}}}=1.
$$


\paragraph{不用Stolz公式的证法}

和上面一样, $x_{n}$单调收敛到0, 且有$x_{n+1}^{-1}-x_{n}^{-1}=\frac{1}{1-x_{n}}\to1$,
$n\to\infty$. 则
$$
\frac{1}{nx_{n}}=\frac{x_{n}^{-1}}{n}=\frac{\left(x_{n}^{-1}-x_{n-1}^{-1}\right)+\left(x_{n-1}^{-1}-x_{n-2}^{-1}\right)+\cdots+\left(x_{2}^{-1}-x_{1}^{-1}\right)+x_{1}^{-1}}{n}\to\lim_{n\to\infty}\left(x_{n}^{-1}-x_{n-1}^{-1}\right)=1.
$$
上面最后一步用到了例2.1.15.

\paragraph{习题2}

设 $\lim_{n\rightarrow\infty}n\left(a_{n}-A\right)=B,k$ 为正整数, 则 
$$
\lim_{n\rightarrow\infty}n\left(\frac{a_{1}+2^{k}a_{2}+\cdots+n^{k}a_{n}}{n^{k+1}}-\frac{A}{k+1}\right)=\frac{B}{k}+\frac{A}{2}.
$$

pf.

\begin{align*}
	\lim_{n\to\infty}\frac{(k+1)(a_{1}+\cdots+n^{k}a_{n})-n^{k+1}A}{(k+1)n^{k}} & =\lim_{n\to\infty}\frac{(k+1)\left((n+1)^{k}a_{n+1}\right)-\left((n+1)^{k+1}-n^{k+1}\right)A}{(k+1)\left((n+1)^{k}-n^{k}\right)}\\
	& =\lim_{n\to\infty}\frac{(k+1)\left[n^{k}+\binom{k}{1}n^{k-1}+o\left(n^{k-1}\right)\right]a_{n+1}-\left(\binom{k+1}{1}n^{k}+\binom{k+1}{2}n^{k-1}+o(n^{k-1})\right)A}{(k+1)\left(\binom{k}{1}n^{k-1}+o(n^{k-1})\right)}\\
	& =\lim_{n\to\infty}\frac{(k+1)\left[n+\binom{k}{1}+o(1)\right]a_{n+1}-\left(\binom{k+1}{1}n+\binom{k+1}{2}+o(1)\right)A}{(k+1)\left(\binom{k}{1}+o(1)\right)}\\
	& =\lim_{n\to\infty}\frac{\boldsymbol{(k+1)}\left({\color{teal}n}+\boldsymbol{k}+{\color{lime}o(1)}\right)\boldsymbol{{\color{teal}a_{n+1}}}-\left({\color{teal}(k+1)n}+{\color{magenta}\frac{k(k+1)}{2}}+{\color{lime}o(1)}\right){\color{teal}A}}{(k+1)\left(k+o(1)\right)}\\
	& =\lim_{n\to\infty}\left(\frac{{\color{teal}n}\left({\color{teal}a_{n+1}-A}\right)}{k+o(1)}-\frac{{\color{magenta}\frac{k(k+1)}{2}A}}{(k+1)(k+o(1))}+\frac{\boldsymbol{k(k+1)a_{n+1}}+{\color{lime}o(1)}\cdot(a_{n+1}-A)}{(k+1)(k+o(1))}\right)\\
	& =\frac{B}{k}-\frac{A}{2}+A
\end{align*}
其中
$$
\lim_{n\to\infty}n(a_{n}-A)=B\Longrightarrow\lim_{n\to\infty}a_{n}=A+\lim_{n\to\infty}(a_{n}-A)=A+\lim_{n\to\infty}\frac{n(a_{n}-A)}{n}=A+\lim_{n\to\infty}\frac{B}{n}+\lim_{n\to\infty}\frac{n(a_{n}-A)-B}{n}=A+0+0
$$


\paragraph{习题6}

设 $a_{1}=1,a_{n+1}=a_{n}+\frac{1}{2a_{n}}$, 证明 (1) $\lim_{n\rightarrow\infty}\frac{a_{n}}{\sqrt{n}}=1$,
(2) $\lim_{n\rightarrow\infty}\frac{a_{n}^{2}-n}{\ln n}=\frac{1}{4}$.

pf. (2) 由Stolz公式
\begin{align*}
	\lim_{n\to\infty}\frac{a_{n}^{2}-n}{\ln n} & =\lim_{n\to\infty}\frac{a_{n+1}^{2}-a_{n}^{2}-1}{\ln(n+1)-\ln n}\\
	& =\lim_{n\to\infty}\frac{a_{n+1}^{2}-a_{n}^{2}-1}{\ln\left(1+\frac{1}{n}\right)}
\end{align*}
这启发我们去简化$a_{n+1}^{2}-a_{n}^{2}$项, 由递推公式
$$
a_{n+1}-a_{n}=\frac{1}{2a_{n}}\Longrightarrow a_{n+1}^{2}-a_{n}^{2}=\frac{a_{n+1}+a_{n}}{2a_{n}}=\frac{a_{n}+\frac{1}{2a_{n}}+a_{n}}{2a_{n}}=1+\frac{1}{4a_{n}^{2}}\Longrightarrow a_{n+1}^{2}-a_{n}^{2}-1=\frac{1}{4a_{n}^{2}}
$$
$$
\lim_{n\to\infty}\frac{a_{n+1}^{2}-a_{n}^{2}-1}{\ln\left(1+\frac{1}{n}\right)}=\lim_{n\to\infty}\frac{\frac{1}{4a_{n}^{2}}}{\frac{1}{n}}=\frac{1}{4}\lim_{n\to\infty}\frac{n}{a_{n}^{2}}=\frac{1}{4}.
$$


\paragraph{习题8}

设 $\lim_{n\rightarrow\infty}\left(a_{n+2}-a_{n}\right)=A$, 证明 (1)
$\lim_{n\rightarrow\infty}\frac{a_{n}}{n}=\frac{A}{2}$, (2) $\lim_{n\rightarrow\infty}\frac{a_{n+1}-a_{n}}{n}=0$.

pf. (1) 用stolz公式
\begin{align*}
	\lim_{n\to\infty}\frac{a_{n}}{n} & =\lim_{n\to\infty}\frac{a_{n+2}-a_{n}}{(n+2)-n}=\lim_{n\to\infty}\frac{a_{n+2}-a_{n}}{2}=\frac{A}{2}.
\end{align*}

(2) 用stolz公式
$$
\lim_{n\to\infty}\frac{a_{n+1}-a_{n}}{n}=\lim_{n\to\infty}\frac{a_{n+3}-a_{n+1}-(a_{n+2}-a_{n})}{(n+2)-n}=\lim_{n\to\infty}\left(\frac{a_{n+3}-a_{n+1}}{2}-\frac{a_{n+2}-a_{n}}{2}\right)=\frac{A}{2}-\frac{A}{2}=0.
$$

若$y_{n}$单调上升趋于无穷, 且
$$
\lim_{n\to\infty}\frac{x_{n+k}-x_{n}}{y_{n+k}-y_{n}}=A,
$$
则
$$
\lim_{n\to\infty}\frac{x_{n}}{y_{n}}=A.
$$


\paragraph{习题9}

设$\lim_{n\to\infty}(a_{n+1}-a_{n})=A$, 则
$$
\lim_{n\to\infty}\frac{a_{n}}{n}=A.
$$

Stolz公式不是万能的, 比如

\paragraph{习题15}

设
$$
\lim_{n\to\infty}a_{n}=A,\quad\lim_{n\to\infty}b_{n}=B,
$$
则
$$
\frac{1}{n}\left(a_{1}b_{n}+a_{2}b_{n-1}+\cdots+a_{n}b_{1}\right)=AB.
$$


\paragraph{证明:}

设$a_{n}=A+\alpha_{n}$, $b_{n}=B+\beta_{n}$, 则问题不妨在$A=B=0$时证明即可.
其实
\begin{align*}
	\frac{1}{n}\sum_{k=1}^{n}a_{k}b_{n+1-k} & =\frac{1}{n}\sum_{k=1}^{n}\left(AB+A\beta_{n+1-k}+B\alpha_{k}+\alpha_{k}\beta_{n+1-k}\right)\\
	& =AB+\frac{A}{n}\sum_{k=1}^{n}\beta_{k}+\frac{B}{n}\sum_{k=1}^{n}\alpha_{k}+\frac{1}{n}\sum_{k=1}^{n}\alpha_{k}\beta_{n+1-k}\\
	& =AB+A\cdot o(1)+B\cdot o(1)+M\cdot o(1)=AB+o(1).
\end{align*}
上式最后用到收敛序列有界的结论.
\section{连续函数}
\subsection{函数的极限}

设$x_{0}\in\RR$, $\delta>0$, $\left(x_{0}-\delta,x_{0}+\delta\right)$称为$x_{0}$的开邻域.

$\left(x_{0}-\delta,x_{0}\right)\cup\left(x_{0},x_{0}+\delta\right)$称为$x_{0}$的去心开邻域.

\paragraph{定义3.1.1}

设$f(x)$定义在$x_{0}$的某个去心开邻域上, 若$\exists A\in\RR$, s.t. $\forall\epsilon>0$,
$\exists\delta=\delta(\epsilon,x_{0})\in(0,\delta_{0})$, 当$0<\left|x-x_{0}\right|<\delta$时,
有
\[
\left|f(x)-A\right|<\epsilon.
\]
则称$f(x)$在$x_{0}$处有极限$A$, 记为
\[
\lim_{x\to x_{0}}f(x)=A,\text{ or }f(x)\to A,\ x\to x_{0}.
\]


\paragraph{注: }

去心邻域说明$f(x)$在$x=x_{0}$处可能没有定义.

可以类似的定义左右极限.

\paragraph{命题3.1.1}

$f$在$x_{0}$处有极限的充要条件是$f$在$x_{0}$的左右极限存在且相等.

\paragraph{命题3.1.2 (夹逼原理)}

设在$x_{0}$的一个空心领域内有
\[
f_{1}(x)\le f(x)\le f_{2}(x).
\]
若$f_{1}$, $f_{2}$在$x_{0}$处的极限存在且等于$A$, 则$f(x)$在$x_{0}$处极限为$A$.

\paragraph{命题3.1.3 (极限唯一性)}

函数极限存在必然唯一.

$\epsilon-\delta$语言是证明函数极限的最简单框架, 其难点仅在于对不等式的掌握情况, 比如重要极限
\[
\lim_{x\to0}\frac{\sin x}{x}=1
\]
对应不等式
\[
\cos x<\frac{\sin x}{x}<1.
\]

同样类似地给出涉及无穷大与无穷远时的函数极限的定义.

\paragraph{习题12}

设 $f,g$ 为两个周期函数, 如果 $\lim_{x\rightarrow+\infty}[f(x)-g(x)]=0$, 则
$f=g$. 

注: $f,g$的周期比可能是无理数, 所以$f-g$可能不是周期函数.

pf. 设$f$的周期为$T$, $g$的周期为$S$. 则
\begin{align*}
	f(x)-g(x) & =f(x+nT)-g(x+nS)\\
	& =f(x+nT)-g(x+nT)\\
	& \quad+g(x+nT{\color{magenta}+nS})-f(x+nS{\color{magenta}+nT})\\
	& \quad+f(x+nS)-g(x+nS)\\
	& =\lim_{n\to\infty}f(x+nT)-g(x+nT)\\
	& \quad+\lim_{n\to\infty}g(x+nT{\color{magenta}+nS})-f(x+nS{\color{magenta}+nT})\\
	& \quad+\lim_{n\to\infty}f(x+nS)-g(x+nS)\\
	& =0+0+0.
\end{align*}


\paragraph{习题13}

设 $\lim_{x\rightarrow0}f(x)=0,\lim_{x\rightarrow0}\frac{1}{x}[f(2x)-f(x)]=0$,
证明 $\lim_{x\rightarrow0}\frac{f(x)}{x}=0$.

pf. $\forall\epsilon>0$, $\exists X>0$, s.t. $\forall x>X$, $\left|f(x)\right|<\epsilon$,
且
\[
\left|f(2x)-f(x)\right|\le\epsilon x.
\]
所以
\[
\left|f(x)-f\left(\frac{x}{2^{n}}\right)\right|\le\epsilon x.
\]
令$n\to\infty$即可.

\subsection{函数极限的性质}

\paragraph{定理3.1.5 (Heine, 归结原则)}

设$f$定义在$x_{0}$的某个去心领域上, $f$在$x_{0}$处极限为$A$的充要条件是$\forall x_{n}\to x_{0}$,
($n\to\infty$), 且$x_{n}\ne x_{0}$, ($\forall n$), 均有
\[
\lim_{n\to\infty}f(x_{n})=A.
\]


\paragraph{证明:}

$\Longrightarrow$: 是容易的; $\Longleftarrow$: 用反证法, 则
\[
\exists\epsilon_{0}>0,\ s.t.\ \forall\delta>0,\ \exists x_{\delta},\ s.t.\ 0<\left|x_{\delta}-x_{0}\right|<\delta,
\]
但$\left|f(x_{\delta})-A\right|\ge\epsilon_{0}$. 取$\delta=\frac{1}{n}$,
构造子列$(x_{n})\to x_{0}$, 但$\left|f(x_{n})-A\right|\ge\epsilon_{0}$.
矛盾.

Heine定理可改述为: $f(x)$在$x_{0}$处有极限当且仅当, $\forall x_{n}\to x_{0}$,
($x_{n}\ne x_{0}$), $\lim_{n\to\infty}f(x_{n})$存在.

\paragraph{定理3.1.6 (Cauchy准则)}

设$f$在$x_{0}$的空心领域上有定义, 则$f$在$x_{0}$处有极限, iff, $\forall\epsilon>0$,
$\exists\delta>0$, s.t. 当$0<\left|x'-x_{0}\right|<\delta$, $0<\left|x''-x_{0}\right|<\delta$时,
有
\[
\left|f(x')-f(x'')\right|<\epsilon.
\]


\paragraph{注:}

对于无穷远处极限有限时, Cauchy准则仍然成立.

给出Cauchy准则的否定表述.

\paragraph{定理3.1.7 (单调有界原理)}

设$f$定义在$(x_{0}-\delta,x_{0})$上, 若$f$单调上升有上界, 或$f$单调下降有下界, 则$f$在$x_{0}$有左极限.

\paragraph{定理3.1.8}

(1). (局部有界原理) 若$f$在$x_{0}$处有有限极限, 则$f$在$x_{0}$的某空心领域内有界.

(2). (保序性) 
\begin{align*}
	\lim_{x\to x_{0}}f(x)=A,\ \lim_{x\to x_{0}}g(x)=B,\ f(x)\ge g(x), & \Longrightarrow A\ge B.\\
	\lim_{x\to x_{0}}f(x)=A,\ \lim_{x\to x_{0}}g(x)=B,\ A>B, & \Longrightarrow\exists U_{x_{0}}^{\circ}\ s.t.\ f(x)>g(x),\ \forall x\in U_{x_{0}}^{\circ}.
\end{align*}

(3). (四则运算)

\paragraph{定理3.1.9 (复合函数极限)}

设$f(y)\to A$, $y\to y_{0}$; $g(x)\to y_{0}$, $x\to x_{0}$, 且$\exists U_{x_{0}}^{\circ}$,
s.t. $\forall x\in U_{x_{0}}^{\circ}$, $g(x)\ne y_{0}$, 则$f(g(x))\to A$,
$x\to x_{0}$.

这个定理说明极限定义中去心领域的重要性, 定理中$y_{0}\not\in g(U_{x_{0}}^{\circ})$不可以弱化为:
存在收敛于$x_{0}$的序列$\left(x_{n}\right)$, 使得$g(x_{n})\ne y_{0}$, $\forall n$.
比如
\[
f(y)=\begin{cases}
	1, & y\ne0,\\
	0, & y=0,
\end{cases}\quad g(x)\equiv0,\ y_{0}=0.
\]

当$f$在$y_{0}$处连续时, 这个去心领域的条件又可以去掉, 这说明研究连续函数是有价值的.

\subsection{无穷小量与无穷大量的阶}

\paragraph{定义3.2.1 (无穷小量与无穷大量)}

若函数$f$在$x_{0}$处的极限是$0$, 则称$f$在$x\to x_{0}$时为无穷小量, 记为$f(x)=o(1)$,
($x\to x_{0}$); 若$x\to x_{0}$时, $\left|f\right|\to+\infty$, 则称$f$在$x\to x_{0}$时为无穷大量.

在无穷远处也可以定义无穷小量和无穷大量, 数列也可以定义无穷小量和无穷大量.

\paragraph{定理3.2.1 (等价代换)}

设$x\to x_{0}$时, $f\sim f_{1}$, $g\sim g_{1}$, 若$\frac{f_{1}}{g_{1}}$在$x_{0}$处有极限,
则$\frac{f}{g}$在$x_{0}$处有极限, 且极限相等.

几个常用的等价代换:
\[
\tan x\sim\sin x\sim x\sim e^{x}-1\sim\ln(1+x).
\]


\paragraph{无穷小量的性质:}

习题4. 设$f(x)=o(1)$, ($x\to x_{0}$), 证明, 当$x\to x_{0}$时, 有

(1) $o(f(x))+o(f(x))=o(f(x))$.

(2) $o(cf(x))=o(f(x))$, 其中$c$是常数.

(3) $g(x)\cdot o(f(x))=o(f(x)g(x))$, 其中$g(x)$是有界函数.

(4) $[o(f(x))]^{k}=o\left(f^{k}(x)\right)$.

\subsection{连续函数}

用来刻画连续变化的量

\paragraph{定义3.3.1 (连续性)}

若$f$在$x_{0}$的某领域上有定义, 且$f$在$x_{0}$处的极限是$f(x_{0})$, 则称$f$在点$x_{0}$处连续,
$x_{0}$称为$f$的连续点. 类似地可以定义左右连续. 在定义域上每一点都连续的函数称为连续函数.

$f$在$x_{0}$处下半连续: $\forall\epsilon>0$, $\exists\delta>0$, 当$\left|x-x_{0}\right|<\delta$时,
总有$f(x)>f(x_{0})-\epsilon$.

$f$在$x_{0}$处上半连续: $\forall\epsilon>0$, $\exists\delta>0$, 当$\left|x-x_{0}\right|<\delta$时,
总有$f(x)<f(x_{0})+\epsilon$.

\paragraph{连续函数的基本性质:}

(1). 保持四则运算;

(2). 若$f,g$连续, 则$\max\left\{ f,g\right\} $和$\min\left\{ f,g\right\} $均连续.

\paragraph{定理3.3.2 (复合函数连续性)}

设$f$在$y_{0}$处连续, $g(x)\to y_{0}$, $x\to x_{0}$, 则
\[
\lim_{x\to x_{0}}f(g(x))=f\left(\lim_{x\to x_{0}}g(x)\right)=f(y_{0}).
\]
当$g$在$x_{0}$处连续时, $f(g(x))$在$x_{0}$处连续.

\paragraph{定义3.3.2}

设$x_{0}$是$f(x)$的间断点, 如果$f(x_{0}-0)$, $f(x_{0}+0)$均存在且有限, 则称$x_{0}$是第一类间断点,
否则, 称为第二类间断点. 按照左右极限不相等和相等来区分跳跃间断点和可去间断点.

\paragraph{命题3.3.3}

设$f(x)$在$[a,b]$上单调, $x_{0}\in\left(a,b\right)$是$f(x)$的间断点, 则$x_{0}$是跳跃间断点.

\paragraph{命题3.3.4}

设$f(x)$定义在区间$I$上的单调函数, 则$f(x)$的间断点至多可数.

\paragraph{证明:}

间断点$x$与开区间$\left(f(x_{0}-0),f(x_{0}+0)\right)$一一对应且至多可数.

\paragraph{命题3.3.5}

若$f(x)$定义在区间$I$上严格单调, 则$f(x)$连续当且仅当$f(I)$也是区间.

\paragraph{证明:}

$\Longrightarrow$: 用介值定理. $\Longleftarrow$: 反证法, 则有$x_{0}$使得$\left(f(x_{0}-0),f(x_{0}+0)\right)$不在区间$f(I)$内,
矛盾.

\paragraph{推论3.3.6}

定义在区间$I$上的严格单调连续函数$f(x)$一定可逆, 且其逆严格单调连续.

\subsection{闭区间上连续函数的性质}

依赖实数系的基本性质

\paragraph{定理3.4.1 (有界性定理)}

设$f\in C[a,b]$, 则$f$在$[a,b]$上有界.

\paragraph{证法一:}

反证法, 取$\left|f(x_{n})\right|\ge n$, 由聚点定理$\Longrightarrow x_{n_{k}}\to x_{0}\in[a,b]$,
$f$连续使得$f(x_{n_{k}})\to f(x_{0})$有界, 矛盾.

\paragraph{证法二:}

连续点$x$处有领域$U_{x}(\delta_{x})$, 使得其上$\left|f-f(x)\right|\le1$, 这样的$\left(U_{x}\right)$形成$[a,b]$的覆盖,
用有限覆盖定理.

\paragraph{定理3.4.2 (最值定理)}

设$f(x)\in C[a,b]$, 则$f(x)$在$[a,b]$上取到最值.

\paragraph{证法一:}

$f$有界, $[a,b]$闭, 所以逼近上确界的点列$\left\{ x_{n}\right\} $有收敛子列$\left\{ x_{n_{k}}\right\} $,
再由$f$连续得到最值点.

\paragraph{证法二:}

反证法, 设$f<M\coloneqq\sup f$, 构造$F(y)=\frac{1}{M-f(y)}\in C[a,b]$,
同样有界$F(y)<K$, $K>0$, 则$\frac{1}{M-f(y)}<K$得出
\[
\sup_{x}f(x)=M>f(y)+\frac{1}{K}\Longrightarrow\sup_{x}f(x)\ge\sup_{y}f(y)+\frac{1}{K}>\sup_{x}f(x).
\]

一般区间上连续函数最值判别法:

\paragraph{命题5.1.3}

设$f\in C(\RR)$, 且
\[
\lim_{x\to-\infty}f(x)=\lim_{x\to+\infty}f(x)=+\infty\text{ (或}-\infty\text{)}.
\]
则$f$在$\RR$上达到最小(大)值.

\paragraph{定理3.4.3 (零点定理, Bolzano)}

设$f\in C[a,b]$, $f(a)f(b)<0$, 则存在$\xi\in\left(a,b\right)$, s.t.
$f(\xi)=0$.

\paragraph{证法一:}

区间二分法+区间套定理.

\paragraph{证法二:}

用连续函数的保号性, 构造
\[
A=\left\{ x\in[a,b]:f(x)<0\right\} ,
\]
取$\xi=\sup A$, 则有$x_{n}\in A$, $x_{n}\to\xi$, $f(x_{n})<0$, 所以$f(\xi)\le0$.
反之, 在$(\xi,b)$上$f\ge0$, 取$x_{n}\downarrow\xi$, 则$f(x_{n})\ge0\Longrightarrow f(\xi)=0$.

\paragraph{定理3.4.4 (介值定理)}

设$f\in C[a,b]$, $\mu$严格介于$f(a)$与$f(b)$之间, 则存在$\xi\in(a,b)$, s.t.
$f(\xi)=\mu$.

\paragraph{推论3.4.5}

设$f(x)$是$[a,b]$上的连续函数, 则$f([a,b])=[m,M]$, 其中$m,M$是$f$在$[a,b]$上的最小值,
最大值.

\paragraph{推论3.4.6}

设区间$I$上, $f(x)\in C(I)$, 则$f(I)$是区间. (可以退化为单点集)

\paragraph{注: }

区间$I$可以无界, 可以是开集.

\paragraph{推论3.4.7}

设$f(x)$是区间$I$上的连续函数, 则$f(x)$可逆的充要条件是$f(x)$严格单调.

\subsection{一致连续性}

\paragraph{定义3.4.1 (一致连续)}

设$f(x)$定义在区间$I$上, 若$\forall\epsilon>0$, $\exists\delta=\delta(\epsilon)>0$,
s.t. 当$x_{1},x_{2}\in I$, $\left|x_{1}-x_{2}\right|<\delta$时, 有$\left|f(x_{1})-f(x_{2})\right|<\epsilon$,
则称$f(x)$在$I$中一致连续.

否定表述: $f(x)$在$I$中不一致连续: $\exists\epsilon_{0}>0$, 以及$\left(a_{n}\right)$,
$\left(b_{n}\right)\subseteq I$, 且$a_{n}-b_{n}\to0$, ($n\to\infty$).
有$\left|f(a_{n})-f(b_{n})\right|\ge\epsilon_{0}$.

\paragraph{定理3.4.9 (Cantor定理)}

闭区间上, 连续函数一致连续.

\paragraph{证法一:}

(反证), $\exists\epsilon_{0}>0$, $\left(a_{n}\right)$, $\left(b_{n}\right)\subseteq[a,b]$,
$a_{n}-b_{n}\to0$, $n\to\infty$, 且$\left|f(a_{n})-f(b_{n})\right|\ge\epsilon_{0}$.
用聚点定理取$\left(b_{n}\right)$的收敛子列$b_{n_{k}}\to x_{0}\in[a,b]$, 则$a_{n_{k}}\to x_{0}$,
取极限.

\paragraph{证法二:}

用连续性构造有限覆盖开集.

\paragraph{定义3.4.2 (振幅, 连续性模)}

设$f(x)$在$x_{0}$的开邻域内有定义, 称
\[
\omega_{f}(x_{0},r)=\sup\left\{ \left|f(x')-f(x'')\right|:x',x''\in B_{x_{0}}(r)\right\} ,\quad r>0.
\]
为$f$在$(x_{0}-r,x_{0}+r)$上的振幅, 显然, $\omega_{f}(x_{0},r)$关于$r\to0+$递减,
故
\[
\omega_{f}(x_{0})=\lim_{r\to0+}\omega_{f}(x_{0},r)
\]
存在, (不一定有限). 称为$f$在点$x_{0}$处的振幅.

注: 定义提到的是两种振幅, 分别表征函数$f$在\textquotedbl 区间\textquotedbl 上和在\textquotedbl 一点\textquotedbl 上的振幅.

\paragraph{命题3.4.10}

$f(x)$在$x_{0}$处连续的充要条件是$\omega_{f}(x_{0})=0$.

\paragraph{命题3.4.11}

$f(x)$在$I$中一致连续的充要条件是
\[
\lim_{r\to0+}\omega_{f}(r)=0,
\]
其中
\[
\omega_{f}(r)=\sup\left\{ \left|f(x')-f(x'')\right|:\forall x',x''\in I,\ \left|x'-x''\right|\le r\right\} .
\]


\subsection{连续函数的积分}

\paragraph{积分定义:}

设$f\in C[a,b]$, 直线$x=a$, $x=b$, $y=0$与曲线$f(x)$的图像在平面上所围成图形的面积用$\int_{a}^{b}f(x)\ud x$来表示,
称为$f$在$[a,b]$上的积分.

\paragraph{命题3.5.1}

设$f\in C[a,b]$, $f_{n}(x)$是分段线性函数, 是将$[a,b]$进行$n$等分, 分点$x_{i}=a+\frac{i}{n}(b-a)$,
在$x\in[x_{i-1},x_{i}]$时
\[
f_{n}(x)=l_{i}(x)=f(x_{i-1})+\frac{f(x_{i})-f(x_{i-1})}{x_{i}-x_{i-1}}(x-x_{i-1}).
\]
则$\forall\epsilon>0$, $\exists N=N(\epsilon)$, 当$n>N$时, 
\[
\left|f(x)-f_{n}(x)\right|<\epsilon,\quad\forall x\in[a,b].
\]
即$f_{n}(x)\rightrightarrows f(x)$. 进而
\begin{align*}
	\int_{a}^{b}f(x)\ud x & =\lim_{n\to\infty}\int_{a}^{b}f_{n}(x)\ud x\\
	& =\lim_{n\to\infty}\sum_{i=1}^{n}\frac{1}{2}\left[f(x_{i-1})+f(x_{i})\right]\cdot\frac{b-a}{n}\\
	& =\lim_{n\to\infty}\sum_{i=1}^{n}f(x_{i})\Delta x_{i}.
\end{align*}


\paragraph{积分的基本性质}

约定$\int_{a}^{a}f(x)\ud x=0$, $\int_{b}^{a}f(x)\ud x=-\int_{a}^{b}f(x)\ud x$.

(1) (线性性) $f,g\in C[a,b]$, $\alpha,\beta\in\RR$, 则
\[
\int_{a}^{b}\left(\alpha f(x)+\beta g(x)\right)\ud x=\alpha\int_{a}^{b}f(x)\ud x+\beta\int_{a}^{b}g(x)\ud x.
\]

(2) 若$\left|f(x)\right|\le M$, $\forall x\in[a,b]$, 则
\[
\left|\int_{a}^{b}f(x)\ud x\right|\le M(b-a).
\]

(3) (保序性) 若$f\ge g$, $f,g\in C[a,b]$, 则$\int_{a}^{b}f\ge\int_{a}^{b}g$.
特别地, 
\[
\left|\int_{a}^{b}f(x)\ud x\right|\le\int_{a}^{b}\left|f(x)\right|\ud x.
\]

(4) (区间可加性) 设$f\in C(I)$, $a,b,c\in I$, 则
\[
\int_{a}^{c}f(x)\ud x=\int_{a}^{b}f(x)\ud x+\int_{b}^{c}f(x)\ud x.
\]

(5) 若$f(x)\in C[a,b]$, 非负, 则$\int_{a}^{b}f(x)\ge0$, 等号仅当$f\equiv0$时取到.

\paragraph{例3.5.4}

设$f\in C[a,b]$, $c\in[a,b]$, 定义$F(x)=\int_{c}^{x}f(t)\ud t$, $x\in[a,b]$,
则$F$是Lipschitz函数.

\paragraph{证明:}

\[
\left|F(x_{2})-F(x_{1})\right|\le M\int_{x_{1}}^{x_{2}}\ud t.
\]


\paragraph{命题3.5.2 (积分中值定理)}

设$f,g\in C[a,b]$, 若$g$不变号, 则$\exists\xi\in[a,b]$, s.t.
\[
\int_{a}^{b}f(x)g(x)\ud x=f(\xi)\int_{a}^{b}g(x)\ud x.
\]


\paragraph{证明:}

用连续函数介值定理.

\paragraph{例3.5.10}

设$f\in C[0,a]$, 定义
\[
f_{0}(x)=f(x),\quad f_{n}(x)=\int_{0}^{x}f_{n-1}(t)\ud t,\quad n=1,2,\cdots.
\]
证明: 存在$\xi=\xi_{n,x}\in[0,x]$, s.t. $f_{n}(x)=f(\xi)\frac{x^{n}}{n!}$.

\paragraph{证明:}

\[
m\frac{x^{n}}{n!}\le\int_{0}^{x}f_{n-1}(t)\ud t\le M\frac{x^{n}}{n!},
\]
用介值定理.

\paragraph{例3.5.12}

求连续函数$f$满足
\[
f(x+y)=f(x)+f(y),\qquad\forall x,y\in\RR.
\]


\paragraph{证明:}

此题的条件可以弱化为$f$是可积函数. 取积分
\[
\int_{0}^{y}f(x+t)\ud t=\int_{x}^{x+y}f(t)\ud t=\int_{0}^{y}f(x)\ud t+\int_{0}^{y}f(t)\ud t,
\]
所以
\[
\int_{0}^{x+y}f(t)\ud t=yf(x)+\int_{0}^{x}f(t)\ud t+\int_{0}^{y}f(t)\ud t.
\]
交换$x,y$的位置即得$yf(x)=xf(y)$, 再取$y=1$.

\paragraph{例3.5.15}

设$f\in C[a,b]$, $g$是周期为$T$的连续函数, 则
\[
\lim_{n\to\infty}\int_{a}^{b}f(x)g(nx)\ud x=\frac{1}{T}\int_{a}^{b}f(x)\ud x\int_{0}^{T}g(x)\ud x.
\]


\subsection{作业}

16. 设 $0<a<b,$ $a_{1}=a$, $b_{1}=b$. 

(1) 如果 $a_{n+1}=\sqrt{a_{n}b_{n}}$, $b_{n+1}=\frac{1}{2}\left(a_{n}+b_{n}\right)$,
则 $\left\{ a_{n}\right\} $ 和 $\left\{ b_{n}\right\} $ 收敛于同一极限. 

(2) 如果 $a_{n+1}=\frac{1}{2}\left(a_{n}+b_{n}\right),b_{n+1}=\frac{2a_{n}b_{n}}{a_{n}+b_{n}}$,
则 $\left\{ a_{n}\right\} $ 和 $\left\{ b_{n}\right\} $ 收敛于同一极限.

pf. (1). 因为$a_{1},b_{1}>0$, 由$a_{n},b_{n}>0$可以得到, 
\[
a_{n+1}=\sqrt{a_{n}b_{n}}>0,\ b_{n+1}=\frac{a_{n}+b_{n}}{2}>0.
\]
即由归纳法, $\left\{ a_{n}\right\} $, $\left\{ b_{n}\right\} $均是正数数列.

由均值不等式, 
\[
a_{n+1}=\sqrt{a_{n}b_{n}}\le\frac{a_{n}+b_{n}}{2}=b_{n+1},\quad n\ge1.
\]
又由$a_{1}=a<b=b_{1}$, 故对于任意的$n\in\NN$, 有$a_{n}\le b_{n}$. 于是
\[
b_{n+1}=\frac{a_{n}+b_{n}}{2}\le\frac{b_{n}+b_{n}}{2}=b_{n},
\]
知$\left\{ b_{n}\right\} $单调递减; 同理, 由
\[
a_{n+1}=\sqrt{a_{n}b_{n}}\ge\sqrt{a_{n}a_{n}}=a_{n},
\]
知$\left\{ a_{n}\right\} $单调上升. 于是
\[
a_{1}\le a_{2}\le\cdots\le a_{n}\le b_{n}\le b_{n-1}\le\cdots\le b_{1}.
\]
这说明$\left\{ a_{n}\right\} $是单调上升的有界序列, 上界是$b_{1}$; $\left\{ b_{n}\right\} $是单调下降的有界序列,
下界是$a_{1}$.

由于单调有界序列必然收敛, 可设$a_{n}\to a$, $b_{n}\to b$, ($n\to\infty$). 对$b_{n+1}=\frac{a_{n}+b_{n}}{2}$两边同时取极限$\lim_{n\to\infty}$得到
\[
b=\frac{a+b}{2}\Longleftrightarrow a=b.
\]
即$\left\{ a_{n}\right\} $和$\left\{ b_{n}\right\} $收敛于同一极限.

13. 设 $f(x)$ 是 $(a,b)$ 中定义的无第二类间断点的函数, 如果对任意的两点 $x$, $y\in(a,b)$,
均有 
\[
f\left(\frac{x+y}{2}\right)\leqslant\frac{1}{2}[f(x)+f(y)],
\]
则 $f$ 为 $(a,b)$ 中的连续函数.

pf. 用反证法, 若$f(x)$在$(a,b)$上不连续, 则存在$x_{0}\in(a,b)$, 使得$f$在$x_{0}$处为第一类间断点(因为$f$没有第二类间断点).

若$f$在$x_{0}$处为跳跃间断点, 则$f(x_{0}-0)$和$f(x_{0}+0)$均存在且不相等, 不妨设$f(x_{0}-0)<f(x_{0}+0)$,
取$\epsilon<\frac{f(x_{0}+0)-f(x_{0}-0)}{4}$, 则存在$\delta>0$, 使得对于任何$x\in(x_{0}-\delta,x_{0})$,
\[
f(x_{0}-0)-\epsilon<f(x)<f(x_{0}-0)+\epsilon;
\]
且对于任何$y\in(x_{0},x_{0}+\delta)$,
\[
f(x_{0}+0)+\epsilon>f(y)>f(x_{0}+0)-\epsilon.
\]
特别的
\[
f\left(x_{0}+\frac{\delta}{8}\right)=f\left(\frac{x_{0}-\frac{\delta}{4}+x_{0}+\frac{\delta}{2}}{2}\right)>f(x_{0}+0)-\epsilon>\frac{3f(x_{0}+0)+f(x_{0}-0)}{4}.
\]
而
\[
\frac{1}{2}\left(f\left(x_{0}-\frac{\delta}{4}\right)+f\left(x_{0}+\frac{\delta}{2}\right)\right)<\frac{1}{2}\left(f(x_{0}-0)+\epsilon+f(x_{0}+0)+\epsilon\right)\le\frac{3f(x_{0}+0)+f(x_{0}-0)}{4},
\]
这与
\[
f\left(\frac{x_{0}-\frac{\delta}{4}+x_{0}+\frac{\delta}{2}}{2}\right)\le\frac{1}{2}\left(f\left(x_{0}-\frac{\delta}{4}\right)+f\left(x_{0}+\frac{\delta}{2}\right)\right)
\]
矛盾.

若$f$在$x_{0}$处是跳跃间断点, 则$f(x_{0}\pm0)$均存在且不等于$f(x_{0})$. 取$x=x_{0}-t$,
$y=x_{0}+t$, 并令$t\to0$, 则
\[
f\left(\frac{x+y}{2}\right)=f(x_{0})\le\frac{f(x_{0}+t)+f(x_{0}-t)}{2}\to\lim_{x\to x_{0}}f(x),\qquad t\to0.
\]
即$f(x_{0})<\lim_{x\to x_{0}}f(x)$. 取$x=x_{0}$, $y=x_{0}+t$, 并令$t\to0+$,
则
\[
f\left(\frac{t}{2}+x_{0}\right)=f\left(\frac{x+y}{2}\right)\le\frac{f(x_{0})+f(x_{0}+t)}{2}\Longrightarrow\lim_{x\to x_{0}}f(x)\le\frac{f(x_{0})+\lim_{x\to x_{0}}f(x)}{2}\Longrightarrow\lim_{x\to x_{0}}f(x)\le f(x_{0}).
\]
又矛盾.

15. 设 $f$ 为 $\mathbb{R}$ 上的连续函数, 如果对任意的 $x,y\in\mathbb{R}$ 均有 $f(x+y)=f(x)f(y)$,
则要么 $f$ 恒为零, 要么存在常数 $a>0$, 使得 $f(x)=a^{x}$, $\forall x\in\mathbb{R}$.

pf. 若$f$不恒为零, 则由于$f$连续, $f$在$\RR$上保持符号. 不然则由连续函数的介值定理, 存在$x_{0}$使得$f(x_{0})=0$,
则对于任意的$x\in\RR$,
\[
f(x)=f(x_{0})f(x-x_{0})\equiv0.
\]
与$f$不恒为零矛盾.

$f$不能恒为负值, 不然$f(2x)=f(x)^{2}\Longrightarrow f(2x)>0$导致$f$在$\RR$上不再保持符号,
与上述推导矛盾. 所以$f$在$\RR$恒为正.

取$g(x)=\ln f(x)$, 由于$f$的连续性, $g(x)$也是$\RR$上的连续函数. 并有
\[
g(x+y)=\ln f(x+y)=\ln\left(f(x)f(y)\right)=\ln f(x)+\ln f(y)=g(x)+g(y).
\]
由于$g$连续, 所以$g(x)=g(1)x$. 令$c=g(1)$, $a=f(1)>0$, 即有$\ln f(x)=cx=x\ln f(1)=\ln a^{x}$,
$f(x)=a^{x}$. (14题的结论是经典结论, 任何时候都可以直接拿来用)

1. 设 $f$ 在 $[a,+\infty)$ 中连续, 且 $\lim_{x\rightarrow+\infty}f(x)=A$,
则 $f$ 在 $[a,+\infty)$ 中有界, 且最大值和最小值中的一个必定能被 $f$ 达到.

pf1. 考虑函数
\[
g(t)\coloneqq\sup_{x\ge t}f(x),
\]
则$g(t)$是$[a,+\infty)$上连续的递减函数. 且
\[
\lim_{t\to\infty}g(t)=\limsup_{x\to\infty}f(x)=\alpha.
\]
所以$g(t)\ge\alpha$.

如果存在$t$使得$g(t)>\alpha$, 也即$\sup_{x\ge t}f(x)>\alpha$, 取$\epsilon<\sup_{x\ge t}f(x)-\alpha$,
则有$X>t$, s.t. $\forall x>X$, 
\[
\left|f(x)-\alpha\right|<\epsilon\Longrightarrow f(x)<\alpha+\epsilon\Longrightarrow\sup_{x>X}f(x)\le\alpha+\epsilon<\sup_{x\ge t}f(x),
\]
这说明$f(x)$在$[t,\infty)$上的上确界不可能在$[X,\infty)$上取到, 即
\[
\sup_{x\ge t}f(x)=\sup_{t\le x\le X}f(x)\Longrightarrow\sup_{x\ge a}f(x)=\sup_{a\le x\le X}f(x)
\]
即$f(x)$在$[a,X]$上存在最大值, 也即$f(x)$在$[a,\infty)$上存在最大值.

若对于任何$t$, $g(t)=\alpha$. 也即
\[
\sup_{x\ge t}f(x)=\alpha,\quad\forall t\ge a.
\]
当$f(x)$不为常数时, 则必然有$f(x_{0})<\alpha$. 取$\epsilon<\alpha-f(x_{0})$,
则有$X>x_{0}$, s.t. $\forall x>X$
\[
\left|f(x)-\alpha\right|<\epsilon\Longrightarrow f(x)>\alpha-\epsilon>f(x_{0}).
\]
所以$f(x)$必然在$[a,X]$中取到最小值.

pf2. $f(x)$在$[a,\infty)$不恒为常数.

因为$f(x)\to\alpha$, $x\to\infty$, 则有点列$\left\{ x_{n}\right\} $,
使得$f(x_{n})\to\alpha$, $f(x_{1})\ne\alpha$(因为$f$不恒为常数).

因为$f(x_{1})\ne\alpha$, 则取$\epsilon<\left|f(x_{1})-\alpha\right|$,
有$X>x_{1}$, s.t. $\forall x>X$,
\[
\left|f(x)-\alpha\right|<\epsilon<\left|f(x_{1})-\alpha\right|,
\]
这意味着在$[a,X]$内存在点$x_{1}$的函数值$f(x_{1})$远离$(\alpha-\epsilon,\alpha+\epsilon)$.

当$f(x_{1})$>$\alpha$时, $f(x)$在$[a,X]$上存在最大值$M\ge f(x_{1})$, 而在$(X,\infty)$,
$f(x)\le\alpha+\epsilon<f(x_{1})\le M$. 最大值存在性得证.

当$f(x_{1})$<$\alpha$时, $f(x)$在$[a,X]$上存在最小值$m\le f(x_{1})$, 而在$(X,\infty)$,
$f(x)\ge\alpha-\epsilon>f(x_{1})\ge m$. 最小值存在性得证.

3. 设$b,\alpha>0$, 求积分$\int_{0}^{b}x^{\alpha}\ud x$, 利用积分计算极限
\[
\lim_{n\to\infty}\frac{1^{\alpha}+2^{\alpha}+\cdots+n^{\alpha}}{n^{\alpha+1}}.
\]

8. 设$f\in C[a,b]$, 如果对于任意的$g\in\left\{ g\in C[a,b]:g(a)=g(b)=0\right\} $,
均有$\int_{a}^{b}f(x)g(x)\ud x=0$, 则$f\equiv0$.

此题类比变分基本定理.

9. 设$f\in C[a,b]$, 如果对于任意的$g\in\left\{ g\in C[a,b]:\int_{a}^{b}g(x)\ud x=0\right\} $,
均有$\int_{a}^{b}f(x)g(x)\ud x=0$, 则$f=C$为常函数.

提示: 设$f$的平均值为$C$, 考虑$g=f-C$和$g^{2}$的积分.

12. 设$f\in C[a,b]$, 则
\[
\lim_{n\to+\infty}\left(\int_{a}^{b}\left|f(x)\right|^{n}\ud x\right)^{1/n}=\max_{x\in[a,b]}\left|f(x)\right|.
\]

14. 设$f,g\in C[a,b]$, 且$f,g>0$, 证明
\[
\lim_{n\to\infty}\frac{\int_{a}^{b}f^{n+1}(x)g(x)\ud x}{\int_{a}^{b}f^{n}(x)g(x)\ud x}=\max_{x\in[a,b]}f(x).
\]

17. 设$f(x)\in C[0,+\infty)$严格单调递增, 则
\[
F(x)=\begin{cases}
	\frac{1}{x}\int_{0}^{x}f(t)\ud t, & x>0,\\
	f(0), & x=0
\end{cases}
\]
也是严格单调递增连续函数.

18. 设$f(x)\in C[a,b]$, $f>0$, 则
\[
\lim_{r\to0+}\left(\frac{1}{b-a}\int_{a}^{b}f^{r}(x)\ud x\right)^{1/r}=\exp\left(\frac{1}{b-a}\int_{a}^{b}\ln f(x)\ud x\right),
\]
并用H\"{o}lder不等式说明上式左端关于$r$单调递增.
\section{微分及其逆运算}
\subsection{可导与可微}

研究函数的局部性质

\paragraph{定义4.1.1 (导数)}

设$f$在$x_{0}$附近有定义, 如果极限
\[
\lim_{x\to x_{0}}\frac{f(x)-f(x_{0})}{x-x_{0}}
\]
存在且有限, 则称$f$在$x_{0}$处可导, 极限称为$f$在$x_{0}$处的导数, 记为$f'(x_{0})$,
$\frac{\ud f}{\ud x}(x_{0})$.

用$\epsilon-\delta$语言表述.

\paragraph{命题4.1.1}

设$f$在$x_{0}$处可导, 则$f$在$x_{0}$处连续.

\paragraph{命题4.1.2 (导数的运算法则)}

设$f,g$在$x$处可导, 则$fg$也在$x$处可导; 对于任意的常数$\alpha,\beta$, $\alpha f+\beta g$也在$x$处可导.
且

(1). $(\alpha f+\beta g)'=\alpha f'+\beta g'$, (线性性);

(2). $(fg)'=f'g+fg'$. (导性).

\paragraph{推论4.1.3}

设$f,g$在$x_{0}$处可导, $g(x_{0})\ne0$, 则$\frac{f}{g}$在$x_{0}$处也可导,
且
\[
\left(\frac{f}{g}\right)'=\frac{f'g-fg'}{g^{2}}.
\]

可以用导数表述曲线在一点处的切线和法线, 事实上仅仅是用导数给出了切线和法线的定义, 属于先有导数的概念才有的切线和法线的概念.

\paragraph{定义4.1.2 (微分)}

设$f$在点$x_{0}$附近有定义, 如果存在常数$A$, 使得
\[
f(x)=f(x_{0})+A(x-x_{0})+o(x-x_{0}),
\]
则称$f$在$x_{0}$处可微, 线性映射$x\mapsto Ax$称为$f$在$x_{0}$处的微分, 记为$df(x_{0})$.

\paragraph{命题4.1.4}

设$f$在$x_{0}$附近有定义, 则$f$在$x_{0}$处可导当且仅当$f$在$x_{0}$处可微, 且微分的斜率就是导数$f'(x_{0})$.

可导对应函数在一点差商存在极限, 可微对应函数的局部线性化, 产生线性变换的主项. $f$在$x$处的微分是一个斜率为$f'(x)$的线性映射,
当$x$变化时, 线性映射也变化, 即$x\mapsto df(x)$是一个新的映射, 记为$df$, 称为$f$的外微分或全微分. 

$x$的全微分$dx$把任意点$x$映为$x$处的恒等映射. 

因为$df(x)$和$dx$都是线性变换, 所以有$df=f'(x)dx$.

把形如$fdx$的表达式($f$为函数)称为$1$次微分形式.

可以把微分运算看做升维运算, 把$x$和$dx$看做两个独立的变量, $dx$就是$\Delta x$, 它与$x$的选取无关.
$df$把单变量函数$f$映射为二元函数$df(x)=f'(x)dx$.

\paragraph{命题4.1.5 (链式法则)}

设$g$在$x_{0}$处可导, $f$在$g(x_{0})$处可导, 则复合函数$f\circ g=f(g)$在$x_{0}$处可导,
且
\[
\left(f(g)\right)'(x_{0})=f'\left(g(x_{0})\right)g'(x_{0}).
\]

证明依赖于下式
\[
\begin{aligned}f(g(x)) & =f\left(g\left(x_{0}\right)\right)+f^{\prime}\left(g\left(x_{0}\right)\right)\left(g(x)-g\left(x_{0}\right)\right)+o\left(g(x)-g\left(x_{0}\right)\right)\\
	& =f\left(g\left(x_{0}\right)\right)+f^{\prime}\left(g\left(x_{0}\right)\right)g^{\prime}\left(x_{0}\right)\left(x-x_{0}\right)+f^{\prime}\left(g\left(x_{0}\right)\right)o\left(x-x_{0}\right)+o\left(x-x_{0}\right)\\
	& =f\left(g\left(x_{0}\right)\right)+f^{\prime}\left(g\left(x_{0}\right)\right)g^{\prime}\left(x_{0}\right)\left(x-x_{0}\right)+o\left(x-x_{0}\right)
\end{aligned}
\]


\paragraph{命题4.1.6 (反函数求导法则)}

设$f$在$x_{0}$附近有定义, 且反函数为$g$. 若$f$在$x_{0}$处可导, 且导数非零, 则$g$在$y_{0}=f(x_{0})$处可导,
且
\[
g'(y_{0})=\frac{1}{f'(x_{0})}.
\]

这个定理并没有要求$f$在$x_{0}$附近每点上都连续. 导数$f'(x_{0})\ne0$的条件不能省掉, 否则考虑$f(x)=x^{3}$.

\paragraph{命题4.1.8}

设$f,g$可微, 则

(1) $d(\alpha f+\beta g)=\alpha df+\beta dg$, 其中$\alpha,\beta$为常数;

(2) $d(fg)=gdf+fdg$;

(3) $d(f/g)=\frac{gdf-fdg}{g^{2}}$, 其中$g\ne0$.

\paragraph{命题4.1.9}

设$f,g$均可微, 且复合函数$f(g)$有定义, 则
\[
d\left(f(g)\right)=f'(g)dg.
\]


\subsection{高阶导数}

本节没有太多复杂的知识点, 仅做一些结论的罗列.

\paragraph{定义4.2.1 (高阶导数)}

设$f$在$x_{0}$附近可导, 如果导数$f'$在$x_{0}$处仍可导, 则称$f$在$x_{0}$处2阶可导.
记为
\[
f''(x_{0})=(f')'(x_{0}),
\]
并称为$f$在$x_{0}$处的2阶导数. 

一般地, 如果$f$在$x_{0}$附近$n$ ($n\ge1$) 阶可导, 且$n$阶导函数$f^{(n)}$在$x_{0}$处可导,
则称$f$在$x_{0}$处$n+1$阶可导, 记为
\[
f^{(n+1)}(x_{0})=\left(f^{(n)}\right)'(x_{0}),
\]
称为$f$的$n+1$阶导数.

\paragraph{注:}

$f$的2阶导数要求$f'$在$x_{0}$附近有定义, 也就是对于$x_{0}$附近的$x$能够计算$f'(x)$的值.
另有一种用差分方法定义的二阶导数, 比如
\[
\lim_{h\to0}\frac{f(x+h)-2f(x)+f(x-h)}{h^{2}}
\]
它避免了计算$f'(x)$, 而且允许一阶导数不存在, 而仅存在二阶导数. 一般用于推广导数的概念.

\paragraph{定义4.2.2}

若$f$在区间$I$上的每点都$n$阶可导, 则称$f$在$I$中$n$阶可导; 如果$f$可导, 且导函数$f'$连续,
则称$f$(1阶)连续可导, 记为$f\in C^{1}(I)$; 一般$n$阶连续可导记为$f\in C^{n}(I)$.
如果$f$有任意阶导数, 则称$f$是光滑的, 记为$f\in C^{\infty}(I)$.

\paragraph{例4.2.3}

可微函数的导函数不一定连续.

尽管导函数连续性丧失, 但仍有介值定理成立, 也就是Darboux介值定理.

\paragraph{例4.2.4}

设$k=1,2,\cdots$, 则函数
\[
f(x)=\begin{cases}
	x^{2k+1}\sin\frac{1}{x}, & x\neq0\\
	0, & x=0
\end{cases}
\]
有$f\in C^{k}\setminus C^{k+1}$.

\paragraph{例4.2.5}

证明函数
\[
f(x)=\begin{cases}
	0, & x\leqslant0\\
	e^{-\frac{1}{x}}, & x>0
\end{cases}
\]
是光滑函数.

\paragraph{命题4.2.1}

设$f,g$均为$n$阶可导函数, 则

(1) $(\alpha f+\beta g)^{(n)}=\alpha f^{(n)}+\beta g^{(n)}$, $\forall\alpha,\beta\in\RR$;

(2) (Leibniz) 
\[
(fg)^{(n)}=\sum_{k=0}^{n}\binom{n}{k}f^{(n-k)}g^{(k)}.
\]


\subsection{不定积分}

\paragraph{命题4.3.1}

设$f$为区间$I$上的可微函数, 则$f'=0$当且仅当$f=C$.

此定理使用中值定理证明最为简洁, 书中使用的方法可以归为极端原理.

\paragraph{定义4.3.1 (原函数)}

方程$F'(x)=f(x)$的一个可微解$F$称为函数$f$的一个原函数.

\paragraph{定义4.3.2 (不定积分)}

设函数$f$在区间$I$上有原函数, 用记号$\int f(x)\ud x$表示$f$的原函数的一般表达式, 则
\[
\int f(x)dx=F(x)+C,\quad x\in I,
\]
其中$C$为常数.

\paragraph{定理4.3.2 (Newton-Leibniz)}

区间$I$中的连续函数都有原函数. 设$f$连续, $a\in I$, 则
\[
F(x)=\int_{a}^{x}f(t)\ud t,\quad x\in I
\]
是$f$的一个原函数.

需要注意Darboux介值定理, 将来会遇到有间断点的可积函数, 其变限积分在间断点处不可微, 所以不能形成一般非连续函数的原函数.

此定理称为微积分基本定理, 它有其它形式:

设$f\in C(I)$, $F$为$f$的任一原函数, 则存在常数$C$, 使得, $F(x)=\int_{a}^{x}f(t)dt+C$,
所以
\[
\int_{a}^{b}f(x)dx=F(b)-F(a)=F\mid_{a}^{b}.
\]

另有一种表述是当$G$连续可微时, 
\[
\int_{a}^{b}G'(x)dx=G(b)-G(a)=G\mid_{a}^{b}.
\]
上式对于$C^{1}$函数总是对的, 但需要注意Volterra函数, 它在定义的区间上处处可导, 且导函数有界, 但导函数不可定积分.

\paragraph{命题4.3.3 (不定积分的线性性质)}

设$f,g$在区间$I$上均有原函数, 则
\[
\int[\alpha f(x)+\beta g(x)]dx=\alpha\int f(x)dx+\beta\int g(x)dx,
\]
其中$\alpha,\beta$为常数.

\paragraph{命题4.3.4}

设$f$的原函数为$F$, 若$f$可逆, 且$g=f^{-1}$, 则
\[
\int g(x)dx=xg(x)-F(g(x))+C.
\]


\subsection{积分的计算}

\paragraph{命题4.4.1 (换元积分法, 变量替换法)}

设$f(u)$是区间$J$上有定义的函数, $u=\phi(x)$是区间$I$中的可微函数, 且$\phi(I)\subset J$.

(1) 设$f$在$J$上的原函数是$F$, 则$F(\phi)$是$f(\phi)\phi'$在区间$I$上的原函数,
即
\[
\int f(\phi(x))\phi^{\prime}(x)dx=\int f(u)du+C=F(\phi(x))+C;
\]

(2) 设$\phi$可逆, 且其逆可微, $\phi(I)=J$. 如果$f(\phi(x))\phi'(x)$有原函数$G$,
则$f$有原函数$G(\phi^{-1}(u))$, 即
\[
\int f(u)du=G(\phi^{-1}(u))+C.
\]


\paragraph{命题4.4.2 (分部积分法)}

设$u(x),v(x)$在区间$I$中可微, 若$u'(x)v(x)$有原函数, 则$u(x)v'(x)$也有原函数, 且
\[
\int u(x)v'(x)dx=u(x)v(x)-\int u'(x)v(x)dx.
\]


\paragraph{例4.4.10}

设$a\ne0$, 求不定积分$I=\int\ue^{ax}\cos bx\ud x$和$J=\int\ue^{ax}\sin bx\ud x$.

\paragraph{例4.4.18}

求不定积分
\[
\int\frac{1-r^{2}}{1-2r\cos x+r^{2}}dx,\qquad0<r<1.
\]

不能用初等函数表述的不定积分:

\[
e^{\pm x^{2}},\sin(x^{2}),\cos(x^{2}),\frac{\sin x}{x},\frac{\cos x}{x},\sqrt{1-k^{2}\sin^{2}x},\frac{1}{\sqrt{1-k^{2}\sin^{2}x}},\ (0<k<1).
\]


\subsection{作业}

19. 设$a_{ij}(x)$均为可导函数, 求行列式函数$\mathrm{det}\left(a_{ij}(x)\right)_{n\times n}$的导数.

20. Riemann函数$R(x)$处处不可导.

11. 通过对 $(1-x)^{n}$ 求导并利用二项式定理证明等式 
\[
\sum_{k=0}^{n}(-1)^{k}C_{n}^{k}k^{m}=\begin{cases}
	0, & m=0,1,\cdots,n-1,\\
	(-1)^{n}n!, & m=n.
\end{cases}
\]

10. 求不定积分的递推公式 ($a\ne0$):
\[
I_{n}=\int\frac{dx}{(ax^{2}+bx+c)^{n}}.
\]

11. 设$a,b>0$, 求不定积分的递推公式:
\[
I_{mn}=\int\frac{dx}{(x+a)^{m}(x+b)^{n}}.
\]

7. 设$f$在$(0,+\infty)$上可导, 且
\[
2f(x)=f(x^{2}),\quad\forall x>0.
\]
证明$f(x)=c\ln x$.

hint: 设$g(x)=f(\ue^{x})$. 题目条件可以弱化为$f$仅在$x=1$处可导. 
\[
g(x)=\frac{g\left(\frac{x}{2^{n}}\right)}{x/2^{n}}x=g'(0)x
\]

可导性不能省去, 否则考虑$\max\left\{ 0,c\ln x\right\} $作为函数的另一个解.

\subsection{简单的微分方程}

例4.5.4 微分方程能够解出通解的表达式通常和朗斯基行列式有关.
\section{微分中值定理和Taylor展开}
\subsection{函数极值}

\paragraph{定义5.1.1 (极值点)}

设$f$定义在$I$上, $x_{0}\in I$, 若存在$\delta>0$, s.t.
\[
f(x)\ge f(x_{0}),\qquad\forall x\in(x_{0}-\delta,x_{0}+\delta)\cap I,
\]
则称$x_{0}$是$f$在$I$上的极小值点, $f(x_{0})$称为极小值.

若$x_{0}\in I$, 且$\forall x\in I$, $f(x)\ge f(x_{0})$, 则称$x_{0}$是$f$在区间$I$上的最小值点,
$f(x_{0})$称为函数$f$在区间$I$上的最小值.

\paragraph{定理5.1.1 (Fermat定理)}

设$x_{0}$是$f$在$I$上的极值点, 且$x_{0}$是内点, 若$f$在$x_{0}$处可导, 则$f'(x_{0})=0$.

\paragraph{注:}

由于极值点定义是在$(x_{0}-\delta,x_{0}+\delta)\cap I$上给出的, 所以以上定理要加上$x_{0}$是内点.

\paragraph{证明:}

使用极限的保号性, 判断
\[
f'(x_{0})=\lim_{x\to x_{0}^{-}}\frac{f(x)-f(x_{0})}{x-x_{0}}=\lim_{x\to x_{0}^{+}}\frac{f(x)-f(x_{0})}{x-x_{0}}
\]
的符号.

满足$f'(x_{0})=0$的点称为$f$的驻点, 临界点.

若$f'(x_{0})\ge0$, 则$\exists\delta>0$, s.t. $\forall x\in(x_{0}-\delta,x_{0}+\delta)$,
有$(x-x_{0})(f(x)-f(x_{0}))\ge0$, 但$f$在$x_{0}$附近不单调.

\paragraph{定理5.1.2 (Darboux)}

设$f$为$[a,b]$上的可导函数, 则$f'$可以取到$f'_{+}(a)$和$f'_{-}(b)$之间的任意值.

\paragraph{证明:}

设$k$介于$f'_{+}(a)$与$f'_{-}(b)$之间, 定义$g(x)=f(x)-kx$, 则
\[
g'_{+}(a)g'_{-}(b)=(f'_{+}(a)-k)(f'_{-}(b)-k)\le0.
\]
若上式等于零, 命题显然. 若上式小于零, 不妨设$g'_{+}(a)>0$, 此时$x=a$不是最大值点; 并有$g'_{-}(b)<0$,
此时$x=b$不是最大值点.

从而$g(x)$只能在$[a,b]$内取到最大值, 由Fermat定理, 存在$\xi\in(a,b)$, s.t. $g'(\xi)=0$.

\paragraph{注:}

导函数有介值定理, 但导函数可以不连续, 比如$x^{2}\sin\frac{1}{x}$, $x\in[-\pi,\pi]$.

Darboux定理说明, 若$g'$在任何点处不为零, 则$g'$不变号.

Darboux定理的使用条件必须是区间内每点处可导.

\paragraph{例5.3.4}

设$f(x)$在$\RR$上二阶可导, 若$f$有界, 证明: $\exists\xi\in\RR$, s.t. $f''(\xi)=0$.

\paragraph{证明:}

(反证法), $\forall x\in\RR$, $f''(x)\ne0$, 由Darboux定理, $f''$不变号, 从而$f'$单调.

不妨设$f'$单调上升, 取$x_{0}\in\RR$, $f'(x_{0})\ne0$. 因为

\[
f'(x_{0})>0\Longrightarrow f(x)-f(x_{0})=\int_{x_{0}}^{x}f'(t)\ud t\ge f'(x_{0})(x-x_{0})\to+\infty,\quad x\to+\infty,
\]
与$f$有界矛盾; 而且
\[
f'(x_{0})<0\Longrightarrow f(x_{0})-f(x)=\int_{x}^{x_{0}}f'(t)\ud t\le f'(x_{0})(x_{0}-x)\to-\infty,\quad x\to-\infty,
\]
也与$f$有界矛盾.

\subsection{微分中值定理}

\paragraph{定理5.2.1 (Rolle)}

设$f\in C[a,b]$, 在$(a,b)$上可微, 且$f(a)=f(b)$, 则存在$\xi\in(a,b)$, s.t.
$f'(\xi)=0$.

\paragraph{定理5.2.2 (Lagrange)}

设$f\in C[a,b]$, 在$(a,b)$上可微, 则存在$\xi\in(a,b)$, s.t.
\[
f'(\xi)=\frac{f(b)-f(a)}{b-a}.
\]

证明是构造性的, 对
\[
F(x)=f(x)-\left[f(a)+\frac{f(b)-f(a)}{b-a}(x-a)\right]
\]
使用Rolle定理.

\paragraph{定理5.2.3 (Cauchy)}

设$f,g\in C[a,b]$, 在$(a,b)$上可微, 且$\forall x\in(a,b)$, $g'(x)\ne0$,
则
\[
\exists\xi\in(a,b),\ s.t.\ \frac{f(b)-f(a)}{g(b)-g(a)}=\frac{f'(\xi)}{g'(\xi)}.
\]


\paragraph{证明:}

($g(a)\ne g(b)$), 对
\[
F(x)=f(x)-\left[f(a)+\frac{f(b)-f(a)}{g(b)-g(a)}(g(x)-g(a))\right]
\]
用Rolle定理.

\paragraph{几何意义:}

定义参数曲线$\overrightarrow{r}(t)=(g(t),f(t))$, $A=\overrightarrow{r}(a)$,
$B=\overrightarrow{r}(b)$. 则$\frac{f(b)-f(a)}{g(b)-g(a)}$表示直线$\ell_{AB}$的斜率.
Cauchy定理指出, $\exists\xi$使得$\overrightarrow{r}(\xi)$处的切线方向$\overrightarrow{r}'(\xi)\parallel\ell_{AB}$,
而$\overrightarrow{r}'(\xi)=(g'(\xi),f'(\xi))$, 有
\[
k_{AB}=\frac{f'(\xi)}{g'(\xi)}.
\]


\paragraph{注:}

由于$\forall x\in(a,b)$, $g'(x)\ne0$, 由Darboux介值定理$g'(x)$不变号, $g(x)$其实是单调可逆的,
这可以给出另一种证法:

取$A=g(a)$, $B=g(b)$, 不妨设$A<B$, 则$f(g^{-1}(y))\in C[A,B]$, 由复合函数求导与反函数求导法则
\[
\frac{f(b)-f(a)}{g(b)-g(a)}=\frac{f(g^{-1}(B))-f(g^{-1}(A))}{B-A}=\frac{\ud}{\ud x}f(g^{-1}(x))\mid_{x=\zeta}=f'(g^{-1}(\zeta))\cdot\frac{1}{g'(g^{-1}(\zeta))}=\frac{f'(\xi)}{g'(\xi)},
\]
其中$g(\xi)=\zeta\in[A,B]$. 

\paragraph{例5.2.4}

设$f(x)\in C[a,b]$, 在$(a,b)$二阶可导, 若$f(a)=f(b)=0$, 则对于任意的$c\in[a,b]$,
存在$\xi\in(a,b)$, s.t.
\[
f(c)=\frac{f''(\xi)}{2}(c-a)(c-b).
\]


\paragraph{证: (K值法)}

设$K$满足$f(c)=K(c-a)(c-b)$. 则$f(x)-K(x-a)(x-b)$有三个零点$a,b,c$. 故存在$\xi\in(a,b)$,
s.t. $f''(\xi)-2K=0$.

\paragraph{证法二:}

构造
\[
F(x)=f(x)-\frac{f(c)}{(c-a)(c-b)}(x-a)(x-b)
\]
也有三个零点$a,b,c$.

\paragraph{注: Lagrange插值公式}

经过$\left(x_{i},f(x_{i})\right)_{i=1}^{n}$的$n-1$次多项式有如下形式
\[
p_{n-1}(x)=\sum_{i=1}^{n}\prod_{j\ne i}\frac{(x-x_{j})}{(x_{i}-x_{j})}f(x_{i}).
\]
用K值法证明:
\[
f(x)-p_{n-1}(x)=\frac{1}{n!}f^{(n)}(\xi)\prod_{i=1}^{n}(x-x_{i}),\quad\xi\in(a,b).
\]


\paragraph{例5.2.5}

证明Legendre (勒让德)多项式$\frac{d^{n}}{dx^{n}}\left(x^{2}-1\right)^{n}$在$(-1,1)$上有$n$个不同实根,
其中$n\ge1$.

\paragraph{证明:}

多项式$\left(x^{2}-1\right)^{n}$的直到$n-1$次导数总有$\pm1$作为其零点, 用Rolle定理,
在每次求导时会多出现一个零点.

\subsection{单调函数}

\paragraph{命题5.3.2}

设$f\in C[a,b]$, 在$(a,b)$上可微, 则$f$单调当且仅当$f'$不变号.

\paragraph{证明:}

$\Longrightarrow$: 用极限的保号性; $\Longleftarrow$: 用Lagrange中值定理.

\paragraph{命题5.3.3 (反函数定理)}

设$f$为区间$I$上的可微函数, 若$f'\ne0$, $\forall x\in I$. 则$f$可逆且反函数可微.

\paragraph{证明:}

用反证法+Lagrange定理, $f$是单射, 从而可逆, 由$f$连续得到$f$单调. 并且
\[
\left(f^{-1}(y)\right)'=\frac{1}{f'(x)}.
\]


\paragraph{命题5.3.4}

设$\delta>0$, $f\in C(x_{0}-\delta,x_{0}+\delta)$, 在$(x_{0}-\delta,x_{0})\cup(x_{0},x_{0}+\delta)$上可微,
若
\[
f'(x)\le0,\ x\in(x_{0}-\delta,x_{0});\quad f'(x)\ge0,\ x\in(x_{0},x_{0}+\delta).
\]
则$x_{0}$为$f$的极小值点. 反之为极大值点.

\paragraph{命题5.3.5}

设$f$在内点$x_{0}$处二阶可导, 且$f'(x_{0})=0$, 则若$f''(x_{0})>0$, 则$x_{0}$为$f$的(严格)极小值点.

\paragraph{证明: }

有高阶导数的定义要求$f'(x)$在$x_{0}$附近可计算, 再由极限保号性, $\exists\delta$使得
\[
x\in(x_{0}-\delta,x_{0})\text{时, }f'(x)<0,
\]
\[
x\in(x_{0},x_{0}+\delta)\text{时, }f'(x)>0.
\]


\paragraph{例5.3.4}

设$f(x)$在$\RR$上二阶可导, 若$f$有界, 证明: $\exists\xi\in\RR$, s.t. $f''(\xi)=0$.

\paragraph{证明: (反证法)}

对于任意的$x\in\RR$, $f''(x)\ne0$. 由Darboux定理, 有$f''$不变号, 所以$f'$单调,
不妨设$f'$单调上升, 取$x_{0}\in\RR$, $f'(x_{0})\ne0$.

当$f'(x_{0})>0$时, $f(x)-f(x_{0})=\int_{x_{0}}^{x}f'(t)\ud t\ge f'(x_{0})(x-x_{0})\to+\infty$,
$x\to+\infty$. 与$f$有界矛盾.

当$f'(x_{0})<0$时, $f(x_{0})-f(x)=\int_{x}^{x_{0}}f'(t)\ud t\le f'(x_{0})(x_{0}-x)\to-\infty$,
$x\to-\infty$. 与$f$有界矛盾.

\subsection{凸函数}

\paragraph{定义5.4.1 (凸函数)}

设$f$在$I$上有定义, 若$\forall a,b\in I$, $a<b$, 有
\[
f(x)\le l(x)=f(a)+\frac{f(b)-f(a)}{b-a}(x-a),\quad\forall x\in[a,b].
\]
则称$f$为$I$中的凸函数; 相应的给出凹函数的定义. 若上式取严格不等号, 则对应严格凸函数.

定义中的不等式可以等价地写成
\[
f(ta+(1-t)b)\le tf(a)+(1-t)f(b),\qquad\forall t\in(0,1).
\]


\paragraph{例5.4.1}

用凸函数证明Young不等式.

\paragraph{证明:}

$e^{x}$是凸函数, $p>1$, 且$\frac{1}{p}+\frac{1}{q}=1$, 则
\[
ab=e^{\ln ab}=e^{\frac{1}{p}\ln a^{p}+\frac{1}{q}\ln b^{q}}\le\frac{1}{p}e^{\ln a^{p}}+\frac{1}{q}e^{\ln b^{q}}=\frac{a^{p}}{p}+\frac{b^{q}}{q}.
\]


\paragraph{定理5.4.1 (Jensen不等式)}

设$f$是定义在$I$上的函数, 则$f$凸当且仅当$\forall x_{i}\in I$, $\lambda_{i}\ge0$,
($i=1,2,\cdots,n$), 且$\sum_{i=1}^{n}\lambda_{i}=1$, 有
\[
f\left(\sum_{i=1}^{n}\lambda_{i}x_{i}\right)\le\sum_{i=1}^{n}\lambda_{i}f(x_{i}).
\]

对比习题3.3的13题.

13: 若$f$没有第二类间断点, 且$\forall x,y\in(a,b)$均有
\[
f\left(\frac{x+y}{2}\right)\le\frac{f(x)+f(y)}{2},
\]
则$f\in C(a,b)$, 由此可证$f$在$(a,b)$上是凸的.

这要依赖$f$的连续性和实数的完备性, 比如考虑证明$f\left(\frac{x+y+z}{3}\right)\le\frac{f(x)+f(y)+f(z)}{3}$.
但下面的证明更精巧.

\paragraph{命题5.4.3}

设$f\in C(I)$, 则$f$凸当且仅当$\forall x_{1}<x_{2}\in I$, 有$f\left(\frac{x_{1}+x_{2}}{2}\right)\le\frac{f(x_{1})+f(x_{2})}{2}$.

\paragraph{证明:}

取$a,b\in I$, 证明$f(x)$在$(a,b)$上位于$l(x)=f(a)+\frac{f(b)-f(a)}{b-a}(x-a)$之下即可.

\[
g(x)\coloneqq f(x)-l(x)\quad x\in[a,b]\Longrightarrow g(x)\in C[a,b].
\]
取$M=\max_{x\in[a,b]}g(x)=g(x_{0})$, 则当$x_{0}$靠近$a$时, $x_{0}\le\frac{a+b}{2}$,
$2x_{0}-a\in[a,b]$. 所以
\[
M=g(x_{0})=g\left(\frac{a+(2x_{0}-a)}{2}\right)\le\frac{g(a)+g(2x_{0}-a)}{2}\le M,
\]
等号成立, 故$M=g(a)=0$.

\paragraph{推论5.4.4}

设$f$在$I$上凸, 若$f$在$I$内达到最大值, 则$f$为常数.

\paragraph{证明:}

设$f$在$x_{0}$达到最大值, 则$\forall a,b\in I$, $\exists t\in(0,1)$, s.t.
\[
x_{0}=ta+(1-t)b\Longrightarrow f(x_{0})=\max f\le tf(a)+(1-t)f(b)\le\max f.
\]
等号成立. $f(x_{0})=f(a)=f(b)$.

\paragraph{命题 5.4.2 (连续性)}

设$f$在$I$中凸, 若$[a,b]\subseteq I$, $a,b\in I^{\circ}$, 则$f\in\mathrm{Lip}[a,b]$,
从而连续.

\paragraph{证明:}

取$[a,b]\subseteq[a',b']\subseteq I$, $a',b'\in I^{\circ}$. 注意使用
\[
\left(\frac{f(a)-f(a')}{a-a'}\le\right)\frac{f(y)-f(x)}{y-x}\le\frac{f(b)-f(y)}{b-y}\le\frac{f(b')-f(b)}{b'-b}.
\]


\paragraph{命题5.4.5 (导数性质)}

设$f$在$I$中凸, $x$为$I$的内点, 则$f$在$x$处左右导数存在, 且
\[
f_{-}'(x)\le f_{+}'(x).
\]


\paragraph{证明:}

使用单调有界函数的极限存在. 设$x_{0}<x_{1}<x_{2}$, 证明: $k_{01}\le k_{02}\le k_{12}$.

\paragraph{命题3.3.4}

设$f(x)$是定义在区间$I$上的单调函数, 则$f$的间断点至多可数.

由上面的命题, 凸函数的不可微点至多可数.

\paragraph{命题5.4.6}

设$f$在$I$上可微, 则

(1) $f$凸当且仅当$f'$单调上升.

(2) $f$凸当且仅当, $\forall x_{0},x\in I$, 有$f(x)\ge f'(x_{0})(x-x_{0})+f(x_{0})$. 

\paragraph{证明:}

(1) $\Longrightarrow$: 同前; $\Longleftarrow$: 用中值定理.
\[
a<x<b\Longrightarrow\frac{f(x)-f(a)}{x-a}=f'(\xi_{1})\le f'(\xi_{2})=\frac{f(b)-f(x)}{b-x}\Longrightarrow f\text{ 凸.}
\]

(2) $\Longrightarrow$: 
\[
x>x_{0}\Longrightarrow f'(x_{0})\le\frac{f(x)-f(x_{0})}{x-x_{0}};
\]
\[
x<x_{0}\Longrightarrow f'(x_{0})\ge\frac{f(x_{0})-f(x)}{x_{0}-x};
\]

$\Longleftarrow$: 
\[
a<x<b\Longrightarrow\frac{f(x)-f(a)}{x-a}\le f'(x)\le\frac{f(b)-f(x)}{b-x}\Longrightarrow f(x)\le\lambda f(a)+(1-\lambda)f(b)
\]
且$x=\lambda a+(1-\lambda)b$.

\paragraph{命题5.4.7}

设$f\in C(I)$, 若$f_{-}'$存在且单调上升, 则$f$是凸函数.

\paragraph{证明:}

设$x_{0}\in I$, 记$L(x)=f_{-}'(x_{0})(x-x_{0})+f(x_{0})$, $g(x)=f(x)-L(x)$.

则$g_{-}'(x)=f_{-}'(x)-f_{-}'(x_{0})$.

\[
\Longrightarrow\begin{cases}
	x\le x_{0}\Longrightarrow g_{-}'(x)\le0\Longrightarrow g(x)\ge g(x_{0}),\ x\to x_{0}^{-}\Longrightarrow g(x)\searrow,\ x\to x_{0}^{-}.\\
	x\ge x_{0}\Longrightarrow g_{-}'(x)\ge0\Longrightarrow g(x)\nearrow,\ x\to x_{0}^{+}.
\end{cases}
\]
所以$\min g=g(x_{0})=0$. 所以$f(x)\ge L(x)=f_{-}'(x_{0})(x-x_{0})+f(x_{0})$,
$x\in I$.

设$x_{i}\in I$, $\lambda_{i}\ge0$, $\sum_{i=1}^{n}\lambda_{i}=1$,
记$x_{0}\in\sum_{i=1}^{n}\lambda_{i}x_{i}$, 所以
\[
\sum_{i=1}^{n}\lambda_{i}f(x_{i})\ge f_{-}'(x_{0})\sum_{i=1}^{n}\lambda_{i}(x_{i}-x_{0})+f(x_{0})\sum_{i=1}^{n}\lambda_{i}=f(x_{0}).
\]


\paragraph{命题5.4.8}

若$f$在$I$中二阶可导, 则$f$凸当且仅当$f''\ge0$.

\subsection{函数作图}

若$\lim_{x\to x_{0}^{+}}f(x)=\infty$或$\lim_{x\to x_{0}^{-}}f(x)=\infty$,
则称$x=x_{0}$为$f$的垂直渐近线.

若
\[
\lim_{x\to+\infty}\left[f(x)-(ax+b)\right]=0\ \text{或}\ \text{\ensuremath{\lim_{x\to-\infty}\left[f(x)-(ax+b)\right]=0}},
\]
则称$y=ax+b$为$f$在无穷远处的渐近线.

\subsection{L'H\^{o}pital法则}

\paragraph{定理5.6.1 (L'H\^{o}pital法则)}

设$f,g$在$(a,b)$中可导, 且$g'(x)\ne0$, $\forall x\in(a,b)$, 又设
\[
\lim_{x\to a+}f(x)=0=\lim_{x\to a+}g(x),
\]
若极限
\[
\lim_{x\to a+}\frac{f'(x)}{g'(x)}\text{ 存在(或为}\infty\text{)}.
\]
则
\[
\lim_{x\to a+}\frac{f(x)}{g(x)}=\lim_{x\to a+}\frac{f'(x)}{g'(x)}.
\]


\paragraph{定理5.6.2 (L'H\^{o}pital法则)}

设$f,g$在$(a,b)$中可导, 且$g'(x)\ne0$, $\forall x\in(a,b)$, 又设
\[
\lim_{x\to a+}g(x)=\infty,
\]
若极限
\[
\lim_{x\to a+}\frac{f'(x)}{g'(x)}=l
\]
存在, (或为$\infty$), 则
\[
\lim_{x\to a+}\frac{f(x)}{g(x)}=\lim_{x\to a+}\frac{f'(x)}{g'(x)}=l.
\]


\paragraph{证明:}

只证明$l<\infty$的情况, $\forall\epsilon>0$, $\exists\eta>0$, s.t.
\[
l-\frac{\epsilon}{2}<\frac{f'(x)}{g'(x)}<l+\frac{\epsilon}{2},\qquad\forall x\in(a,a+\eta).
\]
取$c=a+\eta$, 由Cauchy中值定理, $\exists\xi\in(x,c)$, s.t.
\[
\frac{f(x)-f(c)}{g(x)-g(c)}=\frac{f'(\xi)}{g'(\xi)}\Longrightarrow\frac{f(x)}{g(x)}=\frac{f'(\xi)}{g'(\xi)}+\frac{f(c)}{g(x)}-\frac{f'(\xi)}{g'(\xi)}\cdot\frac{g(c)}{g(x)},\quad\xi\in(x,c)\subseteq(a,a+\eta).
\]
由$\lim_{x\to a+}g(x)=\infty$, 故存在$\delta<\eta$, s.t.
\[
\left|\frac{f(x)}{g(x)}-l\right|<\epsilon,\qquad\forall x\in(a,a+\delta).
\]

注: 当$\lim_{x\to a+}\frac{f'(x)}{g'(x)}$不存在时, $\lim_{x\to a+}\frac{f(x)}{g(x)}$仍然可能存在.
比如求
\[
\lim_{x\to0+}\int_{0}^{x}\cos\frac{1}{t}\ud t.
\]
用L'H\^{o}pital法则有如下过程:
\[
f'_{+}(0)=\underset{x\to0+}{\mathop{\lim}}\frac{f(x)-f(0)}{x-0}=\underset{x\to0+}{\mathop{\lim}}\frac{\int_{0}^{x}\cos\frac{1}{t}\ud t}{x}=\underset{x\to0+}{\mathop{\lim}}\cos\frac{1}{x}
\]
不存在, 但是做变量替换$s=\frac{1}{t}$, $y=\frac{1}{x}$之后, 
\[
\underset{x\to0+}{\mathop{\lim}}\frac{\int_{0}^{x}\cos\frac{1}{t}\ud t}{x}=\lim_{y\to+\infty}y\int_{y}^{\infty}\frac{\cos s}{s^{2}}\ud s=y\cdot\left(\frac{1}{2}\int_{y}^{y+\pi}\frac{\cos s}{s^{2}}\ud s+\frac{1}{2}\int_{y+\pi}^{\infty}\frac{\cos s}{s^{2}}\ud s+\frac{1}{2}\int_{y}^{\infty}\frac{\cos s}{s^{2}}\ud s\right).
\]
显然, $\int_{y}^{y+\pi}\frac{\cos s}{s^{2}}\ud s=O\left(\frac{1}{y^{2}}\right)$.
而
\[
\int_{y+\pi}^{\infty}\frac{\cos s}{s^{2}}\ud s+\int_{y}^{\infty}\frac{\cos s}{s^{2}}\ud s=\int_{y}^{\infty}\cos s\left(\frac{1}{s^{2}}-\frac{1}{(s+\pi)^{2}}\right)\ud s=\int_{y}^{\infty}\cos s\left(\frac{2s\pi+\pi^{2}}{s^{2}(s+\pi)^{2}}\right)\ud s=O\left(\frac{1}{y^{2}}\right).
\]
所以应当有
\[
\lim_{y\to+\infty}y\int_{y}^{\infty}\frac{\cos s}{s^{2}}\ud s=0.
\]
或者对上式用分部积分
\[
\int_{y}^{\infty}\frac{\cos s}{s^{2}}\ud s=\frac{\sin s}{s^{2}}\mid_{y}^{\infty}+\int_{y}^{\infty}\frac{2\sin s}{s^{3}}\ud s=-\frac{\sin y}{y^{2}}+O\left(\int_{y}^{\infty}\frac{2}{s^{3}}\ud s\right)=O\left(\frac{1}{y^{2}}\right).
\]
再或者用Riemann-Lebesgue引理
\[
\lim_{y\to+\infty}y\int_{y}^{\infty}\frac{\cos s}{s^{2}}\ud s=\lim_{y\to+\infty}\int_{1}^{\infty}\frac{\cos yt}{t^{2}}\ud t=0.
\]

再或者在计算导数前按以下过程分部积分将积分改写为
\[
\begin{aligned}\int_{\varepsilon}^{x}\cos\left(t^{-1}\right)dt & =-\int_{\varepsilon}^{x}t^{2}d\sin\left(t^{-1}\right)\\
	& =-x^{2}\sin\left(x^{-1}\right)+\varepsilon^{2}\sin\left(\varepsilon^{-1}\right)+2\int_{\varepsilon}^{x}t\sin\left(t^{-1}\right)dt\\
	& \stackrel{\varepsilon\rightarrow0}{\longrightarrow}-x^{2}\sin\left(x^{-1}\right)+2\int_{0}^{x}t\sin\left(t^{-1}\right)dt.
\end{aligned}
\]


\paragraph{例5.6.4}

设$f$在$(a,+\infty)$上可微.

(1). 若$\lim_{x\to+\infty}x\cdot f'(x)=1$, 则$\lim_{x\to+\infty}f(x)=+\infty$.

(2). 若存在$\alpha>0$, s.t.
\[
\lim_{x\to+\infty}\left(\alpha\cdot f(x)+x\cdot f'(x)\right)=\beta,
\]
则
\[
\lim_{x\to+\infty}f(x)=\frac{\beta}{\alpha}.
\]


\paragraph{证明:}

(1). 当$x\to+\infty$, $\ln x\to+\infty$, 则
\[
\lim_{x\to+\infty}\frac{f(x)}{\ln x}=\lim_{x\to+\infty}\frac{f'(x)}{\frac{1}{x}}=1\Longrightarrow f(x)\to+\infty,\ (x\to+\infty).
\]

(2). 当$\alpha>0$时, $x^{\alpha}\to+\infty$, ($x\to+\infty$), 则
\[
\lim_{x\to+\infty}f(x)=\lim_{x\to+\infty}\frac{x^{\alpha}f(x)}{x^{\alpha}}=\lim_{x\to+\infty}\frac{\alpha x^{\alpha-1}f(x)+x^{\alpha}f'(x)}{\alpha x^{\alpha-1}}=\frac{\beta}{\alpha}.
\]


\subsection{Taylor展开}

(1). 若$f(x)$在$x_{0}$处连续, 则
\[
f(x)-f(x_{0})=o(1).
\]

(2). 若$f(x)$在$x_{0}$处可微, 则
\[
f(x)-\left[f(x_{0})+f'(x_{0})(x-x_{0})\right]=o(x-x_{0}),\quad(x\to x_{0}).
\]
注意, 这里$f$在$x_{0}$处可微, 但没说在$x_{0}$附近可微, 不能用L'H\^{o}pital法则.

(3). 若$f(x)$在$x_{0}$处二阶可微, 则
\[
f(x)-\left[f(x_{0})+f'(x_{0})(x-x_{0})+\frac{1}{2}f''(x_{0})(x-x_{0})^{2}\right]=o\left((x-x_{0})^{2}\right),\quad(x\to x_{0}).
\]


\paragraph{定理5.7.1 (带Peano余项的Taylor公式)}

设$f$在$x_{0}$处$n$阶可导, 则
\[
f(x)=f(x_{0})+f'(x_{0})(x-x_{0})+\frac{1}{2!}f''(x_{0})(x-x_{0})^{2}+\cdots+\frac{1}{n!}f^{(n)}(x_{0})(x-x_{0})^{n}+o((x-x_{0})^{n}),\quad(x\to x_{0}).
\]


\paragraph{证明: }

(归纳法+中值定理) 记
\[
R_{n}(x)=f(x)-\left[f(x_{0})+f'(x_{0})(x-x_{0})+\cdots+\frac{f^{(n)}(x_{0})}{n!}(x-x_{0})^{n}\right],
\]
则
\[
\frac{R_{k+1}(x)}{(x-x_{0})^{k+1}}\to\frac{R'_{k+1}(x)}{(k+1)(x-x_{0})^{k}}=\frac{o((x-x_{0})^{k})}{(k+1)(x-x_{0})^{k}}=o(1).
\]


\paragraph{定理5.7.2 (Taylor)}

设$f$在$(a,b)$上$n+1$阶可导, $x_{0},x\in(a,b)$. 则存在$\xi,\zeta\in(x,x_{0})$或$(x_{0},x)$,
s.t. Taylor展开的余项
\[
R_{n}(x)=\frac{f^{(n+1)}(\xi)}{(n+1)!}(x-x_{0})^{n+1},
\]
称为Lagrange余项, 以及
\[
R_{n}(x)=\frac{1}{n!}f^{(n+1)}(\zeta)(x-\zeta)^{n}(x-x_{0}),
\]
称为Cauchy余项.

\paragraph{证明:}

取
\[
F(t)=f(t)+\sum_{k=1}^{n}\frac{f^{(k)}(t)}{k!}(x-t)^{k},\quad t\in(a,b).
\]

对$t$求导, 得到
\[
F'(t)=\frac{1}{n!}f^{(n+1)}(t)(x-t)^{n}.
\]
所以
\[
F(x)-F(x_{0})=R_{n}(x).
\]

Cauchy余项: 用Lagrange中值定理, $\exists\zeta=x_{0}+\theta(x-x_{0})$, ($0<\theta<1$),
s.t.
\[
R_{n}(x)=F'(\zeta)(x-x_{0}).
\]

Lagrange余项: 用Cauchy微分中值定理, $\exists\xi=x_{0}+\eta(x-x_{0})$, ($0<\eta<1$),
s.t.
\[
\frac{R_{n}(x)}{(x-x_{0})^{n+1}}=\frac{F(x)-F(x_{0})}{G(x)-G(x_{0})}=\frac{F'(\xi)}{G'(\xi)}.
\]

上面的证明给出Taylor展开的积分余项公式
\[
f(x)=f(x_{0})+\sum_{k=1}^{n}\frac{f^{(k)}(x_{0})}{k!}(x-x_{0})^{k}+\int_{x_{0}}^{x}\frac{f^{(n+1)}(t)}{n!}(x-t)^{n}\ud t.
\]


\paragraph{应用}

证明: 
\[
\int_{0}^{1}(1-t^{2})^{n}\ud t=\frac{(2n)!!}{(2n+1)!!}.
\]


\paragraph{证明:}

设$f(x)=(1+x)^{2n+1}$, $f(x)$在$x=0$处Taylor展开:
\[
(1+x)^{2n+1}=\sum_{k=0}^{n}\binom{2n+1}{k}x^{k}+\frac{1}{n!}\int_{0}^{x}\frac{(2n+1)!}{n!}(1+t)^{n}(x-t)^{n}\ud t,
\]
令$x=1$, 所以
\[
\int_{0}^{1}(1-t^{2})^{n}\ud t=\frac{(2n)!!}{(2n+1)!!}.
\]
这个积分可以换元法求解:
\[
\int_{0}^{1}(1-t^{2})^{n}\ud t=\int_{0}^{\pi/2}\cos^{2n+1}x\ud x=\frac{1}{2}B\left(\frac{1}{2},\frac{2n+2}{2}\right).
\]


\paragraph{定理5.7.4 (Taylor系数的唯一性)}

设$f$在$x_{0}$处$n$阶可导, 且
\[
f(x)=\sum_{k=0}^{n}a_{k}(x-x_{0})^{k}+o((x-x_{0})^{n})\quad(x\to x_{0}),
\]
则
\[
a_{k}=\frac{1}{k!}f^{(k)}(x_{0}),\quad k=0,1,\cdots,n.
\]


\paragraph{证明:}

给出Taylor展开的Peano余项表示, 两者作差比阶.

\paragraph{命题5.7.5}

设$f(x)$在$x=0$处的Taylor展开为$\sum_{n=0}^{\infty}a_{n}x^{n}$, 则

(1). $f(-x)$的Taylor展开为$\sum_{n=0}^{\infty}(-1)^{n}a_{n}x^{n}$;

(2). $f(x^{k})$的Taylor展开为$\sum_{n=0}^{\infty}a_{n}x^{kn}$, 其中$k\in\NN_{+}$;

(3). $x^{k}f(x)$的Taylor展开为$\sum_{n=0}^{\infty}a_{n}x^{k+n}$, 其中$k\in\NN_{+}$;

(4). $f'(x)$的Taylor展开为$\sum_{n=1}^{\infty}na_{n}x^{n-1}=\sum_{n=0}^{\infty}(n+1)a_{n+1}x^{n}$;

(5). $\int_{0}^{x}f(t)\ud t$的Taylor展开为$\sum_{n=0}^{\infty}\frac{a_{n}}{n+1}x^{n+1}$;

(6). 如果$g(x)$在$x=0$处的Taylor展开为$\sum_{n=0}^{\infty}b_{n}x^{n}$,
则$\lambda f(x)+\mu g(x)$的Taylor展开为$\sum_{n=0}^{\infty}(\lambda a_{n}+\mu b_{n})x^{n}$,
其中$\lambda,\mu\in\RR$.

\[
\ln(1-x)=-\left(x+\frac{x^{2}}{2}+\cdots+\frac{x^{n}}{n}\right)-\int_{0}^{x}\frac{t^{n}}{1-t}\ud t.
\]
\[
\arctan x=\left(x-\frac{x^{3}}{3}+\frac{x^{5}}{5}-\cdots+(-1)^{n-1}\frac{x^{2n-1}}{2n-1}\right)+(-1)^{n}\int_{0}^{x}\frac{t^{2n}}{1+t^{2}}\ud t=\frac{\ui}{2}\ln\frac{1-\ui x}{1+\ui x}.
\]


\paragraph{例5.7.5 }

Taylor展开收敛, 但不收敛到函数本身的例子.

定义
\[
\phi(x)=\begin{cases}
	0, & x\le0;\\
	\ue^{-\frac{1}{x}}, & x>0.
\end{cases}
\]
在$x=0$处展开的Taylor级数恒为$0$.

\subsection{Taylor公式和微分学的应用}

\paragraph{Thm. 5.8.1 (函数极值的判断)}

设$f$在$x_{0}$处$n$阶可导, 且
\[
f'(x_{0})=f''(x_{0})=\cdots=f^{(n-1)}(x_{0})=0,\quad f^{(n)}(x_{0})\ne0,
\]
则

(1). $n$为偶数, 若$f^{(n)}(x_{0})<0$, 则$x_{0}$为极大值点; 若$f^{(n)}(x_{0})>0$,
则$x_{0}$为极小值点.

(2). $n$为奇数时, $x_{0}$不是极值点.

\paragraph{Thm. 5.8.2 (Jensen不等式的余项)}

设$f\in C[a,b]$, 在$(a,b)$上二阶可导. 当$x_{i}\in[a,b]$, ($1\le i\le n$)时,
$\exists\xi\in(a,b)$, s.t.
\[
f\left(\sum_{i=1}^{n}\lambda_{i}x_{i}\right)-\sum_{i=1}^{n}\lambda_{i}f(x_{i})=-\frac{1}{2}f''(\xi)\sum_{i<j}\lambda_{i}\lambda_{j}(x_{i}-x_{j})^{2},
\]
其中$\lambda_{i}\ge0$, $\sum_{i=1}^{n}\lambda_{i}=1$.

\paragraph{证明: }

记
\[
\overline{x}=\sum_{i=1}^{n}\lambda_{i}x_{i}\in[a,b].
\]
则
\[
f(x_{i})=f(\overline{x})+f'(x)(x_{i}-\overline{x})+\frac{1}{2}f''(\xi_{i})(x_{i}-\overline{x})^{2},\quad\xi_{i}\in(a,b).
\]
所以
\[
\sum_{i=1}^{n}\lambda_{i}f(x_{i})=f(\overline{x})+\frac{1}{2}\sum_{i=1}^{n}f''(\xi_{i})\lambda_{i}(x_{i}-\overline{x})^{2}.
\]
而
\[
\sum_{i=1}^{n}\lambda_{i}(x_{i}-\overline{x})^{2}=\frac{1}{2}\sum_{i,j=1}^{n}\lambda_{i}\lambda_{j}(x_{i}-x_{j})^{2}.
\]
所以
\[
\frac{m}{4}\sum_{i,j=1}^{n}\lambda_{i}\lambda_{j}(x_{i}-x_{j})^{2}\le\sum_{i=1}^{n}\lambda_{i}f(x_{i})-f(\overline{x})\le\frac{M}{4}\sum_{i,j=1}^{n}\lambda_{i}\lambda_{j}(x_{i}-x_{j})^{2}.
\]
用Darboux定理.

用于求极限

\paragraph{例5.8.1}

求
\[
\lim_{x\rightarrow\infty}\left[x-x^{2}\ln\left(1+\frac{1}{x}\right)\right].
\]


\paragraph{解:}

\[
\ln\left(1+\frac{1}{x}\right)=\frac{1}{x}-\frac{1}{2x^{2}}+o\left(\frac{1}{x^{2}}\right)\quad(x\rightarrow\infty),
\]
所以
\[
x-x^{2}\ln\left(1+\frac{1}{x}\right)=\frac{1}{2}+o(1)\rightarrow\frac{1}{2}(x\rightarrow\infty).
\]


\paragraph{例5.8.2}

设$f$在$0$附近二阶可导, 且$\left|f''\right|\le M$, $f(0)=0$, 则
\[
\lim_{n\rightarrow\infty}\sum_{k=1}^{n}f\left(\frac{k}{n^{2}}\right)=\frac{1}{2}f^{\prime}(0).
\]


\paragraph{解:}

\[
f\left(\frac{k}{n^{2}}\right)=f(0)+f^{\prime}(0)\frac{k}{n^{2}}+R_{k,n},
\]
其中
\[
\left|R_{k,n}\right|=\frac{1}{2}\left|f^{\prime\prime}\left(\xi_{k,n}\right)\right|\left(\frac{k}{n^{2}}\right)^{2}\leqslant\frac{1}{2}M\frac{k^{2}}{n^{4}},
\]
所以
\[
\begin{aligned}\sum_{k=1}^{n}f\left(\frac{k}{n^{2}}\right) & =f^{\prime}(0)\frac{1}{n^{2}}\sum_{k=1}^{n}k+\sum_{k=1}^{n}R_{k,n}\\
	& =f^{\prime}(0)\frac{n+1}{2n}+o(1)\quad(n\rightarrow\infty),
\end{aligned}
\]


\paragraph{Stirling公式}

\[
n!=\sqrt{2\pi n}\left(\frac{n}{e}\right)^{n}e^{\frac{\theta_{n}}{12n}},\quad0<\theta_{n}<1.
\]


\subsection{作业}

8. 设$f(x)$在$\RR$上可微, 且$\lim_{x\to-\infty}f'(x)=A$, $\lim_{x\to+\infty}f'(x)=B$.
证明, 如果$A\ne B$, 则任给$\theta\in(0,1)$, 都有$\xi\in\RR$, 使得
\[
f'(\xi)=\theta A+(1-\theta)B.
\]

9. 设$f(x)$在区间$I$中$n$阶可微, $x_{1},x_{2},\cdots,x_{k}$为$I$中的点. 证明存在$\xi\in I$,
s.t.
\[
\frac{1}{k}\left(f^{(n)}(x_{1})+f^{(n)}(x_{2})+\cdots+f^{(n)}(x_{k})\right)=f^{(n)}(\xi).
\]

10. 设$f(x)$在区间$I$中可微, $x_{0}\in I$. 如果$\lim_{x\to x_{0}}f'(x)$存在,
则$f'(x)$在$x_{0}$处连续.

11. 设$f(x)$在$(a,b)$中可微, 如果$f'(x)$为单调函数, 则$f'(x)$在$(a,b)$中连续.

3. 设$f(x)$在$[a,b]$上二阶可导, 且$f(a)=f(b)=0$, $f'_{+}(a)f'_{-}(b)>0$.
证明: 存在$\xi\in(a,b)$, s.t. $f''(\xi)=0$.

6. 设$f(x)$为$[a,b]$上的三阶可导函数, 且$f(a)=f'(a)=f(b)=0$, 证明, 对于任意的$c\in[a,b]$,
存在$\xi\in(a,b)$, s.t.
\[
f(c)=\frac{f'''(\xi)}{6}(c-a)^{2}(c-b).
\]

8. 设$f(x)$在区间$[a,b]$上可导, 且存在$M>0$, 使得
\[
\left|f'(x)\right|\le M,\quad\forall x\in[a,b].
\]
证明:
\[
\left|f(x)-\frac{f(a)+f(b)}{2}\right|\le\frac{M}{2}(b-a),\quad\forall x\in[a,b].
\]

10. 设$f$在$(a,b)$上可微, 且$a<x_{i}\le y_{i}<b$, $i=1,2,\cdots,n$.
证明: 存在$\xi\in(a,b)$, s.t.
\[
\sum_{i=1}^{n}[f(y_{i})-f(x_{i})]=f'(\xi)\sum_{i=1}^{n}(y_{i}-x_{i}).
\]

11. 设$f$在$[a,+\infty)$上可微, 且
\[
f(a)=0,\quad\left|f'(x)\right|\le\left|f(x)\right|,\quad\forall x\in[a,+\infty).
\]
证明: $f\equiv0$.

提示考虑$A=\left\{ x\in[a,+\infty):f(x)=0\right\} $和$\sup A$.

11. 设$f(x)$在$\RR$上二阶可导, 若
\[
\lim_{x\to\infty}\frac{f(x)}{\left|x\right|}=0,
\]
则存在$\xi\in\RR$, 使得$f''(\xi)=0$.

用Darboux定理

12. 设$f(0)=0$, $f'(x)$严格单调递增, 则$\frac{f(x)}{x}$在$(0,+\infty)$上也严格单调递增.

11. 证明, 定义在$\RR$上的有界凸函数是常数函数.

12. 设$f(x)\in C(I)$, 若$\forall x_{0}\in I$, $\exists\delta>0$,
s.t. $f(x)$在$(x_{0}-\delta,x_{0}+\delta)$上凸, 则$f(x)$在$I$中凸.

13. 设$f$为区间$I$上的凸函数, $x_{0}$为$I$的内点. 若$f'_{-}(x_{0})\le k\le f'_{+}(x_{0})$,
则
\[
f(x)\ge k(x-x_{0})+f(x_{0}),\quad\forall x\in I.
\]
并证明Jensen不等式.

15. 设$f\in C[a,b]$凸, 证明Hadamard不等式
\[
f\left(\frac{a+b}{2}\right)\le\frac{1}{b-a}\int_{a}^{b}f(x)\ud x\le\frac{f(a)+f(b)}{2}.
\]

16. (Schwarz symmetric derivative, Riemann derivative) 设$f\in C(\RR)$,
若$\forall x\in\RR$, 均有
\[
\lim_{h\to0}\frac{f(x+h)+f(x-h)-2f(x)}{h^{2}}=0,
\]
证明$f(x)$为线性函数.

连续性是必要的, 否则考虑符号函数. 这个极限不能被改善成
\[
\lim_{h\to0}\frac{f(x+2h)-2f(x+h)+f(x)}{h^{2}}=0,
\]
这只需要改变符号函数在0点处的值为1, 使其在0点处右连续.

提示: 证明$\forall\epsilon>0$, $f(x)+\epsilon x^{2}$是凸的, $f(x)-\epsilon x^{2}$是凹的,
$f(x)$在$[a,b]$上位于直线$[f(a)+\epsilon a^{2},f(b)+\epsilon b^{2}]$与直线$[f(a)-\epsilon a^{2},f(b)-\epsilon b^{2}]$之间.

6. 是否存在$\RR$上的凸函数, 使得$f(0)<0$, 且
\[
\lim_{\left|x\right|\to+\infty}(f(x)-\left|x\right|)=0\ ?
\]

5. 设$f$在点$x_{0}$处2阶可导, 且$f''(x_{0})\ne0$. 由微分中值定理, 当$h$充分小时, 存在$\theta=\theta(h)$,
($0<\theta<1$), s.t.
\[
f(x_{0}+h)-f(x_{0})=f'(x_{0}+\theta h)h,
\]
证明:
\[
\lim_{h\to0}\theta=\frac{1}{2}.
\]

7. 设$f(x)$在$(a,+\infty)$中可微, 且
\[
\lim_{x\to+\infty}\left[f(x)+f'(x)\right]=l,
\]
证明:
\[
\lim_{x\to+\infty}f(x)=l.
\]

9. 设$f''(x_{0})$存在, $f'(x_{0})\ne0$, 求极限
\[
\lim_{x\to x_{0}}\left[\frac{1}{f(x)-f(x_{0})}-\frac{1}{f'(x_{0})(x-x_{0})}\right].
\]

10. 设$a_{1}\in(0,\pi)$, $a_{n+1}=\sin a_{n}$, ($n\ge1$). 证明:
\[
\lim_{n\to\infty}\sqrt{n}a_{n}=\sqrt{3}.
\]

2. 设 $f(x)$ 是 $x$ 的 $n$ 次多项式, 则 $f(x)$ 在 $x=x_{0}$ 处的Taylor展开的
Peano 余项 $R_{n}(x)$ 恒为零. (提示: 考虑其它余项公式.)

10. 设 $f(x),g(x)$ 在 $(-1,1)$ 中无限次可微, 且 
\[
\left|f^{(n)}(x)-g^{(n)}(x)\right|\le n!|x|,\quad\forall x\in(-1,1),n=0,1,2,\cdots.
\]
证明 $f(x)=g(x)$.

12. 设 $f$ 在 $x_{0}$ 附近可以表示为 
\[
f(x)=\sum_{k=0}^{n}a_{k}\left(x-x_{0}\right)^{k}+o\left(\left(x-x_{0}\right)^{n}\right)\quad\left(x\rightarrow x_{0}\right),
\]
则 $f(x)$ 是否在 $x_{0}$ 处 $n$ 阶可导? 

13. 设 $f(x)$ 在 $[a,b]$ 上二阶可导, 且 $f^{\prime}(a)=f^{\prime}(b)=0$.
证明, 存在 $\xi\in(a,b)$, 使得
\[
\left|f^{\prime\prime}(\xi)\right|\ge\frac{4}{(b-a)^{2}}|f(b)-f(a)|.
\]

4. 设 $f(x)$ 在 $x_{0}$ 的一个开邻域内 $n+1$ 次连续可微, 且 $f^{(n+1)}\left(x_{0}\right)\neq0$,
其 Taylor 公式为 
\[
f\left(x_{0}+h\right)=f\left(x_{0}\right)+f^{\prime}\left(x_{0}\right)h+\cdots+\frac{1}{n!}f^{(n)}\left(x_{0}+\theta h\right)h^{n},
\]
其中 $0<\theta<1$. 证明 $\lim_{h\rightarrow0}\theta=\frac{1}{n+1}$.

8. 设 $a_{1}\in\mathbb{R},a_{n+1}=\arctan a_{n}(n\geqslant1)$. 求极限
$\lim_{n\rightarrow\infty}na_{n}^{2}$.

10. 设 $f$ 在 $\mathbb{R}$ 上二阶可导, 且 
\[
M_{0}=\sup_{x\in\mathbb{R}}|f(x)|<\infty,\quad M_{2}=\sup_{x\in\mathbb{R}}\left|f^{\prime\prime}(x)\right|<\infty.
\]
证明 $M_{1}=\sup_{x\in\mathbb{R}}\left|f^{\prime}(x)\right|<\infty$,
且 $M_{1}^{2}\leqslant2M_{0}\cdot M_{2}$. (提示: 考虑 $f(x\pm h)$ 的 Taylor
展开.)
\section{Riemann积分}

\subsection{Riemann可积}

设定义在$[a,b]$区间上的函数$f(x)$, 将$[a,b]$分割为
\[
\pi:a=x_{0}<x_{1}<\cdots<x_{n}=b,
\]
近似第$i$个小梯形的面积为$f(\xi_{i})\Delta x_{i}$, 其中$\xi_{i}\in[x_{i-1},x_{i}]$,
$\Delta x_{i}=x_{i}-x_{i-1}$. 用$\sum_{i=1}^{n}f(\xi_{i})\Delta x_{i}$表示曲边梯形ABCD的面积的近似值,
称为$f$在$[a,b]$上的Riemann和. 若
\[
\lim_{\norm{\pi}\to0}\sum_{i=1}^{n}f(\xi_{i})\Delta x_{i}
\]
存在, 其中$\norm{\pi}=\max_{1\le i\le n}\left\{ \Delta x_{i}\right\} $为分割的模.
则记为$\int_{a}^{b}f(x)\ud x$.

\paragraph{定义6.1.1 (Riemann积分)}

设$f$定义在$[a,b]$上, 若存在$I\in\RR$, s.t. $\forall\epsilon>0$, $\exists\delta>0$,
对任何分割$\pi$, 只要$\norm{\pi}<\delta$, 就有
\[
\left|\sum_{i=1}^{n}f(\xi_{i})\Delta x_{i}-I\right|<\epsilon,\quad\forall\xi_{i}\in[x_{i-1},x_{i}],\ i=1,2,\cdots,n,
\]
则称$f$在$[a,b]$上Riemann可积或可积, $I$为$f$在$[a,b]$上的(定)积分, 记为
\[
I=\int_{a}^{b}f(x)\ud x=\lim_{\norm{\pi}\to0}\sum_{i=1}^{n}f(\xi_{i})\Delta x_{i}.
\]
其中$f$称为被积函数, $[a,b]$称为积分区间, $a,b$分别称为积分下限与积分上限.

\paragraph{定理6.1.1 (可积的必要条件)}

若$f$在$[a,b]$上可积, 则$f$在$[a,b]$上有界, 反之不然.

有界函数未必可积: Dirichlet函数$D(x)$, 对于任意的分割$\xi_{i}\in[x_{i-1},x_{i}]\setminus\QQ$,
积分和为$0$; 当$\xi_{i}\in[x_{i-1},x_{i}]\cap\QQ$时, 积分和为$1$. 所以$D(x)$的积分和没有极限.

对于分割
\[
\pi:a=x_{0}<x_{1}<\cdots<x_{n}=b,
\]
记
\[
M_{i}=\sup_{x\in[x_{i-1},x_{i}]}f(x),\quad m_{i}=\inf_{x\in[x_{i-1},x_{i}]}f(x),
\]
令
\[
S=\sum_{i=1}^{n}M_{i}\cdot\Delta x_{i},\quad s=\sum_{i=1}^{n}m_{i}\cdot\Delta x_{i},
\]
称$S$是$f$关于$\pi$的Darboux上和, 简称上和, 记为$S(\pi)$或$S(\pi,f)$. $s$称为Darboux下和,
简称下和, 记为$s(\pi)$或$s(\pi,f)$. 

称
\[
\omega_{i}=M_{i}-m_{i}=\sup_{x\in\left[x_{i-1},x_{i}\right]}f(x)-\inf_{x\in\left[x_{i-1},x_{i}\right]}f(x)
\]
为$f$在$[x_{i-1},x_{i}]$上的振幅. 则
\[
S-s=\sum_{i=1}^{n}\omega_{i}\cdot\Delta x_{i}.
\]


\paragraph{引理6.1.2}

设分割$\pi'$是从$\pi$添加$k$个分点得到的, 则
\[
\begin{aligned}S(\pi)\geqslant & S\left(\pi^{\prime}\right)\geqslant S(\pi)-(M-m)k\|\pi\|,\\
	s(\pi)\leqslant & s\left(\pi^{\prime}\right)\leqslant s(\pi)+(M-m)k\|\pi\|.
\end{aligned}
\]
即, 对于给定的分割, 增加分点时下和不减, 上和不增.

\paragraph{证明:}

只需要对$k=1$进行即可.

\paragraph{推论6.1.3}

对于任何两个分割$\pi_{1}$和$\pi_{2}$, 有
\[
s(\pi_{1})\le S(\pi_{2}).
\]


\paragraph{定理6.1.4 (Darboux)}

\[
\lim_{\|\pi\|\rightarrow0}S(\pi)=\inf_{\pi}S(\pi),\quad\lim_{\|\pi\|\rightarrow0}s(\pi)=\sup_{\pi}s(\pi).
\]

称$\inf_{\pi}S(\pi)$为$f$在$[a,b]$上的上积分, $\sup_{\pi}s(\pi)$为$f$在$[a,b]$上的下积分.

\paragraph{定理6.1.5 (可积的充要条件)}

设$f$在$[a,b]$上有界, 则以下命题等价:

(1) $f$在$[a,b]$上Riemann可积.

(2) $f$在$[a,b]$上的上积分和下积分相等.

(3) 
\[
\lim_{\norm{\pi}\to0}\sum_{i=1}^{n}\omega_{i}\cdot\Delta x_{i}=0.
\]

(4) $\forall\epsilon>0$, 存在$[a,b]$的分割$\pi$, s.t.
\[
S(\pi)-s(\pi)=\sum_{i=1}^{n}\omega_{i}\cdot\Delta x_{i}<\epsilon.
\]


\paragraph{推论6.1.6}

(1) 设$[\alpha,\beta]\subseteq[a,b]$, 如果$f$在$[a,b]$上可积, 则$f$在$[\alpha,\beta]$上也可积.

(2) 设$c\in(a,b)$, 若$f$在$[a,c]$及$[c,b]$上都可积, 则$f$在$[a,b]$上可积.

\paragraph{例6.1.1}

设$f,g$在$[a,b]$上均可积, 则$fg$在$[a,b]$上也可积.

注意,

\[
\begin{aligned}\omega_{i}(fg) & =\sup_{x^{\prime},x^{\prime\prime}\in\left[x_{i-1},x_{i}\right]}\left|f\left(x^{\prime}\right)g\left(x^{\prime}\right)-f\left(x^{\prime\prime}\right)g\left(x^{\prime\prime}\right)\right|\\
	& =\sup_{x^{\prime},x^{\prime\prime}\in\left[x_{i-1},x_{i}\right]}\left|f\left(x^{\prime}\right)g\left(x^{\prime}\right)-f\left(x^{\prime}\right)g\left(x^{\prime\prime}\right)+f\left(x^{\prime}\right)g\left(x^{\prime\prime}\right)-f\left(x^{\prime\prime}\right)g\left(x^{\prime\prime}\right)\right|\\
	& \leqslant\sup_{x^{\prime},x^{\prime\prime}\in\left[x_{i-1},x_{i}\right]}\left[\left|f\left(x^{\prime}\right)\right|\left|g\left(x^{\prime}\right)-g\left(x^{\prime\prime}\right)\right|+\left|g\left(x^{\prime\prime}\right)\right|\left|f\left(x^{\prime}\right)-f\left(x^{\prime\prime}\right)\right|\right]\\
	& \leqslant K\left(\omega_{i}(g)+\omega_{i}(f)\right),
\end{aligned}
\]
并用前面的定理6.1.5 (3).

\paragraph{定理6.1.7 (可积函数类)}

(1) 若$f\in C[a,b]$, 则$f$在$[a,b]$上可积;

(2) 若有界函数$f$只在$[a,b]$上有限个点处不连续, 则$f$可积;

(3) 若$f$在$[a,b]$上单调, 则$f$可积.

\paragraph{证明:}

(2) 主要依赖以下不等式
\[
\begin{aligned}S(\pi)-s(\pi) & \leqslant\frac{\varepsilon}{2(b-a)}(b-a)+2M\sum_{i=1}^{N}2\rho\\
	& \leqslant\frac{\varepsilon}{2}+2M\cdot2N\rho<\varepsilon.
\end{aligned}
\]

(3) 主要依赖以下不等式
\[
\begin{aligned}\sum_{i=1}^{n}\omega_{i}\cdot\Delta x_{i} & =\sum_{i=1}^{n}\left(f\left(x_{i}\right)-f\left(x_{i-1}\right)\right)\cdot\Delta x_{i}\\
	& \leqslant\sum_{i=1}^{n}\left(f\left(x_{i}\right)-f\left(x_{i-1}\right)\right)\cdot\norm{\pi}\\
	& =\left(f\left(x_{n}\right)-f\left(x_{0}\right)\right)\norm{\pi}\\
	& =(f(b)-f(a))\norm{\pi}<\varepsilon.
\end{aligned}
\]

设$f$为$[a,b]$上定义的函数, 若存在$[a,b]$上的分割
\[
\pi:a=x_{0}<x_{1}<x_{2}<\cdots<x_{n}=b,
\]
使得$f$在每个小区间$(x_{i-1},x_{i})$上均为常数, 则称$f$为阶梯函数.

\paragraph{推论6.1.8}

阶梯函数均为可积函数.

\paragraph{定理6.1.9 (Riemann)}

设$f$在$[a,b]$上有界, 则$f$可积的充要条件是$\forall\epsilon,\eta>0$, 存在$[a,b]$的分割$\pi$,
s.t.
\[
\sum_{\omega_{i}\ge\eta}\Delta x_{i}<\epsilon.
\]


\paragraph{例6.1.3}

设$f\in C[a,b]$, $\phi$在$[\alpha,\beta]$上可积, $\phi([\alpha,\beta])\subseteq[a,b]$.
则$f\circ\phi$在$[\alpha,\beta]$上仍可积.

\paragraph{证明:}

$f$ 在 $[a,b]$ 上一致连续. $\forall\varepsilon>0$, $\exists\delta>0$,
当 $x,y\in$ $[a,b]$, $|x-y|<\delta$ 时, $|f(x)-f(y)|<\frac{\varepsilon}{2(\beta-\alpha)}$.
因为 $\phi$ 在 $[\alpha,\beta]$ 上可积, 则存在 $[\alpha,\beta]$ 的分割 $\pi:\alpha=t_{0}<t_{1}<\cdots<t_{m}=\beta$,
使得 
\[
\sum_{\omega_{i}(\phi)\geqslant\delta}\Delta t_{i}<\frac{\varepsilon}{4K+1},
\]
其中 $K=\max_{x\in[a,b]}|f(x)|$. 于是 
\[
\begin{aligned}\sum_{i=1}^{m}\omega_{i}(f\circ\varphi)\cdot\Delta t_{i} & =\sum_{\omega_{i}(\phi)\geqslant\delta}\omega_{i}(f\circ\varphi)\cdot\Delta t_{i}+\sum_{\omega_{i}(\phi)<\delta}\omega_{i}(f\circ\varphi)\cdot\Delta t_{i}\\
	& \leqslant2K\cdot\sum_{\omega_{i}(\phi)\geqslant\delta}\Delta t_{i}+\frac{\varepsilon}{2(\beta-\alpha)}\cdot\sum_{\omega_{i}(\phi)<\delta}\Delta t_{i}\\
	& \leqslant2K\cdot\frac{\varepsilon}{4K+1}+\frac{\varepsilon}{2(\beta-\alpha)}\cdot(\beta-\alpha)<\varepsilon.
\end{aligned}
\]

两个可积函数的复合不可积的例子: 
\[
f(x)=\begin{cases}
	1, & x\ne0,\\
	0, & x=0.
\end{cases}\quad g(x)=R(x)\Longrightarrow f\circ g(x)=D(x).
\]

可积函数复合连续函数不可积的例子:

\[
f(x)=\begin{cases}
	0, & 0\le x<1,\\
	1, & x=1.
\end{cases}
\]
设$A$为$[0,1]$上有正测度的类Cantor集, $(a_{i},b_{i})$, ($i\in\NN_{+}$)为$A$的邻接区间.
\[
g(x)=\begin{cases}
	1, & x\in A,\\
	1-\frac{1}{2}(b_{i}-c_{i})+\left|x-\frac{1}{2}(a_{i}+b_{i})\right|, & x\in(a_{i},b_{i}),\ i\in\NN_{+}.
\end{cases}
\]
则
\[
f\circ g(x)=\begin{cases}
	1, & x\in A,\\
	0, & x\in[0,1]\setminus A.
\end{cases}
\]


\paragraph{定理6.1.10 (Lebsegue)}

有界函数$f$在$[a,b]$上Riemann可积的充要条件是它的不连续点集$D_{f}$为零测集. 其中$D_{f}=\bigcup_{n=1}^{\infty}D_{\frac{1}{n}}$,
而
\[
D_{\delta}=\left\{ x\in[a,b]:\omega(f,x)\ge\delta\right\} ,\quad\omega(f,x)=\lim_{r\rightarrow0^{+}}\sup\left\{ \left|f\left(x_{1}\right)-f\left(x_{2}\right)\right|:x_{1},x_{2}\in(x-r,x+r)\cap[a,b]\right\} .
\]


\subsection{定积分的性质}

线性性质, 积分区间可加性, 保号性, 绝对值不等式.

\paragraph{定理6.2.3 (积分第一中值定理)}

设 $f,g$ 在 $[a,b]$ 上可积, 且 $g(x)$ 不变号, 则存在 $\mu$, $\inf_{x\in[a,b]}f(x)\leqslant\mu\leqslant\sup_{x\in[a,b]}f(x)$,
使得 
\[
\int_{a}^{b}f(x)g(x)dx=\mu\cdot\int_{a}^{b}g(x)dx.
\]


\paragraph{引理 6.2.4. }

如果 $f(x)$ 在 $[a,b]$ 上可积, 令 
\[
F(x)=\int_{a}^{x}f(t)dt,\quad x\in[a,b],
\]
则 $F$ 是 $[a,b]$ 上的连续函数.

注: 尽管这个变限积分常被用来和Newton-Leibnitz公式混用来求定积分, 但是这并不表示$F$是$f$的原函数. 根据导函数的介值定理,
如果$F$是$f$的原函数, 则$f$不能有间断点, 这对于可积函数$f$是条件不足的.

\paragraph{定理 6.2.5 (积分第二中值定理). }

设 $f$ 在 $[a,b]$ 上可积. 

(1) 如果 $g$ 在 $[a,b]$ 上单调递减, 且 $g(x)\geqslant0$, $\forall x\in[a,b]$,
则存在 $\xi\in[a,b]$ 使得 
\[
\int_{a}^{b}f(x)g(x)dx=g(a)\cdot\int_{a}^{\xi}f(x)dx.
\]

(2) 如果 $g$ 在 $[a,b]$ 上单调递增, 且 $g(x)\geqslant0$, $\forall x\in[a,b]$,
则存在 $\eta\in[a,b]$ 使得 
\[
\int_{a}^{b}f(x)g(x)dx=g(b)\cdot\int_{\eta}^{b}f(x)dx.
\]

(3) 一般地, 如果 $g$ 为 $[a,b]$ 上的单调函数, 则存在 $\zeta\in[a,b]$, 使得 
\[
\int_{a}^{b}f(x)g(x)dx=g(a)\cdot\int_{a}^{\zeta}f(x)dx+g(b)\cdot\int_{\zeta}^{b}f(x)dx.
\]


\paragraph{例 6.2.2. }

\paragraph{设 $\beta\geqslant0,b>a>0$, 证明 
	\[
	\left|\int_{a}^{b}e^{-\beta x}\frac{\sin x}{x}dx\right|\leqslant\frac{2}{a}.
	\]
	证明. }

对 $g(x)=\frac{e^{-\beta x}}{x}$, $f(x)=\sin x$ 用积分第二中值公式, 存在 $\xi\in[a,b]$,
使得 
\[
\int_{a}^{b}e^{-\beta x}\frac{\sin x}{x}dx=\frac{e^{-\beta a}}{a}\cdot\int_{a}^{\xi}\sin xdx=\frac{e^{-\beta a}}{a}(\cos a-\cos\xi)
\]
这说明 
\[
\left|\int_{a}^{b}e^{-\beta x}\frac{\sin x}{x}dx\right|\leqslant2\frac{e^{-\beta a}}{a}\leqslant\frac{2}{a}.
\]


\paragraph{例 6.2.3. }

证明 $\lim_{A\rightarrow\infty}\int_{0}^{A}\frac{\sin x}{x}dx$ 存在. 

\paragraph{证明. }

在上例中取 $\beta=0$, 则当 $B>A>0$ 时, 有
\[
\left|\int_{0}^{B}\frac{\sin x}{x}dx-\int_{0}^{A}\frac{\sin x}{x}dx\right|=\left|\int_{A}^{B}\frac{\sin x}{x}dx\right|\leqslant\frac{2}{A}\rightarrow0\quad(A\rightarrow\infty),
\]


\paragraph{例3.5.15}

设$f\in C[a,b]$, $g$是周期为$T$的连续函数, 则
\[
\lim_{n\to\infty}\int_{a}^{b}f(x)g(nx)\ud x=\frac{1}{T}\int_{a}^{b}f(x)\ud x\int_{0}^{T}g(x)\ud x.
\]


\paragraph{例 6.2.6 (Riemann-Lebesgue)}

设 $f(x)$ 为 $[a,b]$ 上的可积函数, 则
\[
\lim_{\lambda\rightarrow+\infty}\int_{a}^{b}f(x)\sin\lambda xdx=0,\quad\lim_{\lambda\rightarrow+\infty}\int_{a}^{b}f(x)\cos\lambda xdx=0.
\]


\paragraph{证明. }

以第一个极限为例. 因为 $f$ 可积, 故任给 $\varepsilon>0$, 存在 $[a,b]$ 的分割 
\[
\pi:a=x_{0}<x_{1}<x_{2}<\cdots<x_{n}=b,
\]
使得 
\[
\sum_{i=1}^{n}\omega_{i}(f)\Delta x_{i}<\frac{1}{2}\varepsilon.
\]
又因为 $f$ 有界, 故存在 $K$, 使得 $|f(x)|\leqslant K$, $\forall x\in[a,b]$.
于是当 $\lambda>\frac{4nK}{\varepsilon}$ 时, 有 
\[
\begin{aligned}\left|\int_{a}^{b}f(x)\sin\lambda xdx\right| & =\left|\sum_{i=1}^{n}\int_{x_{i-1}}^{x_{i}}f(x)\sin\lambda xdx\right|\\
	& =\left|\sum_{i=1}^{n}\int_{x_{i-1}}^{x_{i}}\left[f(x)-f\left(x_{i-1}\right)\right]\sin\lambda xdx+\sum_{i=1}^{n}\int_{x_{i-1}}^{x_{i}}f\left(x_{i-1}\right)\sin\lambda xdx\right|\\
	& \leqslant\sum_{i=1}^{n}\int_{x_{i-1}}^{x_{i}}\left|f(x)-f\left(x_{i-1}\right)\right|dx+\sum_{i=1}^{n}\left|f\left(x_{i-1}\right)\right|\left|\int_{x_{i-1}}^{x_{i}}\sin\lambda xdx\right|\\
	& \leqslant\sum_{i=1}^{n}\omega_{i}(f)\Delta x_{i}+\sum_{i=1}^{n}K\frac{1}{\lambda}\left|\cos\lambda x_{i-1}-\cos\lambda x_{i}\right|\\
	& <\frac{1}{2}\varepsilon+\frac{2nK}{\lambda}<\varepsilon.
\end{aligned}
\]


\subsection{微积分基本公式}

\paragraph{定理 6.3.1 (微积分基本定理). }

设 $f$ 在 $[a,b]$ 上可积, 且在 $x_{0}\in[a,b]$ 处连续, 则 $F(x)=\int_{a}^{x}f(t)dt$
在 $x_{0}$ 处可导, 且 
\[
F^{\prime}\left(x_{0}\right)=f\left(x_{0}\right).
\]

这个定理说明变限积分是函数$f$的原函数的条件是$f$在$[a,b]$上连续, 而不能有第一类间断点. 但第二类间断点是可以有的.

\paragraph{定理 6.3.3 (Newton-Leibniz 公式). }

设 $F$ 在 $[a,b]$ 上可微, 且 $F^{\prime}=f$ 在 $[a,b]$ 上 Riemann 可积,
则 
\[
\int_{a}^{b}f(x)dx=F(b)-F(a).
\]
(此式又写为 $\int_{a}^{b}F^{\prime}(x)dx=F(b)-F(a)=\left.F(x)\right|_{a}^{b}$)

注: 可微函数的导函数不一定是可积的, 如函数 
\[
F(x)=\begin{cases}
	x^{2}\sin\frac{1}{x^{2}}, & x\neq0\\
	0, & x=0
\end{cases}
\]
在 $[0,1]$ 上可微. 进一步还可以构造导函数有界但不可积的例子.

\paragraph{例 6.3.2. }

设 $f$ 在 $[a,b]$ 上连续可微, $f(a)=0$, 则 
\[
\int_{a}^{b}f^{2}(x)dx\leqslant\frac{(b-a)^{2}}{2}\int_{a}^{b}\left[f^{\prime}(x)\right]^{2}dx.
\]


\paragraph{证明:}

\[
\begin{aligned}f^{2}(x) & =(f(x)-f(a))^{2}=\left[\int_{a}^{x}f^{\prime}(t)dt\right]^{2}\\
	& \leqslant\int_{a}^{x}\left[f^{\prime}(t)\right]^{2}dt\int_{a}^{x}1^{2}dt\quad(\text{Cauchy}-\text{Schwarz})\\
	& \leqslant(x-a)\int_{a}^{b}\left[f^{\prime}(t)\right]^{2}dt.
\end{aligned}
\]


\subsection{定积分的近似计算}

\paragraph{不等式1}

设$f$可微, 且$\left|f'(x)\right|\le M$, 则
\[
\left|\int_{a}^{b}f(x)\ud x-f\left(\frac{a+b}{2}\right)(b-a)\right|\le\frac{M}{4}(b-a)^{2}.
\]


\paragraph{证明:}

\[
\begin{aligned}\left|\int_{a}^{b}f(x)dx-f\left(\frac{a+b}{2}\right)(b-a)\right| & =\left|\int_{a}^{b}\left(f(x)-f\left(\frac{a+b}{2}\right)\right)dx\right|=\left|\int_{a}^{b}f^{\prime}(\xi)\left(x-\frac{a+b}{2}\right)dx\right|\\
	& \leqslant\int_{a}^{b}\left|f^{\prime}(\xi)\right|\left|x-\frac{a+b}{2}\right|dx\leqslant M\int_{a}^{b}\left|x-\frac{a+b}{2}\right|dx\\
	& =\frac{M}{4}(b-a)^{2}.
\end{aligned}
\]


\paragraph{不等式2}

设$f$二阶可微, 且$\left|f''(x)\right|\le M$, $\forall x\in[a,b]$. 则
\[
\left|\int_{a}^{b}f(x)\ud x-f\left(\frac{a+b}{2}\right)(b-a)\right|\le\frac{1}{24}M(b-a)^{3}.
\]


\paragraph{证明:}

用Taylor展开
\[
f(x)=f\left(\frac{a+b}{2}\right)+f^{\prime}\left(\frac{a+b}{2}\right)\left(x-\frac{a+b}{2}\right)+\frac{1}{2}f^{\prime\prime}(\xi)\left(x-\frac{a+b}{2}\right)^{2},
\]
两边积分, 得
\[
\int_{a}^{b}f(x)dx=f\left(\frac{a+b}{2}\right)(b-a)+\frac{1}{2}\int_{a}^{b}f^{\prime\prime}(\xi)\left(x-\frac{a+b}{2}\right)^{2}dx,
\]
所以
\[
\left|\int_{a}^{b}f(x)dx-f\left(\frac{a+b}{2}\right)(b-a)\right|\leqslant\frac{1}{2}M\int_{a}^{b}\left(x-\frac{a+b}{2}\right)^{2}dx=\frac{1}{24}M(b-a)^{3}.
\]


\paragraph{注:}

使用带积分型余项的Taylor公式
\[
f(x)=f\left(\frac{a+b}{2}\right)+f'\left(\frac{a+b}{2}\right)\left(x-\frac{a+b}{2}\right)+\int_{\frac{a+b}{2}}^{x}\frac{f''(t)}{1!}\left(t-\frac{a+b}{2}\right)\ud t.
\]
两边同时积分得到
\[
\int_{a}^{b}f(x)\ud x=f\left(\frac{a+b}{2}\right)(b-a)+\int_{a}^{b}\int_{\frac{a+b}{2}}^{x}f''(t)\left(t-\frac{a+b}{2}\right)\ud t\ud x,
\]
后者可以通过交换积分次序化简为
\begin{align*}
	\int_{a}^{b}\int_{\frac{a+b}{2}}^{x}f''(t)\left(t-\frac{a+b}{2}\right)\ud t\ud x & =\int_{a}^{b}f''(t)\left(t-\frac{a+b}{2}\right)\min\left\{ t-a,b-t\right\} \ud t\\
	& =-\int_{a}^{\frac{a+b}{2}}(t-a)\cdot f''(t)\left(t-\frac{a+b}{2}\right)\ud t+\int_{\frac{a+b}{2}}^{b}(b-t)\cdot f''(t)\left(t-\frac{a+b}{2}\right)\ud t\\
	& =-\int_{a}^{\frac{a+b}{2}}(t-a)\left(t-\frac{a+b}{2}\right)(f''(t)+f''(a+b-t))\ud t.
\end{align*}
所以有下面的恒等式
\[
\int_{a}^{b}f(x)\ud x=f\left(\frac{a+b}{2}\right)(b-a)-\int_{a}^{\frac{a+b}{2}}(t-a)\left(t-\frac{a+b}{2}\right)(f''(t)+f''(a+b-t))\ud t.
\]
而
\[
\left|\int_{a}^{\frac{a+b}{2}}(t-a)\left(t-\frac{a+b}{2}\right)(f''(t)+f''(a+b-t))\ud t\right|\le2M\left|\int_{a}^{\frac{a+b}{2}}(t-a)\left(t-\frac{a+b}{2}\right)\ud t\right|=\frac{M}{24}(b-a)^{3}.
\]


\paragraph{不等式3}

设$f\in C[a,b]$, 若$f$二阶可微, 且$\left|f''(x)\right|\le M$, $\forall x\in[a,b]$,
则
\[
\left|\int_{a}^{b}f(x)\ud x-\frac{f(a)+f(b)}{2}(b-a)\right|\le\frac{M}{12}(b-a)^{3}.
\]


\paragraph{证明:}

\begin{align*}
	\int_{a}^{b}f(x)\ud x & =(x-a)f(x)\mid_{a}^{b}-\int_{a}^{b}(x-a)f'(x)\ud(x-b)\\
	& =(b-a)f(b)+\int_{a}^{b}(x-b)\left(f'(x)+(x-a)f''(x)\right)\\
	& =(b-a)f(b)+(x-b)f(x)\mid_{a}^{b}-\int_{a}^{b}f(x)\ud x+\int_{a}^{b}(x-a)(x-b)f''(x)\ud x\\
	& =(b-a)(f(b)+f(a))-\int_{a}^{b}f(x)\ud x+\int_{a}^{b}(x-a)(x-b)f''(x)\ud x,
\end{align*}
所以
\[
\left|\int_{a}^{b}f(x)\ud x-\frac{f(a)+f(b)}{2}(b-a)\right|=\frac{1}{2}\left|\int_{a}^{b}(x-a)(x-b)f''(x)\ud x\right|\le\frac{M}{12}(b-a)^{3}.
\]

上面的等式的一些应用, 取$a=n$, $b=n+1$, 则有
\[
\int_{n}^{n+1}f(t)\ud t=\frac{f(n)+f(n+1)}{2}-\frac{1}{2}\int_{0}^{1}x(1-x)f''(x+n)\ud x.
\]
对$n$做累和, 
\[
\sum_{k=1}^{n}f(k)=\int_{1}^{n}f(t)\ud t+\frac{f(1)+f(n)}{2}+\frac{1}{2}\int_{0}^{1}x(1-x)\sum_{k=1}^{n-1}f''(x+n)\ud x.
\]
当取$f(x)=\frac{1}{x}$时, 得到
\[
H_{n}=\ln n+\frac{n+1}{2n}+\frac{1}{2}\int_{0}^{1}x(1-x)\sum_{k=1}^{n-1}\frac{2}{(x+n)^{3}}\ud x=\ln n+O(1).
\]
当取$f(x)=\ln x$时, 得到
\[
\ln n!=\ln\left(\sqrt{n}\left(\frac{n}{e}\right)^{n}\right)+1-\frac{1}{2}\int_{0}^{1}x(1-x)\sum_{k=1}^{n-1}\frac{1}{(x+n)^{2}}\ud x=\ln\left(\sqrt{n}\left(\frac{n}{e}\right)^{n}\right)+O(1),
\]
所以极限
\[
\lim_{n\to\infty}\frac{n!}{\sqrt{n}\left(\frac{n}{e}\right)^{n}}=C.
\]


\paragraph{注:}

若$f(x)\in C^{4}[a,b]$, 则
\[
\int_{a}^{b}f(x)dx-\frac{f(a)+f(b)}{2}(b-a)=\frac{1}{24}\int_{a}^{b}f^{(4)}(x)(x-a)^{2}(x-b)^{2}dx-\frac{1}{12}(b-a)^{2}\left[f^{\prime}(b)-f^{\prime}(a)\right].
\]


\paragraph{关于习题3}

是否存在常数 $C$, 使得对于满足条件 $\left|f^{\prime\prime\prime}(x)\right|\leqslant M$
的任意函数 $f$ 有如下估计: 
\[
\left|\int_{a}^{b}f(x)dx-\frac{f(a)+f(b)}{2}(b-a)\right|\leqslant CM(b-a)^{4}.
\]


\paragraph{解:}

条件存在. 不等式相当于
\[
\frac{1}{2}\left|\int_{a}^{b}(x-a)(x-b)f''(x)\ud x\right|\le CM(b-a)^{4}.
\]

由于$\left|f'''\right|\le M$, 所以$-M(x-a)\le f''(x)-f''(a)\le M(x-a)$.
所以
\[
\int_{a}^{b}(x-a)(x-b)f''(x)\ud x\le\int_{a}^{b}(x-a)(x-b)(f''(a)-M(x-a))\ud x=\frac{M}{12}(b-a)^{4}-\frac{f''(a)}{6}(b-a)^{3},
\]
\[
\int_{a}^{b}(x-a)(x-b)f''(x)\ud x\ge\int_{a}^{b}(x-a)(x-b)(f''(a)+M(x-a))\ud x=-\frac{M}{12}(b-a)^{4}-\frac{f''(a)}{6}(b-a)^{3}.
\]
当$f''(a)=0$时, 上面的$C$存在, 而一般情况的$f$, 上面的$C$是不存在的.

当问题加上对于任意的$a,b\in D_{f}$时, 常数$C$也是不存在的. 这相当于对于任意的$a,b$, 
\[
\frac{1}{2}\left|\int_{a}^{b}(x-a)(x-b)f''(x)\ud x\right|\le CM(b-a)^{4}.
\]
由介值定理, 存在$\xi\in(a,b)$, s.t. 
\[
\left|f''(\xi)\right|(b-a)^{3}\le CM(b-a)^{4}.
\]
令$b\to a^{+}$, 得到$f''(a)\equiv0$, 也与$f$的任意性矛盾.

\subsection{作业}

6. 设 $f(x)$ 为 $[0,1]$ 上的非负可积函数, 且 $\int_{0}^{1}f(x)dx=0$. 证明, 任给
$\varepsilon>0$, 均存在子区间 $[\alpha,\beta]$, 使得 $f(x)<\varepsilon$,
$\forall x\in[\alpha,\beta]$.

8. 设 $f(x)$ 在 $[a,b]$ 上可积, 且存在常数 $C>0$, 使得 $|f(x)|\geqslant C(a\leqslant x\leqslant b)$.
证 明 $\frac{1}{f}$ 在 $[a,b]$ 上也是可积的.

11. 设 $f(x)>0$ 为 $[a,b]$ 上的可积函数, 证明 $\int_{a}^{b}f(x)dx>0$.

2. 设 $f(x)$ 是 $[a,b]$ 上定义的函数. 如果 $f^{2}(x)$ 可积, 则 $|f(x)|$ 也可积.

6. 设 $f(x)\geqslant0$ 在 $[a,b]$ 上可积, $\lambda\in\mathbb{R}$, 则
\[
\left(\int_{a}^{b}f(x)\cos\lambda xdx\right)^{2}+\left(\int_{a}^{b}f(x)\sin\lambda xdx\right)^{2}\leqslant\left[\int_{a}^{b}f(x)dx\right]^{2}.
\]
(提示: $f=\sqrt{f}\cdot\sqrt{f}$, 用Cauchy-Schwarz不等式.)

9. 设 $f(x)$ 为 $[0,1]$ 上的连续函数, 则 $\lim_{n\rightarrow+\infty}n\int_{0}^{1}x^{n}f(x)dx=f(1)$.
(提示: $nx^{n}$ 在 $[0,1]$ 上积分趋于 1 , 在 $[0,\delta]$ 上很小, 如果 $0<\delta<1$.)

11. 设 $f(x)$ 为 $[a,b]$ 上的可积函数, 则任给 $\varepsilon>0$, 存在连续函数 $g(x)$,
使得 $\inf f\leqslant g(x)\leqslant\sup f$, 且 
\[
\int_{a}^{b}|f(x)-g(x)|dx<\varepsilon.
\]

12. 设 $f(x)$ 在 $[c,d]$ 上可积, 设 $[a,b]\subset(c,d)$, 则 
\[
\lim_{h\rightarrow0}\int_{a}^{b}|f(x+h)-f(x)|dx=0.
\]
\section{定积分的应用和推广}
\subsection{定积分的应用}

\paragraph{曲线的长度}

设$I=[\alpha,\beta]$, 映射$\sigma:I\to\RR^{2}$, $t\mapsto\left(x(t),y(t)\right)$,
$t\in I$. 

如果$x(t)$, $y(t)$为连续函数, 则称$\sigma$为$\RR^{2}$上的连续曲线. 

如果$x(t)$, $y(t)\in C^{1}$, 则称$\sigma$为$C^{1}$曲线.

定义$\sigma$的长度为
\[
L(\sigma)=\int_{\alpha}^{\beta}\left[\left(x'(t)\right)^{2}+\left(y'(t)\right)^{2}\right]^{1/2}\ud t.
\]


\paragraph{例7.1.1. }

求摆线
\[
\left(x(t),y(t)\right)=\left(a(t-\sin t),a(1-\cos t)\right),\quad a>0.
\]
一拱的长度.

\[
\begin{aligned}l & =\int_{0}^{2\pi}\left[\left(x^{\prime}(t)\right)^{2}+\left(y^{\prime}(t)\right)^{2}\right]^{\frac{1}{2}}dt\\
	& =\int_{0}^{2\pi}a\left[(1-\cos t)^{2}+\sin^{2}t\right]^{\frac{1}{2}}dt\\
	& =2a\int_{0}^{2\pi}\sin\frac{t}{2}dt=8a.
\end{aligned}
\]


\paragraph{简单图形的面积}

\[
S=\int_{a}^{b}f(x)dx.
\]
当$f$变号时, 上式称为代数面积和.

设平面曲线$\sigma$的极坐标方程为$r=r(\theta)$, $r(\theta)\in C[\alpha,\beta]$,
\[
S=\frac{1}{2}\int_{\alpha}^{\beta}r^{2}(\theta)d\theta.
\]

设曲线$\sigma$上的点满足$\sigma(t)=\left(x(t),y(t)\right)$, $t\in[\alpha,\beta]$.
则$\sigma$, $x=a$, $x=b$和$y=0$围成的曲边梯形的面积为
\[
S=\int_{\alpha}^{\beta}y(t)x'(t)dt.
\]
面积公式也可以改写成
\[
S=\frac{1}{2}\left|\int_{\alpha}^{\beta}\left[y(t)x'(t)-y'(t)x(t)\right]\right|\ud t.
\]


\paragraph{旋转曲面的面积}

设$\sigma$为平面曲线$\sigma(t)=(x(t),y(t))$, $t\in[\alpha,\beta${]},
$y(t)\ge0$. $\sigma$绕$x$轴旋转所得曲面的面积为
\[
S=\int_{\alpha}^{\beta}2\pi y(t)\left[\left(x^{\prime}(t)\right)^{2}+\left(y^{\prime}(t)\right)^{2}\right]^{\frac{1}{2}}dt.
\]


\paragraph{简单立体的体积}

(1) 平行截面之间的立体体积

设$\Omega$是$\RR^{3}$中的一块立体区域, 夹在平面$x=a$和$x=b$, ($a<b$) 之间. 记$S(x)$为$x\in[a,b]$处垂直于$x$轴的平面截$\Omega$的截面面积函数.
如果$S(x)\in C[a,b]$, 则$\Omega$的体积为
\[
V=\int_{a}^{b}S(x)\ud x.
\]

(2) 旋转体体积

设$f\in C[a,b]$, 
\[
\Omega=\left\{ (x,y,z)\mid x\in[a,b],y\in[-\left|f(x)\right|,\left|f(x)\right|],\left|z\right|\le\sqrt{x^{2}+y^{2}}\right\} .
\]
则
\[
V(\Omega)=\int_{a}^{b}\pi f^{2}(x)\ud x.
\]


\subsection{广义积分}

\paragraph{定义 7.2.1 (无穷积分). }

设 $a\in\mathbb{R}$, 定义在 $[a,+\infty)$ 中的函数 $f$ 如果在任何有限区间 $[a,A]$
上都是 Riemann 可积的, 且极限 
\[
\lim_{A\rightarrow+\infty}\int_{a}^{A}f(x)dx
\]
存在 (且有限), 则称无穷积分 $\int_{a}^{+\infty}f(x)dx$ 存在或收敛, 记为 
\[
\int_{a}^{+\infty}f(x)dx=\lim_{A\rightarrow+\infty}\int_{a}^{A}f(x)dx
\]
否则就称无穷积分 $\int_{a}^{+\infty}f(x)dx$ 不存在或发散.

类似的,我们可以定义无穷积分$\int_{-\infty}^{a}f(x)\ud x$和$\int_{-\infty}^{\infty}f(x)\ud x$.
如果极限
\[
\lim_{A\to+\infty}\int_{-A}^{A}f(x)\ud x
\]
存在, 它和上面定义的无穷积分是不等价的, 称为Cauchy主值积分, 记为
\[
(V.P.)\int_{-\infty}^{\infty}f(x)\ud x\coloneqq\lim_{A\to+\infty}\int_{-A}^{A}f(x)\ud x.
\]


\paragraph{无穷积分的Cauchy准则}

$f(x)$在$[a,+\infty)$上的积分收敛, 当且仅当, 对于任意的$\epsilon>0$, 存在$M=M(\epsilon)$,
使得对于任何$B>A>M$时, 有
\[
\left|\int_{A}^{B}f(x)\ud x\right|<\epsilon.
\]


\paragraph{例7.2.1}

无穷积分$\int_{1}^{+\infty}\frac{1}{x^{p}}\ud x$, ($p\in\RR$) 仅在$p>1$时收敛.

\paragraph{例7.2.2}

$\int_{-\infty}^{\infty}\frac{1}{1+x^{2}}\ud x=\arctan x\mid_{-\infty}^{\infty}=\frac{\pi}{2}-\left(-\frac{\pi}{2}\right)=\pi$.

\paragraph{定义7.2.2 (瑕积分)}

设函数$f$在任何区间$[a',b]$, ($a<a'<b$) 上均Riemann可积, 如果极限
\[
\lim_{a'\to a+}\int_{a'}^{b}f(x)\ud x
\]
存在且有限, 则称瑕积分$\int_{a}^{b}f(x)\ud x$存在或收敛, 记为
\[
\int_{a}^{b}f(x)\ud x=\lim_{a'\to a+}\int_{a'}^{b}f(x)\ud x,
\]
否则称瑕积分
\[
\int_{a}^{b}f(x)\ud x=\lim_{a'\to a+}\int_{a'}^{b}f(x)\ud x,
\]
否则称瑕积分$\int_{a}^{b}f(x)\ud x$不存在或发散. 

如果$f$在$a$附近无界, 则$f$在$[a,b]$上不是Riemann可积的, 称$a$为$f$的瑕点.

无穷积分和瑕积分统称为广义积分, 也称为反常积分.

\paragraph{例7.2.3}

瑕积分$\int_{0}^{1}\frac{1}{x^{p}}\ud x$仅在$p<1$时收敛.

运算法则: 分部积分, 变量替换, 积分区间可加性, 线性性质.

\paragraph{例7.2.5}

$\int_{0}^{1}\ln x\ud x=\lim_{\epsilon\to0+}\int_{\epsilon}^{1}\ln x\ud x=\lim_{\epsilon\to0+}\left(x\ln x-x\right)\mid_{\epsilon}^{1}=-1$.

\paragraph{例7.2.6}

求Fresnel积分$\int_{0}^{+\infty}\cos(x^{2})\ud x$.

\[
\int_{1}^{+\infty}\cos\left(x^{2}\right)dx=\frac{1}{2}\int_{1}^{+\infty}\frac{\cos t}{\sqrt{t}}dt.
\]
\[
\begin{aligned}\left|\int_{A}^{B}\frac{\cos t}{\sqrt{t}}dt\right| & =\left|\frac{\sin t}{\sqrt{t}}\right|_{A}^{B}+\frac{1}{2}\int_{A}^{B}\frac{\sin t}{t^{\frac{3}{2}}}dt\mid\\
	& \leqslant\frac{1}{\sqrt{A}}+\frac{1}{\sqrt{B}}+\frac{1}{2}\int_{A}^{B}t^{-\frac{3}{2}}dt=\frac{2}{\sqrt{A}}\rightarrow0\quad(B>A\rightarrow+\infty).
\end{aligned}
\]

\begin{align*}
	\int_{0}^{+\infty}\cos(x^{2})\ud x & =\frac{1}{2}\int_{0}^{+\infty}\frac{\cos t}{\sqrt{t}}\ud t\\
	& =\frac{1}{2}\int_{0}^{+\infty}\cos t\left(\frac{2}{\sqrt{\pi}}\int_{0}^{+\infty}\ue^{-tx^{2}}\ud x\right)\ud t\\
	& =\frac{1}{\sqrt{\pi}}\int_{0}^{+\infty}\int_{0}^{+\infty}\ue^{-tx^{2}}\cos t\ud t\ud x\\
	& =\frac{1}{\sqrt{\pi}}\int_{0}^{+\infty}\frac{x^{2}}{1+x^{4}}\ud x=\frac{1}{2\sqrt{\pi}}\int_{0}^{\pi/2}\tan^{1/2}t\ud t\\
	& =\frac{1}{2\sqrt{\pi}}\cdot\frac{1}{2}B\left(\frac{3/2}{2},\frac{1/2}{2}\right)=\frac{1}{4\sqrt{\pi}}\cdot\frac{\Gamma(1/4)\Gamma(3/4)}{\Gamma(1)}\\
	& =\frac{1}{4\sqrt{\pi}}\cdot\frac{\pi}{\sin\frac{\pi}{4}}=\frac{\sqrt{2\pi}}{4}.
\end{align*}
其中
\[
\int\ue^{-tx^{2}}\cos t\ud t=\frac{\ue^{-tx^{2}}\left(-\cos t\cdot x^{2}+\sin t\right)}{1+x^{4}},
\]
所以
\[
\int_{0}^{+\infty}\ue^{-tx^{2}}\ud x=\frac{1}{\sqrt{t}}\int_{0}^{+\infty}\ue^{-x^{2}}\ud x=\frac{\sqrt{\pi}}{2\sqrt{t}}.
\]
综上, 并类似的证明
\[
\int_{\RR}\cos(x^{2})\ud x=\sqrt{\frac{\pi}{2}},\qquad\int_{\RR}\sin(x^{2})\ud x=\sqrt{\frac{\pi}{2}}.
\]


\subsubsection{作业}

7. 设 $f(x)>0$, 如果 $f(x)$ 在 $[a,+\infty)$ 上广义可积, 则 $\int_{a}^{+\infty}f(x)dx>0$. 

在$[a,A]$上, $f(x)$的不连续点集是零测集, 存在不连续点集的至多可数个开区间$\left\{ I_{i}\right\} $,
使得
\[
\sum\left|I_{i}\right|\le\epsilon.
\]
因$A-a>\epsilon$, 所以存在连续点, 设为$x_{0}$, 则存在$\delta>0$, 使得任何$\left|x-x_{0}\right|<\delta$,
$f(x)>0$, 与$\int_{a}^{A}f(x)\ud x=0$矛盾.

8. 设 $f(x)$ 在 $[a,+\infty)$ 上广义可积, 如果 $f(x)$ 在 $[a,+\infty)$ 中一致连续,
则 
\[
\lim_{x\rightarrow+\infty}f(x)=0.
\]
(提示: 先用 Cauchy 准则和中值定理找收敛子列.) 对于任何$\epsilon>0$, 存在$\delta>0$, 只要$\left|x-y\right|<\delta$,
就有$\left|f(x)-f(y)\right|<\epsilon$. 又对于这样小的$\epsilon$, 存在$M>0$,
s.t. 对于任何$A>M$,
\[
\left|\int_{A}^{A+\delta}f(x)\ud x\right|<\epsilon.
\]
则存在$\xi\in(A,A+\delta)$, 使得
\[
\left|f(\xi)\right|<\epsilon.
\]
所以对于任何$x\in(A,A+\delta)$, 有
\[
\left|f(x)\right|\le\left|f(\xi)\right|+\left|f(x)-f(\xi)\right|\le\epsilon+\epsilon
\]

9. 设 $f(x)$ 在 $[a,+\infty)$ 上广义可积, 如果 $f(x)$ 在 $[a,+\infty)$ 中可导,
且导函数 $f^{\prime}(x)$ 有界, 则 $\lim_{x\rightarrow+\infty}f(x)=0$. (提示:
用上一题.)

11. 举例说明, 当无穷积分 $\int_{a}^{+\infty}f(x)dx$ 收敛, 且 $f(x)$ 为正连续函数时,
无穷积分 $\int_{a}^{+\infty}f^{2}(x)dx$ 不一定收敛.

磨光函数
\[
\sum_{k=1}^{n}n\chi_{[n,n+1/n^{3}]}(x)
\]

12. 举例说明, 当无穷积分 $\int_{a}^{+\infty}f(x)dx$ 收敛, 且 $f(x)$ 为正连续函数时,
不一定有 
\[
\lim_{x\rightarrow+\infty}f(x)=0.
\]

比如图形类似于下式的函数
\[
\sum_{k=1}^{\infty}n\chi_{[n,n+1/n]}
\]


\subsection{广义积分的收敛判别法}

\paragraph{Thm 7.3.1.}

设 $f\geqslant0$, 则无穷积分 $\int_{a}^{+\infty}f(x)dx$ 收敛当且仅当 
\[
F(A)=\int_{a}^{A}f(x)dx
\]
是 $A\in[a,+\infty)$ 的有界函数; 对瑕积分有完全类似的结果.

\paragraph{定理 7.3.2. (比较判别法)}

设 $0\leqslant f\leqslant Mg,M>0$ 为常数, 则当无穷积分 $\int_{a}^{+\infty}g(x)dx$
收敛时, 无穷积分 $\int_{a}^{+\infty}f(x)dx$ 也收敛; 当无穷积分 $\int_{a}^{+\infty}f(x)dx$
发散时, 无穷积分 $\int_{a}^{+\infty}g(x)dx$ 也发散; 瑕积分有完全类似的结果.

\paragraph{$M$的求法}

求$\lim_{x\to+\infty}\frac{f(x)}{g(x)}=l$.

$0<l<\infty$时, $\int_{a}^{+\infty}f(x)\ud x$与$\int_{a}^{+\infty}g(x)\ud x$同收敛.

$l=0$时, $\int_{a}^{+\infty}g(x)\ud x$收敛可以推出$\int_{a}^{+\infty}f(x)\ud x$收敛.
注: $0\le f\le Mg$不可省, 否则取$f(x)=\left|\frac{\sin x}{x\ln x}\right|$,
$g(x)=\frac{\sin x}{x}$.

$l=+\infty$时, $\int_{a}^{+\infty}g(x)\ud x$发散, 则$\int_{a}^{+\infty}f(x)\ud x$发散.

\paragraph{Cauchy判别法}

将$f$与$x^{-p}$比较, (无穷积分) (瑕积分有类似结论)

1. 若$p>1$, 且存在$c>0$, s.t. $0\le f(x)\le\frac{c}{x^{p}}$, ($\forall x\ge x_{0}$),
则$\int_{a}^{+\infty}f(x)\ud x$收敛.

2. 若$p\le1$, 且存在$c>0$, s.t. $f(x)\ge\frac{c}{x^{p}}$, ($\forall x\ge x_{0}$),
则$\int_{a}^{+\infty}f(x)\ud x$发散.

设$\lim_{x\to\infty}x^{p}f(x)=l$.

3. 若$p>1$, $0\le l<+\infty$, 则$\int_{a}^{+\infty}f(x)\ud x$收敛.

4. 若$p\le1$, $0<l\le+\infty$, 则$\int_{a}^{+\infty}f(x)\ud x$发散.

\paragraph{例7.3.3}

判断$\int_{1}^{+\infty}\frac{1}{x(1+\ln x)}\ud x$的敛散性.

对于一般函数$f$, 定义
\[
f^{+}(x)=\max\{0,f(x)\},\quad f^{-}(x)=\max\{0,-f(x)\}.
\]
若$\int f^{\pm}$收敛, 则$f$收敛. (绝对收敛)

若$\int f$收敛, 但$\int\left|f\right|$发散. (条件收敛)

\paragraph{例7.3.5}

判断$\int_{1}^{+\infty}\cos x^{p}\ud x$, ($p>1$), 的敛散性.

\[
\int_{1}^{\infty}\cos t\cdot t^{\frac{1}{p}-1}\ud t,
\]
取$B>A\gg1$, 则
\[
\left|\int_{A}^{B}\frac{\cos t}{t^{1-\frac{1}{p}}}\ud t\right|=\left|\frac{1}{A^{1-1/p}}\int_{A}^{\xi}\cos t\ud t+\frac{1}{B^{1-1/p}}\int_{\xi}^{B}\cos t\ud t\right|\le\frac{4}{A^{1-1/p}}\to0,\quad A\to+\infty.
\]
但
\[
\left|\cos x^{p}\right|\ge\cos^{2}x^{p}=\frac{1}{2}\left(1+\cos2x^{p}\right),
\]
反证$\int\left|\cos x^{p}\right|$不收敛.

\paragraph{定理 $7.3.3$ (Dirichlet). }

设 $F(A)=\int_{a}^{A}f(x)dx$ 在 $[a,+\infty)$ 中有界, 函数 $g(x)$ 在 $[a,+\infty)$
中单调, 且 $\lim_{x\rightarrow+\infty}g(x)=0$, 则积分 $\int_{a}^{+\infty}f(x)g(x)dx$
收敛.

pf. $\left|F(A)\right|\le C$, $\forall A\ge a$. 所以
\[
\left|\int_{A}^{B}f(x)\ud x\right|\le2C,\quad\forall A,B\ge a.
\]
$g(x)=o(1)$, $x\to+\infty$. 所以$\forall\epsilon>0$, $\exists M>0$,
s.t. $\forall x>M$, $\left|g(x)\right|\le\frac{\epsilon}{4C}$.
\begin{align*}
	\left|\int_{A}^{B}f(x)g(x)\ud x\right| & =\left|g(A)\int_{A}^{\xi}f(x)\ud x+g(B)\int_{\xi}^{B}f(x)\ud x\right|\\
	& \le\frac{\epsilon}{4C}\cdot2C\cdot2=\epsilon.
\end{align*}


\paragraph{例7.3.6.}

判断积分$\int_{0}^{+\infty}\frac{\sin x}{x^{p}}\ud x$, ($0<p<2$) 的敛散性.

pf. $\frac{\sin x}{x^{p}}\sim\frac{1}{x^{p-1}}$, ($x\to0+$). 所以$\int_{0}^{1}\frac{\sin x}{x^{p}}\ud x$与$\int_{0}^{1}x^{1-p}\ud x$同敛散,
($p<2$).

$\int_{1}^{A}\sin x\ud x$有界, $\frac{1}{x^{p}}\searrow0$, 由Dirichlet判别法,
$\int_{1}^{\infty}\frac{\sin x}{x^{p}}\ud x$收敛.

$0<p\le1$时, $\int_{1}^{\infty}\left|\frac{\sin x}{x^{p}}\right|\ud x$发散.

$1<p<2$时, $\int_{1}^{\infty}\left|\frac{\sin x}{x^{p}}\right|\ud x$收敛.

\paragraph{定理 7.3.4 (Abel). }

如果广义积分 $\int_{a}^{+\infty}f(x)dx$ 收敛, 函数 $g(x)$ 在 $[a,+\infty)$
中单调有界, 则积分 $\int_{a}^{+\infty}f(x)g(x)dx$ 也收敛.

pf. $g$有界, 所以$\left|g(x)\right|\le c$, $\forall x\in[a,+\infty)$.

$\int f$收敛, 所以$\forall\epsilon$>0, $\exists M>0$, s.t. $\forall B>A>M$,
$\left|\int_{A}^{B}f(x)dx\right|\le\frac{\epsilon}{2c}$. 
\[
\begin{aligned}\left|\int_{A}^{B}f(x)g(x)dx\right| & =\left|g(A)\int_{A}^{\xi}f(x)dx+g(B)\int_{\xi}^{B}f(x)dx\right|\\
	& \leqslant c\left|\int_{A}^{\xi}f(x)dx\right|+c\left|\int_{\xi}^{B}f(x)dx\right|\\
	& \leqslant c\frac{\epsilon}{2c}+c\frac{\epsilon}{2c}=\epsilon.
\end{aligned}
\]


\paragraph{例7.3.7}

设 $a\geqslant0$, 研究积分 $\int_{0}^{+\infty}e^{-ax}\frac{\sin x}{x}dx$
的敛散性.

pf. $e^{-ax}\searrow0$, $\int_{0}^{A}\frac{\sin x}{x}$有界, 由Dirichlet判别法,
$\int_{0}^{\infty}e^{-ax}\frac{\sin x}{x}$收敛.

$e^{-ax}\searrow$有界, $\int_{0}^{\infty}\frac{\sin x}{x}$收敛, 由Abel判别法,
$\int_{0}^{\infty}e^{-ax}\frac{\sin x}{x}$收敛.

\subsubsection{作业}

5. 设 $f(x)$ 在 $[1,+\infty)$ 中连续, 如果 $\int_{1}^{+\infty}f^{2}(x)dx$
收敛, 则 $\int_{1}^{+\infty}\frac{f(x)}{x}dx$ 绝对收敛. (提示: 用Cauchy不等式.)

6. 设 $f(x)$ 在 $[a,A](A<\infty)$ 上均可积. 如果 $\int_{a}^{+\infty}|f(x)|dx$
收敛, 且 $\lim_{x\rightarrow+\infty}f(x)=L$, 证明 $L=0$, 且 $\int_{a}^{+\infty}f^{2}(x)dx$
也收敛.

7. 设 $f(x)$ 在 $[a,+\infty)$ 中单调递减, 且 $\int_{a}^{+\infty}f(x)dx$
收敛, 证明 $\lim_{x\rightarrow+\infty}xf(x)=0$. (提示: 在区间 $[A/2,A]$ 上估计积分.)

8. 设 $f(x)$ 在 $[a,+\infty)$ 中单调递减趋于零, 且 $\int_{a}^{+\infty}\sqrt{f(x)/x}dx$
收敛, 则 $\int_{a}^{+\infty}f(x)dx$ 也收敛. (提示: 利用上题, 比较被积函数.)

pf. 取$\epsilon=1/2$, 则存在$X>a$, s.t. $\forall x>X$, 
\[
\frac{x}{2}\cdot\sqrt{\frac{f(x)}{x}}<\int_{x/2}^{x}\sqrt{\frac{f(t)}{t}}dt<\epsilon=\frac{1}{2},
\]
即$\sqrt{xf(x)}<1$. 所以$f(x)<\sqrt{\frac{f(x)}{x}}$, $\forall x>X$.

9. 设 $\int_{a}^{+\infty}f(x)dx$ 收敛, 如果 $x\rightarrow+\infty$ 时 $xf(x)$
单调递减趋于零, 则 
\[
\lim_{x\rightarrow+\infty}xf(x)\ln x=0.
\]

pf. 对于任意的$\epsilon>0$, 因为$\int_{a}^{+\infty}f(x)dx$收敛, 则存在$M>\max(0,a)$,
s.t. 对于任意的$x>M$, 有
\[
xf(x)\int_{\sqrt{x}}^{x}\frac{1}{t}dt\le\int_{\sqrt{x}}^{x}tf(t)\cdot\frac{1}{t}dt=\int_{\sqrt{x}}^{x}f(t)dt<\frac{\epsilon}{2}.
\]
即$xf(x)\ln x<\epsilon$.

10. 设 $f(x)>0$ 在 $[0,+\infty)$ 中连续, 且 $\int_{0}^{+\infty}\frac{dx}{f(x)}$
收敛, 证明 
\[
\lim_{\lambda\rightarrow+\infty}\frac{1}{\lambda}\int_{0}^{\lambda}f(x)dx=+\infty.
\]

pf. Cauchy不等式:
\[
\int_{0}^{\infty}\frac{dx}{f(x)}\int_{0}^{\lambda}f(x)dx\ge\left(\int_{0}^{\lambda}dx\right)^{2}=\lambda^{2}.
\]

11. 研究广义积分 
\[
\int_{2}^{+\infty}\frac{dx}{x^{p}\ln^{q}x}dx\quad(p,q\in\mathbb{R})
\]
的敛散性.

pf. $p>1$收敛, $p<1$发散.

$p=1$, $q>1$时收敛; $p=1$, $q\le1$时发散.

\subsection{广义积分的例子}

\paragraph{例7.4.1}

求
\[
I=\int_{0}^{+\infty}e^{-ax}\sin bxdx,\qquad(a>0).
\]

pf. 原函数可求
\[
F(x)=-\frac{a\sin bx+b\cos bx}{a^{2}+b^{2}}e^{-ax},
\]
\[
I=F(\infty)-F(0)=\frac{b}{a^{2}+b^{2}}.
\]

令$b=n$, $a=1$, 
\[
\int_{0}^{\infty}e^{-x}\sin nxdx=\frac{n}{n^{2}+1},
\]
所以
\[
\int_{0}^{\infty}e^{-x}\frac{\sin nx}{n}dx=\frac{1}{n^{2}+1},
\]
所以
\begin{align*}
	\sum_{n=1}^{\infty}\frac{1}{n^{2}+1} & =\int_{0}^{\infty}e^{-x}\left\langle \frac{\pi-x}{2}\right\rangle dx\\
	& =\sum_{k=0}^{\infty}\int_{2k\pi}^{2(k+1)\pi}e^{-x}\frac{\pi-(x-2k\pi)}{2}dx\\
	& =\sum_{k=0}^{\infty}\int_{0}^{2\pi}e^{-(x+2k\pi)}\frac{\pi-x}{2}dx\\
	& =\int_{0}^{2\pi}\frac{e^{\pi}}{e^{\pi}-e^{-\pi}}e^{-x}\frac{\pi-x}{2}dx\\
	& =\frac{1}{2}\frac{e^{\pi}}{e^{\pi}-e^{-\pi}}\cdot\left((\pi-1)+e^{-2\pi}(\pi+1)\right)=\frac{\pi\coth\pi-1}{2}.
\end{align*}


\paragraph{例7.4.2}

求
\[
I=\int_{-\pi}^{+\pi}\frac{1-r^{2}}{1-2r\cos x+r^{2}}dx,\qquad(0<r<1).
\]

pf. 令$t=\tan\frac{x}{2}$, 则
\[
I=\int_{\RR}\frac{2(1-r^{2})}{(1-r)^{2}+(1+r)^{2}t^{2}}dt=2\arctan\left(\frac{1+r}{1-r}t\right)\mid_{-\infty}^{\infty}=2\pi.
\]
另外注意
\[
\sum_{n\in\ZZ}r^{\left|n\right|}e^{inx}=\frac{1-r^{2}}{1-2r\cos x+r^{2}}.
\]


\paragraph{例7.4.3}

求
\[
I=\int_{0}^{\infty}\frac{1}{1+x^{4}}dx.
\]

pf. $\alpha>1$. 取$x^{\alpha/2}=\tan t$, 则$x=\left(\tan t\right)^{2/\alpha}$.
\[
I=\int_{0}^{\infty}\frac{1}{1+x^{\alpha}}dx=\int_{0}^{\pi/2}\frac{2}{\alpha}\left(\tan t\right)^{2/\alpha-1}dt=\frac{2}{\alpha}\cdot\frac{1}{2}B\left(\frac{1}{\alpha},1-\frac{1}{\alpha}\right)=\frac{1}{\alpha}\Gamma\left(\frac{1}{\alpha}\right)\Gamma\left(1-\frac{1}{\alpha}\right)=\frac{\pi}{\alpha\sin\frac{\pi}{\alpha}}.
\]


\paragraph{例7.4.5}

求
\[
I=\int_{0}^{+\infty}\frac{1}{\left(1+x^{2}\right)^{n}}dx\qquad(n\geqslant1).
\]

pf. 令$x=\tan t$, 
\[
\int_{0}^{\infty}\frac{1}{(1+x^{2})^{\alpha}}dx=\int_{0}^{\pi/2}\cos^{2\alpha-2}tdt=\frac{1}{2}B\left(\frac{1}{2},\frac{2\alpha-1}{2}\right).
\]
当$\alpha=n$时, $I=\frac{\pi}{2}\cdot\frac{(2n-3)!!}{(2n-2)!!}$.

\paragraph{例7.4.8}

求
\[
\int_{0}^{\infty}\frac{\sin^{2}x}{x^{2}}dx.
\]

pf. 注意$\sin^{2}nx-\sin^{2}(n-1)x=\sin x\cdot\sin\left[(2n-1)x\right]$,
则
\[
\frac{\sin^{2}nx}{\sin^{2}x}=\sum_{k=1}^{n}\frac{\sin(2k-1)x}{\sin x}.
\]
又注意
\[
\sin(2n-1)x-\sin(2n-3)x=2\sin x\cos\left[2(n-1)x\right],
\]
\[
\sin(2n-1)x=2\sin x\left(\frac{1}{2}+\sum_{k=1}^{n-1}\cos2kx\right),
\]
所以
\[
\int_{0}^{\pi/2}\frac{\sin^{2}nx}{\sin^{2}x}=\sum_{k=1}^{n}\int_{0}^{\pi/2}\frac{\sin(2k-1)x}{\sin x}dx=\sum_{k=1}^{n}\int_{0}^{\pi/2}1+2\sum_{j=1}^{k-1}\cos2jxdx=n\cdot\frac{\pi}{2}.
\]
最后由Riemann Lebesgue引理
\begin{align*}
	\int_{0}^{\infty}\frac{\sin^{2}x}{x^{2}}dx & =\lim_{n\to\infty}\int_{0}^{\frac{n\pi}{2}}\frac{\sin^{2}x}{x^{2}}dx\\
	& =\lim_{n\to\infty}\int_{0}^{\pi/2}\frac{\sin^{2}nx}{nx^{2}}dx\\
	& =\lim_{n\to\infty}\left(\int_{0}^{\pi/2}\frac{\sin^{2}nx}{n\sin^{2}x}dx+\int_{0}^{\pi/2}\frac{\sin^{2}nx}{n}\left(\frac{1}{x^{2}}-\frac{1}{\sin^{2}x}\right)dx\right)\\
	& =\frac{\pi}{2}+O\left(\frac{1}{n}\right).
\end{align*}

也可以不用Riemann Lebesgue引理, 需要用夹逼原理, 使用不等式
\[
x-\frac{x^{3}}{3!}\le\sin x\le x,\qquad\forall x\in\left[0,\frac{\pi}{2}\right].
\]
则
\begin{align*}
	\frac{1}{n}\int_{0}^{\pi/2}\frac{\sin^{2}nx}{x^{2}}dx & \le\frac{1}{n}\int_{0}^{\pi/2}\frac{\sin^{2}nx}{\sin^{2}x}dx=\frac{\pi}{2}\le\frac{1}{n}\int_{0}^{\pi/2}\frac{\sin^{2}nx}{\left(x-\frac{x^{3}}{3!}\right)^{2}}dx\\
	& \le\frac{1}{n}\int_{0}^{\delta}\frac{\sin^{2}nx}{x^{2}}\left(1-\frac{\delta^{2}}{6}\right)^{-2}dx+\frac{1}{n}\int_{\delta}^{\pi/2}\frac{1}{x^{2}}\left(1-\frac{(\pi/2)^{2}}{6}\right)^{-2}dx\qquad\forall\delta>0.
\end{align*}
由于
\[
\frac{\pi}{2}\ge\frac{1}{n}\int_{0}^{\pi/2}\frac{\sin^{2}nx}{x^{2}}dx\ge\left(1-\frac{\delta^{2}}{6}\right)^{2}\left[\frac{\pi}{2}-\frac{1}{n}\int_{\delta}^{\pi/2}\frac{1}{x^{2}}\left(1-\frac{\pi^{2}}{24}\right)^{-2}dx\right],\qquad\forall\delta>0.
\]
所以
\[
\frac{\pi}{2}\ge I\ge\left(1-\frac{\delta^{2}}{6}\right)^{-2}\frac{\pi}{2},\qquad\forall\delta>0.
\]

同上面类似的过程有Dirichlet积分
\begin{align*}
	\int_{0}^{\infty}\frac{\sin x}{x}dx & =\lim_{n\to\infty}\int_{0}^{\frac{(2n-1)\pi}{2}}\frac{\sin x}{x}dx\\
	& =\lim_{n\to\infty}\int_{0}^{\pi/2}\frac{\sin(2n-1)x}{x}dx\\
	& =\lim_{n\to\infty}\left[\int_{0}^{\pi/2}\frac{\sin(2n-1)x}{\sin x}dx+\int_{0}^{\pi/2}\sin(2n-1)x\cdot\left(\frac{1}{x}-\frac{1}{\sin x}\right)dx\right]=\frac{\pi}{2}.
\end{align*}


\paragraph{例7.4.9.}

求Euler积分
\[
I=\int_{0}^{\pi/2}\ln\sin xdx.
\]

pf.
\begin{align*}
	I\xlongequal{x=2t}2\int_{0}^{\pi/4}\ln\sin2tdt & =2\int_{0}^{\pi/4}\ln2+\ln\sin t+\ln\cos tdt\\
	& =2\ln2\cdot\frac{\pi}{4}+2\int_{0}^{\pi/4}\ln\sin tdt+2\int_{0}^{\pi/4}\ln\cos tdt\\
	& =\frac{\pi}{2}\ln2+2I.
\end{align*}
所以$I=-\frac{\pi}{2}\ln2$.

\subsection{作业}

8. (Frullani积分)设 $f(x)$ 在 $[0,+\infty)$ 上连续, 且对任意 $c>0$, 积分 $\int_{c}^{+\infty}\frac{f(x)}{x}$
收敛, 则 
\[
\int_{0}^{+\infty}\frac{f(\alpha x)-f(\beta x)}{x}dx=f(0)\ln\frac{\beta}{\alpha},\quad(\alpha,\beta>0).
\]

9. (Frullani积分) 设 $f(x)$ 是定义在 $(0,+\infty)$ 上的函数, 如果对于任意 $b>a>0$,
积分 $\int_{a}^{b}\frac{f(x)}{x}dx$ 收敛, 且 
\[
\lim_{x\rightarrow0^{+}}f(x)=L,\quad\lim_{x\rightarrow+\infty}f(x)=M,
\]
则 
\[
\int_{0}^{+\infty}\frac{f(\alpha x)-f(\beta x)}{x}dx=(L-M)\ln\frac{\beta}{\alpha},\quad(\alpha,\beta>0).
\]

pf. 
\begin{align*}
	\int_{\epsilon}^{M}\frac{f(\alpha x)-f(\beta x)}{x}dx & =\int_{\alpha\epsilon}^{\alpha M}\frac{f(x)}{x}dx-\int_{\beta\epsilon}^{\beta M}\frac{f(x)}{x}dx\\
	& =\int_{\alpha\epsilon}^{\beta\epsilon}\frac{f(x)}{x}dx-\int_{\alpha M}^{\beta M}\frac{f(x)}{x}dx\qquad(\alpha<\beta)\\
	& =\int_{\alpha\epsilon}^{\beta\epsilon}\frac{f(0)+o(1)}{x}dx-\int_{\alpha M}^{\beta M}\frac{f(\infty)+o(1)}{x}dx\\
	& =L\ln\frac{\beta}{\alpha}-M\ln\frac{\beta}{\alpha}.
\end{align*}

9.5 如果$f$在(0,+$\infty)$上连续, $\lim_{x\to+\infty}f(x)=f(+\infty)$存在,
且$\int_{0}^{1}\frac{f(x)}{x}\ud x$收敛, 则有
\[
\int_{0}^{+\infty}\frac{f(\alpha x)-f(\beta x)}{x}\ud x=-f(+\infty)\ln\frac{\beta}{\alpha}.
\]

10. 计算下列积分 $(a,b>0)$ : 

(1) $\int_{0}^{+\infty}\left(\frac{x}{e^{x}-e^{-x}}-\frac{1}{2}\right)\frac{dx}{x^{2}}$; 

(2) $\int_{0}^{+\infty}\frac{b\sin ax-a\sin bx}{x^{2}}dx$.

pf. (1) 注意到
\[
\frac{1}{x^{2}}\left(\frac{x}{e^{x}-e^{-x}}-\frac{1}{2}\right)=\frac{1}{x}\left(\frac{e^{x}+1-1}{e^{2x}-1}-\frac{1}{2x}\right)=\frac{1}{x}\left(\frac{1}{e^{x}-1}-\frac{1}{e^{2x}-1}-\frac{1}{x}+\frac{1}{2x}\right).
\]
如令$f(x)=\frac{1}{e^{x}-1}-\frac{1}{x}$, 则由Frullani积分, 
\[
\int_{0}^{+\infty}\left(\frac{x}{e^{x}-e^{-x}}-\frac{1}{2}\right)\frac{dx}{x^{2}}=f(0+)\ln2=-\frac{1}{2}\ln2.
\]
\section{数项级数}

\subsection{级数收敛与发散的概念}

无穷级数
\[
\sum_{n=1}^{\infty}a_{n}=a_{1}+a_{2}+\cdots+a_{n}+\cdots
\]
其中
\[
S_{n}=\sum_{k=1}^{n}a_{k}=a_{1}+\cdots+a_{n}
\]
称为级数的第$n$个部分和.

\paragraph{级数收敛的必要条件:}

如果 $\sum_{n=1}^{\infty}a_{n}$ 收敛, 则通项 $a_{n}\rightarrow0$, $(n\rightarrow\infty)$.
(否定表述)

\paragraph{级数收敛的Cauchy准则:}

$\sum_{n=1}^{\infty}a_{n}$ 收敛 $\Longleftrightarrow$ 任给 $\varepsilon>0$,
存在 $N=N(\epsilon)$, 当 $n>N$ 时 
\[
\left|a_{n+1}+a_{n+2}+\cdots+a_{n+p}\right|<\varepsilon,\quad\forall p\geqslant1.
\]
(否定表述)

\paragraph{例 8.1.2. }

判断级数 $\sum_{n=1}^{\infty}\frac{1}{n^{2}}$ 的敛散性.

\paragraph{例 8.1.3. }

判断级数 $\sum_{n=1}^{\infty}\frac{1}{n}$ 的敛散性 (调和级数).

\paragraph{例 8.1.4. }

判断级数 $\sum_{n=1}^{\infty}\sin n$ 的敛散性.

pf. 
\[
\sin(n+1)=\sin n\cdot\cos1+\cos n\cdot\sin1\Longrightarrow\cos n\to0,\ n\to\infty
\]
这与
\[
\sin^{2}n+\cos^{2}n=1
\]
矛盾.

\paragraph{例 8.1.5. }

设 $q>0$, 则当 $q<1$ 时, $\sum_{n=1}^{\infty}q^{n}$ 收敛; $q\geqslant1$
时, $\sum_{n=1}^{\infty}q^{n}$ 发散(几何级数).

\paragraph{Thm8.1.1}

(1) 如果 $\sum_{n=1}^{\infty}a_{n}$ 和 $\sum_{n=1}^{\infty}b_{n}$ 均收敛,
则 $\sum_{n=1}^{\infty}\left(\lambda a_{n}+\mu b_{n}\right)$ 也收敛,
且 $\sum_{n=1}^{\infty}\left(\lambda a_{n}+\mu b_{n}\right)=\lambda\sum_{n=1}^{\infty}a_{n}+\mu\sum_{n=1}^{\infty}b_{n},\quad(\lambda,\mu\in\mathbb{R})$ 

(2) 级数的敛散性与其有限项的值无关.

3. 设级数 $\sum_{n=1}^{\infty}a_{n}$ 的部分和为 $S_{n}$. 如果 $S_{2n}\rightarrow S$,
且 $a_{n}\rightarrow0$, 则 $\sum_{n=1}^{\infty}a_{n}$ 收敛. 

条件$a_{n}\to0$不能舍去, 比如$1-1+1-1+1-1+\cdots$.

4. 设级数 $\sum_{n=1}^{\infty}\left|a_{n+1}-a_{n}\right|$ 收敛, 则数列 $\left\{ a_{n}\right\} $
收敛. (提示: 用 Cauchy 准则.) 

5. 设数列 $na_{n}$ 收敛, 且级数 $\sum_{n=2}^{\infty}n\left(a_{n}-a_{n-1}\right)$
收敛, 证明级数 $\sum_{n=1}^{\infty}a_{n}$ 也是收敛的. 

6. 证明, 如果级数 $\sum_{n=1}^{\infty}a_{n}^{2}$ 收敛, 则 $\sum_{n=1}^{\infty}\frac{a_{n}}{n}$
也收敛. (提示: 用平均值不等式.)

8. 设 $\sum_{n=1}^{\infty}a_{n}$ 为发散级数, 则 $\sum_{n=1}^{\infty}\min\left\{ a_{n},1\right\} $
也发散.

hint: 反证法, 比Cauchy收敛准则的否定表述更好写证明过程.

\subsection{正项级数收敛与发散的判别法}

\paragraph{基本判别法}

$\sum_{n=1}^{\infty}a_{n}$ 收敛 $\Longleftrightarrow\left\{ S_{n}\right\} $
收敛 $\Longleftrightarrow\left\{ S_{n}\right\} $ 有上界.

\paragraph{定理 8.2.1 (比较判别法). }

设 $\sum_{n=1}^{\infty}a_{n}$ 和 $\sum_{n=1}^{\infty}b_{n}$ 为正项级数,
如果存在常数 $M>0$, 使得 
\[
a_{n}\leqslant Mb_{n},\quad\forall n\geqslant1.
\]
则 (1) $\sum_{n=1}^{\infty}b_{n}$ 收敛时 $\sum_{n=1}^{\infty}a_{n}$
也收敛; (2) $\sum_{n=1}^{\infty}a_{n}$ 发散时 $\sum_{n=1}^{\infty}b_{n}$
也发散.

\paragraph{例 8.2.3. }

判别 $\sum_{n=1}^{\infty}\left[\frac{1}{n}-\ln\left(1+\frac{1}{n}\right)\right]$
的敛散性. 

解. 根据 Taylor 展开, 
\[
0<\frac{1}{n}-\ln\left(1+\frac{1}{n}\right)=\frac{1}{2}\frac{1}{n^{2}}+o\left(\frac{1}{n^{2}}\right).
\]
因此 
\[
\lim_{n\rightarrow\infty}\left[\frac{1}{n}-\ln\left(1+\frac{1}{n}\right)\right]/\frac{1}{n^{2}}=\frac{1}{2},
\]
而 $\sum_{n=1}^{\infty}\frac{1}{n^{2}}$ 收敛, 故原级数收敛.

\paragraph{Cauchy判别法或根值判别法}

如果$n$充分大时, $\sqrt[n]{a_{n}}\le q<1$, 则$\sum_{n=1}^{\infty}a_{n}$收敛;

\paragraph{例 8.2.4. }

设 $p\in\mathbb{R}$, 判别级数 $\sum_{n=1}^{\infty}\left(1-\frac{p}{n}\right)^{n^{2}}$
的敛散性. 

解. 因为 
\[
\sqrt[n]{a_{n}}=\left(1-\frac{p}{n}\right)^{n}\rightarrow e^{-p},
\]
故 $p>0$ 时原级数收敛; $p<0$ 时级数发散. 显然, $p=0$ 时级数也发散.

\paragraph{d'Alembert判别法或比值判别法}

如果$n$充分大时, $\frac{a_{n+1}}{a_{n}}\le q<1$, 则$\sum_{n=1}^{\infty}a_{n}$收敛;

\paragraph{例 8.2.5. }

设 $x>0$, 判别级数 $\sum_{n=1}^{\infty}n!\left(\frac{x}{n}\right)^{n}$
的敛散性. 

解. 因为 
\[
\frac{a_{n+1}}{a_{n}}=\frac{x}{\left(1+\frac{1}{n}\right)^{n}}\rightarrow\frac{x}{e},
\]
故 $0<x<e$ 时级数收敛; $x>e$ 时级数发散. $x=e$ 时, 
\[
\frac{a_{n+1}}{a_{n}}=e/\left(1+\frac{1}{n}\right)^{n}\geqslant1,
\]
故此时级数也发散.

\paragraph{定理 8.2.2 (积分判别法). }

设 $f(x)$ 是定义在 $[1,+\infty)$ 上的非负单调递减函数, 记 $a_{n}=f(n)$, $(n\geqslant1)$.
则级数 $\sum_{n=1}^{\infty}a_{n}$ 的敛散性与广义积分 $\int_{1}^{+\infty}f(x)dx$
的敛散性相同.

证明. 令 
\[
F(x)=\int_{1}^{x}f(t)dt,\quad\forall x\geqslant1.
\]
因为 $f$ 为单调递减函数, 故当 $n\leqslant x\leqslant n+1$ 时 
\[
a_{n+1}=f(n+1)\leqslant f(x)\leqslant f(n)=a_{n},
\]
这说明 
\[
a_{n+1}\leqslant\int_{n}^{n+1}f(t)dt\leqslant a_{n},
\]
从而有 
\[
S_{n}\leqslant a_{1}+F(n),\quad F(n)\leqslant S_{n-1}.
\]
其中 $S_{n}=\sum_{k=1}^{n}a_{k}$ 为级数的部分和. 因为 $S_{n}$ 及 $F(n)$ 关于
$n$ 都是单调递增的, 二者同时有界或无界, 即 $\sum_{n=1}^{\infty}a_{n}$ 与 $\int_{1}^{+\infty}f(x)dx$
同敛散.

\paragraph{例 8.2.6. }

设 $s\in\mathbb{R}$, 判断级数 $\sum_{n=1}^{\infty}\frac{1}{n^{s}}$ 的敛散性.

\paragraph{定理 8.2.3 (Kummer). }

设 $\sum_{n=1}^{\infty}a_{n}$, $\sum_{n=1}^{\infty}b_{n}$ 为正项级数,
如果 $n$ 充分大时 

(1) $\frac{1}{b_{n}}\cdot\frac{a_{n}}{a_{n+1}}-\frac{1}{b_{n+1}}\geqslant\lambda>0$,
则 $\sum_{n=1}^{\infty}a_{n}$ 收敛; 

(2) $\frac{1}{b_{n}}\cdot\frac{a_{n}}{a_{n+1}}-\frac{1}{b_{n+1}}\leqslant0$
且 $\sum_{n=1}^{\infty}b_{n}$ 发散, 则 $\sum_{n=1}^{\infty}a_{n}$ 发散.

pf. (1) 条件可改写为 
\[
a_{n+1}\leqslant\frac{1}{\lambda}\left(\frac{a_{n}}{b_{n}}-\frac{a_{n+1}}{b_{n+1}}\right),\quad\forall n\geqslant N\text{. }
\]

这说明当 $n\geqslant N$ 时 
\[
\begin{aligned}S_{n+1} & =S_{N}+\sum_{k=N}^{n}a_{k+1}\\
	& \leqslant S_{N}+\frac{1}{\lambda}\sum_{k=N}^{n}\left(\frac{a_{k}}{b_{k}}-\frac{a_{k+1}}{b_{k+1}}\right)\\
	& =S_{N}+\frac{1}{\lambda}\left(\frac{a_{N}}{b_{N}}-\frac{a_{n+1}}{b_{n+1}}\right)\\
	& \leqslant S_{N}+\frac{1}{\lambda}\frac{a_{N}}{b_{N}}
\end{aligned}
\]
即 $\left\{ S_{n}\right\} $ 有上界, 从而 $\sum_{n=1}^{\infty}a_{n}$ 收敛. 

(2) 由 
\[
\frac{1}{b_{n}}\cdot\frac{a_{n}}{a_{n+1}}-\frac{1}{b_{n+1}}\leqslant0
\]
可知 
\[
\frac{a_{n}}{b_{n}}\leqslant\frac{a_{n+1}}{b_{n+1}},
\]
即 $\left\{ \frac{a_{n}}{b_{n}}\right\} $ 关于 $n$ 单调递增, 从而 $a_{n}\geqslant\frac{a_{1}}{b_{1}}b_{n}$,
因此由 $\sum_{n=1}^{\infty}b_{n}$ 发散知 $\sum_{n=1}^{\infty}a_{n}$ 也发散.

\paragraph{Kummer判别法推d'Alembert判别法}

取$b_{n}=1$, 则当$n$充分大, 有$\frac{a_{n+1}}{a_{n}}\le q<1$时, 有
\[
\frac{1}{b_{n}}\cdot\frac{a_{n}}{a_{n+1}}-\frac{1}{b_{n+1}}\ge\frac{1}{q}-1>0,
\]
所以由Kummer判别法有$\sum a_{n}$收敛. 当$n$充分大满足$\frac{a_{n+1}}{a_{n}}\ge1$时,
有
\[
\frac{1}{b_{n}}\cdot\frac{a_{n}}{a_{n+1}}-\frac{1}{b_{n+1}}\le1-1=0,
\]
再由Kummer判别法和$\sum1$发散, 知$\sum a_{n}$发散.

\paragraph{Kummer判别法证明Raabe判别法}

取$b_{n}=\frac{1}{n}$, 当$n$充分大时, 如果有$n\left(\frac{a_{n}}{a_{n+1}}-1\right)\ge\mu>1$,
则
\[
\frac{1}{b_{n}}\cdot\frac{a_{n}}{a_{n+1}}-\frac{1}{b_{n+1}}\ge n\left(1+\frac{\mu}{n}\right)-(n+1)=\mu-1>0,
\]
由Kummer判别法, $\sum a_{n}$收敛.

当$n$充分大时, 如果有$n\left(\frac{a_{n}}{a_{n+1}}-1\right)\le1$, 则
\[
\frac{1}{b_{n}}\cdot\frac{a_{n}}{a_{n+1}}-\frac{1}{b_{n+1}}\le n\left(1+\frac{1}{n}\right)-(n+1)=0,
\]
由Kummer判别法以及$\sum\frac{1}{n}$是发散的, 所以$\sum a_{n}$也发散.

\paragraph{Kummer判别法推出Gauss判别法}

取$b_{n}=\frac{1}{n\ln n}$, 当$n>N$充分大时, 如果有$\theta>1$使得
\[
\frac{a_{n}}{a_{n+1}}=1+\frac{\theta}{n}+o\left(\frac{1}{n\ln n}\right).
\]
则
\begin{align*}
	\frac{1}{b_{n}}\cdot\frac{a_{n}}{a_{n+1}}-\frac{1}{b_{n+1}} & ={\color{magenta}n\ln n\left(1+\frac{\theta}{n}+o\left(\frac{1}{n\ln n}\right)\right)}-(n+1)\ln(n+1)\\
	& ={\color{magenta}(n+1)\ln n+(\theta-1)\ln n+o(1)}-n\ln(n+1)-\ln(n+1)\\
	& =(\theta-1)\ln n+o(1)-(n+1)\left(\frac{1}{n}+o\left(\frac{1}{n}\right)\right)\\
	& \ge(\theta-1)\ln N>0
\end{align*}
所以$\sum a_{n}$收敛; 当$n>N$充分大时, 如果有$\theta\le1$使得
\[
\frac{a_{n}}{a_{n+1}}=1+\frac{\theta}{n}+o\left(\frac{1}{n\ln n}\right).
\]
则
\begin{align*}
	\frac{1}{b_{n}}\cdot\frac{a_{n}}{a_{n+1}}-\frac{1}{b_{n+1}} & ={\color{magenta}n\ln n\left(1+\frac{\theta}{n}+o\left(\frac{1}{n\ln n}\right)\right)}-(n+1)\ln(n+1)\\
	& ={\color{magenta}(n+1)\ln n+(\theta-1)\ln n+o(1)}-n\ln(n+1)-\ln(n+1)\\
	& =(\theta-1)\ln n+o(1)-(n+1)\left(\frac{1}{n}+o\left(\frac{1}{n}\right)\right)\\
	& \le0
\end{align*}
由Kummer判别法以及$\sum\frac{1}{n\ln n}$发散, 所以$\sum a_{n}$也发散.

\paragraph{例 8.2.8. }

判别下列级数的敛散性: 

(1) $\sum_{n=1}^{\infty}\frac{n!}{(\alpha+1)(\alpha+2)\cdots(\alpha+n)}$,
$(\alpha>0)$; 

(2) $\sum_{n=1}^{\infty}\left(\frac{(2n-1)!!}{(2n)!!}\right)^{s}\cdot\frac{1}{2n+1}$.

\paragraph{例 8.2.9 (Cauchy 凝聚判别法). }

设 $a_{n}$ 单调递减趋于零. 则 $\sum_{n=1}^{\infty}a_{n}$ 收敛当且仅当 $\sum_{k=0}^{\infty}2^{k}a_{2^{k}}$
收敛.

6. 设 $a_{n}>0$, $S_{n}=a_{1}+a_{2}+\cdots+a_{n}$, 证明 

(1) 级数 $\sum_{n=1}^{\infty}\frac{a_{n}}{S_{n}^{2}}$ 总是收敛的; 

(2) 级数 $\sum_{n=1}^{\infty}\frac{a_{n}}{\sqrt{S_{n}}}$ 收敛当且仅当 $\sum_{n=1}^{\infty}a_{n}$
收敛. 

\[
\sum\frac{S_{n}-S_{n-1}}{S_{n}^{2}}\le\sum\frac{S_{n}-S_{n-1}}{S_{n}S_{n-1}};
\]
\[
\sum\frac{a_{n}}{\sqrt{S_{n}}}\le\frac{1}{\sqrt{a_{1}}}\sum a_{n};
\]
\[
\sum\frac{a_{n}}{\sqrt{S_{n}}}=\sum\left(\sqrt{S_{n}}-\frac{S_{n-1}}{\sqrt{S_{n}}}\right)\ge\sum\left(\sqrt{S_{n}}-\sqrt{S_{n-1}}\right)
\]

7. 设正项级数 $\sum_{n=1}^{\infty}a_{n}$ 发散, 试用积分判别法证明 $\sum_{n=1}^{\infty}\frac{a_{n+1}}{S_{n}}$
也发散, 其中 $S_{n}$ 为 $\sum_{n=1}^{\infty}a_{n}$ 的部分和. 

hint1: 取$\epsilon=\frac{1}{2}$, 则对于任何$n$, 存在$m>n$, s.t. $S_{m+1}>2S_{n}$.
则
\[
\sum_{k=n}^{m}\frac{a_{k+1}}{S_{k}}\ge\sum_{k=n}^{m}\frac{a_{k+1}}{S_{k+1}}\ge\frac{\sum_{k=n}^{m}a_{k+1}}{S_{m+1}}=\frac{S_{m+1}-S_{n}}{S_{m+1}}\ge\frac{1}{2}=\epsilon,
\]
由Cauchy收敛判别法即得.

hint2: 
\[
\frac{a_{n+1}}{S_{n}}=\frac{S_{n+1}-S_{n}}{S_{n}}\ge\int_{S_{n}}^{S_{n+1}}\frac{dx}{x}.
\]

8. 判断下列级数的敛散性: 

(1) $\sum_{n=1}^{\infty}\frac{n!e^{n}}{n^{n+p}}$; 

(2) $\sum_{n=1}^{\infty}\frac{p(p+1)\cdots(p+n-1)}{n!}\cdot\frac{1}{n^{q}}\quad(p>0,q>0)$; 

(3) $\sum_{n=1}^{\infty}\frac{(2n-1)!!}{(2n)!!}$; 

(4) $\sum_{n=1}^{\infty}\frac{\sqrt{n!}}{(a+\sqrt{1})(a+\sqrt{2})\cdots(a+\sqrt{n})}(a>0)$. 

9. 设 $a_{n}>0,S_{n}=a_{1}+a_{2}+\cdots+a_{n}$, 证明 

(1) 当 $\alpha>1$ 时, 级数 $\sum_{n=1}^{\infty}\frac{a_{n}}{S_{n}^{\alpha}}$
总是收敛的; 

(2) 当 $\alpha\leqslant1$ 时, 级数 $\sum_{n=1}^{\infty}\frac{a_{n}}{S_{n}^{\alpha}}$
收敛当且仅当 $\sum_{n=1}^{\infty}a_{n}$ 收敛.

(1) hint: $\frac{a_{n}}{S_{n}^{\alpha}}=\frac{S_{n}-S_{n-1}}{S_{n}^{\alpha}}\le\int_{S_{n-1}}^{S_{n}}\frac{dx}{x^{\alpha}}$,
($\alpha>1$).

(2) hint: 如果$\sum_{n=1}^{\infty}a_{n}$ 发散, 类似第7题的证明.

10. 设 $a_{n}>0$ 关于 $n$ 单调递增. 证明级数 $\sum_{n=1}^{\infty}\left(1-\frac{a_{n}}{a_{n+1}}\right)$
收敛当且仅当级数 $\sum_{n=1}^{\infty}\left(\frac{a_{n+1}}{a_{n}}-1\right)$
收敛. 

\[
\sum_{n=1}^{\infty}\left(1-\frac{a_{n}}{a_{n+1}}\right)\le\sum_{n=1}^{\infty}\left(\frac{a_{n+1}}{a_{n}}-1\right);
\]


\paragraph{Sapagof 判别法: }

设正数数列 $\left\{ a_{n}\right\} _{n=1}^{\infty}$ 单调递减,则 $\lim_{n\rightarrow\infty}a_{n}=0$
的充要条件是 $\sum_{n=1}^{\infty}\left(1-\frac{a_{n+1}}{a_{n}}\right)$
发散。 

上述断言等价于: 单调递增数列 $\left\{ a_{n}\right\} _{n=1}^{\infty}$ 与级数 $\sum_{n=1}^{\infty}\left(1-\frac{a_{n}}{a_{n+1}}\right)$
同敛散.

由此可以得到: 设正项级数 $\sum_{n=1}^{\infty}a_{n}$ 的前 $n$ 项部分和为 $S_{n}$,
那么级数 $\sum_{n=1}^{\infty}a_{n}$ 和 $\sum_{n=1}^{\infty}\frac{a_{n}}{S_{n}}$
同敛散. 

设 $p>1$, 正项级数 $\sum_{n=1}^{\infty}a_{n}$ 的前 $n$ 项部分和为 $S_{n}$,
那么级数 $\sum_{n=1}^{\infty}\frac{a_{n}}{S_{n}^{p}}$ 始终是收敛的.

11. 设 $\sum_{n=1}^{\infty}a_{n}$ 为正项级数, 且 
\[
\frac{a_{n}}{a_{n+1}}=1+\frac{1}{n}+\frac{\alpha_{n}}{n\ln n}.
\]
如果 $n$ 充分大时 $\alpha_{n}\geqslant\mu>1$, 则 $\sum_{n=1}^{\infty}a_{n}$
收玫; 如果 $n$ 充分大时 $\alpha_{n}\leqslant1$, 则 $\sum_{n=1}^{\infty}a_{n}$
发散. 这个结果称为 Bertrand 判别法.

hint: 取\textbf{$b_{n}=\frac{1}{n\ln n}$}, 则
\[
\frac{1}{b_{n}}\cdot\frac{a_{n}}{a_{n+1}}-\frac{1}{b_{n+1}}=\alpha_{n}-1+o(1).
\]
然后用Kummer判别法.

17. 设 $a_{n}>0$, $\sum_{n=1}^{\infty}\frac{1}{a_{n}}$ 收敛. 证明级数 $\sum_{n=1}^{\infty}\frac{n}{a_{1}+a_{2}+\cdots+a_{n}}$
也收敛.

hint: 将$a_{n}$递增重排.

以上收敛判别法列表也可以参考:

\url{https://en.wikipedia.org/wiki/Convergence_tests}

至今, 这些收敛判别法全部失效的正项级数是存在的, 比如Flint Hills级数
\[
\sum_{k=1}^{\infty}\frac{1}{n^{3}\sin^{2}n}
\]
它的收敛性涉及到$\pi$的无理测度的大小, 见

\url{https://math.stackexchange.com/questions/162573}

\subsection{一般级数收敛与发散判别法}

级数是正负交替出现的称为交错级数

\paragraph{定理 8.3.1 (Leibniz). }

设 $a_{n}$ 单调递减趋于 0 , 则级数 $\sum_{n=1}^{\infty}(-1)^{n-1}a_{n}$ 收敛.

pf. 使用Cauchy收敛准则.
\[
\begin{aligned}S_{n+p}-S_{n} & =(-1)^{n}\cdot a_{n+1}+(-1)^{n+1}a_{n+2}+\cdots+(-1)^{n+p-1}a_{n+p}\\
	& =(-1)^{n}\left[a_{n+1}-a_{n+2}+a_{n+3}-a_{n+4}+\cdots+(-1)^{p-1}a_{n+p}\right].
\end{aligned}
\]
因此当 $p=2k-1$ 时, 
\[
\begin{aligned}(-1)^{n}\left(S_{n+p}-S_{n}\right) & =a_{n+1}-\left(a_{n+2}-a_{n+3}\right)-\left(a_{n+4}-a_{n+5}\right)-\cdots\leqslant a_{n+1},\\
	(-1)^{n}\left(S_{n+p}-S_{n}\right) & =\left(a_{n+1}-a_{n+2}\right)+\left(a_{n+3}-a_{n+4}\right)+\cdots+a_{n+2k-1}\\
	& \geqslant0
\end{aligned}
\]
这说明 
\[
\left|S_{n+p}-S_{n}\right|\leqslant a_{n+1}\rightarrow0,\quad(n\rightarrow\infty).
\]
当 $p=2k$ 时, 类似地可证上式仍成立. 因此原级数收敛.

\paragraph{例 8.3.1. }

级数 $\sum_{n=1}^{\infty}(-1)^{n-1}\frac{1}{\sqrt{n}}$ 收敛.

\paragraph{引理 8.3.2 (分部求和). }

设 $\left\{ a_{k}\right\} ,\left\{ b_{k}\right\} $ 为数列, 则 
\[
\sum_{k=m}^{n-1}a_{k+1}\left(b_{k+1}-b_{k}\right)+\sum_{k=m}^{n-1}b_{k}\left(a_{k+1}-a_{k}\right)=a_{n}b_{n}-a_{m}b_{m}.
\]


\paragraph{推论 8.3.3 (Abel 变换). }

设 $a_{i},b_{i}(i\geqslant1)$ 为两组实数, 如果约定 $b_{0}=0$, 记 
\[
B_{0}=0,B_{k}=b_{1}+b_{2}+\cdots+b_{k}\quad(k\geqslant1),
\]
则有 
\[
\sum_{i=m+1}^{n}a_{i}b_{i}=\sum_{i=m+1}^{n-1}\left(a_{i}-a_{i+1}\right)B_{i}+a_{n}B_{n}-a_{m+1}B_{m},\quad\forall m\geqslant0.
\]


\paragraph{推论 8.3.4 (Abel 引理). }

设 $a_{1},a_{2},\cdots,a_{n}$ 为单调数列, 且 $\left|B_{i}\right|\leqslant M$,
$(i\geqslant1)$, 则 
\[
\left|\sum_{i=m+1}^{n}a_{i}b_{i}\right|\leqslant2M\left(\left|a_{n}\right|+\left|a_{m+1}\right|\right),\quad\forall m\geqslant0.
\]


\paragraph{定理 8.3.5 (Dirichlet). }

设数列 $\left\{ a_{n}\right\} $ 单调趋于 0 , 级数 $\sum_{n=1}^{\infty}b_{n}$
的部分和有界, 则级数 $\sum_{n=1}^{\infty}a_{n}b_{n}$ 收敛.

\paragraph{证明. }

由假设, 存在 $M>0$ 使得 
\[
\left|\sum_{i=1}^{n}b_{i}\right|\leqslant M,\quad\forall n\geqslant1.
\]
由 Abel 变换及其推论, 
\[
\left|\sum_{i=n+1}^{n+p}a_{i}b_{i}\right|\leqslant2M\left(\left|a_{n+1}\right|+\left|a_{n+p}\right|\right)\leqslant4M\left|a_{n+1}\right|\rightarrow0.
\]
由 Cauchy 准则知级数 $\sum_{n=1}^{\infty}a_{n}b_{n}$ 收敛.

\paragraph{定理 8.3.6 (Abel). }

如果 $\left\{ a_{n}\right\} $ 为单调有界数列, $\sum_{n=1}^{\infty}b_{n}$
收敛, 则级数 $\sum_{n=1}^{\infty}a_{n}b_{n}$ 收敛.

\paragraph{证明. }

$\left\{ a_{n}\right\} $ 单调有界意味着极限 $\lim_{n\rightarrow\infty}a_{n}=a$
存在. 于是 $\left\{ a_{n}-a\right\} $ 单调趋于 0 . 由 Dirichlet 判别法, $\sum_{n=1}^{\infty}\left(a_{n}-a\right)b_{n}$
收敛. 从而级数 
\[
\sum_{n=1}^{\infty}a_{n}b_{n}=\sum_{n=1}^{\infty}\left(a_{n}-a\right)b_{n}+\sum_{n=1}^{\infty}a\cdot b_{n}
\]
也收敛.

\paragraph{例 8.3.3. }

判断级数 $\sum_{n=1}^{\infty}\frac{1}{n}\sin nx$ 的敛散性.

解. $a_{n}=\frac{1}{n}$ 单调递减趋于 $0,b_{n}=\sin nx$. 利用公式 
\[
2\sin\frac{x}{2}\cdot\sin kx=\cos\left(k-\frac{1}{2}\right)x-\cos\left(k+\frac{1}{2}\right)x
\]
得 
\[
\sum_{k=1}^{n}b_{n}=\begin{cases}
	0, & x=2k\pi,\\
	\frac{\cos\frac{x}{2}-\cos\left(n+\frac{1}{2}\right)x}{2\sin\frac{x}{2}}, & x\neq2k\pi.
\end{cases}
\]
即 $b_{n}$ 的部分和总是有界的. 故由 Dirichlet 判别法知, 原级数收敛.

\paragraph{定义 8.3.1 (绝对收敛). }

如果 $\sum_{n=1}^{\infty}\left|a_{n}\right|$ 收敛, 则称 $\sum_{n=1}^{\infty}a_{n}$
绝对收敛 (此时, 由于 
\[
\left|a_{n+1}+\cdots+a_{n+p}\right|\leqslant\left|a_{n+1}\right|+\cdots+\left|a_{n+p}\right|\rightarrow0,
\]
故 $\sum_{n=1}^{\infty}a_{n}$ 的确为收敛级数). 如果 $\sum_{n=1}^{\infty}a_{n}$
收敛而 $\sum_{n=1}^{\infty}\left|a_{n}\right|$ 发散, 则称 $\sum_{n=1}^{\infty}a_{n}$
条件收敛.

\paragraph{例 8.3.5. }

判断级数 $\sum_{n=1}^{\infty}(-1)^{n-1}\frac{x^{n}}{n}$, $(x\in\mathbb{R})$,
的敛散性. 

\paragraph{解. }

令 $a_{n}=\frac{|x|^{n}}{n}$, 则 $\sqrt[n]{a_{n}}\rightarrow|x|$.
故 $|x|<1$ 时原级数绝对收敛; 而 $|x|>1$ 时显然发散. $x=1$ 时级数条件收敛; $x=-1$ 时级数发散.

\subsubsection{作业}

6. 设 $\sum_{n=1}^{\infty}a_{n}$ 绝对收敛, $\left\{ b_{n}\right\} $ 为有界数列,
则 $\sum_{n=1}^{\infty}a_{n}b_{n}$ 也是绝对收敛的. 

6'. 证明或否定: 设 $\sum_{n=1}^{\infty}a_{n}$ 收敛, $\left\{ b_{n}\right\} $
为有界数列, 则 $\sum_{n=1}^{\infty}a_{n}b_{n}$ 也是收敛的. 

7. 如果 $\sum_{n=1}^{\infty}a_{n}$ 收敛, 则 $\sum_{n=1}^{\infty}a_{n}^{3}$
是否也收敛? 证明你的结论. 

比如
\[
a_{n}=\frac{\cos\frac{n\pi}{3}}{\sqrt[3]{n}}.
\]

8. 设级数 $\sum_{n=1}^{\infty}\left|a_{n+1}-a_{n}\right|$ 收敛且 $a_{n}$
极限为零, 级数 $\sum_{n=1}^{\infty}b_{n}$ 的部分和有界, 则级数 $\sum_{n=1}^{\infty}a_{n}b_{n}$
收敛. (提示: Abel 求和.)

9. 设级数 $\sum_{n=1}^{\infty}\left|a_{n+1}-a_{n}\right|$ 收敛, 级数 $\sum_{n=1}^{\infty}b_{n}$
收敛, 则级数 $\sum_{n=1}^{\infty}a_{n}b_{n}$ 也收敛. (用Abel求和)

10. 设 $\left\{ a_{n}\right\} $ 单调递减趋于零. 证明下面的级数是收敛的: 
\[
\sum_{n=1}^{\infty}(-1)^{n}\frac{a_{1}+a_{2}+\cdots+a_{n}}{n}.
\]

用9.

11. 设 $\sum_{n=1}^{\infty}a_{n}$ 收敛, 证明 
\[
\lim_{n\rightarrow\infty}\frac{a_{1}+2a_{2}+\cdots+na_{n}}{n}=0.
\]

考虑把$a_{n}$用部分和$S_{n}$表示的问题. 或者使用$\epsilon-N$法+Abel变换.

本题不能使用Stolz公式, 因为有反例如下:
\[
a_{n}=\begin{cases}
	\frac{1}{m^{2}}, & n=m^{2};\\
	\frac{1}{n^{3}}, & n\ne m^{2}.
\end{cases}
\]

12. 设 $a_{n}>0,na_{n}$ 单调趋于 $0,\sum_{n=1}^{\infty}a_{n}$ 收敛. 证明
$n\ln n\cdot a_{n}\rightarrow0$.

类比下题:

9. 设 $\int_{a}^{+\infty}f(x)dx$ 收敛, 如果 $x\rightarrow+\infty$ 时 $xf(x)$
单调递减趋于零, 则 
\[
\lim_{x\rightarrow+\infty}xf(x)\ln x=0.
\]

pf. 对于任意的$\epsilon>0$, 因为$\int_{a}^{+\infty}f(x)dx$收敛, 则存在$M>\max(0,a)$,
s.t. 对于任意的$x>M$, 有
\[
xf(x)\int_{\sqrt{x}}^{x}\frac{1}{t}dt\le\int_{\sqrt{x}}^{x}tf(t)\cdot\frac{1}{t}dt=\int_{\sqrt{x}}^{x}f(t)dt<\frac{\epsilon}{2}.
\]
即$xf(x)\ln x<\epsilon$.

\subsection{数项级数的进一步讨论}

\subsubsection{级数求和与求极限的可交换性}

\paragraph{定义 8.4.1 (级数的一致收敛). }

一列收敛级数 $\sum_{j=1}^{\infty}a_{ij}=A_{i}$ 关于 $i$ 一致收敛是指, 任给 $\varepsilon>0$,
存在 $N$, 当 $n>N$ 时, 
\[
\left|\sum_{j=1}^{n}a_{ij}-A_{i}\right|<\varepsilon,\quad\forall i\geqslant1.
\]
当且仅当有Cauchy准则: $\forall\epsilon>0$, $\exists N$, 当$m,n>N$时, 
\[
\left|\sum_{j=n}^{m}a_{ij}\right|<\varepsilon,\quad\forall i\geqslant1.
\]

给出级数一致收敛的否定表述.

\paragraph{定理 8.4.1. }

设一列级数 $\sum_{j=1}^{\infty}a_{ij}=A_{i}$ 关于 $i$ 一致收敛, 如果 $\lim_{i\rightarrow\infty}a_{ij}=a_{j}$
$(j\geqslant1)$, 则极限 $\lim_{i\rightarrow\infty}A_{i}$ 存在, 级数 $\sum_{j=1}^{\infty}a_{j}$
收敛, 且 
\[
\lim_{i\rightarrow\infty}A_{i}=\sum_{j=1}^{\infty}a_{j}
\]
或改写为 
\[
\lim_{i\rightarrow\infty}\sum_{j=1}^{\infty}a_{ij}=\sum_{j=1}^{\infty}\lim_{i\rightarrow\infty}a_{ij}.
\]

pf: 1. $\sum_{j\ge1}a_{j}$收敛

证明. 由一致收敛的定义, 任给 $\varepsilon>0$, 存在 $N_{0}$, 当 $n\geqslant N_{0}$
时, 
\[
\left|\sum_{j=1}^{n}a_{ij}-A_{i}\right|<\frac{1}{4}\varepsilon,\quad\forall i\geqslant1.
\]
因此, 当 $m>n\geqslant N_{0}$ 时 
\[
\left|\sum_{j=n+1}^{m}a_{ij}\right|\leqslant\left|\sum_{j=1}^{m}a_{ij}-A_{i}\right|+\left|\sum_{j=1}^{n}a_{ij}-A_{i}\right|<\frac{1}{2}\varepsilon,\quad\forall i\geqslant1.
\]
在上式中令 $i\rightarrow\infty$, 得 
\[
\left|\sum_{j=n+1}^{m}a_{j}\right|\leqslant\frac{1}{2}\varepsilon,
\]
由 Cauchy 准则即知级数 $\sum_{j=1}^{\infty}a_{j}$ 收敛, 且在上式中令 $m\rightarrow\infty$
可得 
\[
\left|\sum_{j=n+1}^{\infty}a_{j}\right|\leqslant\frac{1}{2}\varepsilon,\quad\forall n\geqslant N_{0}
\]

2. 证$\lim_{i\to\infty}A_{i}=\sum_{j\ge1}a_{j}$.

对于 $j=1,2,\cdots,N_{0}$, 因为 $a_{ij}\rightarrow a_{j}$, 故存在 $N$,
当 $i>N$ 时, 
\[
\left|a_{ij}-a_{j}\right|<\frac{\varepsilon}{4N_{0}},\quad j=1,2,\cdots,N_{0}.
\]
因此, 当 $i>N$ 时, 有 
\[
\begin{aligned}\left|A_{i}-\sum_{i=1}^{\infty}a_{j}\right| & \leqslant\left|A_{i}-\sum_{j=1}^{N_{0}}a_{ij}\right|+\left|\sum_{j=1}^{N_{0}}a_{ij}-\sum_{j=1}^{N_{0}}a_{j}\right|+\left|\sum_{j=N_{0}+1}^{\infty}a_{j}\right|\\
	& <\frac{1}{4}\varepsilon+N_{0}\frac{\varepsilon}{4N_{0}}+\frac{1}{2}\varepsilon=\varepsilon
\end{aligned}
\]
这说明 $\left\{ A_{i}\right\} $ 的极限存在且极限为 $\sum_{j=1}^{\infty}a_{j}$.

\paragraph{推论 8.4.2. (控制收敛定理)}

设 $\lim_{i\rightarrow\infty}a_{ij}=a_{j}(j\geqslant1)$, $\left|a_{ij}\right|\leqslant b_{j}(i\geqslant1)$,
且 $\sum_{j=1}^{\infty}b_{j}$ 收敛(控制级数), 则级数 $\sum_{j=1}^{\infty}a_{j}$
收敛, 且 
\[
\sum_{j=1}^{\infty}a_{j}=\sum_{j=1}^{\infty}\lim_{i\rightarrow\infty}a_{ij}=\lim_{i\rightarrow\infty}\sum_{j=1}^{\infty}a_{ij}.
\]


\paragraph{证明. }

由 $a_{ij}\rightarrow a_{j}$, 且 $\left|a_{ij}\right|\leqslant b_{j}$
知 $\left|a_{j}\right|\leqslant b_{j}$, $j=1,2,\cdots$. 因为级数 $\sum_{j=1}^{\infty}b_{j}$
收敛, 故级数 $\sum_{j=1}^{\infty}a_{j}$ 绝对收敛. 任给 $\varepsilon>0$, 存在
$N$, 当 $n>N$ 时, 
\[
0\leqslant\sum_{j=n+1}^{\infty}b_{j}<\varepsilon.
\]
此时, 对任意 $i\geqslant1$, 有 
\[
\left|\sum_{j=1}^{n}a_{ij}-\sum_{j=1}^{\infty}a_{ij}\right|=\left|\sum_{j=n+1}^{\infty}a_{ij}\right|\leqslant\sum_{j=n+1}^{\infty}b_{j}<\varepsilon,
\]
从而级数 $\sum_{j=1}^{\infty}a_{ij}$ 关于 $i$ 是一致收敛的. 由上一定理知本推论结论成立.

\paragraph{推论 8.4.3. (Fubini)}

设 $\sum_{i=1}^{\infty}\left|a_{ij}\right|\leqslant A_{j}(j\geqslant1)$,
且 $\sum_{j=1}^{\infty}A_{j}$ 收敛, 则对任意 $i\geqslant1$, 级数 $\sum_{j=1}^{\infty}a_{ij}$
收敛, 且 
\[
\sum_{i=1}^{\infty}\sum_{j=1}^{\infty}a_{ij}=\sum_{j=1}^{\infty}\sum_{i=1}^{\infty}a_{ij}.
\]


\paragraph{证明. }

1. 首先, 由题设知, $\left|a_{ij}\right|\leqslant A_{j}$, $j=1,2,\cdots$.
这说明, 对任意 $i\geqslant1$, 级数 $\sum_{j=1}^{\infty}a_{ij}$ 是绝对收敛的. 

2. 因为 
\[
\left|\sum_{i=1}^{k}a_{ij}\right|\leqslant\sum_{i=1}^{k}\left|a_{ij}\right|\leqslant A_{j},\quad j\geqslant1.
\]
故$\left(\sum_{i=1}^{k}a_{ij}\right)_{kj}$满足上一推论, 有 
\[
\begin{aligned}\sum_{i=1}^{\infty}\sum_{j=1}^{\infty}a_{ij} & =\lim_{k\rightarrow\infty}\sum_{i=1}^{k}\sum_{j=1}^{\infty}a_{ij}\\
	& {\color{magenta}=\lim_{k\rightarrow\infty}\sum_{j=1}^{\infty}\sum_{i=1}^{k}a_{ij}}\\
	& {\color{magenta}=\sum_{j=1}^{\infty}\lim_{k\rightarrow\infty}\sum_{i=1}^{k}a_{ij}}\\
	& =\sum_{j=1}^{\infty}\sum_{i=1}^{\infty}a_{ij}.
\end{aligned}
\]
这就证明了本推论.

上述推论的条件中$\sum_{i\ge1}\left|a_{ij}\right|$的绝对值不能省去, 比如设
\[
a_{ij}=\begin{cases}
	\frac{1}{2^{j-i}}, & j>i;\\
	-1, & i=j;\\
	0, & j<i.
\end{cases}
\]
也即
\[
\left(a_{ij}\right)=\begin{pmatrix}-1 & \frac{1}{2} & \frac{1}{4} & \frac{1}{8} & \frac{1}{16} & \cdots\\
	0 & -1 & \frac{1}{2} & \frac{1}{4} & \frac{1}{8} & \cdots\\
	0 & 0 & -1 & \frac{1}{2} & \frac{1}{4} & \cdots\\
	0 & 0 & 0 & -1 & \frac{1}{2} & \cdots\\
	\vdots & \vdots & \vdots & \vdots & \vdots & \ddots
\end{pmatrix}.
\]
则
\[
\sum_{i\ge1}a_{ij}=-\frac{1}{2^{j-1}},\qquad\sum_{j\ge1}a_{ij}\equiv0\ \Longrightarrow\ 0=\sum_{i\ge1}\sum_{j\ge1}a_{ij}\ne\sum_{j\ge1}\sum_{i\ge1}a_{ij}=-2.
\]


\paragraph{例 8.4.1. }

设 $\sum_{n=2}^{\infty}\left|a_{n}\right|$ 收敛, 记 $f(x)=\sum_{n=2}^{\infty}a_{n}x^{n}$,
$x\in[-1,1]$. 则 
\[
\sum_{n=1}^{\infty}f\left(\frac{1}{n}\right)=\sum_{n=2}^{\infty}a_{n}\zeta(n),
\]
其中 $\zeta(s)$ 是 Riemann-Zeta 函数.

\paragraph{证明.}

\[
\sum_{n=2}^{\infty}\frac{\left|a_{n}\right|}{m^{n}}\le\frac{1}{m^{2}}\sum_{n=2}^{\infty}\left|a_{n}\right|\Longrightarrow\sum_{m=1}^{\infty}\sum_{n=2}^{\infty}\frac{\left|a_{n}\right|}{m^{n}}\le\sum_{m=1}^{\infty}\left(\frac{1}{m^{2}}\sum_{n=2}^{\infty}\left|a_{n}\right|\right)<+\infty.
\]
由Fubini定理, $\sum_{n=2}^{\infty}\sum_{m=1}^{\infty}\frac{a_{n}}{m^{n}}$收敛,
且
\begin{align*}
	\sum_{n=2}^{\infty}\sum_{m=1}^{\infty}\frac{a_{n}}{m^{n}} & =\sum_{m=1}^{\infty}\sum_{n=2}^{\infty}\frac{a_{n}}{m^{n}}=\sum_{m=1}^{\infty}f\left(\frac{1}{m}\right)\\
	& =\sum_{n=2}^{\infty}a_{n}\zeta(n).
\end{align*}
所以
\begin{align*}
	\sum_{n=2}^{\infty}\frac{1}{2^{n}}\zeta(n) & =\sum_{n=2}^{\infty}\sum_{m=1}^{\infty}\frac{1}{2^{n}}\cdot\frac{1}{m^{n}}=\sum_{m=1}^{\infty}\sum_{n=2}^{\infty}\frac{1}{m^{n}}\cdot\frac{1}{2^{n}}=\sum_{m=1}^{\infty}\frac{1}{2^{2}m^{2}}\cdot\frac{1}{1-\frac{1}{2m}}\\
	& =\sum_{m=1}^{\infty}\left(\frac{1}{2m-1}-\frac{1}{2m}\right)=\ln2.
\end{align*}


\subsubsection{级数的乘积}

设$\sum_{n=0}^{\infty}a_{n}$和$\sum_{n=0}^{\infty}b_{n}$之积为$\sum_{n=0}^{\infty}c_{n}$,
称为Cauchy乘积
\[
c_{n}=\sum_{i+j=n}a_{i}b_{j},\qquad n\ge0.
\]


\paragraph{定理 8.4.4 (Cauchy). }

如果 $\sum_{n=0}^{\infty}a_{n}$ 和 $\sum_{n=0}^{\infty}b_{n}$ 绝对收敛,
则它们的乘积级数也绝对收敛, 且 
\[
\sum_{n=0}^{\infty}c_{n}=\left(\sum_{n=0}^{\infty}a_{n}\right)\left(\sum_{n=0}^{\infty}b_{n}\right).
\]


\paragraph{定理 8.4.5 (Mertens). }

如果 $\sum_{n=0}^{\infty}a_{n}$ 和 $\sum_{n=0}^{\infty}b_{n}$ 收敛, 且至少其中一个级数绝对收敛,
则它们的乘积级数也收敛, 且 
\[
\sum_{n=0}^{\infty}c_{n}=\left(\sum_{n=0}^{\infty}a_{n}\right)\left(\sum_{n=0}^{\infty}b_{n}\right).
\]


\paragraph{证明. }

不妨设 $\sum_{n=0}^{\infty}a_{n}$ 绝对收敛. 分别记 
\[
A_{n}=\sum_{k=0}^{n}a_{k},\quad B_{n}=\sum_{k=0}^{n}b_{k},\quad C_{n}=\sum_{k=0}^{n}c_{k}.
\]
则 $A_{n}\rightarrow A$, $B_{n}\rightarrow B$, 而 
\[
C_{n}=\sum_{i+j\leqslant n}a_{i}b_{j}=a_{0}B_{n}+a_{1}B_{n-1}+\cdots+a_{n}B_{0}=A_{n}B+\delta_{n},
\]
其中 
\[
\delta_{n}=a_{0}\left(B_{n}-B\right)+a_{1}\left(B_{n-1}-B\right)+\cdots+a_{n}\left(B_{0}-B\right).
\]
我们只要证明 $\delta_{n}\rightarrow0$ 即可. 因为 $B_{n}\rightarrow B$, 故 $\left\{ B_{n}\right\} $
关于 $n$ 有界, 从而存在 $K$, 使得 
\[
\left|B_{n}-B\right|\leqslant K,\quad\forall n\geqslant0.
\]
由于 $\sum_{n=0}^{\infty}a_{n}$ 绝对收敛, 故任给 $\varepsilon>0$, 存在 $N_{0}$,
当 $n>N_{0}$ 时 
\[
\left|a_{N_{0}+1}\right|+\cdots+\left|a_{n}\right|<\frac{\varepsilon}{2K+1}.
\]
记 $L=\left|a_{0}\right|+\left|a_{1}\right|+\cdots+\left|a_{N_{0}}\right|$.
由于 $B_{n}-B\rightarrow0$, 故存在 $N_{1}$, 当 $n>N_{1}$ 时 
\[
\left|B_{n}-B\right|<\frac{\varepsilon}{2L+1}.
\]
从而当 $n>N_{0}+N_{1}$ 时, 有 
\[
\begin{aligned}\left|\delta_{n}\right| & \leqslant\sum_{k=0}^{N_{0}}\left|a_{k}\right|\left|B_{n-k}-B\right|+\left(\left|a_{N_{0}+1}\right|+\cdots+\left|a_{n}\right|\right)K\\
	& \leqslant\frac{\varepsilon}{2L+1}\left(\left|a_{0}\right|+\left|a_{1}\right|+\cdots+\left|a_{N_{0}}\right|\right)+\frac{\varepsilon}{2K+1}K\\
	& =\frac{\varepsilon}{2L+1}L+\frac{\varepsilon}{2K+1}K\\
	& <\varepsilon
\end{aligned}
\]
这说明 $\delta_{n}\rightarrow0$, 因而 $C_{n}=A_{n}B+\delta_{n}\rightarrow AB$.

注. 定理中的绝对收敛的条件不能去掉, 反例就是将 $a_{n}$ 和 $b_{n}$ 均取为交错级数 $(-1)^{n-1}\frac{1}{\sqrt{n}}$,
此时所得乘积级数是发散的. 

但是, 如果乘积级数仍然收敛, 则其和等于两个级数和的乘积. 为了说明这一点, 需要下面的引理.

\paragraph{引理 8.4.6 (Abel). }

设级数 $\sum_{n=0}^{\infty}c_{n}=C$ 收敛, 令 
\[
f(x)=\sum_{n=0}^{\infty}c_{n}x^{n},\quad x\in[0,1),
\]
则 $\lim_{x\rightarrow1^{-}}f(x)=C$.

\paragraph{证明. }

级数收敛表明 $\left\{ c_{n}\right\} $ 有界, 因此当 $x\in[0,1)$ 时, $\sum_{n=0}^{\infty}c_{n}x^{n}$
绝对收敛. 记 
\[
C_{-1}=0,\quad C_{n}=\sum_{k=0}^{n}c_{k},\quad n\geqslant0.
\]
则有 
\[
\begin{aligned}\sum_{k=0}^{n}c_{k}x^{k} & =\sum_{k=0}^{n}\left(C_{k}-C_{k-1}\right)x^{k}\\
	& =\sum_{k=0}^{n}C_{k}x^{k}-x\sum_{k=0}^{n-1}C_{k}x^{k}\\
	& =C_{n}x^{n}+(1-x)\sum_{k=0}^{n-1}C_{k}x^{k}\\
	& =C_{n}x^{n}+C\left(1-x^{n}\right)+(1-x)\sum_{k=0}^{n-1}\left(C_{k}-C\right)x^{k}.
\end{aligned}
\]
在上式中令 $n\rightarrow\infty$ 就得到 
\[
f(x)=C+(1-x)\sum_{k=0}^{\infty}\left(C_{k}-C\right)x^{k}.
\]
因为 $C_{k}-C\rightarrow0$, 故任给 $\varepsilon>0$, 存在 $N$, 当 $k>N$
时 
\[
\left|C_{k}-C\right|<\frac{1}{2}\varepsilon.
\]
令 $M=\sum_{k=0}^{N}\left|C_{k}-C\right|$, 则有估计 
\[
|f(x)-C|\leqslant M(1-x)+(1-x)\sum_{k=N+1}\frac{1}{2}\varepsilon x^{k}\leqslant M(1-x)+\frac{1}{2}\varepsilon.
\]
因此, 当 $0<1-x<\frac{\varepsilon}{2M+1}$ 时, 
\[
|f(x)-C|\leqslant M\frac{\varepsilon}{2M+1}+\frac{1}{2}\varepsilon<\varepsilon.
\]
这说明 $\lim_{x\rightarrow1^{-}}f(x)=C$.

\paragraph{定理 8.4.7 (Abel). }

设级数 $\sum_{n=0}^{\infty}a_{n}$, $\sum_{n=0}^{\infty}b_{n}$ 以及它们的乘积
$\sum_{n=0}^{\infty}c_{n}$ 均收敛, 则 
\[
\sum_{n=0}^{\infty}c_{n}=\left(\sum_{n=0}^{\infty}a_{n}\right)\left(\sum_{n=0}^{\infty}b_{n}\right).
\]
证明. 当 $x\in[0,1)$ 时, 级数 $\sum_{n=0}^{\infty}a_{n}x^{n}$ 和 $\sum_{n=0}^{\infty}b_{n}x^{n}$
绝对收敛, 它们的乘积级数为 $\sum_{n=0}^{\infty}c_{n}x^{n}$. 根据 Cauchy 定理, 有 
\[
\sum_{n=0}^{\infty}c_{n}x^{n}=\left(\sum_{n=0}^{\infty}a_{n}x^{n}\right)\left(\sum_{n=0}^{\infty}b_{n}x^{n}\right).
\]
令 $x\rightarrow1^{-}$, 由上述 Abel 引理即得欲证结论.

\subsubsection{乘积级数}

将$\prod_{n=1}^{\infty}p_{n}$称为无穷乘积, 记部分乘积$P_{n}=\prod_{k=1}^{n}p_{k}$,
($n\ge1$). 当$\lim_{n\to\infty}P_{n}$有限且非零时, 称无穷乘积收敛, 否则称它发散.

\paragraph{命题 8.4.8.}

设 $p_{n}>0,\forall n\geqslant1$. 则 

(1) 无穷乘积 $\prod_{n=1}^{\infty}p_{n}$ 收敛当且仅当级数 $\sum_{n=1}^{\infty}\ln p_{n}$
收敛, 且 
\[
\prod_{n=1}^{\infty}p_{n}=e^{\sum_{n=1}^{\infty}\ln p_{n}};
\]

(2) 记 $p_{n}=1+a_{n}$. 如果 $n$ 充分大时 $a_{n}>0$ (或 $a_{n}<0$ ), 则无穷乘积
$\prod_{n=1}^{\infty}p_{n}$ 收敛当且仅当级数 $\sum_{n=1}^{\infty}a_{n}$
收敛; 

(3) 如果级数 $\sum_{n=1}^{\infty}a_{n}$ 和 $\sum_{n=1}^{\infty}a_{n}^{2}$
均收敛, 则无穷乘积 $\prod_{n=1}^{\infty}\left(1+a_{n}\right)$ 也收敛. 

\paragraph{证明. }

(1) 是显然的. (2) 只要利用 
\[
\lim_{n\rightarrow\infty}\frac{\ln\left(1+a_{n}\right)}{a_{n}}=1
\]
以及数项级数的比较判别法即可. 

(3) 则是利用 ($a_{n}$ 不为零时) 
\[
\lim_{n\rightarrow\infty}\frac{\left[a_{n}-\ln\left(1+a_{n}\right)\right]}{a_{n}^{2}}=\frac{1}{2}
\]
以及(1).

\subsubsection{级数重排}

\paragraph{定理 8.4.9 (Riemann). }

如果 $\sum_{n=1}^{\infty}a_{n}$ 为条件收敛的级数, 则可以将它重排为一个收敛级数, 使得重排后的级数和为任意指定的实数.

\paragraph{例.}

求证:
\[
\left(1+\frac{1}{3}-\frac{1}{2}\right)+\left(\frac{1}{5}+\frac{1}{7}-\frac{1}{4}\right)+\cdots+\left(\frac{1}{4n-3}+\frac{1}{4n-1}-\frac{1}{2n}\right)+\cdots=\ln\left(2\sqrt{2}\right).
\]
这说明

\begin{align*}
	\ln2 & ={\color{teal}1-\frac{1}{2}+\frac{1}{3}}{\color{purple}-\frac{1}{4}+\frac{1}{5}}-\frac{1}{6}{\color{purple}+\frac{1}{7}}-\cdots{\color{orange}-\frac{1}{2n}}+\frac{1}{2n+1}-\cdots{\color{orange}+\frac{1}{4n-3}}-\frac{1}{4n-2}{\color{orange}+\frac{1}{4n-1}}-\cdots\\
	& \ \xcancel{=}\left({\color{teal}1+\frac{1}{3}-\frac{1}{2}}\right)+\left({\color{purple}\frac{1}{5}+\frac{1}{7}-\frac{1}{4}}\right)+\cdots+\left({\color{orange}\frac{1}{4n-3}+\frac{1}{4n-1}-\frac{1}{2n}}\right)+\cdots\\
	& =\ln\left(2\sqrt{2}\right).
\end{align*}