\chapter{数学分析}
\section{极限\index{极限}}
\bq{}{}
设函数$\varphi(x)$可导, 且满足$\varphi(0)=0$, 又设$\varphi'(x)$单调减少. 
\begin{enumerate}
\item 证明: 对$x\in(0,1)$, 有$\varphi(1)x<\varphi(x)<\varphi'(0)x$.
\item 若$\varphi(1)\ge0$, $\varphi'(0)\le1$, 任取$x_{0}\in(0,1)$, 令$x_{n}=\varphi(x_{n-1})$,
($n=1,2,\cdots$). 证明: $\lim_{n\to\infty}x_{n}$ 存在, 并求该极限值.
\end{enumerate}
\eq
\ba
对于任意的$x\in[0,1],$在$[0,x]$上用拉格朗日定理,
\bee
\varphi(x)-\varphi(0)=\varphi'(\xi_{1})x<\varphi'(0)x
\eee
在$[x,1]$上用拉格朗日定理
\bee
\varphi(1)-\varphi(x)=\varphi'(\xi_{2})(1-x)<\varphi'(\xi_{1})(1-x)=\varphi'(\xi_{1})-\varphi(x)
\eee
所以$\varphi(1)x<\varphi'(\xi_{1})x=\varphi(x)$.
\ea

\section{导数}

\bq{}{}
函数$f(x)\in C[0,1]$, 在$(0,1)$上可微, 对于任意的$x\in(0,1)$, $\abs{xf'(x)-f(x)+f(x)}<Mx^2$, 问$f'(0)$的存在性.
\eq
\ba
不妨设$f(0)=0$, 定义$h(x)=\frac{f(x)}{x}$($0<x<1$), 即证$\lim_{x\to0}h(x)$存在, 则$\abs{x^2h'(x)}<Mx^2$,
所以$\abs{h'(x)}<M$, 所以若$\{x_n\}\to0$, 则有
\bee
\abs{h(x_m)-h(x_n)}=\abs{h'{\xi}(x_m-x_n)}<M\abs{x_m-x_n}<\varepsilon.
\eee
所以$\{h(x_n)\}$是Cauchy列, 从而$\lim_{x\to0}h(x)$存在.
\ea

\bq{}{}
构造有界单调函数$f(x)$使得对于任意的$x\in\RR$, $f'(x)$存在, 且$\lim_{x\to\pm\i}f'(x)\ne0$.
\eq
\ba
取$a_n=1-2^{-n}$, ($n\in\NN$), $f(n)=a_n$, $f\paren{n+\frac12}=\frac12\paren{a_n+a_{n+1}}$, 
且$f'(n)=0$, $f'\paren{n+\frac12}=1$, 将其它点处可微连接$f(n)$这些离散点, 知$\lim_{x\to\pm\i}f'(x)$不存在, 从而不为0.
\ea

\section{积分}

\bq{}{}
\bee
\int_{-1}^1 \frac{\sqrt{\frac{x+1}{1-x}} \log \left(\frac{2 x^2+2 x+1}{2 x^2-2
		x+1}\right)}{x} \, dx
\eee
\eq
\ba
$4\pi \mathrm{arccot}(\sqrt{\phi}).$
\ea

\bq{}{}
设函数$f(x)$在$[a,b]$上有连续的导数, 且$f(a)=0$, 证明
\bee
\int_{a}^{b}f^{2}(x)\mathrm{d}x\le\frac{(b-a)^{2}}{2}\int_{a}^{b}[f'(x)]^{2}\mathrm{d}x.
\eee
\eq
\ba
令$F(b)=RHS-LHS,$证$F'(b)\ge0$即可.
\ea

\bq{}{}
设函数$f(x)$在$[0,1]$上有二阶连续的导数, 证明:
\begin{enumerate}
\item 对任意$\xi\in\paren{0,\frac{1}{4}}$和$\eta\in\paren{\frac{3}{4},1}$有
\[
\abs{f'(x)}<2\abs{f(\xi)-f(\eta)}+\int_{0}^{1}\abs{f''(x)}\mathrm{d}x\quad x\in[0,1]
\]
\item 当$f(0)=f(1)=0$及$f(x)\ne0$, ($x\in(0,1)$)时有
\[
\int_{0}^{1}\abs{\frac{f''(x)}{f(x)}}\mathrm{d}x\ge4.
\]
\end{enumerate}
\eq
\ba
用中值定理, 
\begin{align*}
\abs{f'(x)}-2\abs{f(\xi)-f(\eta)} & =\abs{f'(x)}-2\abs{f'(\theta)}(-\xi+\eta)\\
 & \le\abs{f'(x)}-\abs{f'(\theta)}\\
 & \le\abs{f'(x)-f'(\theta)}=\abs{\int_{\theta}^{x}f''(t)\mathrm{d}t}\\
 & \le\int_{0}^{1}\abs{f''(t)}\mathrm{d}t
\end{align*}
最后取$f(x_{0})=\max_{x\in[0,1]}f(x)$, 则
\bee
f(x_0)=f'(\xi_1)x_0=f'(\xi_2)(x_0-1),
\eee
所以
\begin{align*}
\int_0^1\abs{\frac{f''}{f}}\ud x & \ge\frac1{\abs{f(x_0)}}\ud x\\
 & \ge \frac1{\abs{f(x_0)}}\abs{\int_{\xi_1}^{\xi_2}f''\ud x}\\
 & = \frac{1}{\abs{f(x_0)}}\abs{f'(\xi_2)-f'(\xi_1)}\\
 & = \frac1{x_0}+\frac1{1-x_0}\ge4.
\end{align*}
\ea

\bq{}{}
设函数$f(x)$在$\left[-\frac1a,a\right]$上连续(其中$a>0$), 且$f(x)\ge0$, $\int_{-\frac1a}^{a}xf(x)\ud x=0$, 求证: $\int_{-\frac1a}^ax^2f(x)\ud x\le\int_{-\frac1a}^af(x)\ud x$.
\eq
\ba
因$(a-x)\paren{x+\frac1a}\ge0$, 对$(a-x)\paren{x+\frac1a}f(x)\ge0$两边同时积分.
\ea

\bq{}{}
设函数$f(x)$在$[0,1]$上连续, $\int_0^1f(x)\ud x=0$, $\int_0^1xf(x)\ud x=1$. 求证: 
\begin{enumerate}[1.]
 \item 存在$\xi\in[0,1]$, 使得$|f(\xi)|\ge4$;
 \item 存在$\eta\in[0,1]$, 使得$\abs{f(\eta)}=4$.
\end{enumerate}
\eq
\ba
用反证法, 
\bee
1=\abs{\int_0^1\paren{x-\frac12}f(x)\ud x}
  \le \int_0^1\abs{x-\frac12}\cdot\abs{f}\ud x
  \le 1.
\eee
等号取不到, 否则, 
\bee
\int_0^1\paren{4-\abs{f}}\abs{x-\frac12}\ud x=0.
\eee
\ea

\bq{}{}
求$\int_0^{2\pi}\frac{\ud \t}{3-\sin 2\t}$.
\eq
\ba
$\int_0^{2\pi}\frac{\ud \t}{3-\sin 2\t}
=\int_0^{2\pi}\frac{\ud \t}{2+2\sin^2\left(\t-\frac{\pi}{4}\right)}
=\frac12\int_0^{2\pi}\frac{\ud \t}{1+\sin^2\t}
=2\int_{0}^{2\pi}\frac{\ud \t}{1+\sin^2\t}
=2\int_0^{+\i}\frac{\ud t}{1+2t^2}=\frac{\sqrt{2}}{2}\pi$.
\ea
\ba
\begin{align*}
 \int_0^{2\pi}\frac{\ud \t}{3-\sin 2\t}
 &= \frac12\int_0^{4\pi}\frac{\ud t}{3-\sin t}\\
 &= \int_0^{2\pi}\frac{\ud t}{3-\sin t}\\
 &= \int_0^{\pi}\frac{\ud t}{3-\sin t}+\int_{0}^{\pi}\frac{\ud t}{3+\sin t}\\
 &= 2\int_{0}^{\pi/2}\frac{\ud x}{3-\sin x}+2\int_{0}^{\pi/2}\frac{\ud x}{3+\sin x}\\
 &= 12\int_{0}^{\pi/2}\frac{\ud x}{9-\sin^2x}\\
 &= 12\int_0^{\pi/2}\frac{\ud\tan x}{8\tan^2x+9}\\
 &= \frac{12}{6\sqrt{2}}\arctan\left(\frac{2\sqrt{2}}{3}\tan x\right)\big\vert_0^{\pi/2}
 =\frac{\sqrt{2}}{2}\pi.
\end{align*}
\ea
\ba
\begin{align*}
 \int_0^{2\pi}\frac{\ud \t}{3-\sin2\t}
 &=\frac12\int_0^{4\pi}\frac{\ud \t}{3\sin\t}\\
 &=\int_0^{2\pi}\frac{\ud \t}{3-\sin\t}\\
 &=\oint_{\abs{z}=1}\frac{\ud z}{\ui z\left(3-\frac{z-z^{-1}}{2i}\right)}\\
 &=2\oint_{\abs{z}=1}\frac{\ud z}{z^2+6\ui z-1}\\
 &=2\oint_{\abs{z}=1}\frac{\ud z}{(z-(-3+2\sqrt{2})\ui)(z-(-3-2\sqrt{2})\ui)}\\
 &=2\cdot2\pi\ui\Res(f(z))\big\vert_{z=(-3+2\sqrt{2})\ui}
 =\frac{\sqrt{2}}{2}\pi.
\end{align*}
\ea
\ba
\begin{align*}
 \int_0^{2\pi}\frac{\ud \t}{3-\sin 2\t}
 &=\int_0^{2\pi}\frac{\ud \tan\t}{3\tan^2\t-2\tan\t+3}\\
 &=\int_0^{\pi/2}+\int_{\pi/2}^{\pi}+\int_{\pi}^{3\pi/2}+\int_{3\pi/2}^{2\pi}=\cdots
\end{align*}
于是
\bee
\int_0^{\pi/2}
=\int_{0}^{\i}\frac{\ud\t}{3t^2-2t+3}
=\frac13\int_0^{+\i}\frac{\ud\left(t-\frac13\right)}{\left(t-\frac13\right)^2+\frac89}
=\frac13\cdot\frac3{\sqrt{8}}\arctan\left(\frac{3\left(t-\frac13\right)}{\sqrt{8}}\right)\big\vert_{0}^{+\i}
=\frac{\sqrt{2}}{4}\left(\frac{\pi}{2}+\arctan\frac{\sqrt{2}}{4}\right).
\eee
其它三个同理. 所以$\sum=\frac{\sqrt{2}}{4}\left(\frac{\pi}{2}\cdot4\right)=\frac{\sqrt{2}}{2}\pi$.
\ea

\bq{}{}
求
\bee
\int_0^{\pi/4}\frac{(\cot x-1)^{p-1}}{\sin^2x}\ln\tan x\ud x=-\frac{\pi}{p}\csc p\pi, \ (-1<p<0).
\eee
\eq
\ba
令$u=\cot x-1$, 原式等价于
\bee
\int_{0}^{+\i}u^{p-1}\ln(u+1)\ud u=\frac{\pi}{p}\csc p\pi,
\eee
而
\bee
\int_{0}^{\i}u^{p-1}\ln(u+1)\ud u
=\int_{0}^{\i}u^{p-1}\int_{1}^{1+u}\frac1y\ud y\ud u
\eee
交换积分次序, 用Beta函数
\ea

\section{级数}



\bq{}{}
证明
\bee
1+x+\frac{x^2}{2!}+\cdots+\frac{x^{2n}}{(2n)!}=0
\eee
无实根.
\eq
\ba
设$-y<0$, 则$y>0$, 所以$1-y+\frac{y^2}{2!}-\frac{y^3}{3!}+\cdots+\frac{y^{2n}}{(2n)!}>\ue^{-y}>0$.
\ea

\bq{F.F.Abi=Khuzam and A.B.Boghossian, Some recent geometric inequalities, AMM Vol 96(1989), No. 7:576-589}{}
函数$f(x)=\cot x-\frac1x$, 则$f^{(k)}(x)<0$, $0<x<\pi$, $k\in\NN$.
\eq
\ba
\bee
\cot x=\frac1x+\sum_{k=1}^{\i}\paren{\frac1{k\pi+x}-\frac1{k\pi-x}}, \quad x\in(0,\pi).
\eee
将真分式展开有
\bee
f(x)=-2x\sum_{k=0}^{\i}c_kx^{2k}, \  x\in(0,\pi), \quad c_k=\sum_{n=1}^{\i}\frac1{(n\pi)^{2k+2}},\ (k\in\NN).
\eee
\ea

\bq{}{}
对$n\in\pNN$, 确定$(0,1)$的子集, 使在此子集上$\paren{\frac{\ud}{\ud x}}^n\paren{\ln x\ln(1-x)}<0$.
\eq
\ba
\bee
f'(x)=(\ln x\ln(1-x))'=\sum_{m=1}^{\i}\frac{(1-x)^{m-1}-x^{m-1}}{m},
\eee
当$n$是偶数时, 所有项都是负的, 当$n$是奇数时, 仅当$1-x<x$即$x>\frac12$时, $f^{(n)}(x)$时负的.
\ea

\bq{}{}
已知$S_n=\frac{n+1}{2^{n+1}}\sum_{i=1}^{n}\frac{2^i}{i}$, 证明$\lim_{n\to\i}S_n$存在并求其值.
\eq
\ba
因$S_{n+1}=\frac{n+2}{2(n+1)}(S_n+1)$, 所以
\bee
S_{n+2}-S_{n+1}=\frac{(n+2)^2(S_{n+1}-S_{n})-S_{n+1}-1}{2(n+1)(n+2)},
\eee
$S_4-S_3=0$, 当$n\ge3$时, $S_n$不增, 所以$S=\lim_{n\to\i}S_{n}$存在, 所以$S=\lim_{n\to\i}\frac{n+2}{2(n+1)}(S+1)$, 
得$S=1$.
\ea

\section{其他}

\bq{}{}
证明:
\bee
\lim_{n\to\infty}\left(\frac12\cdot\frac34\cdots\frac{2n-1}{2n}\right)=0.
\eee
\eq
\ba
用$\frac12\cdot\frac34\cdots\frac{2n-1}{2n}<\frac1{\sqrt{2n+1}}$.
\ea

\bq{}{}
证明:
\bee
0<\ue-\left(1+\frac1n\right)^n<\frac{3}{n}.
\eee
\eq
\ba
先证: $x_n=\left(1+\frac1n\right)^n$单调上升且有界. ($1\cdot x_n\le\left(\frac{1+n(1+1/n)}{n+1}\right)^{n+1}=x_{n+1}$), 则
\begin{align*}
 x_n & = \sum_{k=0}^{n}\binom{n}{k}\frac1{n^k}=1+1+\binom{n}{2}\frac1{n^2}+\cdots\\
 &= 2+\frac1{2!}\left(1-\frac1n\right)+\frac1{3!}\left(1-\frac1n\right)\left(1-\frac2n\right)+\cdots+\frac1{n!}\left(1-\frac1n\right)\cdots\left(1-\frac{n-1}{n}\right)\\
 &< 2+\frac1{2\cdot 1}+\frac1{3\cdot 2}+\cdots+\frac1{n(n-1)}=3.
\end{align*}
再证$y_n=\left(1+\frac1n\right)^{n+1}$单调下降且有界. (用$(1+x)^n>1+nx$, $x>-1$证单调性).

由$\ue=\lim_{n\to\infty}x_n=\lim_{n\to\infty}y_n$, $x_n<\ue<y_n$, (这可以证得{\color{red}{$\frac1{n+1}<\ln\left(1+\frac1n\right)<\frac1n$}}).

故$\ue-x_n<y_n-x_n=\frac{x_n}{n}<\frac3n$.
\ea

\bq{}{}
证明不等式
\bee
\left(\frac{n}{\ue}\right)^n<n!<\ue\left(\frac{n}{2}\right)^n.
\eee
\eq
\ba
用$2<\left(1+\frac1n\right)^n<\ue$及归纳法.
\ea

\bq{}{}
设$a^{[n]}=a(a-h)\cdots[a-(n-1)h]$及$a^{[0]}=1$, 证明:
\bee
(a+b)^{[n]}=\sum_{m=0}^{n}\binom{n}{m}a^{[n-m]}b^{[m]}.
\eee
并由此推出牛顿二项式公式.
\eq

\bq{}{}
证明不等式
\bee
n!<\left(\frac{n+1}{2}\right)^n\quad (n>1).
\eee
\eq
\ba
用均值不等式.
\ea
\ba
用伯努利不等式证$\left(\frac{n+2}{n+1}\right)^{n+1}=\left(1+\frac1{n+1}\right)^{n+1}>2$, ($n\in\pNN$).
\ea

\bq{}{}
设$p_n$($n\in\pNN$)为趋于正无穷的任意数列, 而$q_n$($n\in\pNN$)为趋于负无穷的任意数列($p_n, q_n\not\in[-1,0]$), 求证:
\bee
\lim_{n\to\infty}\left(1+\frac1{p_n}\right)^{p_n}=\lim_{n\to\infty}\left(1+\frac1{q_n}\right)^{q_{n}}=\ue.
\eee
\eq
\ba
注意$[x]\le x<[x]+1$.
\ea

\bq{}{201805317}
已知$\lim_{n\to\infty}\left(1+\frac1n\right)^n=\ue$, 求证:
\bee
\lim_{n\to\infty}\left(1+1+\frac{1}{2!}+\frac{1}{3!}+\cdots+\frac{1}{n!}\right)=\ue.
\eee
并推出
\bee
\ue=1+1+\frac{1}{2!}+\frac{1}{3!}+\cdots+\frac{1}{n!}+\frac{\t_n}{n!n},
\eee
其中$0<\t_n<1$.
\eq

\bq{}{}
证明: $\ue$是无理数.
\eq
\ba
用反证法及\ref{q:201805317}有, 对于任意的$n$, $n!n\cdot\ue$不是整数.
\ea

\bq{}{}
证明不等式:
\begin{enumerate}[(a)]
 \item $\frac{1}{n+1}<\ln\left(1+\frac1n\right)<\frac1n$, ($n\in\pNN$);
 \item $1+\a<\ue^{\a}$, ($\a\ne0, \a\in\RR$).
\end{enumerate}
\eq
\ba
\begin{enumerate}[(a)]
 \item 原式等价于$\left(1+\frac1n\right)^n<\ue<\left(1+\frac1n\right)^{n+1}$;
 \item $\a>-1$时, 用伯努利不等式, $\ue^{\a}>\left(1+\frac1n\right)^{\a n}>1+\a$.
\end{enumerate}
\ea

\bq{}{2018053110}
证明: (在以下各极限均存在的情况下)
\begin{enumerate}[(a)]
 \item $\liminf_{n\to\infty}x_n+\liminf_{n\to\infty}y_n\le\liminf_{n\to\infty}(x_n+y_n)\le\liminf_{n\to\infty}x_n+\limsup_{n\to\infty}y_n$;
 \item $\liminf_{n\to\infty}x_n+\limsup_{n\to\infty}y_n\le\limsup_{n\to\infty}(x_n+y_n)\le\limsup_{n\to\infty}x_n+\limsup_{n\to\infty}y_n$.
\end{enumerate}
\eq
\ba
用$\liminf_{n\to\infty}x_n=-\limsup_{n\to\infty}(-x_n)$.
\ea

\bq{}{}
证明: 若$\lim_{n\to\infty}x_n$存在, 则对于任何数列$y_n$($n\in\pNN$), $\limsup_{n\to\infty}y_n$有限且有:
\begin{enumerate}[(a)]
 \item $\limsup_{n\to\infty}(x_n+y_n)=\lim_{n\to\infty}x_n+\limsup_{n\to\infty}y_n$;
 \item $\limsup_{n\to\infty}x_ny_n=\lim_{n\to\infty}x_n\cdot\limsup_{n\to\infty}y_n$, ($x_n\ge0$).
\end{enumerate}
\eq
\ba
用\ref{q:2018053110}.
\ea

\bq{}{}
证明: 若对于某数列$x_n$($n\in\pNN$), 无论数列$y_n$($n\in\pNN$)如何选取, 以下两个等式中都至少有一个成立:
\begin{enumerate}[(a)]
 \item $\limsup_{n\to\infty}(x_n+y_n)=\limsup_{n\to\infty}x_n+\limsup_{n\to\infty}y_n$.
 \item $\limsup_{n\to\infty}(x_ny_n)=\limsup_{n\to\infty}x_n\cdot\limsup_{n\to\infty}y_n$, ($x_n\ge0$).
\end{enumerate}
则数列$x_n$收敛或发散于正无穷.
\eq

\bq{}{}
证明: 若$x_n>0$($n\in\pNN$)及
\bee
\limsup_{n\to\infty}x_n\cdot\limsup_{n\to\infty}\frac1{x_n}=1.
\eee
则数列$x_n$是收敛的.
\eq

\bq{}{}
证明: 若数列$x_n$($n\in\pNN$)有界, 且
\bee
\lim_{n\to\infty}(x_{n+1}-x_n)=0.
\eee
则此数列的子列极限充满于下极限和上极限
\bee
l=\liminf_{n\to\infty}x_n\  \textrm{和}\ L=\limsup_{n\to\infty}x_n
\eee
之间.
\eq

\bq{}{2018053117}
证明: 若$x_n>0$($n\in\pNN$)且$\lim_{n\to\infty}\frac{x_{n+1}}{x_n}$存在, 则
\bee
\lim_{n\to\infty}\sqrt[n]{x_n}=\lim_{n\to\infty}\frac{x_{n+1}}{x_n}.
\eee
\eq
\ba
用结论: 若$\{x_n\}\to x$, $x_n>0$, 则$\lim_{n\to\infty}\sqrt[n]{x_1x_2\cdots x_n}=\lim_{n\to\infty}x_n=x$.
\ea

\bq{}{}
证明: $\lim_{n\to\infty}\frac{n}{\sqrt[n]{n!}}=\ue$.
\eq
\ba
用\ref{q:2018053117}.
\ea

\bq{数$a$和$b$的算术几何平均值}{}
证明: 由下列各式
\bee
x_1=a,\quad y_1=b,\quad x_{n+1}=\sqrt{x_ny_n},\quad y_{n+1}=\frac{x_n+y_n}{2},
\eee
确定的数列$x_n$和$y_n$($n\in\pNN$)有共同的极限.
\bee
\m(a,b)=\lim_{n\to\infty}x_n=\lim_{n\to\infty}y_n.
\eee
\eq
\ba
用幂平均不等式, $\sqrt{x_{n+1}+y_{n+1}}=\frac{\sqrt{x_n}+\sqrt{y_{n}}}{\sqrt{2}}\le\sqrt{x_n+y_n}$.
即$\{y_n\}$单调有界, 从而有极限, 从而$x_n=2y_{n+1}-y_n$有相同的极限.
\ea

\bq{}{}
设
\bee
f\left(x+\frac1x\right)=x^2+\frac1{x^2}\quad(\abs{x}\ge2),
\eee
求$f(x)$.
\eq
\ba
$x^2-2$, ($\abs{x}\ge\frac52$).
\ea

\bq{}{}
证明: 若
\begin{enumerate}[(1)]
 \item 函数$f(x)$定义于区域$x>a$;
 \item $f(x)$在每一个有限区间$a<x<b$内是有界的;
 \item 对于某一个整数$n$, 存在有限的或无穷的极限
 \bee
 \lim_{x\to+\infty}\frac{f(x+1)-f(x)}{x^n}=l,
 \eee
\end{enumerate}
则
\bee
\lim_{x\to+\infty}\frac{f(x)}{x^{n+1}}=\frac{l}{n+1}.
\eee
\eq
{\color{red}{\bf{能否用Cauchy定理\ref{th:cauchy-theorem}证明它.}}}

\bq{}{}
利用定理
\bt{}{}
设
\bee
\lim_{x\to 0}\frac{\phi(x)}{\psi(x)}=1,
\eee
其中$\psi(x)>0$, 再设当$n\to\infty$时$\a_{mn}\rightrightarrows0$($m=1,2,\cdots,n$), 
换言之, 对于任意$\varepsilon>0$, 存在正整数$N(\varepsilon)$, 当$m=1,2,\cdots,n$且$n>N(\varepsilon)$时, 
$0<\abs{\a_{mn}}<\varepsilon$. 证明:
\bee
\lim_{n\to\infty}[\phi(\a_{1n})+\phi(\a_{2n})+\cdots+\phi(\a_{mn})]
=\lim_{n\to\infty}[\psi(\a_{1n})+\psi(\a_{2n})+\cdots+\psi(\a_{mn})],
\eee
此处同时还要假设上式右端的极限存在.
\et
求:
\begin{enumerate}[(1)]
 \item $\lim_{n\to\infty}\sum_{k=1}^{n}\left(\sqrt[n]{1+\frac{k}{n^2}}-1\right)$;
 \item $\lim_{n\to\infty}\sum_{k=1}^{n}\left(\sin\frac{ka}{n^2}\right)$;
 \item $\lim_{n\to\infty}\sum_{k=1}^{n}\left(a^{\frac{k}{n^2}}-1\right)$, ($a>0$);
 \item $\lim_{n\to\infty}\prod_{k=1}^{n}\left(1+\frac{k}{n^2}\right)$;
 \item $\lim_{n\to\infty}\prod_{k=1}^{n}\cos\frac{ka}{n\sqrt{n}}$.
\end{enumerate}
\eq

\bq{}{}
设函数$f(x)$在区间$(x_0,+\infty)$上连续并有界. 证明: 对于任何数$T$, 可求得数列$x_n\to+\i$, 
使
\bee
\lim_{n\to\infty}[f(x_n+T)-f(x_n)]=0.
\eee
\eq

\bq{}{}
证明: 在有限区间$(a,b)$上有定义且连续的函数$f(x)$, 可用连续的方法延拓到闭区间$[a,b]$上, 
其充分必要条件是函数$f(x)$在区间$(a,b)$上一致连续.
\eq

\bq{}{}
$x_n$满足$x_n^n+x_n-1=0$, $0<x_n<1$, 求$\lim_{n\to\infty}x_n$.
\eq
\ba
$y=x^n+x-1$则有$y'>0$, $y\vert_{x=0}=-1<0$, $y\vert_{x=1}=1>0$.

$x_n$是$x^n+x-1$的唯一零点. 由于
\bee
x_{n}^{n+1}+x_n-1=(x_n-1)(1-x_n)<0
\eee
及$y$的单调性, 知$x_{n+1}$在$x_n$与$1$之间, 故$\{x_n\}$单调有界.
反证$\{x_n\}$的极限$A=1$, 否则$0\le A<1$矛盾.
\ea
\ba
$y^x+y-1=0$是隐函数, 确定$y=f(x)$, $x_n=f(n)$, 求导
\bee
y'=-y^x\cdot\frac{\ln y}{\left(1+\frac{x}{y}\cdot y^x\right)}
\eee
当$x\ge1$时, $y'>0$, $y$单调增加, 以下同上.
\ea

\bq{}{}
若级数$\sum_{n=1}^{\infty}a_n^2$, $\sum_{n=1}^{\infty}b_n^2$都收敛, 则以下不成立的是?

A. $\sum_{n=1}^{\infty}(a_n+b_n)^2$收敛;\qquad\qquad B.$\sum_{n=1}^{\infty}\frac{|a_n|}{n}$收敛;\\
C. $\sum_{n=1}^{\infty}a_nb_n$收敛;\qquad\qquad\qquad D. $\sum_{n=1}^{\infty}a_nb_n$发散.
\eq

\bq{}{}
设$f(x,y)$与$\varphi(x,y)$均为可微函数, 且$\varphi'_y(x,y)\ne0$. 已知$(x_0,y_0)$是$f(x,y)$在约束条件$\varphi(x,y)=0$下的一个极值点,
下列选项正确的是(D)

(A) 若$f'_x(x_0,y_0)=0$, 则$f'_y(x_0,y_0)=0$;\quad (B) 若$f'_{x}(x_0,y_0)=0$, 则$f'_{y}(x_0,y_0)\ne0$;\\
(C) 若$f'_{x}(x_0,y_0)\ne0$, 则$f'_{y}(x_0,y_0)=0$;\quad (D) 若$f'_{x}(x_0,y_0)\ne0$, 则$f'_{y}(x_0,y_0)\ne0$.
\eq

\bq{}{}
\bee
\frac{\int_0^1\frac{1}{\sqrt{1-t^4}}\ud t}{\int_0^1\frac{1}{\sqrt{1+t^4}}\ud t}=\sqrt{2}
\eee
\eq
\ba
用Beta函数.
\ea

\bq{陕西省第七次大学生高等数学竞赛复赛}{}
计算$I=\int_{\pi/8}^{3\pi/8}\frac{\sin^2x}{x(\pi-2x)}\ud x$.
\eq

\bq{陕西省第七次大学生高等数学竞赛复赛}{}
设$\varphi(x)$在$(-\infty, 0]$可导, 且函数
\bee
f(x)=\begin{cases}
      \int_{x}^{0}\frac{\varphi(t)}{t}\ud t, & x<0,\\
      \lim_{n\to\infty}\sqrt[n]{(2x)^n+x^{2n}}, & x\ge0.
     \end{cases}
\eee
在点$x=0$可导, 求$\varphi(0)$, $\varphi'(0^-)$, 并讨论$f'(x)$的存在性.
\eq

\bq{陕西省第七次大学生高等数学竞赛复赛}{}
已知函数$f(x)$与$g(x)$满足$f'(x)=g(x)$, $g'(x)=2\ue^x-f(x)$, 且$f(0)=0$, 求
\bee
\int_0^{\pi}\left(\frac{g(x)}{1+x}-\frac{f(x)}{(1+x)^2}\right)\ud x.
\eee
\eq

\bq{陕西省第七次大学生高等数学竞赛复赛}{}
设$y_1$和$y_2$是方程$y''+p(x)y'+2\ue^xy=0$的两个线性无关解, 而且$y_2=(y_1)^2$. 若有$p(0)>0$,
求$p(x)$及此方程的通解.
\eq

\bq{陕西省第七次大学生高等数学竞赛复赛}{}
设$f(x)$在$\left[-\frac1a,a\right]$($a>0$)上非负可积, 且$\int_{-1/a}^{a}xf(x)\ud x=0$. 求证:
\bee
\int_{-1/a}^{a}x^2f(x)\ud x\le\int_{-1/a}^{a}f(x)\ud x.
\eee
\eq
\ba
分$0<a\le1$和$a>1$两种情况讨论.
\ea

\bq{陕西省第七次大学生高等数学竞赛复赛}{}
设在点$x=0$的某邻域$U$内, $f(x)$可展成泰勒级数, 且对任意正整数$n$, 皆有
\bee
f\left(\frac1n\right)=\frac1{n^2}.
\eee
证明: 在$U$内, 恒有$f(x)=x^2$.
\eq

\bq{陕西省第七次大学生高等数学竞赛复赛}{}
设$\frac{\ud y}{\ud x}=\frac1{1+x^2+y^2}$, 证明: $\lim_{x\to+\infty}y(x)$和$\lim_{x\to-\infty}y(x)$都存在.
\eq
\ba
用单调有界定理.
\ea

\bq{陕西省第七次大学生高等数学竞赛复赛}{}
求幂级数$\sum_{n=1}^{\infty}\left(1+\frac12+\frac13+\cdots+\frac1n\right)x^n$的收敛域与和函数.
\eq

\bq{陕西省第七次大学生高等数学竞赛复赛}{}
求极限
\bee
\lim_{n\to\infty}\sum_{i=1}^n\left(\frac1{(n+i+1)^2}+\frac1{(n+i+2)^2}+\cdots+\frac1{(n+i+i)^2}\right).
\eee
\eq
\ba
用重积分得$\ln\frac{2}{\sqrt{3}}$.
\ea

\bq{}{20170404012}
设$\{f_n(x)\}_{n=1}^{\infty}\subset C_{[a,b]}$, 且$f_n(x)$在$[a,b]$上一致收敛于$f(x)$, 则$\lim_{n\to\infty}\int_{a}^{b}f_n(x)\ud x=\int_a^bf(x)\ud x$.
\eq
\ba
因$f_n(x)\rightrightarrows f(x)$, $x\in[a,b]$, 所以$f\in C_{[a,b]}$且$\forall \varepsilon>0$, $\exists N$, 当$n>N$时, $\forall x\in[a,b]$均有$|f_n(x)-f(x)|<\varepsilon$.
$f(x), f_n(x)$连续必可积, 有
\bee
\left|\int_a^b f_{n}(x)\ud x-\int_a^bf(x)\ud x\right| < (b-a)\varepsilon.
\eee
其实当$a\le x\le b$时, 有$\left|\int_a^xf_n(t)\ud t-\int_a^xf(t)\ud t\right|<(b-a)\varepsilon$对$x$一致成立, 
所以$\int_a^xf_n(x)\ud t\rightrightarrows\int_a^xf(t)\ud t$.
\ea

\bq{}{}
$f_n(x)$在$[a,b]$上都有连续导数, 且$f_n(x)\to f(x)$, $f'_n(x)\rightrightarrows g(x)$, 则$f'(x)=g(x)$, 
即$\frac{\ud}{\ud x}\left(\lim_{n\to\infty}f_n(x)\right)=\lim_{n\to\infty}\frac{\ud}{\ud x}f_n(x)$.
\eq
\ba
因$f'_n\rightrightarrows g$, 所以$g$连续, 可积, 由\ref{q:20170404012}, $\int_a^xg(t)\ud t=\lim_{n\to\infty}\int_a^xf'_n(t)\ud t=f(x)-f(a)$,
所以$f'(x)=g(x)$. 其实$f_n(x)=f_n(a)+\int_a^xf'_n(t)\ud t$在$[a,b]$上也一致收敛.

若\ref{q:20170404012}和$\sum_{k=1}^{\infty}u_k(x)$的前$n$项部分和, 就有函数项级数的相应命题.
\begin{enumerate}[(1). ]
 \item 若$[a,b]$上$\sum_{k=1}^{\infty}u_k(x)$中每项$u_k$均连续, 且$\sum_{k=1}^{\infty} u_k(x)\rightrightarrows f(x)$, 则$f(x)\in C_{[a,b]}$且
 \bee
 \sum_{k=1}^{\infty}\int_a^bu_k(x)\ud x=\int_a^bf(x)\ud x.
 \eee
 \item 若$[a,b]$上, $\sum_{k=1}^{\infty}u_k(x)$的每项都有连续导数$u'_k(x)$且$\sum_{k=1}^{\infty}u'_k(x)\rightrightarrows g(x)$, 
 而$\sum_{k=1}^{\infty}u_k(x)\to f(x)$. 则
 \bee
 \frac{\ud}{\ud x}\left(\sum_{k=1}^{\infty}u_k(x)\right)=\sum_{k=1}^{\infty}u'_k(x).
 \eee
\end{enumerate}
\ea

\bq{}{}
举反例:
\begin{enumerate}[(1)]
 \item 积分的极限不等于极限的积分的函数列.
 \item 导数的极限不等于极限的导数的函数列.
\end{enumerate}
\eq
\ba
(1). 在$[0,1]$上极限函数为0, 但$f_n$在$\left[0,\frac1n\right]$上面积为1的脉冲函数.

(2). $f_n(x)=\frac{x}{1+n^2x^2}$, $x\in[-1,1]$.
\ea

\bq{http://math.stackexchange.com/questions/2143014}{}
证明: 对于任意的$\alpha\in\mathbb{Q}\setminus\mathbb{Z}$, $\sum_{n=1}^{\infty}\ln\left|\frac{\alpha-n}{\alpha+n}\right|$发散.
\eq

\bq{http://math.stackexchange.com/questions/472007}{}
判断$\sum_{n=10}^{\infty}\frac{\sin n}{n+10\sin n}$的敛散性.
\eq
\ba
其实, 若$f(n)$是有界函数, 且级数$\sum_{n=1}^{n}\frac{f(n)}{n}$收敛, $a$是使任意的$n$都有$n+af(n)\ne 0$, 则$\sum_{n=1}^{\infty}\frac{f(n)}{n+af(n)}$.

这是因为
\bee
\sum_{n=1}^{m}\frac{f(n)}{n+af(n)}
  = \sum_{n=1}^{m}\frac{f(n)}{n}+\sum_{n=1}^{m}\frac{-af^2(n)}{n(n+af(n))}
\eee
后一个和式用比较判别法.
\ea

\bq{http://math.stackexchange.com/questions/273559}{}
判断$\sum_{n=1}^{\infty}\frac{\sin^2(n)}{n}$的敛散性.
\eq
\ba
比较判别法, $[k\pi+\frac{\pi}{6}, (k+1)\pi-\frac{\pi}{6}]$中总有至少一个整数.

或用$\sum\frac{\sin^2(n)}{n}=\sum\frac{1}{2n}-\sum\frac{\cos(2n)}{2n}$, 前者发散, 后者用Dirichlet判别法.
\ea

\bq{http://math.stackexchange.com/questions/991652}{}
证明$\int_0^1\left|\frac{1}{x}\sin\frac{1}{x}\right|\ud x$发散.
\eq
\ba
变量替换, 积分化为$\int_1^{\infty}\frac{|\sin x|}{x}\ud x$, 然后在子区间$(k\pi, (k+1)\pi)$上求下界. 
\ea

\bq{http://math.stackexchange.com/questions/620449}{}
证明: 求$p\in\mathbb{R}$, 使积分$\int_0^{\infty}\frac{x^p}{1+x^p}\ud x$发散.
\eq
\ba
$p\ge-1$.
\ea

\bq{http://math.stackexchange.com/questions/596511}{}
计算
\bee
\int_0^{\infty}\frac{\ue^{-x}-\ue^{-2x}}{x}\ud x
\eee
\eq
\ba
用Frullani积分. 结果为$\log 2$. 或用重积分: $\ue^{-x}-\ue^{-2x}=x\int_1^2\ue^{-xt}\ud t$. 这里给出Frullani积分证明过程的做法.
\begin{align*}
 \int_a^b\frac{\ue^{-x}-\ue^{-2x}}{x}\ud x
  & =\int_a^b\frac{\ue^{-x}}{x}\ud x - \int_{2a}^{2b}\frac{\ue^{-x}}{x}\ud x\\
  & =\left(\int_a^b-\int_{2a}^{2b}\right)\frac{\ue^{-x}}{x}\ud x\\
  & = \left(\int_a^{2a}-\int_{b}^{2b}\right)\frac{\ue^{-x}}{x}\ud x\to\log(2)-0, \quad a\to0, b\to\infty
\end{align*}
上式最后一步用$\ue^{-2c}\log(2)\le\int_c^{2c}\frac{\ue^{-x}}{x}\ud x\le\ue^{-c}\log(2)$.
\ea

\bq{http://math.stackexchange.com/questions/590774}{}
已知$a>b>0$, 求$\lim_{t\to0^+}(a^{-t}-b^{-t})\Gamma(t)$.
\eq
\ba
注意到$\Gamma(t)=a^t\int_0^{\infty}\frac{\ue^{-as}}{s^{1-t}}\ud s$, 所以
\bee
\lim_{t\to0^+}(a^{-t}-b^{-t})\Gamma(t)
  = \int_0^{\infty}\frac{\ue^{-ax}-\ue^{-bx}}{x}\ud x
\eee
然后用Frullani积分算得$\log\frac{a}{b}$.
\ea

\bq{http://math.stackexchange.com/questions/590774}{20170626001}
已知$a>b>0$, 求$\int_0^{\infty}\frac{\ue^{-ax}-\ue^{-bx}}{x}\ud x$.
\eq
\ba
可以用Frullani积分. 这里用含参积分求导的方法, 定义$I(t)=\int_0^{\infty}\frac{\ue^{-x}-\ue^{-tx}}{x}\ud x$, 被积函数记为$f(x,t)$, 
由$f(x,t)$和$f_t(x,t)$在定义域均连续, $I(t)$关于$t$收敛且$\int_0^{\infty}f_t(x,t)\ud x$关于$t$一致收敛, 则满足积分号下求导条件, 
所以$\frac{\ud I}{\ud t}=-\frac{1}{a}$, $I=-\log t$.
\ea
\ba
\begin{align*}
\int_{0}^{\infty}\frac{\exp(-ax) - \exp(-bx)}{x}dx &= \lim_{\epsilon\to 0}\int_{\epsilon}^{\infty}\frac{\exp(-ax) - \exp(-bx)}{x}dx\\
&=\lim_{\epsilon\to 0}\left[\int_{\epsilon}^{\infty}\frac{\exp(-ax)}{x}dx - \int_{\epsilon}^{\infty}\frac{\exp(-bx)}{x}dx\right]\\
&=\lim_{\epsilon\to 0}\left[\int_{a\epsilon}^{\infty}\frac{\exp(-t)}{t}dt - \int_{b\epsilon}^{\infty}\frac{\exp(-t)}{t}dt\right]\\
&=\lim_{\epsilon\to 0}\int_{a\epsilon}^{b\epsilon}\frac{\exp(-t)}{t}dt=\lim_{\epsilon\to 0}\int_{a}^{b}\frac{\exp(-\epsilon u)}{u}du
\end{align*}
最后的被积函数一致收敛到$\frac{1}{u}$.
\ea
\ba
用Laplace变换, $F(s)=\int_0^{\infty}f(x)\ue^{-sx}\ud x$, 则
\bee
F(s) = \int_{0}^{\infty} \frac{e^{-bx}-e^{-ax}}{x} e^{-sx} dx \implies  F'(s) = -\int_{0}^{\infty} ({e^{-bx}-e^{-ax}}) e^{-sx} dx.
\eee
计算最后一个积分, 然后求积分并令$s\to0$, 其中积分出来的积分常数用极限$\lim_{s\to\infty}F(s)=0$计算.
\ea

\bq{http://math.stackexchange.com/questions/164400}{}
求$\int_{0}^{\infty}\frac{\ue^{-x}-\ue^{-xt}}{x}\ud x=\log t$, 其中$t>0$.
\eq
\ba
让$u=\ue^{-x}$, 得
\bee
\int_{0}^{1}\frac{u^{t-1}-1}{\log u}\ud u = \int_{0}^{1}\int_{1}^{t}u^{s-1}\ud s\ud u
  =\int_{1}^{t}\int_{0}^{1}u^{s-1}\ud u\ud s=\log t.
\eee
\ea
\ba
同\ref{q:20170626001}的解法一.
\ea
\ba
用重积分求解, $I(t)=\int_{0}^{\infty}\frac{\ue^{-x}-\ue^{-xt}}{x}\ud x=\int_{0}^{\infty}\int_{1}^{t}\ue^{-xs}\ud s\ud x$.
这里验证积分次序可交换, 则
\bee
LHS=\int_{1}^{t}\int_{0}^{\infty}\ue^{-xs}\ud x\ud s=\ln t.
\eee
\ea
\ba
用Laplace变换, $g(s)=L[f(x)]=L\left[\frac{\ue^{-ax}-\ue^{-bx}}{x}\right]$, 则$-g'(s)=L[xf(x)]=\frac{1}{s+a}-\frac{1}{s+b}$.
所以$g(s)=\log\frac{s+b}{s+a}+c$, 由于$g(\infty)=0$, 所以$c=0$. 从而
\bee
\int_{0}^{\infty}\frac{\ue^{-ax}-\ue^{-bx}}{x}\ue^{-sx}\ud x=\log\frac{s+b}{s+a}.
\eee
令$s=0$即可.
\ea
\ba
用Laplace变换, $L[1]=\int_{0}^{\infty}\ue^{-st}\ud t=\frac1s$, 则
\bee
LHS = \int_{0}^{\infty}\int_{0}^{\infty}(\ue^{-x}-\ue^{-xt})\ue^{-xs}\ud s\ud x
  = \int_{0}^{\infty}\left(\frac{1}{s+1}-\frac{1}{s+t}\right)
  = \ln\frac{s+1}{s+t}\Big\vert_{0}^{\infty}
  = \ln t
\eee
\ea

\bq{}{}
求积分
\bee
I(a)=\int_{0}^{\frac{\pi}{2}}\ln\abs{\sin^2x-a}\ud x.
\eee
\eq
\ba
当$a\le0$时, 设$a=-t$,
\begin{align*}
	I(a) & =\int_{0}^{\pi/2}\ln\left(t+\sin^{2}x\right)\ud x\\
	& =\int_{0}^{\pi/2}\left[\ln\left(\sin^{2}x\right)+\int_{0}^{t}\frac{1}{s+\sin^{2}x}\ud s\right]\ud x\\
	& =-\pi\ln2+\int_{0}^{t}\int_{0}^{\pi/2}\frac{1}{s+\sin^{2}x}\ud x\ud s\\
	& =-\pi\ln2+\frac{\pi}{2}\int_{0}^{t}\frac{1}{\sqrt{s(1+s)}}\ud s\\
	& =-\pi\ln2+\pi\arctanh\sqrt{\frac{-a}{1-a}}.
\end{align*}
当$a\ge1$时, 取$a=t+1$, 同上面的解法
\begin{align*}
	I(a) & =\int_{0}^{\pi/2}\ln\left(t+\cos^{2}x\right)\ud x\\
	& =\int_{0}^{\pi/2}\left[\ln\cos^{2}x+\int_{0}^{t}\frac{1}{s+\cos^{2}x}\ud s\right]\ud x\\
	& =-\pi\ln2+\pi\cdot\arctanh\sqrt{\frac{t}{1+t}}=-\pi\ln2+\pi\cdot\arctanh\sqrt{\frac{a-1}{a}}.
\end{align*}
当$0<a<1$时, 
\bee
I(a)=\frac{1}{4}\int_{0}^{2\pi}\ln\left|a-\sin^{2}x\right|\ud x=\frac{1}{4}\int_{0}^{2\pi}\ln\left|\frac{1-2a}{2}-\frac{\cos2x}{2}\right|\ud x.
\eee
令$\cos2\alpha=2a-1$, 则
\bee
I(a)=\frac{1}{4}\int_{0}^{2\pi}\ln\left|\frac{\cos2\alpha+\cos2x}{2}\right|\ud x=\frac{1}{4}\int_{0}^{2\pi}\ln\left|\cos(x+\alpha)\cos(x-a)\right|\ud x=\frac{1}{2}\int_{0}^{2\pi}\ln\left|\cos x\right|\ud x=-\pi\ln2.
\eee
\ea

\bq{http://tieba.baidu.com/p/2686576086}{}
设$f(x)$在实轴$\mathbb{R}$上有二阶导数, 且满足方程
\bee
2f(x)+f''(x)=-xf'(x).
\eee
求证$f(x)$和$f'(x)$都在$\mathbb{R}$上有界.
\eq
\ba
构造$L=f^2+\frac12f'^2$, 研究$L'$. 方程可改成$f''(x)+\frac{x}{2}f'(x)+f(x)=0$.
\ea

\bq{http://tieba.baidu.com/p/3846349760}{}
证明: $\sum_{n=1}^{\infty}\frac{1}{n}$发散.
\eq
\ba
由于$\lim_{n\to\infty}\left(1+\frac{1}{n}\right)^n=\ue$, 且$\left(1+\frac{1}{n}\right)^n$单调增加, 则$\left(1+\frac{1}{n}\right)^n<\ue$, 
于是$\ln\left(1+\frac{1}{n}\right)<\frac{1}{n}$, 从而$1+\frac12+\frac13+\cdots+\frac1n>\ln2+\ln\frac32+\cdots+\ln\left(1+\frac{1}{n}\right)=\ln(n+1)>\ln n$, 
即得.
\ea

\bq{http://tieba.baidu.com/p/4931607145}{}
设$f(x)$在$[a,b]$上可导, $f'(x)$在$[a,b]$上可积. 令
\bee
A_{n}=\sum_{i=1}^{n}f\left(a+i\cdot \frac{b-a}{n}\right)-\int_{a}^{b}f(x)\ud x.
\eee
试证
\bee
\lim_{n\to\infty}nA_n=\frac{b-a}{2}[f(b)-f(a)].
\eee
\eq
\ba
令$x_i=a+i\cdot\frac{b-a}{n}$, 则
\begin{align*}
nA_n & = n\left[\sum_{i=1}^{n}f(x_i)\cdot\frac{b-a}{n}-\sum_{i=1}^{n}\int_{x_{i-1}}^{x_{i}}f(x)\ud x\right]\\
  & = n\sum_{i=1}^{n}\int_{x_{i-1}}^{x_{i}}[f(x_i)-f(x)]\ud x\\
  & = n\sum_{i=1}^{n}\int_{x_{i-1}}^{x_{i}}\frac{f(x_{i})-f(x)}{x_{i}-x}(x_i-x)\ud x.
\end{align*}
在$[x_{i-1}, x_{i}]$上, $(x_i-x)$保号, 而$g(x)=\frac{f(x_i)-f(x)}{x_i-x}$连续(补充定义$g(x_i)=f'(x_i)$). 由积分第一中值定理知, 
存在$\eta_{i}\in(x_{i-1},x_i)$, 使得
\bee
\int_{x_{i-1}}^{x_{i}}\frac{f(x_{i})-f(x)}{x_i-x}(x_i-x)\ud x = g(\eta_i)\int_{x_{i-1}}^{x_i}(x_i-x)\ud x.
\eee
再由Lagrange中值定理, 以上$g(\eta_i)=\frac{f(x_{i})-f(\eta_i)}{\eta_{i}-x}=f'(\xi_{i})$, ($\xi_i\in(\eta_i, x_i)\subset(x_{i-1},x_i)$). 
于是
\begin{align*}
nA_n & = n\sum_{i=1}^{n}f'(\xi_{i})\int_{x_{i-1}}^{x_i}(x_i-x)\ud x = \frac{n}{2}\sum_{i=1}^{n}f'(\xi_i)(x_i-x_{i-1})^2\\
  & = \frac{n}{2}\cdot\frac{b-a}{n}\sum_{i=1}^{n}f'(\xi_i)(x_i-x_{i-1})=\frac{b-a}{2}\sum_{i=1}^{n}f'(\xi_i)(x_i-x_{i-1})\\
  & \to \frac{b-a}{2}\int_a^bf'(x)\ud x = \frac{b-a}{2}(f(b)-f(a)), \textrm{当}n\to\infty.
\end{align*}
\ea

\bq{}{}
$\int_0^1\frac{\sqrt[4]{x(1-x)^3}}{(1+x)^3}\ud x = \frac{3}{64}\sqrt[4]{2}\pi$.
\eq

\bq{}{}
求证:
\bee
\int_0^{\infty}\frac{\sin^3x}{x^3}\ud x=\frac{3\pi}{8}
\eee
\eq
\ba
让$f(y)=\int_{0}^{\infty}\frac{\sin^3yx}{x^3}\ud x$, 判断积分号下可求导, 有
\bee
f''(y)=\frac{3}{4}\int_{0}^{\infty}\frac{-\sin yx+3\sin 3yx}{x}\ud x=\frac{3\pi}{4}\mathrm{sign} y.
\eee
\ea
\ba
用$\sum_{k=1}^{\infty}\frac{\sin k x}{k}=\frac{\pi-x}{2}$, 由
\bee
\frac{\ud^2}{\ud x^2}\frac{\sin^3(kx)}{k^3}=\frac{9\sin(3kx)-3\sin(kx)}{4k}.
\eee
则
\bee
\sum_{k=1}^{\infty}\frac{9\sin(3kx)-3\sin(kx)}{4k}=\frac{3\pi}{4}-3x.
\eee
积分后得:
\bee
\sum_{k=1}^{\infty}\frac{\sin^3(kx)}{k^3}=\frac{3\pi}{8}x^2-\frac{1}{2}x^3.
\eee
取$x=\frac{1}{n}$后两边同乘$n^2$得
\bee
\sum_{k=1}^{\infty}\frac{\sin^3\frac{k}{n}}{(k/n)^3}\frac{1}{n}=\frac{3\pi}{8}-\frac{1}{2n}.
\eee
而广义 Riemann 和显示
\bee
\lim_{n\to\infty}\sum_{k=1}^{\infty}\frac{\sin^3\frac{k}{n}}{(k/n)^3}\frac{1}{n}=\int_{0}^{\infty}\frac{\sin^3x}{x^3}\ud x.
\eee
广义 Riemann 和成立的条件是 $\sum_{k=0}^{\infty}\sup_{x\in[k,k+1]}|f'(x)|<\infty$? 这不等式蕴含 
$\int_{0}^{\infty}|f'|\ud x<+\infty$, $f'\in L^1(0,\infty)$.
\ea
\ba
用 Parseval 定理
\bee
\int_{-\infty}^{\infty}f(x)g(x)\ud x=\frac{1}{2\pi}\int_{-\infty}^{\infty}F(s)G(s)\ud s,
\eee
其中$F(s)=\mathcal{F}[f]$, $G(s)=\mathcal{F}[g]$, $\mathcal{F}[f]=\int_{\mathbb{R}}f(x)\ue^{\ui sx}\ud x$.
若$f(x)=\frac{\sin x}{x}$, 则
\bee
F(s)=
\begin{dcases}
 \pi, \quad |s|\le 1\\
 0, \quad |s|>1,
\end{dcases}
\eee
若$g(x)=\frac{\sin^2x}{x^2}$, 则
\bee
G(s)=
\begin{dcases}
 \pi\left(1-\frac{|s|}{2}\right), |s|\le 2\\
 0, |s|>2,
\end{dcases}
\eee
所以$\int_{\mathbb{R}}\frac{\sin^3x}{x^3}\ud x=\frac{1}{2\pi}\int_{-1}^{1}\pi^2\left(1-\frac{|s|}{2}\right)\ud s=\frac{3\pi}{4}$.
即$\int_{0}^{\infty}\frac{\sin^3x}{x^3}\ud x=\frac{3\pi}{8}$.
\ea
\ba
用 Laplace 变换 $F(s)=\mathcal{L}[f]=\int_{0}^{\infty}f(x)\ue^{-sx}\ud x$, 对于$f(x)=\frac{\sin^3x}{x^3}$有
\bee
F(s)=\frac{\pi s^2}{8}+\frac{3\pi}{8}-\frac{3(s^2-1)}{8}\arctan s+\frac{s^2-9}{8}\arctan\frac{s}{3}+\frac{3s}{8}\ln\frac{s^2+1}{s^2+9}.
\eee
\ea
\ba
用 Laplace 恒等式
\bee
\int_{0}^{\infty}F(u)g(u)\ud u=\int_{0}^{\infty}f(u)G(u)\ud u, F(s)=\mathcal{L}[f(t)], G(s)=\mathcal{L}[g(t)].
\eee
让 $G(u)=\frac{1}{u^3}$ 得 $g(u)=\frac{u^2}{2}$, 让 $f(u)=\sin^3 u$ 得 $F(u)=\frac{6}{(u^2+1)(u^2+9)}$, 则
\bee
\int_{0}^{\infty}\frac{\sin^3 x}{x^3}\ud x=\frac{6}{2}\int_{0}^{\infty}\frac{u^2}{(u^2+1)(u^2+9)}\ud u = \frac{3\pi}{8}.
\eee
\ea
\ba
留数定理, 由 $\sin^3 x=\frac{3\sin x-\sin(3x)}{4}$ 得 $\int_{\mathbb{R}}\left(\frac{\sin x}{x}\right)^3\ud x=\int_{\mathbb{R}}\frac{3\sin x-\sin(3x)}{4x^3}\ud x$.
围道 $\gamma=\gamma_1\cup\gamma_2\cup\gamma_3\cup\gamma_4$, $\gamma_1=[r,R]$, $\gamma_2$ 是以 $(0,0)$ 为心 $R$ 为径的上半圆,
$\gamma_3=[-R,-r]$, $\gamma_4$ 为以 $(0,0)$ 为心 $r$ 为径的上半圆, $\gamma$ 取逆时针方向为正方向. 取 $f(z)=\frac{3\ue^{\ui z}-\ue^{3\ui z}}{z^3}$, 
则 $f$ 在 $\gamma$ 内解析, 由 Cauchy-Goursat 公式 
\be
\oint_{\gamma}f(z)=0\label{Cauchy}, 
\ee
并用 Laurent 展开或留数定理得
\bee
\int_{\gamma_4}f(z)\ud z=-\frac{3\pi\ui}{4}, 
\lim_{R\to\infty}\int_{\gamma_2}f(z)\ud z=0, 
\lim_{{r\to0}\atop{R\to\infty}}\int_{\gamma_1\cup\gamma_3}f(z)\ud z=2\ui\int_{0}^{\infty}\left(\frac{\sin x}{x}\right)^3\ud x.
\eee
代入\ref{Cauchy}即得.
\ea

\bq{}{}
设函数$f:(a,b)\to\mathbb{R}$连续可微, 又设对于任意的$x,y\in(a,b)$, 存在唯一的$z\in(a,b)$使得$\frac{f(y)-f(x)}{y-x}=f'(z)$.
证明: $f(x)$严格凸或严格凹.
\eq
\ba
反证法, 构造$\lambda$的函数
\bee
\Lambda(\lambda)=f(\lambda\alpha+(1-\lambda)\beta)-\lambda f(\alpha)-(1-\lambda)f(\beta), \quad \lambda\in(0,1)
\eee
若结论不真, 则存在$\alpha, \beta\in(a,b)$使$\Lambda(\lambda)$在$(0,1)$的某两点异号, 由于$f$连续, 所以存在$\lambda_0\in(0,1)$使$\Lambda(\lambda_0)=0$. 

设$\gamma=\lambda_0\alpha+(1-\lambda_0)\beta$, 则$\gamma\in(\alpha,\beta)\subset(a,b)$且
$\frac{f(\gamma)-f(\alpha)}{\gamma-\alpha}=\frac{f(\beta)-f(\gamma)}{\beta-\gamma}$. 再由拉格朗日中值定理得出矛盾.
\ea

\newpage
\bq{}{}
设函数$f(x)$在$\mathbb{R}$上无限可微, 且:
\begin{enumerate}[a)]
 \item 存在$L>0$, 使得对于任意的$x\in\mathbb{R}$及$n\in\mathbb{N}$有$|f^{(n)}(x)|\le L$.
 \item $f\left(\frac1n\right)=0$, 对所有$n=1,2,3\cdots$.
\end{enumerate}
求证: $f(x)\equiv 0$.
\eq
\ba
由$f$在$\mathbb{R}$上无限次可微, 且由a)知$f$有在$x=0$处的Taylor展开
\bee
f(x)=f(0)+\frac{f'(0)}{1!}x+\frac{f''(0)}{2!}x^2+\cdots.
\eee
设$N$是使$f^{(N)}(0)\ne0$的最小者, 取正整数$M>1$使$|f^{(N)}(0)|>\frac{L}{M-1}$.
则
\begin{align*}
 0=\left|f\left(\frac{1}{M}\right)\right| &= \left|f^{(N)}(0)\frac{x^N}{N!}+f^{(N+1)}(0)\frac{x^{N+1}}{(N+1)!}+\cdots\right|\\
    &\ge |f^{(N)}(0)|\frac{1}{N!M^N}-L\left(\frac{1}{(N+1)!M^{N+1}}+\frac{1}{(N+2)!M^{N+2}}+\cdots\right)\\
    &\ge |f^{(N)}(0)|\frac{1}{N!M^N}-L\left(\frac{1}{N!M^{N+1}}+\frac{1}{N!M^{N+2}}+\cdots\right)\\
    &= |f^{(N)}(0)|\frac{1}{N!M^N}-\frac{L}{N!}\frac{1}{M^{N+1}}\frac{1}{1-\frac1M}\\
    &= \left(|f^{(N)}(0)|-\frac{L}{M-1}\right)\frac{1}{N!M^N}>0.
\end{align*}
这导致矛盾, 即所有$f^{(n)}(0)=0$, $f(x)\equiv0$.
\ea