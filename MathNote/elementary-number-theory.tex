\chapter{初等数论}
\bq{}{}
求所有的多项式$f(x)$, 满足$f(x^2+1)=f^2(x)+1$, 且$f(0)=0$.
\eq
\ba
求导并让$x=0$.
\ea
\ba
定义数列$\{x_n\}$为$x_0=0$, $x_{n+1}=x_{n}^2+1$, 则$f(x)-x=0$有无穷多个根$\{x_n\}$.
\ea

\bq{}{}
设$f(x)\in\Bbb R[x]$, 如果对任意实数$x$有$f(x)\ge0$, 则$f(x)$是两个实系数多项式的平方和.
\eq
\ba
由于$f=\prod_{b,c}\left[(x-b)^2+c\right]$和$(x^2+y^2)(z^2+w^2)=(xz-yw)^2+(xw+yz)^2$.
\ea

\bq{}{}
证明多项式$x^{2n}-2x^{2n-1}+3x^{2n-2}-\cdots-2nx+(2n+1)$没有实根.
\eq
\ba
当$x\le0$时, $f(x)=x^{2n}-2x^{2n-1}+3x^{2n-2}-\cdots-2nx+(2n+1)>0$; 若$x>0$, 
$(1+x)f(x)=f(x)+xf(x)=x\frac{x^{2n+1}+1}{x+1}+2n+1>0$, 所以对$x>0$时亦有$f(x)>0$.
\ea

\bq{}{}
设$f(x)$是一个整系数多项式, 首项系数为1, 且$f(0)\ne0$. 若$f(x)$仅有一个单根$\alpha$使得$|\alpha|\ge1$, 则$f(x)$在$\Bbb Z$上不可约.
\eq
\ba
反证, $f=gh$, 设$h(\alpha)=0$, 则$|g(0)|\ge1$是其根的模之积, 又小于1, 矛盾.
\ea

\bq{}{}
设$a_1,\cdots,a_n$是互不相同的整数, 证明: 多项式
\bee
(x-a_1)\cdots(x-a_n)-1
\eee
在$\Bbb Z$上不可约.
\eq
\ba
反证, $f=gh$, 因$f(a_i)=-1$, 知$g(a_i)+h(a_i)=0$, 由次数限制而导致矛盾.
\ea

\bq{}{}
给定$2n$个互不相同的复数$a_1, \cdots, a_n, b_1,\cdots, b_n$, 将其按下列规则填入$n\times n$方格中: 
第$i$行第$j$列相交处方格内填$a_i+b_j$, ($i,j=1,\cdots,n$), 证明: 若各列数乘积相等, 则各行数的积也相等.
\eq
\ba
设各列的积都为$c$, 则$f(x)=(x+a_1)\cdots(x+a_n)-c$, 有$n$个根$b_1,\cdots, b_n$.
所以$f(x)=\prod_{j}(x-b_j)$, 所以$f(-a_i)=(-1)^n\prod_{j}(a_{i}+b_j)=-c$,
即$\prod_j(a_i+b_j)=(-1)^{n+1}c$, ($\forall i$).
\ea

\bq{}{}
设$a,b,c$为整数, $abc\ne0$, 求证:
\bee
[(a,b),(b,c),(c,a)]=([a,b],[b,c],[c,a]).
\eee
\eq

\bq{}{}
设$\{F_n\}$是符合$F_1=F_2=1$的裴波那契数列, 若$(m,n)=d$, 则$(F_m,F_n)=F_d$;
反之, 若$(F_m, F_n)=F_d$, 则$d=(m,n)$或$d=1$, $(m,n)=2$或$d=2$, $(m,n)=1$.
\eq
\ba
先证$F_q=F_kF_{q-k+1}+F_{k-1}F_{q-k}$, 后证$m=nq+r$, $q\ge0$, $1\le r\le n-1$时, 
$(F_m,F_n)=(F_n,F_r)$, 即$(F_m,F_n)=F_{(m,n)}$, 于是$F_d=F_{(m,n)}$的解即为结论的三种情况.
\ea

\bq{}{}
当$a,b$满足什么条件时, $3\mid n(an+1)(bn+1)$对任意$n$成立.
\eq
\ba
根据同余理论, 只需让$n$分别取$1,2,3$分别代入上式, 可得$3\mid ab+1$, 即$ab\equiv 2\pmod 3$
\ea

\bq{}{}
$a,b$正整数, $d=(a,b)$, 证$S=\{ma+nb\}$, ($m,n$遍历正整数)包含$d$的大于$ab$的所有倍数.
\eq
\ba
设$t>ab$是$d$的倍数且$ax+by=t$, 由B\'{e}zout等式, 左式有解$x=x_0+br$, $y=y_0-ar$, 调整$r$使$0<x<b$, 得$-\frac{x_0}{b}<r<\frac{b-x_0}{b}$,
于是$y=y_0-ar>y_0-a\frac{b-x_0}{b}=\frac{t-ab}{b}>0$.
\ea

\bq{}{}
$4k-1$形素数无穷.
\eq
\ba
反证: $p_1, p_2, \cdots, p_r$为$r$个有限$4k-1$形素数, 考虑$p_1^2p_2^2\cdots p_r^2-1$的因子中必有$4k-1$形素因子.
另外也可考虑$4p_1p_2\cdots p_r-1$.
\ea

\bq{}{}
$6k-1$形素数无穷.
\eq
\ba
想法同上, 考虑$6p_1p_2\cdots p_{r}-1$中必有$6k-1$形素数, 却不是$p_1,p_2,\cdots p_r$中的一个.
\ea

\bq{}{}
$4k+1$形素数无穷.
\eq
\ba
$4(p_1p_2\cdots p_r)^2+1$的素因子均可写成$4k+1$形素数, 但不是$p_1, \cdots, p_r$中的任一个.
\ea

\bq{}{20170116002}
假设自然数$N$有形如$4n-1$的因子, 则$N$必有形如$4n-1$的素因子.
\eq

\bq{}{20170116003}
对每个素数$p=4n-1$和整数$a$, $p^2$不可能整除$a^2+1$.
\eq
\ba
反证, 若$p^2\mid a^2+1$, $a^2\equiv-1\pmod{p^2}$, 说明$a$的阶等于$4$. $a$可以看成群$\mathbb{Z}_{p^2}^{*}$中的元素, 
由Euler定理, 群$\mathbb{Z}_{p^2}^{*}$的阶为$p(p-1)=2(4n-1)(2n-1)$. 在由Cauchy定理, 群中元素的阶必整除群的阶, 
所以有$4\mid2(4n-1)(2n-1)$. 矛盾.
\ea

\bq{}{}
证明$a^2+1$没有$4n-1$的因子, $a, n$是任何正整数.
\eq
\ba
假设自然数$N=a^2+1$有形如$4n-1$形因子, 则由问题\ref{q:20170116002}知$N$必有形如$4n-1$的素因子, 记为$p$.
而$N$可以写成两个整数的平方和($N=a^2+1^2$), 所以由问题\ref{un:20170116001}知$N$的素因子分解中$p$出现次数为偶数.
从而必有$p^2$整除$N=a^2+1$, 与问题\ref{q:20170116003}矛盾. 所以$N$不可能有形如$4n-1$的因子.
\ea

\bq{}{}
$a,b$正整数, $a+b=57$, $[a,b]=680$, 求$a,b$.
\eq
\ba
用$(a,b)[a,b]=ab=(a+b,b)[a,b]$, 所以$(57,b)\cdot 680=ab$, 然后$57=3\times19$, 分四种情况讨论$(57,b)$的值,
并计算出$ab$的值联合$a+b$的值用Vieta定理.
\ea

\bq{}{}
$a,b,c,d\in\mathbb{Z}$, $a-c\mid ab+cd$, 则$a-c\mid ad+bc$.
\eq

\bq{}{}
$a,b\in\mathbb{Z}$且$\frac{b+1}{a}+\frac{a+1}{b}\in\mathbb{Z}$, 则$(a,b)\le\sqrt{a+b}$.
\eq

\bq{}{}
$a,b$为大于$1$的正整数, $(a,b)=1$, 则有唯一一对整数$r,s$使得$ar-bs=1$, 且$0<r<b$, $0<s<a$.
\eq

\bq{}{}
$q$-进表示$n$中. $n=\sum_{i=0}^{k}a_iq^i$, 有$a_{i}=\left[\frac{n}{q^i}\right]-q\left[\frac{n}{q^{i+1}}\right]$.
\eq

\bq{}{}
$p^{\alpha_{p}}\| n!$, 则$\alpha_p=\sum_{i=1}^{\infty}\left[\frac{n}{p^i}\right]$, 设$n=\sum_{i=0}^{k}a_iq^i$, 则$a_{i}=\left[\frac{n}{q^i}\right]-q\left[\frac{n}{q^{i+1}}\right]$.
而$S_{p}(n)=\sum_{i=0}^{\infty}a_{i}=(n+\alpha_{p})-q\alpha_{p}$, 所以$\alpha_{p}=\sum_{i=1}^{\infty}\left[\frac{n}{p^i}\right]=\frac{n-S_{p}(n)}{p-1}$.
其中$S_{p}(n)$为$n$的$p$-进展开的各个数字值之和.
\eq

\bq{}{}
若$2^m+1$是素数, 则$m$是$2$的幂, 从而是Fermat数$F_{n}$.

$a,m>1$, $a^m-1$是素数, 则$a=2$且$m$是素数. 从而是梅森Mersenne数.
\eq

\bq{偶完全数}{}
$\sigma(n)=2n$则称$n$是完全数, 证明偶完全数形如$2^k(2^{k+1}-1)$.
\eq
\ba
$\sigma$是积性函数, 设$n=2^p\cdot q$, 则$(2^{p+1}-1)\sigma(q)=2^{p+1}q$, 即$\frac{q}{\sigma(q)}=\frac{2^{p+1}-1}{2^{p+1}}$且$2^{p+1}-1\mid q$.
设$q=(2^{p+1}-1)k$.

(1). 若$2^{p+1}-1$是合数, $q$的最小质因子$p_1<2^{p+1}-1$, 所以$q=p_{1}^{\alpha_1}w$. 这与$\frac{q}{\sigma(q)}=\frac{p_1^{\alpha_1}w}{\sum_{i}p^{i}f(w)}<\frac{p_1}{p_1+1}<\frac{2^{p+1}-1}{2^{p+1}}$.

(2). 若$2^{p+1}-1$是素数, 若$k=1$, 命题显然成立. 否则$q=(2^{p+1}-1)^{\alpha}w$.
\begin{itemize}
 \item 若$w=1$, $\alpha>1$, $\frac{q}{\sigma(q)}=\frac{(2^{p+1}-1)^{\alpha}}{\sum_{i}(2^{p+1}-1)^i}<\frac{2^{p+1}-1}{2^{p+1}}$. 这导致矛盾.
 \item 若$w>1$, 则$\frac{q}{\sigma(q)}=\frac{(2^{p+1}-1)^{\alpha}}{\sum_{i}(2^{p+1}-1)^i}\frac{w}{\sigma(w)}<\frac{2^{p+1}-1}{2^{p+1}}$, 又矛盾.
\end{itemize}
故$w=1$, $\alpha=1$.
\ea

\bq{}{}
若$m>2$, 则$\varphi(m)$是偶数.
\eq
\ba
由Euler公式, $(-1)^{\varphi(m)}\equiv1\pmod{m}$.
\ea
\ba
若$i$与$m$互素, 则$m-i$与$m$也互素, 所以与$m$互素的数总是成对出现. 若有$i=m-i$, 这表明$m$是$i$的倍数, 则$i$不可能是与$m$互素的$\varphi(m)$个数的任一个.
\ea

\bq{}{}
设$a,b\in\mathbb{Z}$, 且$p$为素数, 若$a^p\equiv b^p\pmod{p}$, 则$a^p\equiv b^p\pmod{p^2}$.
\eq
\ba
Fermat定理$a\equiv b\pmod(p)$, 所以$a\equiv b\pmod{p}$, 于是可设$a=b+mp$, 并用二项式定理证得.
\ea

\bq{}{}
设$p$是素数, $k\in\mathbb{N}_{+}$, 则
\begin{itemize}
 \item $p\vert \binom{p}{k}$, $k=1,\cdots, p-1$.
 \item $\binom{p-1}{k}\equiv(-1)^k\pmod{p}$, $k=0,\cdots,p-1$.
 \item $\binom{k}{p}\equiv\left[\frac{k}{p}\right]\pmod{p}$.
\end{itemize}
\eq
\ba
\begin{itemize}
 \item $p\vert k\binom{p}{k}=p\binom{p-1}{k-1}$, $(k,p)=1$.
 \item $\binom{p}{k}-\binom{p-1}{k}=\binom{p-1}{k-1}\Longrightarrow\binom{p-1}{k}\equiv-\binom{p-1}{k-1}\pmod{p}$, 用归纳法.
 \item $\binom{k}{p}=\frac{k(k-1)\cdots(k-p+1)}{p!}$, $k, k-1, \cdots, k-p+1$中必有$p$的倍数, 设为$k-i$, \\
 则$\binom{k}{p}=\frac{k-i}{p}\cdot\frac{k(k-1)\cdots(k-p+1)}{(k-i)(p-1)!}=\left[\frac{k}{p}\right]\frac{k(k-1)\cdots(k-p+1)}{k-i}\cdot\frac1{(p-1)!}\equiv\left[\frac{k}{p}\right]\frac{-1}{-1}\pmod{p}$.

 最后一步用Wilson公式.
\end{itemize}
\ea

\bq{}{20170513001}
素数$p\ge5$, 则$\frac1{1^2}+\frac1{2^2}+\cdots+\frac1{(p-1)^2}\equiv0\pmod{p}$.
\eq
\ba
$1,2,\cdots, p-1$是模$p$的缩系. 所以$\frac11,\frac12,\cdots,\frac1{p-1}$也是模$p$的缩系.
故对$p\ge5$有
\bee
\sum_{k=1}^{p-1}\left(\frac1{k}\right)^2\equiv\sum_{k=1}^{p-1}k^2=\frac{(p-1)p(2p-1)}{6}\equiv0\pmod{p}.
\eee
\ea
\ba
$\frac11, \frac12, \cdots, \frac1{p-1}$是模$p$的缩系, 对任意$a$, $p\nmid a$, $\frac{a}{1}, \frac{a}{2}, \cdots, \frac{a}{p-1}$也是模$p$的缩系. 故
\bee
\sum_{k=1}^{p-1}\left(a\cdot\frac1{k}\right)^2\equiv\sum_{k=1}^{p-1}\left(\frac{1}{k}\right)^2\pmod{p}\Longrightarrow(a^2-1)\sum_{k=1}^{p-1}\left(\frac{1}{k}\right)^2\equiv0\pmod{p}.
\eee
对$p\ge5$可取$a$满足$p\nmid a$且$p\nmid a^2-1$即得.
\ea

\bq{}{}
若 $p>3$, 证明
\bee
p^2\mid(p-1)!\left(1+\frac{1}{2}+\cdots+\frac{1}{p-1}\right).
\eee
\eq
\ba
用\ref{q:20170513001}, 知
\bee
\sum_{i=1}^{p-1}\frac{1}{i(p-i)}\equiv\sum_{i=1}^{\frac{p-1}{2}}\frac{2}{i(p-i)}\equiv 0\pmod{p}.
\eee
从而
\bee
(p-1)!\sum_{i=1}^{p-1}\frac{1}{i}\equiv(p-1)!\sum_{i=1}^{\frac{p-1}{2}}\frac{p}{(p-i)i}\equiv0\pmod{p^2}.
\eee
\ea
\ba
设
\be\label{20170513002}
(x-1)(x-2)\cdots(x-p+1)=x^{p-1}-s_1x^{p-2}+\cdots+s_{p-1},
\ee
其中 $s_{p-1}=(p-1)!$, $s_{p-2}=(p-1)!\left(1+\frac{1}{2}+\cdots+\frac{1}{p-1}\right)$. 因
\bee
s_1x^{p-2}+\cdots+s_{p-2}x=(x-1)(x-2)\cdots(x-p+1)-x^{p-1}+1\equiv0\pmod{p}
\eee
有 $p-1$ 个根, 所以 $s_1, \cdots, s_{p-2}$ 都是 $p$ 的倍数. 对\ref{20170513002}中取 $x=p$ 有
\bee
p^{p-2}-s_1p^{p-3}+\cdots+s_{p-3}p-s_{p-2}=0.
\eee
所以 $p^2\mid s_{p-2}$.
\ea

\bq{}{}
$n$是偶数, $a_1, \cdots, a_n$与$b_1,\cdots, b_n$都是模$n$的完系, 证$a_1+b_1,\cdots, a_n+b_n$不是模$n$的完系.
\eq
\ba
反证法, $\sum a_i\equiv\frac{n(n-1)}{2}\equiv\sum b_i\pmod{n}$, 若$a_1+b_1,\cdots, a_n+b_n$是模$n$的完系.
则$\sum (a_i+b_i)\equiv\frac{n(n-1)}{2}$, 于是$n(n-1)\equiv\frac{n(n-1)}{2}\pmod{n}$, 由于$n(n-1)\equiv0\pmod{n}$,
故$\frac{n(n-1)}{2}\equiv0\pmod{n}$, 即$n\vert\frac{n(n-1)}{2}$, 但$(n,n-1)=1$, 所以$n\lvert\frac{n}{2}$不可能.
\ea

\bq{}{}
设$a,b$为正整数, $n$是正整数, 证明
\bee
n!\mid b^{n-1}a(a+b)\cdots(a+(n-1)b).
\eee
\eq
\ba
只需证对任意的素数$p$, 若$p^{\alpha}\| n!$, 则$p^{\alpha}\mid b^{n-1}a(a+b)\cdots(a+(n-1)b)$.

(1). 若$p\mid b$, 则由于$\alpha=\sum_{i\ge1}\left[\frac{n}{p^i}\right]<\frac{n}{p-1}\le n$, 所以$p^{\alpha}|b^{n-1}$.

(2). 若$p\nmid b$, 则$(p,b)=1$,  从而有$b_1b\equiv 1\pmod{p}$, 于是
\bee
b_{1}^{n}a(a+b)\cdots(a+(n-1)b)\equiv ab_1(ab_1+1)\cdots(ab_1+n-1)\pmod{p}.
\eee
由于$n!\mid ab_1(ab_1+1)\cdots(ab_1+n-1)$, 所以$p^{\alpha}\mid ab_1(ab_1+1)\cdots(ab_1+n-1)$.
\ea

\bq{}{}
设$p$是一个奇素数, 证明
\begin{itemize}
 \item $1^2\cdot3^2\cdots(p-2)^2\equiv(-1)^{\frac{p+1}{2}}\pmod{p}$;
 \item $2^2\cdot4^2\cdots(p-1)^2\equiv(-1)^{\frac{p+1}{2}}\pmod{p}$.
\end{itemize}
\eq
\ba
用Wilson公式, 注意到$k\equiv -(p-k)\pmod{p}$.
\ea
