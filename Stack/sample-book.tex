\chapter{tex笔记}
%%%%%%%%%%%%%%%%%%%%%%%%%%%%%%%%%%%%%%%%%%%%%%%
%%%%%%%%%%%%%%%%%%%%%%%%%%%%%%%%%%%%%%%%%%%%%%%
\section{使用频率较低的符号列表}

\tcbset{skin=enhanced, fonttitle=\bfseries, boxrule=1mm, drop fuzzy shadow}
\begin{tcblisting}{title=特殊符号表}
 \begin{center}
  \begin{tabular}{|c|c|c|c|c|}
    \hline
    $\hbar$ & $\imath$ & $\jmath$ & $\ell$ & $\Im$\\
    \hline
    $\wp$ & $\mho$ & $\prime$ & $\Box$ & $\Diamond$\\
    \hline
    $\bot$ & $\top$ & $\surd$ & $\diamondsuit$ & $\heartsuit$\\
    \hline
    $\clubsuit$ & $\spadesuit$ & $\neg$ & $\lnot$ & $\flat$\\
    \hline
    $\natural$ & $\sharp$ & \dag & \ddag & \S\\
    \hline
    \P & \copyright & \pounds & \textregistered & \\
    \hline
  \end{tabular}
\end{center}
\end{tcblisting}
%%%%%%%%%%%%%%%%%%%%%%%%%%%%%%%%%%%%%%%%%%%%%%%
%%%%%%%%%%%%%%%%%%%%%%%%%%%%%%%%%%%%%%%%%%%%%%%
\section{itemize enumerate}
\begin{tcblisting}{title=列表}
  \begin{enumerate}
  \item This is an example of \ldots
  \item \ldots the usual enumeration.
  \begin{enumerate}[a)]
    \item And this is a \ldots
    \item \ldots couple of \ldots
  \end{enumerate}
    \item 
    \begin{enumerate}[-- i --]
    \item \ldots examples of \ldots
    \item \ldots custom-tailored \ldots
    \item \ldots enumerations.
    \newcounter{enumii_saved}
    \setcounter{enumii_saved}{\value{enumii}}
    \end{enumerate}
    Some general comments
    \begin{enumerate}[-- i --]
    \setcounter{enumii}{\value{enumii_saved}}
    % 如果要换另一个条列式项目, 但编号接续, 使用\newcounter{enumii_saved}来操作
    \item My next point.
    \setcounter{enumii}{7}
    % 使用setcounter{enumii}{数字}来指定编号号码
    \item My eighth point.
    \end{enumerate}
  \end{enumerate}
\end{tcblisting}
%%%%%%%%%%%%%%%%%%%%%%%%%%%%%%%%%%%%%%%%%%%%%%%
%%%%%%%%%%%%%%%%%%%%%%%%%%%%%%%%%%%%%%%%%%%%%%%
\section{tikz}

\begin{tcblisting}{title=画图}
	\begin{tikzcd}
		A \arrow[rd] \arrow[r, "\phi"] & B \\
		& C
	\end{tikzcd}
	\begin{tikzcd}
		A \arrow[r, "\phi"] \arrow[d, red]
		& B \arrow[d, "\psi" red] \\
		C \arrow[r, red, "\eta" blue]
		& D
	\end{tikzcd}
	\begin{tikzcd}
		A \arrow[r, "\phi" near start, "\psi"', "\eta" near end] & B
	\end{tikzcd}
	\begin{tikzcd}
		T
		\arrow[drr, bend left, "x"]
		\arrow[ddr, bend right, "y"]
		\arrow[dr, dotted, "{(x,y)}" description] & & \\
		& X \times_Z Y \arrow[r, "p"] \arrow[d, "q"]
		& X \arrow[d, "f"] \\
		& Y \arrow[r, "g"]
		& Z
	\end{tikzcd}\\
	\begin{tikzcd}[column sep=tiny]
		& \pi_1(U_1) \ar[dr] \ar[drr, "j_1", bend left=20]
		&
		&[1.5em] \\
		\pi_1(U_1\cap U_2) \ar[ur, "i_1"] \ar[dr, "i_2"']
		&
		& \pi_1(U_1) \ast_{ \pi_1(U_1\cap U_2)} \pi_1(U_2) \ar[r, dashed, "\simeq"]
		& \pi_1(X) \\
		& \pi_1(U_2) \ar[ur]\ar[urr, "j_2"', bend right=20]
		&
		&
	\end{tikzcd}
	\begin{tikzcd}
		X \arrow[r, hook] \arrow[dr, dashrightarrow]
		& \bar{X} \arrow[d]\\
		& Y
	\end{tikzcd}
\end{tcblisting}

\begin{tcblisting}{title=画图}
	\begin{tikzcd}
		A \arrow[r, tail, two heads, dashed] & B
	\end{tikzcd}
	\begin{tikzcd}
		A \arrow{d} \arrow{r}[near start]{\phi}[near end]{\psi}
		& B \arrow[red]{d}{\xi} \\
		C \arrow[red]{r}[blue]{\eta}
		& D
	\end{tikzcd}
	\begin{tikzcd}[column sep=small]
					 & A \arrow[dl] \arrow[dr] & \\
		B \arrow{rr} &
			  			   & C
	\end{tikzcd}
	\begin{tikzcd}[row sep=tiny]
		& B \arrow[dd] \\
		A \arrow[ur] \arrow[dr] &
		\\
		& C
	\end{tikzcd}
	% in preamble
	\tikzcdset{
		arrow style=tikz,
		diagrams={>={Straight Barb[scale=0.8]}}
	}
	% in document body
	\begin{tikzcd}
		A \arrow[r, tail] \arrow[rd] & B \arrow[d, two heads]\\
		& D
	\end{tikzcd}
\end{tcblisting}

\begin{tcblisting}{title=画图}
	\begin{tikzcd}
		A \arrow[to=Z, red] \arrow[to=2-2, blue]
		& B \\
		|[alias=Z]| C
		& D
		\arrow[from=ul, to=1-2, purple]
	\end{tikzcd}
	\begin{tikzcd}[column sep=scriptsize]
		A \arrow[dr] \arrow[rr, ""{name=U, below, draw=red}]{}
		& & B \arrow[dl] \\
		& C \arrow[Rightarrow, from=U, "\psi"]
	\end{tikzcd}
	\begin{tikzcd}
		A \arrow[r, bend left=50, ""{name=U, below, draw=red}]
		\arrow[r, bend right=50, ""{name=D, draw=red}]
		& B
		\arrow[Rightarrow, from=U, to=D]
	\end{tikzcd}
	\begin{tikzcd}
		A \arrow[r] \arrow[d] \arrow[dr, phantom, "\ulcorner", very near start]
		& B \arrow[d] \\
		C \arrow[r]
		& D
	\end{tikzcd}
	\begin{tikzcd}
		A \arrow[r, red, shift left=1.5ex] \arrow[r]
		\arrow[dr, blue, shift right=1.5ex] \arrow[dr]
		& B \arrow[d, purple, shift left=1.5ex] \arrow[d]\\
		& C
	\end{tikzcd}
	\begin{tikzcd}
		A \arrow[r]
		& B \arrow[r, shift left]
		\arrow[r, shift right]
		& C \arrow[r]
		\arrow[r, shift left=2]
		\arrow[r, shift right=2]
		& \cdots
	\end{tikzcd}
	\begin{tikzcd}
		A \arrow[r, yshift=0.7ex] \arrow[r, yshift=-0.7ex]
		& B \arrow[d, xshift=0.7ex] \arrow[d, xshift=-0.7ex] \\
		& C
	\end{tikzcd}
\end{tcblisting}

\begin{tcblisting}{title=画图}
	\begin{tikzcd}[cells={nodes={draw=gray}}]
		A \arrow[r, black]
\arrow[r, blue, end anchor=north east]
			\arrow[r,
red, start anchor={[xshift=-1ex]},
end anchor={[yshift=2ex]north east}]
		& B
	\end{tikzcd}
	\begin{tikzcd}
		A \arrow[r, shift left]
\ar[r, shorten=2mm, shift right]
		& B
	\end{tikzcd}
	\begin{tikzcd}
		A \arrow[dr] & B \arrow[dl, crossing over] \\
		C
	& D
	\end{tikzcd}
	\begin{tikzcd}
		A \arrow[r, "/" marking]
\arrow[rd, "\circ" marking]
		& B \\
		& C
	\end{tikzcd}
	\begin{tikzcd}
		A \arrow[r, "\phi" description] & B
	\end{tikzcd}
	\begin{tikzcd}
		A \arrow[dr, controls={+(1.5,0.5) and +(-1,0.8)}]
		\arrow[dr, dashed, to path=|- (\tikztotarget)]
		& \\
		& B \arrow[loop right]
	\end{tikzcd}
	\begin{tikzcd}
		A \arrow[r]
& B \arrow[r]
		\arrow[d, phantom, ""{coordinate, name=Z}]
		& C \arrow[dll,
"\delta",
		rounded corners,
		to path={ -- ([xshift=2ex]\tikztostart.east)
			|- (Z) [near end]\tikztonodes
			-| ([xshift=-2ex]\tikztotarget.west)
			-- (\tikztotarget)}] \\
		D \arrow[r]
& E \arrow[r]
	& F
	\end{tikzcd}
\end{tcblisting}

\begin{tcblisting}{title=画图}
	\begin{tikzpicture}[commutative diagrams/every diagram]
		\matrix[matrix of math nodes, name=m, commutative diagrams/every cell] {
			X & \bar X \\
			& Y
			\\};
		\path[commutative diagrams/.cd, every arrow, every label]
		(m-1-1) edge[commutative diagrams/hook] (m-1-2)
		edge[commutative diagrams/dashed] (m-2-2)
		(m-1-2) edge (m-2-2);
	\end{tikzpicture}
	\begin{tikzcd}[ampersand replacement=\&]
		A \oplus B \ar[r, "{\begin{pmatrix} e & f \\ g & h \end{pmatrix}}"]
		\& C \oplus D
	\end{tikzcd}
	\tikzset{
		math to/.tip={Glyph[glyph math command=rightarrow]},
		loop/.tip={Glyph[glyph math command=looparrowleft, swap]},
		weird/.tip={Glyph[glyph math command=Rrightarrow, glyph length=1.5ex]},
		pi/.tip={Glyph[glyph math command=pi, glyph length=1.5ex, glyph axis=0pt]},
	}
	\begin{tikzpicture}[line width=rule_thickness]
		\draw[loop-math to, bend left] (0,2) to (1,2);
		\draw[math to-weird] (0,1) to (1,1);
		\draw[pi-pi] (0,0) to (1,0);
	\end{tikzpicture}
\end{tcblisting}

\begin{tcblisting}{title=画图}
	\begin{dependency}
		\begin{deptext}
			My \& dog \& also \& likes \& eating \& sausage \\
		\end{deptext}
		\depedge{5}{6}{dobj}
	\end{dependency}\\
	\begin{dependency}
		\begin{deptext}
			My \& dog \& also \& likes \& eating \& sausage \\
		\end{deptext}
		\depedge{2}{1}{poss}
		\depedge{4}{2}{nsubj}
		\depedge{4}{3}{advmod}
		\depedge{4}{5}{xcomp}
		\depedge{5}{6}{dobj}
	\end{dependency}\\
	\begin{dependency}[theme=night]
		\begin{deptext}[column sep=.5cm, row sep=.1ex]
			PRP\$ \& NN \& RB \&[.5cm] VBZ \& VBG \& NN \\
			My \& dog \& also \& likes \& eating \& sausage \\
		\end{deptext}
		\deproot{4}{root}
		\depedge{2}{1}{poss}
		\depedge{4}{2}{nsubj}
		\depedge{4}{3}{advmod}
		\depedge{4}{5}{xcomp}
		\depedge{5}{6}{dobj}
	\end{dependency}
\end{tcblisting}

\begin{tcblisting}{title=画图}
  \begin{tikzpicture}
   \draw[gray, thick] (-1,2) -- (1,-2);
   \draw[gray, thick] (-1,-1) -- (2,2);
   \filldraw[black] (0,0) circle (2pt) node[anchor=west] {Intersection point};
  \end{tikzpicture}
  \begin{tikzpicture}
    \draw (-2,0) -- (2,0);
    \filldraw [gray] (0,0) circle (2pt);
    \draw (-2,-2) .. controls (0,0) .. (2,-2);
    \draw (-2,2) .. controls (-1,0) and (1,0) .. (2,2);
  \end{tikzpicture}
  \begin{tikzpicture}
    \filldraw[color=red!60, fill=red!5, very thick](-1,0) circle (1.5);
    \fill[blue!50] (2.5,0) ellipse (1.5 and 0.5);
    \draw[ultra thick, ->] (6.5,0) arc (0:220:1);
  \end{tikzpicture}
\end{tcblisting}

\begin{tcblisting}{title=画图}
  \begin{tikzpicture}
    \filldraw[color=red!60, fill=red!50, very thick](1,1) rectangle (0.5,1.5);
    \draw[blue, very thick] (0,0)rectangle (3,2);
    \draw[orange, ultra thick] (4,0) -- (6,0) -- (5.7,2) -- cycle;
  \end{tikzpicture}
\end{tcblisting}

%% 关联图
\tikz[remember picture] \node[fill=green!30] (n3) {$\sum_{n=1}^{\infty}\frac{1}{n^2}=\frac{\pi^2}{6}$};

\tikz[remember picture]
\node[
	color=red!20,
%	circle,
	draw,
%	label=angle:text,
	fill=red,
	] (nodename) {
	contents
};

%
here is some text\quad\qquad\tikz[remember picture]
\node[
color=yellow!10,
%	circle,
draw,
%	label=angle:text,
fill=blue!20,
] (n1) {
	$\sum_{n=1}^{\infty}$
};

\tikz[remember picture]
\draw[overlay,->,very thick,red,opacity=.5]
(nodename) to[bend left] (n1);

here is a circle I will type more words: \tikz[remember picture] \node[fill=red!50] (n1) {$f(x)=\sin x$};

here is a node: \tikz[remember picture] \node[fill=blue!50] (n2) {$f(x)=\sin x$};

\begin{tikzpicture}[remember picture,overlay]
	\draw[->,very thick] (n1) -- (n2);
\end{tikzpicture}

\begin{tikzpicture}[remember picture]
	\node (c) [fill=yellow!20] {Big circle};
	
	\draw[overlay,->,very thick,red,opacity=.5]
	(c) to[bend right] (n1) (n1) -| (n2) (n3) to[bend left] (n1);
\end{tikzpicture}

\begin{tcblisting}{title=270226}
  \definecolor{myred}{RGB}{183,18,52}
  \definecolor{myyellow}{RGB}{254,213,1}
  \definecolor{myblue}{RGB}{0,80,198}
  \definecolor{mygreen}{RGB}{0,155,72}
  \begin{tikzpicture}[
    line join=round,
    y={(-0.86cm,0.36cm)},x={(1cm,0.36cm)}, z={(0cm,1cm)},
    arr/.style={-latex,ultra thick,line cap=round,shorten <= 1.5pt}
  ]
  \def\Side{2}
  \coordinate (A1) at (0,0,0);
  \coordinate (A2) at (0,\Side,0);
  \coordinate (A3) at (\Side,\Side,0);
  \coordinate (A4) at (\Side,0,0);
  \coordinate (B1) at (0,0,\Side);
  \coordinate (B2) at (0,\Side,\Side);
  \coordinate (B3) at (\Side,\Side,\Side);
  \coordinate (B4) at (\Side,0,\Side);

  \fill[myyellow] (A2) -- (A3) -- (B3) -- (B2) -- cycle;
  \fill[mygreen]  (A2) -- (A3) -- (A4) -- (A1) -- cycle;
  \fill[myred](A3) -- (B3) -- (B4) -- (A4) -- cycle;
  \fill[myblue]   (A1) -- (A2) -- (B2) -- (B1) -- cycle;

  \draw (A2) -- (A1) -- (A4);
  \draw (B2) -- (B1) -- (B4) -- (B3) -- cycle;
  \draw (A1) -- (B1);
  \draw (A2) -- (B2);
  \draw (A4) -- (B4);

  \draw[thin] (A3) -- (B3);
  \draw[thin] (A3) -- (A4);

  \path[arr] 
    (A1) edge (A2)
    (B2) edge (A2)
    (B1) edge (B2)
    (B1) edge (A1)
    (B4) edge (A4)
    (B3) edge (A3)
    (B4) edge (B3)
    (A4) edge (A3);

  \node[below] at (A1) {$A$};
  \node[below] at (A2) {$B$};
  \node[below] at (A3) {$C$};
  \node[below] at (A4) {$D$};
  \node[above] at (B1) {$E$};
  \node[above] at (B2) {$F$};
  \node[above] at (B3) {$G$};
  \node[above] at (B4) {$H$};
  \end{tikzpicture}
\end{tcblisting}

\begin{tcblisting}{title=根据三点画弧}
  \begin{tikzpicture}
    \tkzDefPoint(1,2){A}
    \tkzDefPoint(3,4){B}
    \tkzDefPoint(2,4){C}
    \tkzCircumCenter(A,B,C)\tkzGetPoint{O}
    \tkzDrawArc(O,C)(A)
  \end{tikzpicture}
\end{tcblisting}


\begin{tcblisting}{title=字体}
  $\mathscr{ABCDEFGHIJKLMNOPQRSTUVWXYZ}$\\
  $\mathbb{ABCDEFGHIJKLMNOPQRSTUVWXYZ}$\\
  $\mathcal{ABCDEFGHIJKLMNOPQRSTUVWXYZ}$\\
  $\mathfrak{ABCDEFGHIJKLMNOPQRSTUVWXYZ}$
\end{tcblisting}

\section{pstricks}
{\black this is black.}
{\darkgray this is darkgray.}
{\gray this is gray.}
{\lightgray this is lightgray.}
{\white this is white.}

{\red this is red.}
{\green this is green.}
{\blue this is blue.}
{\cyan this is cyan.}
{\magenta this is magenta.}
{\yellow this is yellow.}

%{\psset{linecolor=green,linestyle=dotted}\psline(8,7)}

%\begin{tcblisting}
%	
%\end{tcblisting}
\newpage

