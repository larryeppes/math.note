%% LyX 2.3.1 created this file.  For more info, see http://www.lyx.org/.
%% Do not edit unless you really know what you are doing.
\documentclass[a4paper,UTF8]{article}
\usepackage[T1]{fontenc}
\usepackage{CJKutf8}
\usepackage{geometry}
\geometry{verbose,tmargin=3cm,bmargin=3cm,lmargin=2cm}
\usepackage{calc}
\usepackage{amsmath}
\usepackage{amsthm}
\usepackage{amssymb}
\usepackage{esint}

\makeatletter
%%%%%%%%%%%%%%%%%%%%%%%%%%%%%% Textclass specific LaTeX commands.
\theoremstyle{plain}
\newtheorem{thm}{\protect\theoremname}[section]
\ifx\proof\undefined
\newenvironment{proof}[1][\protect\proofname]{\par
	\normalfont\topsep6\p@\@plus6\p@\relax
	\trivlist
	\itemindent\parindent
	\item[\hskip\labelsep\scshape #1]\ignorespaces
}{%
	\endtrivlist\@endpefalse
}
\providecommand{\proofname}{Proof}
\fi
\theoremstyle{definition}
\newtheorem{defn}[thm]{\protect\definitionname}
\theoremstyle{plain}
\newtheorem{lem}[thm]{\protect\lemmaname}
\theoremstyle{plain}
\newtheorem{fact}[thm]{\protect\factname}
\theoremstyle{plain}
\newtheorem{prop}[thm]{\protect\propositionname}

%%%%%%%%%%%%%%%%%%%%%%%%%%%%%% User specified LaTeX commands.
% 如果没有这一句命令,XeTeX会出错,原因参见
% http://bbs.ctex.org/viewthread.php?tid=60547
\DeclareRobustCommand\nobreakspace{\leavevmode\nobreak\ }
\usepackage{tkz-euclide}
\usetkzobj{all}

\def\degree{^\circ}
\def\bt{\begin{theorem}}
\def\et{\end{theorem}}
\def\bl{\begin{lemma}}
\def\el{\end{lemma}}
\def\bc{\begin{corrolary}}
\def\ec{\end{corrolary}}
\def\ba{\begin{proof}[解]}
\def\ea{\end{proof}}
\def\ue{\mathrm{e}}
\def\ud{\,\mathrm{d}}
\def\GF{\mathrm{GF}}
\def\ui{\mathrm{i}}
\def\Re{\mathrm{Re}}
\def\Im{\mathrm{Im}}
\def\Res{\mathrm{Res}}
\def\diag{\,\mathrm{diag}\,}
\def\be{\begin{equation}}
\def\ee{\end{equation}}
\def\bee{\begin{equation*}}
\def\eee{\end{equation*}}
\def\sumcyc{\sum\limits_{cyc}}
\def\prodcyc{\prod\limits_{cyc}}
\def\i{\infty}
\def\a{\alpha}
\def\b{\beta}
\def\g{\gamma}
\def\d{\delta}
\def\l{\lambda}
\def\m{\mu}
\def\t{\theta}
\def\p{\partial}
\def\wc{\rightharpoonup}
\def\sline{\noindent\rule[0.25\baselineskip]{\textwidth}{1pt}}

%%%%%%%%%%%%%%%%%%%%%%%%%%%%%%%%%%%%%%%%%%%%%%%%%%%%%%%%%%%%%%%%%%%%%%%%%%%%%%%%%%%
%%%%%%%%%%%%%%%%%%%%%%%%%%%%%%%%%%%%%%%%%%%%%%%%%%%%%%%%%%%%%%%%%%%%%%%%%%%%%%%%%%%
%%%%%%%%%%%%%%%%%%%%%%%%%%%%%%%%%%%%%%%%%%%%%%%%%%%%%%%%%%%%%%%%%%%%%%%%%%%%%%%%%%%

\providecommand{\abs}[1]{\left\lvert#1\right\rvert}
\providecommand{\norm}[1]{\lVert#1\rVert}
\providecommand{\paren}[1]{\left(#1\right)}
\providecommand{\bb}[1]{\mathbb{#1}}
\providecommand{\mc}[1]{\mathcal{#1}}

%%%%%%%%%%%%%%%%%%%%%%%%%%%%%%%%%%%%%%%%%%%%%%%%%%%%%%%%%%%%%%%%%%%%%%%%%%%%%%%%%%%
%%%%%%%%%%%%%%%%%%%%%%%%%%%%%%%%%%%%%%%%%%%%%%%%%%%%%%%%%%%%%%%%%%%%%%%%%%%%%%%%%%%
%%%%%%%%%%%%%%%%%%%%%%%%%%%%%%%%%%%%%%%%%%%%%%%%%%%%%%%%%%%%%%%%%%%%%%%%%%%%%%%%%%%

\newcommand{\ZZ}{\mathbb{Z}}
\newcommand{\NN}{\mathbb{N}}
\newcommand{\QQ}{\mathbb{Q}}
\newcommand{\RR}{\mathbb{R}}
\newcommand{\CC}{\mathbb{C}}
\newcommand{\TT}{\mathbb{T}}
\newcommand{\pNN}{\mathbb{N}_{+}}

\makeatother

\providecommand{\definitionname}{定义}
\providecommand{\factname}{事实}
\providecommand{\lemmaname}{引理}
\providecommand{\propositionname}{命题}
\providecommand{\theoremname}{定理}

\begin{document}
\begin{CJK}{UTF8}{gbsn}%
\title{数学笔记}
\maketitle

\section{report}

\subsection{2019-05-04}

investigate functional

\[
I:\mathcal{A}\to\RR,\omega\mapsto\int_{U}F\left(D\omega\right)\ud x.
\]
\[
\mathcal{A}\equiv\left\{ \omega\in W^{1,q}\left(U;\RR^{m}\right)\mid\omega=g\ \text{ on \ \ensuremath{\partial U}}\right\} ,\quad1<q<\infty,
\]
in which
\[
g:\partial U\to\RR^{m},
\]
and
\[
F:M^{m\times n}\to\RR
\]
be given smooth function. 

for $P\in M^{m\times n}$, $\alpha>0$, $\beta\ge0$, suppose $F\left(P\right)\ge\alpha\left|P\right|^{q}-\beta$.

$\omega\in\mathcal{A}$, so $\omega:U\to\RR^{m},\ x\mapsto\left(\omega^{1}\left(x\right),\cdots,\omega^{m}\left(x\right)\right)$.
and denote
\[
D\omega=\left(\begin{array}{ccc}
\omega_{x_{1}}^{1} & \cdots & \omega_{x_{n}}^{1}\\
\vdots & \ddots & \vdots\\
\omega_{x_{1}}^{m} & \cdots & \omega_{x_{n}}^{m}
\end{array}\right)=\frac{\partial\omega}{\partial\left(x_{1},\cdots,x_{n}\right)}.
\]

\sline

existence problem for a minimizer of $I\left[\cdot\right]$ in $\mathcal{A}$
turns the weak lower semicontinuity of $I\left[\cdot\right]$.

$I\left[\cdot\right]$在class $\mathcal{A}$ 中最小值存在问题转化为$I\left[\cdot\right]$的弱下半连续性问题.

$F$的什么样的非线性结构条件可以推出$I\left[\cdot\right]$的弱下半连续性?

\sline

suppose $u$ is a smooth minimizer, $v=\left(v^{1},\cdots,v^{m}\right)$
is Lipschitz function with compact support in $U$, then
\[
i\left(t\right)\equiv I\left[u+tv\right]=\int_{U}F\left(Du+tDv\right)\ud x
\]
with
\[
0\le i''\left(0\right)=\int_{U}\frac{\partial^{2}F}{\partial p_{i}^{k}\partial p_{j}^{l}}\left(Du\right)v_{x_{i}}^{k}v_{x_{j}}^{l}\ud x
\]
select
\[
v\left(x\right)\equiv\varepsilon\zeta\left(x\right)\rho\left(\frac{x\cdot\xi}{\varepsilon}\right)\eta
\]
send $\varepsilon\to0$
\[
\frac{\partial^{2}F}{\partial p_{i}^{k}\partial p_{j}^{l}}\left(Du\left(x\right)\right)\eta_{k}\eta_{l}\xi_{i}\xi_{j}\ge0,\ x\in U,\eta\in\RR^{m},\xi\in\RR^{n}.
\]
concern $F$ satisfy \textbf{the Hadamard-Legedre inequality}
\[
\left(\eta\otimes\xi\right)^{T}D^{2}F\left(P\right)\left(\eta\otimes\xi\right)\ge0\quad\left(P\in M^{m\times n},\eta\in\RR^{m},\xi\in\RR^{n}\right).
\]
$\eta\otimes\xi=\left(\eta_{k}\xi_{i}\right)_{m\times n}$, $P=\left(p_{j}^{i}\right)_{m\times n}$,
$D^{2}F\left(P\right)=\left(\frac{\partial^{2}F}{\partial p_{i}^{k}\partial p_{j}^{l}}\right)_{\left(m\times n\right)\times\left(m\times n\right)}$,
$F$ called \textbf{rank-one convex}.

\sline
\begin{thm}
If $F:M^{m\times n}\to\RR$ is rank-one convex, then for each $P\in M^{m\times n}$,
$\eta\in\RR^{m}$, $\xi\in\RR^{n}$, scalar function
\[
f\left(t\right)\equiv F\left(P+t\left(\eta\otimes\xi\right)\right)\quad\left(t\in\RR\right)
\]
is convex.
\end{thm}
\begin{proof}
$f''\left(t\right)=\left(\eta\otimes\xi\right)^{T}D^{2}F\left(P+t\left(\eta\otimes\xi\right)\right)\left(\eta\otimes\xi\right)\ge0$.
\end{proof}
\textbf{the reverse does not imply $F$ is convex}.

\sline

lower semi-continuous of $I\left[\cdot\right]$ implies $F$ is rank-one
convex.

$P\in M^{m\times n}$, $U=Q=\left(0,1\right)^{n}\subset\RR^{n}$,
$v\in C_{c}^{\infty}\left(Q;\RR^{m}\right)$, $k\in\NN_{+}$, 
\[
Q=\bigcup_{l=1}^{2^{kn}}Q_{l}
\]
for each $Q_{l}$ with side length $\frac{1}{2^{k}}$, centered at
$x_{l}$.
\[
u_{k}\left(x\right)=\frac{1}{2^{k}}v\left(2^{k}\left(x-x_{l}\right)\right)+Px\quad x\in Q_{l},
\]
\[
u\left(x\right)=Px
\]
then $u_{k}\rightharpoonup u$ in $W^{1,q}\left(U;\RR^{m}\right)$,
since $\norm{u_{k}-u}_{L^{p}}\ll\frac{C}{2^{k}}$, 
\[
Du_{k}-Du=\left(v_{x^{j}}^{i}\left(2^{k}\left(x-x_{l}\right)\right)\right)_{m\times n}\rightharpoonup\left(0\right)_{m\times n}=\boldsymbol{0}.
\]
we can use the following theorem

\noindent\fbox{\begin{minipage}[t]{1\columnwidth - 2\fboxsep - 2\fboxrule}%
\begin{thm}
$f$ and $g$ are continuous 1-periodic functions. Then
\[
\lim_{n\to\infty}\int_{0}^{1}f\left(x\right)g\left(nx\right)\ud x=\int_{\left[0,1\right]}f\ud x\int_{\left[0,1\right]}g\ud x.
\]
\end{thm}
%
\end{minipage}}

$u\left(x\right)=Px$ is a minimizer on $Q$ subject to its own boundary
values. If $I\left[u\right]\le\liminf_{k\to\infty}I\left[u_{k}\right]$,
then 
\[
\int_{Q}F\left(p\right)\ud x=L^{n}\left(Q\right)F\left(P\right)\le\int_{Q}F\left(P+Dv\right)\ud x.
\]

that is the following theorem:
\begin{thm}
if the functional $I\left[\cdot\right]$ is lower semicontinuous with
respect to weak convergence in $W^{1,q}\left(U;\RR^{m}\right)$, then
$F:M^{m\times n}\to\RR$ is quasiconvex, moreover, $F$ is automatically
rank-one convex.
\end{thm}
%
\begin{defn}
A function $F:M^{m\times n}\to\RR$ is called quasiconvex provided
\[
L^{n}\left(Q\right)F\left(P\right)=\int_{Q}F\left(P\right)\ud x\le\int_{Q}F\left(P+Dv\right)\ud x,\quad\forall P\in M^{m\times n},v\in C_{c}^{\infty}\left(Q;\RR^{m}\right).
\]
其中$Q$是$\RR^{n}$中的单位开立方体.
\end{defn}
%
\sline

\textbf{open question}: rank-one convex is quasiconvex function? ref{[}13{]}
and {[}93{]}.
\begin{thm}
if $F$ is quadratic, then $F$ is quasiconvex iff $F$ is rank-one
convex.
\end{thm}
%
\begin{thm}
convex function is quasiconvex and rank-one convex, by Jensen's inequality.

\[
F\left(\fint_{Q}P+Dv\ud x\right)\le\fint_{Q}F\left(P+Dv\right)\ud x.
\]
\end{thm}
%
\sline

growth condition: $0\le F\left(P\right)\le C\left(1+\left|P\right|^{q}\right)$,
$P\in M^{m\times n}$, $C$ be constant.
\[
\left|P\right|=\sum_{i=1}^{m}\sum_{j=1}^{n}\left|p_{ij}\right|
\]

\begin{lem}
$F$ is rank-one convex and verifies growth condition 
\[
0\le F\left(P\right)\le C\left(1+\left|P\right|^{q}\right),
\]
$P\in M^{m\times n}$. Then
\[
\left|DF\left(P\right)\right|\le C\left(1+\left|P\right|^{q-1}\right)\qquad P\in M^{m\times n}
\]
for $C$ be constant. 
\end{lem}
\begin{proof}
define $f\left(t\right)\equiv F\left(P+t\left(\eta\otimes\xi\right)\right)\quad\left(t\in\RR\right)$,
$\eta=e_{k}$, $\xi=e_{i}$. then $F$ rank-one convex implies $f$
is a convex function, then
\[
f'\left(0\right)\le f'\left(\delta\right)=\frac{f\left(r\right)-f\left(0\right)}{r}\le\frac{C}{r}\max_{B\left(r\right)}\left|f\right|,\quad r>0
\]
\[
f'\left(0\right)=DF\left(P\right)
\]
\[
\max_{B\left(r\right)}\left|f\right|=F\left(P+te_{ki}\right)\le\max_{B\left(r\right)}C\left(1+\left|P+te_{ki}\right|^{q}\right)\le C\left(1+\left|P\right|^{q}+r^{q}\right)
\]
select $r=1+\left|P\right|$, but not $r=\left|P\right|-1$.
\end{proof}
%
注意, 这一引理是针对rank-one convex而言的, 从而quasiconvex自然也有此引理成立.

\sline

\subsection{2019-04-05}

Concentration and sobolev inequalities

Gagliardo-Nirenberg-Sobolev inequality 可以得出$W^{1,p}\left(U\right)$嵌入到$L^{p^{*}}\left(U\right)$
for $1\le p<n$, $p^{*}=\frac{pn}{n-p}$.

$W^{1,p}\left(U\right)$可以紧嵌入到$L^{q}\left(U\right)$中, for $1\le q<p^{*}$.
\begin{thm}
(Estimates for $W^{1,p}$, $1\le p<n$)
\end{thm}
%
\begin{thm}
(\textbf{Rellich-Kondrachov Compactness Theorem}). Assume $U$ bounded
open subset of $\RR^{n}$, and $\partial U$ is $C^{1}$. Suppose
$1\le p<n$. Then
\[
W^{1,p}\left(U\right)\Subset L^{q}\left(U\right)
\]
for each $1\le q<p^{*}$.
\end{thm}
%
\begin{proof}
根据定理 Estimates for $W^{1,p}$, $1\le p<n$. and since $U$ is Bounded
and $\partial U$ is $C^{1}$, so
\[
W^{1,p}\left(U\right)\subset L^{q}\left(U\right),\quad\norm{u}_{L^{q}\left(U\right)}\le C\norm{u}_{W^{1,p}\left(U\right)}.
\]
所以只需证紧嵌入. 设$\left(u_{m}\right)_{1}^{\infty}$ 是$W^{1,p}\left(U\right)$中的有界序列,
现求它在$L^{q}\left(U\right)$中的收敛子列.

用延拓定理对函数序列$\left(u_{m}\right)$延拓到$V$上, $V$是有界开集 of $\RR^{n}$,
and
\[
\sup_{m}\norm{u_{m}}_{W^{1,p}\left(V\right)}<\infty.
\]
then smooth the functions $u_{m}^{\varepsilon}:=\eta_{\varepsilon}*u_{m}$
($\varepsilon>0$, $m=1,2,\cdots$), 
\[
u_{m}^{\varepsilon}\to u_{m}\ \text{in }L^{1}\left(V\right)\text{ as }\varepsilon\to0,\text{ uniformly in }m.
\]
这是计算技巧, 再由插值不等式
\[
u_{m}^{\varepsilon}\to u_{m}\ \text{in }L^{q}\left(V\right)\text{ as }\varepsilon\to0,\text{ uniformly in }m.
\]

然后, for each fixed $\varepsilon>0$, the sequence $\left(u_{m}^{\varepsilon}\right)_{1}^{\infty}$
is uniformly bounded and equicontinuous. (计算技术)

对序列$\left(u_{m}^{\varepsilon}\right)_{1}^{\infty}$使用Arzela-Ascoli紧性判别法,
构造出子列$\left(u_{m_{j}}\right)_{j=1}^{\infty}$满足
\[
\limsup_{j,k\to\infty}\norm{u_{m_{j}}-u_{m_{k}}}_{L^{q}\left(V\right)}\le\delta.
\]
最后取$\delta=1,\frac{1}{2},\frac{1}{3},\cdots$, 并利用对角线原理构造子列$\left(u_{m_{l}}\right)_{l=1}^{\infty}$满足
\[
\limsup_{l,k\to\infty}\norm{u_{m_{l}}-u_{m_{k}}}_{L^{q}\left(V\right)}=0.
\]
\end{proof}
但是一般从$W^{1,q}\left(U\right)$到$L^{q^{*}}\left(U\right)$的嵌入不是紧嵌入,
其中$1\le q<n$.

\subsubsection{定义}
\begin{defn}
Let $X$ and $Y$ be Banach spaces, $X\subset Y$. say $X$ is \textbf{compactly
embedded} in $Y$, $X\Subset Y,$ provided
\end{defn}
\begin{enumerate}
\item $\norm{x}_{Y}\le C\norm{x}_{X}$ ($x\in X$) for some constant $C$;
\item each bounded sequence in $X$ is precompact in $Y$.
\end{enumerate}
%
\begin{defn}
Let $\left(X,\tau\right)$ be a topological space, and let $\Sigma$
be a $\sigma$-algebra containing $\tau$ (namely, containing the
Borel $\sigma$-algebra). A collection of measures $M$ on $\Sigma$
is called \textbf{tight} if for every $\varepsilon>0$ there is compact
set $K_{\varepsilon}\subset X$ s.t. for each measure $\mu\in M$,
we have that $\left|\mu\left(X\setminus K_{\varepsilon}\right)\right|<\varepsilon$.
\end{defn}
%
\begin{defn}
Let $X$ be a non-empty set, $\mathcal{A}$ is a $\sigma$-algebra
on $X$. Given two measure $\mu$ and $\nu$ on $\mathcal{A}$, say
$\nu$ has \textbf{the Radon-Nikodym property relative to} $\mu$,
if there exists a measureable function $f:X\to[0,\infty]$, s.t.
\[
\nu\left(A\right)=\int_{A}f\ud\mu,\quad\forall A\in\mathcal{A}.
\]
say $f$ is a \textbf{density} for $\nu$ relative to $\mu$. 
\end{defn}

\subsubsection{定理}
\begin{thm}
设$n\ge3$, (为了防止$2^{*}=\infty$, 其实$2^{*}=\frac{2n}{n-2}$),
\[
\begin{array}{cc}
f_{k}\to f\ \text{ strongly in }L_{loc}^{2}\left(\RR^{n}\right), & \left|f_{k}\right|^{2^{*}}\to\nu\ \text{ in }\mathcal{M}\left(\RR^{n}\right),\\
Df_{k}\wc Df\ \text{ in }L^{2}\left(\RR^{n};\RR^{n}\right), & \left|Df_{k}\right|^{2}\wc\mu\ \text{ in }\mathcal{M}\left(\RR^{n}\right).
\end{array}
\]
则
\end{thm}
\begin{enumerate}
\item There exists an at most countable index set $J$, and distinct points
$\left\{ x_{j}\right\} _{j\in J}\subset\RR^{n}$, and nonnegative
weights $\left\{ \mu_{j},\nu_{j}\right\} _{j\in J}$ s.t.
\[
\begin{array}{cc}
\nu=\left|f\right|^{2^{*}}+\sum_{j\in J}\nu_{j}\delta_{x_{j}}, & \mu\ge\left|Df\right|^{2}+\sum_{j\in J}\mu_{j}\delta_{x_{j}}\end{array}.
\]
\item $\nu_{j}\le C_{2}^{2^{*}}\mu_{j}^{2^{*}/2}$, for all $j\in J$. here
$C_{2}$ is the optimal constant for the Gagliardo-Nirenberg-Sobolev
inequality.
\item if $f\equiv0$ and $\nu\left(\RR^{n}\right)^{1/2^{*}}\ge C_{2}\mu\left(\RR^{n}\right)^{1/2}$,
then $\nu$ is concentrated at a single point.
\end{enumerate}
这个定理是为了展示$W^{1,2}\left(U\right)$不能紧嵌入到$L^{2^{*}}\left(U\right)$中,
这个定理指明了这种非紧性的构造方法.
\begin{proof}
first assume $f\equiv0$, then $Df\equiv0$. 

$\mu,\nu$都是非负有限测度, 这是条件中的弱收敛保证的.

consider 
\[
D=\left\{ x\in\RR^{n}\mid\mu\left(\left\{ x\right\} \right)>0\right\} 
\]
then $D$ is at most countable for $\mu$ is positive finite measure.
then suppose $\left(x_{j}\right)_{j\in J}=D$, $\mu_{j}=\mu\left(\left\{ x_{j}\right\} \right)$,
for subadditivity of $\mu$
\[
\mu\ge\sum_{j\in J}\mu_{j}\delta_{x_{j}}.
\]

by $\left|f_{k}\right|^{2^{*}}\to\nu\ \text{ in }\mathcal{M}\left(\RR^{n}\right)$
and $\left|Df_{k}\right|^{2}\wc\mu\ \text{ in }\mathcal{M}\left(\RR^{n}\right)$,
for all $\phi\left(x\right)\in C_{c}^{\infty}\left(\RR^{n}\right)$,
$\phi f_{k}\in W^{1,2}\left(\RR^{n}\right)$ with compact support.
use the Gagliardo-Nirenberg-Sobolev inequality.
\[
\norm{\phi f_{k}}_{2^{*}}\le C_{2}\norm{D\left(\phi f_{k}\right)}_{2}
\]

by $f_{k}\to0$ in $L_{loc}^{2}$, so
\[
\left(\int_{\RR^{n}}\left|\phi\right|^{2^{*}}\ud\nu\right)^{1/2^{*}}\le C_{2}\left(\int_{\RR^{n}}\left|\phi\right|^{2}\ud\mu\right)^{1/2}
\]

by regularity of $\mu$ and $\nu$ 
\[
\nu\left(E\right)\le C_{2}^{2^{*}}\mu\left(E\right)^{2^{*}/2}
\]
for all $E$ to be Borel set of $\RR^{n}$. then when $\mu\left(E\right)=0$
implies $\nu\left(E\right)=0$, i.e. $\nu\ll\mu$. then 
\[
\nu\left(E\right)=\int_{E}D_{\mu}\nu\left(x\right)\ud\mu
\]
for all $E$ to be Borel set, and 
\[
D_{\mu}\nu\left(x\right)\equiv\lim_{r\to0}\frac{\nu\left(B\left(x,r\right)\right)}{\mu\left(B\left(x,r\right)\right)},\qquad\mu-a.e.\ \ x\in\RR^{n}
\]
因为$\nu\left(E\right)\le C_{2}^{2^{*}}\mu\left(E\right)^{2^{*}/2}$,
所以在$\RR^{n}\setminus D$上$\mu-a.e.$有$D_{\mu}\nu=0$. 从而可令$\nu_{j}=D_{\mu}\nu\left(x_{j}\right)\mu_{j}$.
所以
\[
\nu\left(E\right)=\int_{E}D_{\mu}\nu\left(x\right)\ud\mu=\int_{E\cap D}D_{\mu}\nu\left(x\right)\ud\mu=\sum_{x\in E\cap D}D_{\mu}\nu\left(x_{j}\right)\mu_{j}=\sum_{x\in E}\nu_{j}\delta_{x_{j}}\left(x\right)
\]

从而$\nu_{j}\le C_{2}^{2^{*}}\mu_{j}^{2^{*}/2}$. 最后证(3). 首先有$\nu\left(\RR^{n}\right)=C_{2}^{2^{*}}\mu\left(\RR^{n}\right)^{2^{*}/2}$.
用Holder不等式有
\[
\left(\int_{\RR^{n}}\left|\phi\right|^{2^{*}}\ud\nu\right)^{1/2^{*}}\le C_{2}\left(\int_{\RR^{n}}\left|\phi\right|^{2}\ud\mu\right)^{1/2}\Longrightarrow\left(\int_{\RR^{n}}\left|\phi\right|^{2^{*}}\ud\nu\right)^{1/2^{*}}\le C_{2}\mu\left(\RR^{n}\right)^{1/n}\left(\int_{\RR^{n}}\left|\phi\right|^{2^{*}}\ud\mu\right)^{1/2^{*}}
\]
\textbf{所以$\text{\frame{\ensuremath{\nu}=\ensuremath{C_{2}^{2^{*}}\mu\left(\RR^{n}\right)^{2/\left(n-2\right)}\mu}}}$.
}因此
\[
\left(\int_{\RR^{n}}\left|\phi\right|^{2^{*}}\ud\nu\right)^{1/2^{*}}\le C_{2}\left(\int_{\RR^{n}}\left|\phi\right|^{2}\ud\mu\right)^{1/2}
\]
成为\textbf{
\[
\left(\int_{\RR^{n}}\left|\phi\right|^{2^{*}}\ud\nu\right)^{1/2^{*}}\left(\nu\left(\RR^{n}\right)\right)^{1/n}\le\left(\int_{\RR^{n}}\left|\phi\right|^{2}\ud\nu\right)^{1/2}
\]
}

when $f\not\equiv0$, then $\left|Df_{k}-Df\right|^{2}=\left|Df_{k}\right|^{2}-2Df_{k}Df+\left|Df\right|^{2}\wc\left|Df_{k}\right|^{2}-\left|Df\right|^{2}\wc\mu-\left|Df\right|^{2}$
to be a positive measure, also we can get 
\[
\mu-\left|Df\right|^{2}\ge\sum_{j\in J}\mu_{j}\delta_{x_{j}}.
\]

如何证明, 
\[
\int\phi\left(\left|f_{k}\right|^{2^{*}}-\left|f_{k}-f\right|^{2^{*}}\right)\ud x\to\int\phi\left|f\right|^{2^{*}}\ud x
\]
\end{proof}

\subsubsection{问题}
\begin{enumerate}
\item (solved)为什么Sobolev函数的弱导数满足牛顿莱布尼兹公式: 
\[
\int_{U}\varphi\left(u\left(x+1\right)-u\left(x\right)\right)\ud x=\int_{U}\varphi\int_{0}^{1}Du\left(x+t\right)\ud t\ud x.
\]
\end{enumerate}

\subsubsection{反例}

\section{notes for Toro, Geometric Measure Theory - Recent Applications}

Let $\Omega\subset\RR^{n+1}$ be a bounded domain, $f\in C\left(\partial\Omega\right)$.
The classical Dirichlet problems asks whether there is a function
$u\in C\left(\overline{\Omega}\right)\cap W^{1,2}\left(\Omega\right)$
such that

\[
\begin{cases}
\Delta u=0 & in\ \Omega\\
u=f & on\ \partial\Omega.
\end{cases}
\]
Here $u\in W^{1,2}\left(\Omega\right)$ means: $u$ and its weak derivatives
are in $L^{2}\left(\Omega\right)$ and $\Delta u=0$ is interpreted
in the weak sense; that is, for any $\zeta\in C_{c}^{1}\left(\Omega\right)$,
\[
\int\left\langle \nabla u,\nabla\zeta\right\rangle =0.
\]
The questions here are whether a sol. $u$ exists, how regular it
is, whether there is a formula in terms of $f$.

say $\Omega$ is regular if for all $f\in C\left(\partial\Omega\right)$,
any sol. $u$ is in $C\left(\overline{\Omega}\right)\cap W^{1,2}\left(\Omega\right)$.

a characterization of regular domains using capacity.

the Maximum Principle $\left|u\left(x\right)\right|\le\max_{\partial\Omega}\left|f\right|$.

for $x\in\Omega$, $T_{x}:C\left(\partial\Omega\right)\to\RR$, $f\mapsto u\left(x\right)$.
\[
u\left(x\right)=\int_{\partial\Omega}f\left(q\right)\ud\omega^{x}\left(q\right).
\]


\section{BV space}

\subsection{Kreuzer - Bounded variation and Helly’s selection theorem.pdf}

全体左连续的有界变差函数不是完备的, 有界变差函数的定义不能推广到大于1维.

\section{Sobolev space}

弱导数(Weak Derivative)是一个函数的微分(强微分)概念的推广,它可以作用于那些勒贝格可积(Lebesgue Integrable)的函数,而不必预设函数的可导性(事实上大部分可以弱微分的函数并不可微)。

弱导数作用于那些勒贝格可积的函数,而不必预设函数的可微性。一个典型的勒贝格可积函数的空间是$L^{1}\left(\left[a,b\right]\right)$。在分布中,可以定义一个更一般的微分概念。

令$u$是一个在$L^{1}\left(\left[q,p\right]\right)$中的勒贝格可积的函数,称$v\in L^{1}\left(\left[q,p\right]\right)$是$u$的一个弱导数,如果
\[
\int_{q}^{p}u\left(t\right)\varphi'\left(t\right)\ud t=-\int_{q}^{p}v\left(t\right)\varphi\left(t\right)\ud t,
\]
其中$\varphi$是任意一个连续可微的函数,并且满足$\varphi\left(p\right)=\varphi\left(q\right)=0$。

推广到$n$维的情形,如果$u$和$v$是$L_{loc}^{1}\left(U\right)$中的函数(在某个开集$U\subset\RR^{n}$中局部可积),并且$\alpha$是一个多重指标,那么$v$称为$u$的$\alpha$次弱微分,如果
\[
\int_{U}uD^{\alpha}\varphi=\left(-1\right)^{\alpha}\int_{U}v\varphi,
\]
其中$\varphi\in C_{c}^{\infty}\left(U\right)$是一个任意给定的函数,即给定的支撑集含于$U$的无穷可微的函数。

如果$u$的弱导数存在,一般被记为$D^{\alpha}u$。可以证明,一个函数的弱微分在测度意义是唯一的,即如果有两个不同的弱导数,其仅可能在一个零测集上存在差异.

\section{Holder space}

\section{Lorentz space}

\section{Lipschitz space}

\section{Functional Analysis}

\subsection{稠密性}
\begin{thm}
\textbf{\cite[Appendix, C.4. Theorem 6]{key-1}} If $1\le p<\infty$
and $f\in L_{loc}^{p}\left(U\right)$, $U$ is open set of $\RR^{n}$,
then there is smooth functions $f^{\epsilon}\to f$ in $L_{loc}^{p}\left(U\right)$.

If $f\in C\left(U\right)$, then there is smooth functions such that
$f^{\epsilon}\rightrightarrows f$ on compact subsets of $U$.

for any function $f$, there is smooth functions such that $f^{\epsilon}\to f\ a.e.$
as $\epsilon\to0$.
\end{thm}
\begin{fact}
continuous functions not dense in $L^{\infty}$, i.e. the Heaviside
function.
\end{fact}

\subsection{空间关系}
\begin{prop}
for any bounded open set $U$, there is 
\[
C\left(\overline{U}\right)\subset Lipchitz\subset Holder\subset W^{k,p}.
\]
\end{prop}

\section{论文}

\subsection{古典导数与弱导数}

有弱导数不一定有古典导数. 几乎处处有古典导数, 不一定有弱导数: 除一点有跳跃间断点的$C^{1}$函数均无弱导数. 

没有跳跃间断点, 几乎处处可导函数不一定有弱导数. $I=\left(-1,1\right)$.
\begin{lem}
设函数$f$, 对任意的$\phi$, 有
\[
\int_{I}f\left(x\right)\phi\left(x\right)\ud x=\phi\left(0\right),
\]
则$f$在$I$是非局部Lebesgue可积函数.
\end{lem}
%
\begin{lem}
设函数$f,g\in L_{loc}^{1}\left(I\right)$且在分布意义下相等, 即对于任意的$\phi$,
\[
\int_{I}f\phi\ud x=\int_{I}g\phi\ud x,
\]
则$f$与$g$几乎处处相等. 反之, 若$f$与$g$几乎处处相等且其中之一属于$L_{loc}^{1}\left(I\right)$,
则另一个也属于$L_{loc}^{1}\left(I\right)$且分布意义下必相等.
\end{lem}
%
\begin{lem}
设连续函数$u$几乎处处可导且导函数$\frac{\ud u}{\ud x}\in L_{loc}^{1}\left(I\right)$,
则$u$弱可导, 弱导数$u'$几乎处处等于$\frac{\ud u}{\ud x}$.
\end{lem}
%
\begin{lem}
\label{L7.4}设$u$存在弱导数, 那么存在函数$\widetilde{u}\in C\left(I\right)$,
使得在$I$上几乎处处有$u=\widetilde{u}$且对于任意的$x,y\in I$, 有
\[
\widetilde{u}\left(x\right)-\widetilde{u}\left(y\right)=\int_{y}^{x}u'\left(t\right)\ud t.
\]
\end{lem}
%
\begin{lem}
引理\ref{L7.4}中连续表示$\widetilde{u}$存在弱导数, 且弱导数$\widetilde{u}'=u'\ a.e.$
\end{lem}
%
\begin{lem}
$u$存在弱导数, 则存在零测集$I_{0}$, 使得$u$在$I\setminus I_{0}$上是相对连续的且当$x\in I\setminus I_{0}$,
$u\left(x\right)=\widetilde{u}\left(x\right)$.
\end{lem}
%
\begin{lem}
$u$存在弱导数, 则连续表示$\widetilde{u}$的不可导点是零测集且$\frac{\ud\widetilde{u}}{\ud x}\left(x\right)=u'\left(x\right)\ a.e.$
\end{lem}
%
\begin{lem}
设$u$在点$x_{0}$古典意义下可导, 则$\widetilde{u}$在点$x_{0}$也古典意义下可导且$\frac{\ud\widetilde{u}}{\ud x}\left(x_{0}\right)=\frac{\ud u}{\ud x}\left(x_{0}\right)$.
\end{lem}
%
\begin{thm}
设函数$u$几乎处处可导且$u$弱可导, 则$u'=\frac{\ud u}{\ud x}\ a.e.$
\end{thm}
%
\begin{thm}
设函数$u$定义在$I$上, 则$u$存在弱导数的充要条件是存在连续表示$\widetilde{u}\in C\left(I\right)$,
且$\widetilde{u}$古典意义下几乎处处可导, 且古典意义下的导数$\frac{\ud\widetilde{u}}{\ud x}\in L_{loc}^{1}\left(I\right)$.
\end{thm}
%
\begin{thm}
$u\in W^{1,p}\left(I\right)$的充要条件是存在连续表示$\widetilde{u}\in C\left(I\right)$,
$\widetilde{u}$古典意义下几乎处处可导, 且古典意义下的导数$\frac{\ud\widetilde{u}}{\ud x}\in L^{p}\left(I\right)$.
\end{thm}
在高维空间中上述定理不成立.

\section{术语表}
\begin{enumerate}
\item 集合函数理论
\item 全变差测度
\item 一致$p$可积性
\item 弱完备
\item Gelfand理论
\item 有界正规复值Borel测度
\item 弱可分离
\item Lipschitz空间$\Lambda_{\alpha}$, $\Lambda_{p,\alpha}$
\item Harday空间$H_{p}$
\item 光滑化子
\item BMO
\item Besov $B_{p,q}^{s}$
\item geophysical models
\item Navier Stokes equations
\item rotating fluids
\item Navier Stokes Coriolis equations ($\mathrm{NSC}_{\varepsilon}$)
\item $co\mathrm{M}$: smallest convex set containing $M$
\item $\overline{co}M$: smallest closed convex set containing $M$
\item $ext\ M$: $int\left(M^{c}\right)$
\item Frechet组合
\item Lebesgue积分的性质有哪些
\item Volterra型积分方程
\item Fredholm积分方程
\item Schwartz space
\item 古典分析的内容
\item Levy processes
\item direct limit of topological spaces
\item real-valued Radon measure and signed measures
\item upper integral
\end{enumerate}

\section{method in mathematics}

\subsection{approximation method}

\subsection{increase or decrease dimensions}

\subsection{sliding hump method}

\subsection{solve in more large space}

\subsection{biconditional}

\subsection{construct a counterexample}

\subsection{construct from nothing}

\section{reading}

\subsection{weakly compact and weakly sequentially compact}

\subsubsection{weakly compact from wikipedia}

\textbf{weakly compact cardinal}, an infinite cardinal number on which
every binary relation has an equally large homogeneous subset.

\textbf{weakly compact set}, a compact set in a space with the weak
topology.

weakly compact set, a set that has some but not all of the properties
of compact sets, for example:
\begin{enumerate}
\item \textbf{sequentially compact space}, a set in which every infinite
sequence has a convergent subsequence.
\item \textbf{limit point compact}, a set in which every infinite subset
of $X$ has a limit point.
\end{enumerate}

\subsubsection{compactness in weak{*} topology from MSE/27423}

$X$ be a Banach space, $X^{*}$ be dual space. 

\textbf{Under the weak{*} topology, do compactness and sequential
compactness coincide?}

that is: If a subset of $X^{*}$ weakly{*} compact iff it is weakly{*}
sequentially compact?

\textbf{Is the weak{*} topology on $X^{*}$ Hausdorff? Is the weak
topology on $X$ Hausdorff?}

\paragraph{If a subset of $X^{*}$ is weakly{*} compact, then it is weakly{*}
closed. If a subset of $X$ is weakly compact, then it is weakly closed.}

Let $\mathcal{F}\subset\mathcal{F}'$ 


\subsection{Fourier transforms}

Fourier transform of functions defined on some Abelian group.

denote $\bb T=\RR/\ZZ$, $1\le p\le\infty$, function $f$ defined
on $\bb T$ can be thought of a function defined on $\bb R$ with
$f\left(t+1\right)=f\left(t\right)$ for all $t\in\bb R$. The space
$L^{p}\left(\bb T;\bb C\right)$ can be identified with the spaces
$L^{p}\left(\left[0,1\right];\bb C\right)$, but $\mc C\left(\bb T;\bb C\right)$
is not the same space as $\mc C\left(\left[0,1\right];\bb C\right)$.
\begin{defn}
If $f\in L^{1}\left(\bb T;\bb C\right)$, (or $f\in L^{1}\left(\left[0,1\right];\bb C\right)$)
then its Fourier transform is the sequence $\hat{f}$ defined by
\[
\hat{f}\left(k\right)=\int_{0}^{1}\ue^{-\ui2\pi kt}f\left(t\right)\ud t,\qquad k\in\bb Z.
\]
If $F\in L^{1}\left(\bb R;\bb C\right)$, then its Fourier transform
is the function $\hat{F}$ defined by 
\[
\hat{F}\left(\omega\right)=\int_{\RR}\ue^{-\ui2\pi\omega t}F\left(t\right)\ud t,\qquad\omega\in\RR.
\]
If $\phi\in l^{1}\left(\ZZ;\bb C\right)$, then its Fourier transform
is the function $\hat{\phi}$ defined by 
\[
\hat{\phi}\left(\omega\right)=\sum_{k\in\ZZ}\ue^{-2\pi\ui\omega k}\phi_{k},\qquad\omega\in\RR.
\]
\end{defn}
\begin{thm}
(Riemann-Lebesgure lemma)
\end{thm}
\begin{enumerate}
\item If $f\in L^{1}\left(\bb T;\bb C\right)$, then $\hat{f}\in c_{0}\left(\bb Z;\bb C\right)$.
\item If $F\in L^{1}\left(\bb R;\bb C\right)$, then $\hat{F}\in C_{0}\left(\bb R;\bb C\right)$.
\item If $\phi\in l^{1}\left(\bb Z;\bb C\right)$, then $\hat{\phi}\in C\left(\bb T;\bb C\right)$.
\end{enumerate}
\begin{defn}
\end{defn}

\subsection{questions to be solved}

\subsubsection{question from MSE/3170661: }

More generally, the sum of $p-1$ consecutive Fibonacci numbers is
divisible by the prime $p$ as soon as the polynomial $x^{2}-x-1$
is reducible in $F_{p}[x]$ (and $1$ is not a root, which can never
occur).

\subsubsection{note from MSE/103208}

Radon测度的一些不同定义之间的关系, Radon测度似乎是定义在不同拓扑空间上的Borel sigma代数上的, 比如Hausdorff空间,
局部紧空间, 或者局部紧Hausdorff空间. Can these definitions or most of them be
unified? if the definitions are related in some way?

From ``Measure Theory, Volumes 1-2'' by Valdimir I. Bogachev:
\begin{defn}
Let $X$ be a topological space. A Borel measure $\mu$ on $X$ is
called a Radon measure if for every $B$ in $B\left(X\right)$ and
$\varepsilon>0$, there exists a compact set $K_{\varepsilon}\subset B$
s.t. $\left|\mu\right|\left(B\setminus K_{\varepsilon}\right)<\varepsilon$. 

设$X$是一个拓扑空间, $X$上的Borel测度$\mu$称为是Radon测度, 如果对于任意的$B\in B\left(X\right)$和$\varepsilon>0$,
存在紧集$K_{\varepsilon}\subset B$使得$\left|\mu\right|\left(B\setminus K_{\varepsilon}\right)<\varepsilon$.
\end{defn}
这定义了一个有限符号测度.

\begin{defn}
From Wikipedia

On the Borel $\sigma$-algebra of a Hausdorff topological space $X$,
a measure is called a Radon measure if it is locally finite, and inner
regular.

在Hausdorff拓扑空间$X$上的Borel $\sigma$-代数, 一个测度称为是Radon测度如果它是局部有限的(紧集上有限测度),
并且是内正则的.
\end{defn}
%
\begin{defn}
From ncatlab

If $X$ is a locally compact Hausdorff topological space, a Radon
measure on $X$ is a Borel measure on $X$ that is finite on all compact
set, outer regular on all Borel sets, and inner on open sets.

$X$是局部紧Hausdorff空间, $X$上的Radon测度是一个$X$上的所有紧集上是有限Borel测度, 在所有Borel集上是外正则的,
且在开集上是内正则的.
\end{defn}
%
\begin{defn}
From planetmath

Let $X$ be a Hausdorff space. A Borel measure $\mu$ on $X$ is said
to be a Radon measure if it is finite on compact sets and inner regular
(tight).

设$X$是Hausdorff空间. $X$上的一个Borel测度$\mu$称为是Radon测度, 如果它在紧集上有限并且内正则.
\end{defn}
%
\begin{defn}
From Wikipedia's Radon measures on locally compact spaces

When the underlying measure space is a locally compact topological
space, the definition of a Radon measure can be expressed in terms
of continuous linear functionals on the space of continuous functions
with compact support.

当测度空间的底空间是局部紧拓扑空间, 则Radon测度可以表示为具有紧支集的连续函数空间上的连续线性泛函.
\end{defn}
%
在正测度情况下, 定义1和定义4是等价的. 在局部紧Hausdorff空间中定义2和定义4等价.

在第二可数的局部紧Hausdorff空间中, 每一个局部有限测度都满足定义3和4

在sigma紧的局部紧Hausdorff空间中, 定义3和4等价. 这一等价性的证明可以在 ''\textbf{Arveson -
NOTES ON MEASURE AND INTEGRATION IN LOCALLY COMPAC.pdf}'' 和 ''\textbf{Integral
representation theory: applications to convexity, Banach spaces and
potential theory}''中找到, 而避免使用Riesz定理.

一般定义3和4是不等价的, 甚至是在局部紧度量空间中.

在局部紧Hausdorff空间中, 以下存在双射
\begin{enumerate}
\item 满足定义3的测度
\item 满足定义4的测度
\item 有紧支集的连续函数空间上的正线性泛函
\end{enumerate}
Riesz表示定理给出1等价于3或者2等价于3, 1 and 2 equivalent is in the Schwarz book
mentioned by Joe Lucke; see also Ex 7.14 of ``\textbf{G.B. Folland,
Real Analysis: Modern Techniques and Their Applications}''. 在这个文献中,
Radon测度是按照定义3给出的. 

Do you have an example of a locally compact $\sigma$-compact (non-second
countable) space which admits a locally finite measure which is not
Radon?

For a finite measure on a compact space which is not Radon, I think
you can take the measure $\mu$ on $\left\{ 0,1\right\} ^{\RR}$ s.t.
$\mu\left(A\right)=1$ if $A$ contains $\left\{ 0\right\} ^{S}\times\left\{ 0,1\right\} ^{\RR\setminus S}$
for some countable set $S$, and $\mu\left(A\right)=0$ otherwise.

cylindrical measures?

Schwartz (Radon measures on arbitrary topological spaces and cylindrical
measures, 1973) defines Radon measures as comprising two measures.

The first is the measure given in version 3 above and the second is
the essential measure defined as locally finite, tight measure. He
then shows that each can generate the other. On LCH spaces, version
3 equivalent to version 5. Prinz (Regularity of Riesz measures, 1986,
Proc Amer Math Soc) calls version 3 a \textquotedbl Riesz\textquotedbl{}
measure and the locally finite, tight version a \textquotedbl Radon\textquotedbl{}
measure and refers to Schwartz to give their duality.

\subsubsection{Vitali Hahn Saks笔记}
\begin{thebibliography}{Evans}
\bibitem[Evans]{key-1}Evans, Partial differential equations.
\end{thebibliography}
\end{CJK}

\end{document}
