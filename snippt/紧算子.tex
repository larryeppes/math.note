%% LyX 2.3.6.1 created this file.  For more info, see http://www.lyx.org/.
%% Do not edit unless you really know what you are doing.
\documentclass[a4paper,UTF8]{article}
\usepackage[T1]{fontenc}
\usepackage{CJKutf8}
\usepackage{geometry}
\geometry{verbose,tmargin=2cm,bmargin=2cm,lmargin=2cm,rmargin=2cm}
\usepackage{color}
\usepackage{textcomp}
\usepackage{mathtools}
\usepackage{amsmath}
\usepackage{amsthm}
\usepackage{amssymb}
\usepackage{esint}
\usepackage[unicode=true,pdfusetitle,
 bookmarks=true,bookmarksnumbered=false,bookmarksopen=false,
 breaklinks=false,pdfborder={0 0 0},pdfborderstyle={},backref=false,colorlinks=true]
 {hyperref}

\makeatletter
%%%%%%%%%%%%%%%%%%%%%%%%%%%%%% Textclass specific LaTeX commands.
\theoremstyle{plain}
    \ifx\thechapter\undefined
      \newtheorem{prop}{\protect\propositionname}
    \else
      \newtheorem{prop}{\protect\propositionname}[chapter]
    \fi
\theoremstyle{definition}
    \ifx\thechapter\undefined
      \newtheorem{defn}{\protect\definitionname}
    \else
      \newtheorem{defn}{\protect\definitionname}[chapter]
    \fi
\theoremstyle{plain}
    \ifx\thechapter\undefined
	    \newtheorem{thm}{\protect\theoremname}
	  \else
      \newtheorem{thm}{\protect\theoremname}[chapter]
    \fi
\theoremstyle{plain}
    \ifx\thechapter\undefined
  \newtheorem{cor}{\protect\corollaryname}
\else
      \newtheorem{cor}{\protect\corollaryname}[chapter]
    \fi
\theoremstyle{definition}
    \ifx\thechapter\undefined
      \newtheorem{example}{\protect\examplename}
    \else
      \newtheorem{example}{\protect\examplename}[chapter]
    \fi
\theoremstyle{definition}
\newtheorem*{sol*}{\protect\solutionname}
\theoremstyle{definition}
    \ifx\thechapter\undefined
      \newtheorem{xca}{\protect\exercisename}
    \else
      \newtheorem{xca}{\protect\exercisename}[chapter]
    \fi
\theoremstyle{plain}
    \ifx\thechapter\undefined
      \newtheorem{lem}{\protect\lemmaname}
    \else
      \newtheorem{lem}{\protect\lemmaname}[chapter]
    \fi

%%%%%%%%%%%%%%%%%%%%%%%%%%%%%% User specified LaTeX commands.
% 如果没有这一句命令,XeTeX会出错,原因参见
% http://bbs.ctex.org/viewthread.php?tid=60547
\DeclareRobustCommand\nobreakspace{\leavevmode\nobreak\ }
% \usepackage{tkz-euclide}
% \usetkzobj{all}

\usepackage{amsmath, amsthm, amsfonts, amssymb, mathtools, yhmath, mathrsfs}
% http://ctan.org/pkg/extarrows
% long equal sign
\usepackage{extarrows}

\DeclareMathOperator{\sech}{sech}

%\everymath{\color{blue}\everymath{}}
\everymath\expandafter{\color{blue}\displaystyle}
%\everydisplay\expandafter{\the\everydisplay \color{red}}

\def\degree{^\circ}
\def\bt{\begin{theorem}}
\def\et{\end{theorem}}
\def\bl{\begin{lemma}}
\def\el{\end{lemma}}
\def\bc{\begin{corrolary}}
\def\ec{\end{corrolary}}
\def\ba{\begin{proof}[解]}
\def\ea{\end{proof}}
\def\ue{\mathrm{e}}
\def\ud{\,\mathrm{d}}
\def\GF{\mathrm{GF}}
\def\ui{\mathrm{i}}
\def\Re{\mathrm{Re}}
\def\Im{\mathrm{Im}}
\def\uRes{\mathrm{Res}}
\def\diag{\,\mathrm{diag}\,}
\def\be{\begin{equation}}
\def\ee{\end{equation}}
\def\bee{\begin{equation*}}
\def\eee{\end{equation*}}
\def\sumcyc{\sum\limits_{cyc}}
\def\prodcyc{\prod\limits_{cyc}}
\def\i{\infty}
\def\a{\alpha}
\def\b{\beta}
\def\g{\gamma}
\def\d{\delta}
\def\l{\lambda}
\def\m{\mu}
\def\t{\theta}
\def\p{\partial}
\def\wc{\rightharpoonup}
\def\udiv{\mathrm{div}}
\def\diam{\mathrm{diam}}
\def\dist{\mathrm{dist}}
\def\uloc{\mathrm{loc}}
\def\uLip{\mathrm{Lip}}
\def\ucurl{\mathrm{curl}}
\def\usupp{\mathrm{supp}}
\def\uspt{\mathrm{spt}}

%%%%%%%%%%%%%%%%%%%%%%%%%%%%%%%%%%%%%%%%%%%%%%%%%%%%%%%%%%%%%%%%%%%%%%%%%%%%%%%%%%%
%%%%%%%%%%%%%%%%%%%%%%%%%%%%%%%%%%%%%%%%%%%%%%%%%%%%%%%%%%%%%%%%%%%%%%%%%%%%%%%%%%%
%%%%%%%%%%%%%%%%%%%%%%%%%%%%%%%%%%%%%%%%%%%%%%%%%%%%%%%%%%%%%%%%%%%%%%%%%%%%%%%%%%%

\providecommand{\abs}[1]{\left\lvert#1\right\rvert}
\providecommand{\norm}[1]{\left\Vert#1\right\Vert}
\providecommand{\paren}[1]{\left(#1\right)}

%%%%%%%%%%%%%%%%%%%%%%%%%%%%%%%%%%%%%%%%%%%%%%%%%%%%%%%%%%%%%%%%%%%%%%%%%%%%%%%%%%%
%%%%%%%%%%%%%%%%%%%%%%%%%%%%%%%%%%%%%%%%%%%%%%%%%%%%%%%%%%%%%%%%%%%%%%%%%%%%%%%%%%%
%%%%%%%%%%%%%%%%%%%%%%%%%%%%%%%%%%%%%%%%%%%%%%%%%%%%%%%%%%%%%%%%%%%%%%%%%%%%%%%%%%%

\newcommand{\FF}{\mathbb{F}}
\newcommand{\ZZ}{\mathbb{Z}}
\newcommand{\WW}{\mathbb{W}}
\newcommand{\NN}{\mathbb{N}}
\newcommand{\PP}{\mathbb{P}}
\newcommand{\QQ}{\mathbb{Q}}
\newcommand{\RR}{\mathbb{R}}
\newcommand{\TT}{\mathbb{T}}
\newcommand{\CC}{\mathbb{C}}
\newcommand{\pNN}{\mathbb{N}_{+}}
\newcommand{\cZ}{\mathcal{Z}}
\newcommand{\cS}{\mathcal{S}}
\newcommand{\cX}{\mathcal{X}}
\newcommand{\cW}{\mathcal{W}}

\newcommand{\eqdef}{\xlongequal{\text{def}}}%
\newcommand{\eqexdef}{\xlongequal[\text{存在}]{\text{记为}}}%

\makeatother

\providecommand{\corollaryname}{推论}
\providecommand{\definitionname}{定义}
\providecommand{\examplename}{例}
\providecommand{\exercisename}{练习}
\providecommand{\lemmaname}{引理}
\providecommand{\propositionname}{命题}
\providecommand{\solutionname}{解答}
\providecommand{\theoremname}{定理}

\begin{document}
\begin{CJK}{UTF8}{gbsn}%
\title{紧算子}

\maketitle
紧算子是一种特殊的线性算子.

为什么研究紧算子?

一个主要原因是, 紧算子可以用有限秩算子(finite rank operator)逼近, 便于对算子方程的解做数值逼近.
\begin{prop}
若$H$是无穷维Hilbert空间, 算子$A\in CL(H)$, 且$\norm{A}<1$, 则对于任何$y\in H$,
存在唯一的$x\in H$满足
\[
(I-A)x=y.
\]
这个解$x$可以由Neumann级数
\[
x=(I-A)^{-1}y=(I+A+A^{2}+\cdots)y
\]
给出.
\end{prop}
上面的解有两种缺点:

1. 计算$A^{n}$是不现实的.

2. 级数收敛速度不理想.

\section{紧算子}
\begin{defn}
(紧算子) 设$X,Y$是赋范空间, 线性变换$T:X\to Y$称为是紧算子, 若对于任何有界序列$(x_{n})_{n\in\NN}\subseteq X$,
序列$(Tx_{n})_{n\in\NN}$有收敛子列.
\end{defn}
记从$X$到$Y$的所有紧算子形成集合为$K(X,Y)$.
\begin{thm}
设$X,Y$是赋范空间, $T:X\to Y$是线性变换, 则以下命题等价:

(1). $T$是紧的.

(2). $\overline{T(B)}$是紧的, 其中$B$是$X$中的单位球, 即
\[
B\coloneqq\left\{ x\in X:\norm{x}\le1\right\} .
\]
\end{thm}
\begin{proof}
(1)$\Longrightarrow$(2). 也就是证$\overline{T(B)}$中的序列$(z_{n})_{n\in\NN}$有收敛子列.

(2)$\Longrightarrow$(1). 即证$X$中有界序列$(x_{n})$, 使$(Tx_{n})_{n\in\NN}$有收敛子列.
\end{proof}

\section{紧算子集$K(X,Y)$}
\begin{cor}
$K(X,Y)\subseteq CL(X,Y)$. 
\end{cor}
\begin{proof}
紧算子将$X$中的单位球映为$Y$中的紧集, 从而是$Y$中的有界集, 即这紧算子是有界算子, 线性有界算子是连续的.
\end{proof}
\begin{example}
不是所有连续线性变换都是紧的.
\end{example}
\begin{sol*}
设$X$为任一无穷维内积空间, 比如$l^{2}$. 其上的恒等算子$I\in CL(X)$不是紧算子.

因为$I$映$X$中正交基$(u_{n})_{n\in\NN}$的像没有收敛子列.
\end{sol*}
\begin{defn}
称算子$T$是有限秩算子(finite rank operator), 如果它的值域$\mathrm{ran}(T)$是有限维向量空间.
\end{defn}
\begin{thm}
设$X$是赋范空间, $Y$是内积空间, 若$T\in CL(X,Y)$使$\mathrm{ran}(T)$是有限维的, 则$T$是紧算子.
\end{thm}
\begin{proof}
设有界序列$(x_{n})_{n\in\NN}\subseteq X$, $u_{1},\cdots,u_{m}$为$\mathrm{ran}(T)$的正交基.

则$\left(\left\langle Tx_{n},u_{l}\right\rangle \right)_{n\in\NN}$对于任何$l$有界,
类似聚点定理的证明.
\end{proof}
\begin{example}
$L(\CC^{n},\CC^{m})=CL(\CC^{n},\CC^{m})=K(\CC^{n},\CC^{m})$.
\end{example}
\begin{sol*}
若$A\in\CC^{n\times m}$, 则$T_{A}:\CC^{n}\to\CC^{m}$, $x\mapsto Ax$满足$T_{A}\in CL(\CC^{n},\CC^{m})$,
且因为$\mathrm{ran}T_{A}\subseteq\CC^{m}$, $T_{A}$是有限秩的. 故$T_{A}$是紧算子.

特别地, 恒等算子$I:\CC^{d}\to\CC^{d}$是紧算子.
\end{sol*}
%
\begin{thm}
$K(X,Y)$是$CL(X,Y)$的子空间, 其中$X,Y$均为赋范空间.
\end{thm}
\begin{proof}
1. $0$是紧算子, 因为对于任何有界序列$(x_{n})\subseteq X$, $(0x_{n})_{n\in\NN}=(0)_{n\in\NN}$是收敛的.

2. 若$T,S$均是紧算子, 设$(x_{n})_{n\in\NN}$有界, 则$(Tx_{n})_{n\in\NN}$有子列$(Tx_{n_{k}})_{k\in\NN}$收敛.

$(Sx_{n_{k}})_{k\in\NN}$有子列$(Sx_{n_{k_{l}}})_{l\in\NN}$收敛.

故子列$\left((T+S)x_{n_{k_{l}}}\right)_{l\in\NN}$收敛, 即说明$T+S$是紧算子.

3. 若$T$是紧算子, $\alpha\in\mathbb{K}$, $(x_{n})_{n\in\NN}\in X$为有界序列,
则$(Tx_{n})_{n\in\NN}$有收敛子列$(Tx_{n_{k}})_{k\in\NN}$.

因此$\left((\alpha T)x_{n_{k}}\right)_{k\in\NN}=\left(\alpha(Tx_{n_{k}})\right)_{k\in\NN}$收敛,
故$\alpha T$是紧算子.
\end{proof}
\begin{thm}
设$X$是赋范空间, $Y$是Banach空间, $(T_{n})_{n\in\NN}\subseteq K(X,Y)$在赋范空间$CL(X,Y)$中收敛于$T\in CL(X,Y)$.
则$T$是紧算子, 即$T\in K(X,Y)$, $K(X,Y)$是$CL(X,Y)$的闭线性子空间.
\end{thm}
\begin{proof}
1. 设$\left(x_{n}\right)_{n\in\NN}$是$X$中有界序列, 则由$(T_{n})_{n\in\NN}\subseteq K(X,Y)$,
知

$\left(T_{1}x_{n}\right)_{n\in\NN}$有收敛子列$\left(T_{1}x_{n}^{(1)}\right)_{n\in\NN}$;

$\left(T_{2}x_{n}^{(1)}\right)_{n\in\NN}$有收敛子列$\left(T_{2}x_{n}^{(2)}\right)_{n\in\NN}$;

...

考虑序列$x_{1}$, $x_{2}^{(1)}$, $x_{3}^{(2)}$, $\cdots$, 则
\[
\left\{ x_{k+1}^{(k)},x_{k+2}^{(k+1)},x_{k+3}^{(k+2)},\cdots\right\} \text{是序列}\left\{ x_{k+1}^{(k)},x_{k+2}^{(k)},x_{k+3}^{(k)},\cdots\right\} \text{的子列.}
\]

因$\left(T_{k}x_{n}^{(k)}\right)_{n\in\NN}$收敛, 从而$\left(T_{k}x_{k+n+1}^{(k+n)}\right)_{n\in\NN}$也收敛,
即$\left(T_{k}x_{n+1}^{(n)}\right)_{n\in\NN}$收敛.

2. 对于任意的$m,n\in\NN$, 有
\begin{align*}
\norm{Tx_{n+1}^{(n)}-Tx_{m+1}^{(m)}} & \le\norm{Tx_{n+1}^{(n)}-T_{k}x_{n+1}^{(n)}}+\norm{T_{k}x_{n+1}^{(n)}-T_{k}x_{m+1}^{(m)}}+\norm{T_{k}x_{m+1}^{(m)}-Tx_{m+1}^{(m)}}\\
 & \le\norm{T-T_{k}}\cdot\norm{x_{n+1}^{(n)}}+\norm{T_{k}x_{n+1}^{(n)}-T_{k}x_{m+1}^{(m)}}+\norm{T_{k}-T}\cdot\norm{x_{m+1}^{(m)}}\\
 & \to0,\qquad m,n\to\infty.
\end{align*}
故$\left(Tx_{n+1}^{(n)}\right)_{n\in\NN}$在$Y$中是Cauchy列, 由$Y$是Banach空间,
它在$Y$中收敛.

由$\left(Tx_{n+1}^{(n)}\right)_{n\in\NN}$是$\left(Tx_{n}\right)_{n\in\NN}$的子列,
故$T$是紧算子.
\end{proof}
\begin{cor}
设$X$是赋范空间, $Y$是Hilbert空间, $\left(T_{n}\right)_{n\in\NN}\subseteq CL(X,Y)$是有限秩算子,
且在$CL(X,Y)$中收敛于$T$, 则$T$是紧算子.
\end{cor}
\begin{example}
(什么时候$l^{2}$中的对角算子是紧算子)

设$X,Y=l^{2}$, $\left(\lambda_{n}\right)_{n\in\NN}$在域$\mathbb{K}$中有界,
$\Lambda\in CL(l^{2})$定义为
\[
\Lambda=\mathrm{diag}(\lambda_{1},\lambda_{2},\lambda_{3},\cdots).
\]
则$\norm{\Lambda}=\sup_{n\in\NN}\left|\lambda_{n}\right|$, 下面证明$\Lambda$是紧算子当且仅当$\lim_{n\to\infty}\lambda_{n}=0$.
\end{example}
\begin{sol*}
$\Longleftarrow$: 取$n\in\NN$, 算子$\Lambda_{n}=\mathrm{diag}(\lambda_{1},\cdots,\lambda_{n},0,\cdots)\in CL(l^{2})$,
则$\Lambda_{n}$是有限秩算子, $\mathrm{ran}\Lambda_{n}\subseteq\mathrm{span}\left\{ e_{1},\cdots,e_{n}\right\} $.

从而$\Lambda_{n}$是紧算子, 而
\[
\norm{\Lambda-\Lambda_{n}}=\norm{\mathrm{diag}(0,\cdots,0,\lambda_{n+1},\lambda_{n+2},\cdots)}=\sup_{k:k>n}\left|\lambda_{k}\right|\to0,\quad n\to\infty.
\]
即$\Lambda$是紧算子序列$\left(\Lambda_{n}\right)_{n\in\NN}$的强极限(一致极限, uniform
limit), 故$\Lambda$是紧算子.

$\Longrightarrow$: 反证, 设$\Lambda$是紧算子, 但存在$\epsilon>0$, 使对于任意的$N\in\NN$,
存在$n>N$使得$\left|\lambda_{n}\right|\ge\epsilon$.

取$N=1$, 则存在$n_{1}>1$, 使得$\left|\lambda_{n_{1}}\right|\ge\epsilon$.

取$N=n_{1}$, 则存在$n_{2}>n_{1}$, 使得$\left|\lambda_{n_{2}}\right|\ge\epsilon$.

...

则有$\left(\lambda_{n}\right)_{n\in\NN}$的子列$\left(\lambda_{n_{k}}\right)_{k\in\NN}$,
使$\left|\lambda_{n_{k}}\right|\ge\epsilon$, 对于任何$k\in\NN$. 于是$\left(\Lambda e_{n_{k}}\right)_{k\in\NN}=\left(\lambda_{n_{k}}e_{n_{k}}\right)_{k\in\NN}$没有收敛子列.

与$\Lambda$是紧算子矛盾.
\end{sol*}
\begin{xca}
(Hilbert Schmidt算子是紧算子)

设$H$是Hilbert空间, 有标准正交基$\left\{ u_{1},u_{2},u_{3},\cdots\right\} $. 

设$T\in CL(H)$是Hilbert-Schmidt算子, 即满足$\sum_{n=1}^{\infty}\norm{Tu_{n}}^{2}<+\infty$.

(1) 若$m\in\NN$, 则定义$T_{m}:H\to H$, $x\mapsto\sum_{n=1}^{m}\left\langle x,u_{n}\right\rangle Tu_{n}$.
则$T_{m}\in CL(H)$且满足
\[
\norm{(T-T_{m})x}^{2}\le\norm{x}^{2}\sum_{n=m+1}^{\infty}\norm{Tu_{n}}^{2}.
\]
这是因为$x=\sum_{n}\left\langle x,u_{n}\right\rangle u_{n}$, $Tx=\sum_{n}\left\langle x,u_{n}\right\rangle Tu_{n}$.

(2) 证明每个Hilbert-Schmidt算子$T$是紧算子.

Hint: $T$是有限秩算子序列$\left(T_{m}\right)_{m\in\NN}$的强极限.
\end{xca}
%
\begin{xca}
设$H$是Hilbert空间, $x_{0},y_{0}\in H$给定, 定义$x_{0}\otimes y_{0}:H\to H$,
$x\mapsto\left\langle x,y_{0}\right\rangle x_{0}$.

(1) 证明$x_{0}\otimes y_{0}\in CL(H)$, 且$\norm{x_{0}\otimes y_{0}}\le\norm{x_{0}}\cdot\norm{y_{0}}$.

(2) $x_{0}\otimes y_{0}$是否为紧算子.

(3) 设$A,B\in CL(H)$, 证明$A(x_{0}\otimes y_{0})B=(Ax_{0})\otimes(B^{*}y_{0})$.
\end{xca}
\begin{defn}
(代数中的理想)

代数$R$中的理想$I$是$R$的一个子集, 且满足:

(I1). $0\in I$.

(I2). 若$a,b\in I$, 则$a+b\in I$.

(I3). 若$a\in I$, $r\in R$, 则$ar\in I$且$ra\in I$.
\end{defn}
\begin{thm}
\label{thm:5.5} 设$H$是Hilbert空间, 则

(1). 若$T\in K(H)$是紧算子, $S\in CL(H)$, 则$TS$是紧算子.

(2). 若$T\in CL(H)$是紧算子, 则$T^{*}$是紧算子.

(3). 若$T\in CL(H)$是紧算子, $S\in CL(H)$, 则$ST$是紧算子.

即$K(H)$是$CL(H)$的闭理想.
\end{thm}
\begin{proof}
(2). 用(1)知, $TT^{*}$是紧算子, 并注意以下不等式
\begin{align*}
0\le\norm{T^{*}x_{m}-T^{*}x_{n}}^{2} & =\left\langle TT^{*}(x_{m}-x_{n}),(x_{m}-x_{n})\right\rangle \\
 & \le\norm{TT^{*}(x_{m}-x_{n})}\cdot\norm{x_{m}-x_{n}}.
\end{align*}

(3). 
\[
\left.\begin{array}{c}
T\text{紧}\xrightarrow{(2)}T^{*}\text{紧}\\
S\in K(H)\longrightarrow S^{*}\in CL(H)
\end{array}\right\} \xrightarrow{(1)}T^{*}S^{*}\text{紧}\xrightarrow{(2)}(T^{*}S^{*})^{*}=S^{**}T^{**}=ST\text{紧}.
\]
\end{proof}
\begin{example}
(无穷维Hilbert空间中的紧算子不可逆)

设$H$是无穷维Hilbert空间, $T\in K(H)$, 若$T\in CL(H)$可逆, 则$T^{-1}\in CL(H)$,
由定理\ref{thm:5.5}知
\[
I=TT^{-1}\in K(H).
\]
但$I$在$H$中并不是紧算子, 矛盾.
\end{example}
\begin{xca}
设$T\in CL(H)$, $H$是无穷维Hilbert空间.

\textcolor{red}{(1). 举例$H$和$T$, 使得$T^{2}$是$H$上的紧算子, 但$T$不是紧算子.}

(2). 证明: 若$T$是自伴的, $T^{2}$是紧算子, 则$T$是紧算子. (Hint: 才用定理\ref{thm:5.5}中(2)的证明.)
\end{xca}
%
\begin{xca}
设$H$是无穷维Hilbert空间, $S,T\in CL(H)$, 判断以下命题是否正确:

(1). 若$S,T$均是紧算子, 则$S+T$是紧算子. \textsurd{}

(2). 若$S+T$紧, 则$S$或$T$紧. \texttimes{}

(3). 若$S$或$T$紧, 则$ST$紧. \textsurd{}

(4). 若$ST$紧, 则$S$紧或$T$紧. \texttimes{}
\end{xca}
%
\begin{xca}
设$H$是Hilbert空间, $A\in CL(H)$, 定义$\Lambda\in CL(CL(H))$: $CL(H)\to CL(H)$,
$T\mapsto A^{*}T+TA$. 证明$CL(H)$的子空间$K(H)$是$\Lambda$-不变子空间, 即$\Lambda K(H)\subseteq K(H)$.

Hint: 用定理\ref{thm:5.5}
\end{xca}

\section{紧算子的逼近}

考虑方程$(I-K)x=y$, 其中$K$是Hilbert空间$H$上给定的算子, $y\in H$给定, 求$x\in H$的问题.

用有限秩算子逼近, 设$K_{0}$与$K$接近, $y_{0}$与$y$很接近, 求$x_{0}$的问题$(I-K_{0})x=y_{0}$是容易的.

下面估计$\norm{x-x_{0}}$的大小.
\begin{thm}
设$H$是Hilbert空间, $K\in CL(H)$使$I-K$在$CL(H)$中可逆, $K_{0}\in CL(H)$满足
\[
\epsilon\coloneqq\norm{(K-K_{0})(I-K)^{-1}}.
\]
则对于任意的$y,y_{0}\in H$, 存在唯一的$x,x_{0}\in H$满足:

(a). $(I-K)x=y$;

(b). $(I-K_{0})x_{0}=y_{0}$;

(c). $\norm{x-x_{0}}\le\frac{\norm{(I-K)^{-1}}}{1-\epsilon}\left(\epsilon\norm{y}+\norm{y-y_{0}}\right)$.
\end{thm}
\begin{proof}
由$\norm{(K-K_{0})(I-K)^{-1}}<1$, Neumann级数定理给出$I+(K-K_{0})(I-K)^{-1}$可逆,
故
\[
I-K_{0}=I-K+K-K_{0}=\left(I+(K-K_{0})(I-K)^{-1}\right)(I-K)
\]
可逆, 从而
\[
\norm{(I-K_{0})^{-1}}\le\frac{\norm{(I-K)^{-1}}}{1-\norm{(K-K_{0})(I-K)^{-1}}}=\frac{\norm{(I-K)^{-1}}}{1-\epsilon}.
\]
由
\[
(I-K)^{-1}-(I-K_{0})^{-1}=(I-K_{0})^{-1}(K-K_{0})(I-K)^{-1}
\]
有
\[
\norm{(I-K)^{-1}-(I-K_{0})^{-1}}\le\frac{\norm{(I-K)^{-1}}}{1-\epsilon}\cdot\epsilon,
\]
故(a), (b)解得$x,x_{0}\in H$存在唯一, 且
\[
x-x_{0}=\left((I-K)^{-1}-(I-K_{0})^{-1}\right)y+(I-K_{0})^{-1}(y-y_{0})
\]
有
\[
\norm{x-x_{0}}\le\frac{\epsilon\norm{(I-K)^{-1}}}{1-\epsilon}\cdot\norm{y}+\frac{\norm{(I-K)^{-1}}}{1-\epsilon}\cdot\norm{y-y_{0}}
\]
即得.
\end{proof}
\begin{thm}
(Galerkin逼近)

设$H$是Hilbert空间, $K$是$H$上的紧算子, $\left(P_{n}\right)_{n\in\NN}$为有限秩投影算子($P_{n}^{2}=P_{n}=P_{n}^{*}\in CL(H)$),
且$P_{n}$强收敛于$I$, 即对于任意的$x\in H$, $\lim_{n\to\infty}P_{n}x=x$,
则$P_{n}KP_{n}\to K$ in $CL(H)$.
\end{thm}
\begin{proof}
以此证明:

(1). $P_{n}K\to K$ in $CL(H)$, (投影逼近)(projection approximation)

(2). $KP_{n}\to K$ in $CL(H)$, (sloan approximation)

(3). $P_{n}KP_{n}\to K$ in $CL(H)$. (Galerkin逼近)

(1): 注意
\[
\norm{P_{n}x}^{2}=\left\langle P_{n}x,P_{n}x\right\rangle \le\norm{P_{n}x}\cdot\norm{x}\Longrightarrow\norm{P_{n}}\le1.
\]
反证$P_{n}K\not\to K$ in $CL(H)$, 则$\exists\epsilon>0$, s.t. $\forall N\in\NN$,
$\exists n>N$, s.t. $\norm{P_{n}K-K}>\epsilon$.

从而有$\left(x_{n_{k}}\right)_{k\in\NN}$, 使$\norm{x_{n_{k}}}\le1$,
$\norm{\left(P_{n_{k}}K-K\right)x_{n_{k}}}>\epsilon$, 由$K$紧, 取$Kx_{n_{k_{l}}}\to y$
in $H$. 故
\begin{align*}
\epsilon<\norm{(P_{n_{k}}K-K)x_{n_{k_{l}}}} & =\norm{\left(P_{n_{k_{l}}}-I\right)y+\left(P_{n_{k_{l}}}-I\right)\left(Kx_{n_{k_{l}}}-y\right)}\\
 & \le\norm{\left(P_{n_{k_{l}}}-I\right)y}+\norm{P_{n_{k_{l}}}-I}\cdot\norm{Kx_{n_{k_{l}}}-y}\to0.
\end{align*}
矛盾.

(2): 因$K$是紧算子, 故$K^{*}$紧, 由(1), $P_{n}^{*}K^{*}=P_{n}K^{*}\to K^{*}$
in $CL(H)$. 从而
\[
\norm{KP_{n}-K}=\norm{\left(KP_{n}-K\right)^{*}}=\norm{P_{n}K^{*}-K^{*}}\to0,\qquad n\to\infty.
\]

(3):
\begin{align*}
\norm{P_{n}KP_{n}-K} & =\norm{P_{n}(KP_{n}-K)+P_{n}K-K}\\
 & \le\norm{P_{n}}\cdot\norm{KP_{n}-K}+\norm{P_{n}K-K}\to0.
\end{align*}
\end{proof}

\section{紧算子的谱定理}
\begin{thm}
(紧自伴算子的谱定理)

设$H$是Hilbert空间, $T=T^{*}\in K(H)$有有限秩, 则存在正交特征向量$u_{n}$, $n\in\NN$与相应的特征向量$\lambda_{n}$,
$n\in\NN$, 使得$\lim_{n\to\infty}\lambda_{n}=0$, 且对于任意的$x\in H$,
有
\[
Tx=\sum_{n=1}^{\infty}\lambda_{n}\left\langle x,u_{n}\right\rangle u_{n}.
\]
\end{thm}
先证几个引理.
\begin{lem}
若$T=T^{*}\in CL(H)$, 则
\[
\norm{T}=\sup_{\norm{x}=1}\left|\left\langle Tx,x\right\rangle \right|.
\]
\end{lem}
\begin{proof}
设$M\coloneqq\sup_{\norm{x}=1}\left|\left\langle Tx,x\right\rangle \right|$,
则易证$M\le\norm{T}$. 下证$\norm{T}\le M$.

注意
\begin{align*}
\left\langle T(x+y),x+y\right\rangle  & =\left\langle Tx,x\right\rangle +2\mathrm{Re}\left\langle Tx,y\right\rangle +\left\langle Ty,y\right\rangle \\
\left\langle T(x-y),x-y\right\rangle  & =\left\langle Tx,x\right\rangle -2\mathrm{Re}\left\langle Tx,y\right\rangle +\left\langle Ty,y\right\rangle 
\end{align*}
故
\begin{align*}
4\mathrm{Re}\left\langle Tx,y\right\rangle  & =\left\langle T(x+y),x+y\right\rangle -\left\langle T(x-y),x-y\right\rangle \\
 & \le M\norm{x+y}^{2}+M\norm{x-y}^{2}\\
 & =2M\left(\norm{x}^{2}+\norm{y}^{2}\right).
\end{align*}
取$\theta\in\RR$, 用$\ue^{\ui\theta}y$代替$y$使
\[
\mathrm{Re}\left\langle Tx,y\right\rangle =\left\langle Tx,y\right\rangle \le\frac{M}{2}\left(\norm{x}^{2}+\norm{y}^{2}\right).
\]

当$Tx=0$, 或$x=0$时, $\norm{Tx}\le M\cdot\norm{x}$, 显然成立.

当$Tx\ne0$, 且$x\ne0$时, 则取$y\coloneqq\frac{\norm{x}}{\norm{Tx}}Tx$,
有
\[
\norm{Tx}\cdot\norm{x}\le M\cdot\norm{x}^{2}\Longrightarrow\norm{T}\le M.
\]

特别地, 当$T$是紧算子时, 存在$x\in H$, $\norm{x}=1$使
\[
\left|\left\langle Tx,x\right\rangle \right|=\norm{T}=\sup_{\norm{x}=1}\left\langle Tx,x\right\rangle .
\]
\end{proof}
\begin{lem}
\label{lem:5.2} 若$H$是非平凡Hilbert空间, $T=T^{*}\in K(H)$. 则$\norm{T}$或$-\norm{T}$之一为$T$的特征值.
\end{lem}
\begin{proof}
不妨设$T\ne0$. 由于$T$是自伴的, 故$\left\langle Tx,x\right\rangle \in\RR$,
取$\left(x_{n}\right)_{n\in\NN}$, $\norm{x_{n}}=1$, 不妨证$\left\langle Tx_{n},x_{n}\right\rangle \to\lambda=\norm{T}$,
$n\to\infty$. 则
\begin{align*}
0 & \le\norm{Tx_{n}-\lambda x_{n}}^{2}=\norm{Tx_{n}}^{2}-2\lambda\left\langle Tx_{n},x_{n}\right\rangle +\lambda^{2}\\
 & \le\norm{T}-2\lambda\left\langle Tx_{n},x_{n}\right\rangle +\lambda^{2}\to0,\quad n\to\infty.
\end{align*}

故$Tx_{n}-\lambda x_{n}\to0$, 由$T$紧, $Tx_{n_{k}}\to y$, 则$x_{n_{k}}\to\frac{y}{\lambda}$,
再由$T$连续, 
\[
y=\lim_{k\to\infty}Tx_{n_{k}}=T\frac{y}{\lambda}=\frac{1}{\lambda}Ty.
\]
而
\[
\norm{y}=\lim_{k\to\infty}\norm{\lambda x_{n_{k}}}=\lambda\ne0.
\]
\end{proof}
\begin{lem}
设$H$是Hilbert空间, $T=T^{*}\in CL(H)$, $Y$是$H$中$T$-不变闭子空间, 则

(1). $Y^{\perp}$也是$T$-不变子空间;

(2). $T$在Hilbert空间$Y^{\perp}$上的限制$T\mid_{Y^{\perp}}:Y^{\perp}\to Y^{\perp}$也是自伴的;

(3). 若$T$是紧的, 则$T\mid_{Y^{\perp}}$也紧.
\end{lem}
\begin{proof}
(1). 对于任意的$z\in Y^{\perp}$, 有$Tz\in Y^{\perp}$.

(2). $Y^{\perp}$是Hilbert空间中的闭子空间, 也是Hilbert空间.

(1)推出$Y^{\perp}$是$T$-不变子空间, 所以$T\mid_{Y^{\perp}}$是良定的.

最后由$T$自伴推出$T\mid_{Y^{\perp}}$自伴.

(3). 设$\left(z_{n}\right)_{n\in\NN}\subseteq Y^{\perp}$, $\left(z_{n}\right)_{n\in\NN}$有界,
$\left(Tz_{n_{k}}\right)\subseteq Y^{\perp}$收敛于$z\in H$.

由$Y^{\perp}$是$H$的闭子空间, 所以$z\in Y^{\perp}$.
\end{proof}
%
\begin{proof}
(谱定理的证明)

设$H\coloneqq H$, $T_{1}\coloneqq T$.

则引理\ref{lem:5.2}推出存在$\lambda_{1}$, $u_{1}$, s.t. $\left|\lambda_{1}\right|=\norm{T_{1}}$,
$\norm{u_{1}}=1$, $T_{1}u_{1}=\lambda_{1}u_{1}$.

取$H_{2}\coloneqq\left(\mathrm{span}\left\{ u_{1}\right\} \right)^{\perp}$为$H_{1}$的闭子空间,
是$T$-不变子空间, 让$T_{2}=T\mid_{H_{2}}$, 则$T_{2}$是紧自伴的.

故存在$\lambda_{2}$, $u_{2}$, s.t. $\left|\lambda_{2}\right|=\norm{T_{2}}$,
$\norm{u_{2}}=1$, $T_{2}u_{2}=\lambda_{2}u_{2}$且
\[
\left|\lambda_{2}\right|=\left|\left\langle T_{2}u_{2},u_{2}\right\rangle \right|=\left\langle Tu_{2},u_{2}\right\rangle \le\norm{T}=\left|\lambda_{1}\right|.
\]

综上, $\left\{ u_{1},u_{2}\right\} $是正交的, 且$Tu_{1}=\lambda_{1}u_{1}$,
$Tu_{2}=\lambda_{2}u_{2}$.

取$H_{3}\coloneqq\left(\mathrm{span}\left\{ u_{1},u_{2}\right\} \right)^{\perp}$为$H_{2}$的闭子空间,
是$T$-不变子空间, 让$T_{3}=T\mid_{H_{3}}$, 则$T_{3}$是紧自伴的.

故存在$\lambda_{3}$, $u_{3}$, s.t. $\left|\lambda_{3}\right|=\norm{T_{3}}$,
$\norm{u_{3}}=1$, $T_{3}u_{3}=\lambda_{3}u_{3}$, 且
\[
\left|\lambda_{3}\right|=\left|\left\langle T_{3}u_{3},u_{3}\right\rangle \right|=\left|\left\langle T_{2}u_{3},u_{3}\right\rangle \right|\le\norm{T_{2}}=\left|\lambda_{2}\right|.
\]

如此下去, 有$H_{n}\coloneqq\left(\mathrm{span}\left\{ u_{1},\cdots,u_{n-1}\right\} \right)^{\perp}$,
$H_{n}\ne0$, 否则对于任何$x\in H$,
\[
x-\sum_{k=1}^{n-1}\left\langle x,u_{k}\right\rangle u_{k}\in H_{n}=\left\{ 0\right\} .
\]
故
\[
Tx=\sum_{k=1}^{n-1}\left\langle x,u_{k}\right\rangle Tu_{k},\qquad\forall x\in H.
\]
与$T$是无穷秩算子矛盾.

下面证明$\left|\lambda_{n}\right|\to0$, $n\to\infty$. 否则
\[
\inf_{n\in\NN}\left|\lambda_{n}\right|\eqqcolon\epsilon>0,
\]
则取$n\ne m$时
\[
\norm{Tu_{n}-Tu_{m}}^{2}=\norm{\lambda_{n}u_{n}-\lambda_{m}u_{m}}^{2}=\lambda_{n}^{2}+\lambda_{m}^{2}\ge2\epsilon^{2},
\]
与$T$是紧算子矛盾.

最后证
\[
Tx=\sum_{k=1}^{\infty}\lambda_{k}\left\langle x,u_{k}\right\rangle u_{k},\qquad\forall x\in H.
\]
即, 对于任意的$x$, $x-\sum_{k=1}^{n-1}\left\langle x,u_{k}\right\rangle u_{k}\in H_{n}$与不等式
\begin{align*}
\norm{Tx-\sum_{k=1}^{n-1}\lambda_{k}\left\langle x,u_{k}\right\rangle u_{k}} & =\norm{T\left(x-\sum_{k=1}^{n-1}\left\langle x,u_{k}\right\rangle u_{k}\right)}\\
 & \le\norm{T_{n}}\cdot\norm{x-\sum_{k=1}^{n-1}\left\langle x,u_{k}\right\rangle u_{k}}\\
 & \le\left|\lambda_{n}\right|\cdot\norm{x}\to0,\qquad n\to\infty.
\end{align*}
 
\end{proof}
\begin{xca}
设$H$是无穷维Hilbert空间, $T=T^{*}\in K(H)$是双射, 则$T$的特征向量形成$H$中的一组基.
\end{xca}
\begin{proof}
只需证$T$是无穷秩的, 用线性相关性和双射反证, 然后用紧自伴算子的谱定理.
\end{proof}
\begin{xca}
设$H$是Hilbert空间, 设$T=T^{*}\in K(H)$有无穷秩, 且是正算子, 即
\[
\left\langle Tx,x\right\rangle \ge0,\qquad\forall x\in H.
\]
证明$T$有平方根, 即算子$\sqrt{T}\in CL(H)$使$\left(\sqrt{T}\right)^{2}=T$.
\end{xca}
\begin{proof}
用谱定理, $T$的所有特征根$\lambda_{n}\ge0$, 构造算子$S:H\to H$, 满足对于任意的$x\in H$,
有
\[
Sx\coloneqq\sum_{n=1}^{\infty}\sqrt{\lambda_{n}}\left\langle x,u_{n}\right\rangle u_{n},
\]
则$S^{2}=T$.
\end{proof}
\end{CJK}

\end{document}
